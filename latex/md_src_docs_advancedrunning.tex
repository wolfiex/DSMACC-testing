
\begin{DoxyEnumerate}
\item For any mechanism, when you run {\ttfamily make kpp} it will ask you what .kpp mechanism (from the ./mechanisms folder) you wish to compile.
\item Choose all mechanisms needed-\/ if multiple add a space between each selection.
\item If only one mechanism is selected, and this contains the word \textquotesingle{}organic\textquotesingle{}, then inorganic.\+kpp shall also automatically be included.
\item These is also a custom option available from the makemodeldotkpp.\+py file.
\item If you want to use the {\itshape model.\+kpp} file from the {\itshape src} folder, use `make kpp M\+O\+D\+E\+L\+K\+PP='--custom\textquotesingle{}\`{}, which will copy {\itshape model.\+kpp} from the {\itshape src} folder to the main folder.
\end{DoxyEnumerate}

\subsection*{Running with multiple models}


\begin{DoxyEnumerate}
\item Making a new model model and ensure it works
\item Type {\ttfamily make savemodel name=$<$yournamehere$>$} with what you wish to refer to your model with in the future
\item In your Initial conditions, at the description add your model name after a hyphen, e.\+g., myrun-\/mcm\+\_\+new
\item Run using the saved flag {\ttfamily ./begin.py -\/saved}
\end{DoxyEnumerate}
\begin{DoxyItemize}
\item only use a hyphen when providing a model name
\end{DoxyItemize}

\subsection*{Alias run shortcuts}

For those who do not know, aliases are shortcuts for command line functions and code snipets. For seasoned D\+S\+M\+A\+CC users it may be easier to use these shortcuts than to have to type out the full commands all the time. To use these an alias must be included in your .bashrc or .profile file. Use of combinatory aliases should however be used with caution!

Examples of some potentially useful commands are\+: 106 alias b=\char`\"{}python begin.\+py\char`\"{} 107 alias d=\char`\"{}cd $\sim$/\+D\+S\+M\+A\+C\+C-\/testing\char`\"{} 108 alias k=\char`\"{}make kpp \&\& make\char`\"{} 109 alias ics=\char`\"{}cd $\sim$/\+D\+S\+M\+A\+C\+C-\/testing/\+Init\+Cons\char`\"{} 110 alias m=\char`\"{}cd $\sim$/\+D\+S\+M\+A\+C\+C-\/testing/mechanisms\char`\"{}

\subsection*{Updating the rate constants}

Rate-\/constant simplification through the use of a symbolic engine has been applied. This reduces computation and allows the setting of parameters for fixed numeric constants (see K\+D\+EC bug). An example of some equations below.

\tabulinesep=1mm
\begin{longtabu} spread 0pt [c]{*{3}{|X[-1]}|}
\hline
\rowcolor{\tableheadbgcolor}\textbf{ Rate Cefficient  }&\textbf{ Original-\/eqn  }&\textbf{ Simplified-\/eq   }\\\cline{1-3}
\endfirsthead
\hline
\endfoot
\hline
\rowcolor{\tableheadbgcolor}\textbf{ Rate Cefficient  }&\textbf{ Original-\/eqn  }&\textbf{ Simplified-\/eq   }\\\cline{1-3}
\endhead
krd  &kd0/kdi  &5.\+79e-\/23$\ast$m$\ast$exp(4000\textbackslash{}/temp)   \\\cline{1-3}
ncd  &0.\+75-\/1.\+27$\ast$(log10(fcd))  &1.\+41   \\\cline{1-3}
kbpan  &(kd0$\ast$kdi)$\ast$fd/(kd0+kdi)  &fd$\ast$kd0$\ast$kdi\textbackslash{}/(kd0 + kdi)   \\\cline{1-3}
kr1  &k10/k1i  &3.\+32e-\/18$\ast$m$\ast$temp$\ast$$\ast$(-\/1.\+3)   \\\cline{1-3}
f1  &10$\ast$$\ast$(log10(fc1)/(1+(log10(kr1)/nc1)$\ast$$\ast$2))  &10$\ast$$\ast$(-\/0.\+07\textbackslash{}/(1 + log10(kr1)$\ast$$\ast$2\textbackslash{}/nc1$\ast$$\ast$2))   \\\cline{1-3}
kmt12  &(k120$\ast$k12i$\ast$f12)/(k120+k12i)  &2e-\/12$\ast$f12$\ast$k120\textbackslash{}/(k120 + 2e-\/12)   \\\cline{1-3}
\end{longtabu}


To do this place your new rate file in the src folder, run simplfy\+\_\+rates.\+py and update the constants.\+f90 code to inculde the newly generates .def and .var files 