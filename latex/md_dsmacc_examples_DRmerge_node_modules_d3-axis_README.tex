The axis component renders human-\/readable reference marks for \href{https://github.com/d3/d3-scale}{\tt scales}. This alleviates one of the more tedious tasks in visualizing data.

\subsection*{Installing}

If you use N\+PM, {\ttfamily npm install d3-\/axis}. Otherwise, download the \href{https://github.com/d3/d3-axis/releases/latest}{\tt latest release}. You can also load directly from \href{https://d3js.org}{\tt d3js.\+org}, either as a \href{https://d3js.org/d3-axis.v1.min.js}{\tt standalone library} or as part of \href{https://github.com/d3/d3}{\tt D3 4.\+0}. (To be useful, you’ll also want to use \href{https://github.com/d3/d3-scale}{\tt d3-\/scale} and \href{https://github.com/d3/d3-selection}{\tt d3-\/selection}, but these are soft dependencies.) A\+MD, Common\+JS, and vanilla environments are supported. In vanilla, a {\ttfamily d3} global is exported\+:


\begin{DoxyCode}
<script src="https://d3js.org/d3-axis.v1.min.js"></script>
<script>

var axis = d3.axisLeft(scale);

</script>
\end{DoxyCode}


\href{https://tonicdev.com/npm/d3-axis}{\tt Try d3-\/axis in your browser.}

\subsection*{A\+PI Reference}

Regardless of orientation, axes are always rendered at the origin. To change the position of the axis with respect to the chart, specify a \href{http://www.w3.org/TR/SVG/coords.html#TransformAttribute}{\tt transform attribute} on the containing element. For example\+:


\begin{DoxyCode}
d3.select("body").append("svg")
    .attr("width", 1440)
    .attr("height", 30)
  .append("g")
    .attr("transform", "translate(0,30)")
    .call(axis);
\end{DoxyCode}


The elements created by the axis are considered part of its public A\+PI. You can apply external stylesheets or modify the generated axis elements to \href{https://bl.ocks.org/mbostock/3371592}{\tt customize the axis appearance}.

\href{http://bl.ocks.org/mbostock/3371592}{\tt }

An axis consists of a \href{https://www.w3.org/TR/SVG/paths.html#PathElement}{\tt path element} of class “domain” representing the extent of the scale’s domain, followed by transformed \href{https://www.w3.org/TR/SVG/struct.html#Groups}{\tt g elements} of class “tick” representing each of the scale’s ticks. Each tick has a \href{https://www.w3.org/TR/SVG/shapes.html#LineElement}{\tt line element} to draw the tick line, and a \href{https://www.w3.org/TR/SVG/text.html#TextElement}{\tt text element} for the tick label. For example, here is a typical bottom-\/oriented axis\+:


\begin{DoxyCode}
<g fill="none" font-size="10" font-family="sans-serif" text-anchor="middle">
  <path class="domain" stroke="#000" d="M0.5,6V0.5H880.5V6"></path>
  <g class="tick" opacity="1" transform="translate(0.5,0)">
    <line stroke="#000" y2="6"></line>
    <text fill="#000" y="9" dy="0.71em">0.0</text>
  </g>
  <g class="tick" opacity="1" transform="translate(176.5,0)">
    <line stroke="#000" y2="6"></line>
    <text fill="#000" y="9" dy="0.71em">0.2</text>
  </g>
  <g class="tick" opacity="1" transform="translate(352.5,0)">
    <line stroke="#000" y2="6"></line>
    <text fill="#000" y="9" dy="0.71em">0.4</text>
  </g>
  <g class="tick" opacity="1" transform="translate(528.5,0)">
    <line stroke="#000" y2="6"></line>
    <text fill="#000" y="9" dy="0.71em">0.6</text>
  </g>
  <g class="tick" opacity="1" transform="translate(704.5,0)">
    <line stroke="#000" y2="6"></line>
    <text fill="#000" y="9" dy="0.71em">0.8</text>
  </g>
  <g class="tick" opacity="1" transform="translate(880.5,0)">
    <line stroke="#000" y2="6"></line>
    <text fill="#000" y="9" dy="0.71em">1.0</text>
  </g>
</g>
\end{DoxyCode}


The orientation of an axis is fixed; to change the orientation, remove the old axis and create a new axis.

\label{_axisTop}%
\# d3.{\bfseries axis\+Top}({\itshape scale}) \href{https://github.com/d3/d3-axis/blob/master/src/axis.js#L159}{\tt $<$$>$}

Constructs a new top-\/oriented axis generator for the given \href{https://github.com/d3/d3-scale}{\tt scale}, with empty \href{#axis_ticks}{\tt tick arguments}, a \href{#axis_tickSize}{\tt tick size} of 6 and \href{#axis_tickPadding}{\tt padding} of 3. In this orientation, ticks are drawn above the horizontal domain path.

\label{_axisRight}%
\# d3.{\bfseries axis\+Right}({\itshape scale}) \href{https://github.com/d3/d3-axis/blob/master/src/axis.js#L163}{\tt $<$$>$}

Constructs a new right-\/oriented axis generator for the given \href{https://github.com/d3/d3-scale}{\tt scale}, with empty \href{#axis_ticks}{\tt tick arguments}, a \href{#axis_tickSize}{\tt tick size} of 6 and \href{#axis_tickPadding}{\tt padding} of 3. In this orientation, ticks are drawn to the right of the vertical domain path.

\label{_axisBottom}%
\# d3.{\bfseries axis\+Bottom}({\itshape scale}) \href{https://github.com/d3/d3-axis/blob/master/src/axis.js#L167}{\tt $<$$>$}

Constructs a new bottom-\/oriented axis generator for the given \href{https://github.com/d3/d3-scale}{\tt scale}, with empty \href{#axis_ticks}{\tt tick arguments}, a \href{#axis_tickSize}{\tt tick size} of 6 and \href{#axis_tickPadding}{\tt padding} of 3. In this orientation, ticks are drawn below the horizontal domain path.

\label{_axisLeft}%
\# d3.{\bfseries axis\+Left}({\itshape scale}) \href{https://github.com/d3/d3-axis/blob/master/src/axis.js#L171}{\tt $<$$>$}

Constructs a new left-\/oriented axis generator for the given \href{https://github.com/d3/d3-scale}{\tt scale}, with empty \href{#axis_ticks}{\tt tick arguments}, a \href{#axis_tickSize}{\tt tick size} of 6 and \href{#axis_tickPadding}{\tt padding} of 3. In this orientation, ticks are drawn to the left of the vertical domain path.

\label{__axis}%
\# {\itshape axis}({\itshape context}) \href{https://github.com/d3/d3-axis/blob/master/src/axis.js#L40}{\tt $<$$>$}

Render the axis to the given {\itshape context}, which may be either a \href{https://github.com/d3/d3-selection}{\tt selection} of S\+VG containers (either S\+VG or G elements) or a corresponding \href{https://github.com/d3/d3-transition}{\tt transition}.

\label{_axis_scale}%
\# {\itshape axis}.{\bfseries scale}(\mbox{[}{\itshape scale}\mbox{]}) \href{https://github.com/d3/d3-axis/blob/master/src/axis.js#L120}{\tt $<$$>$}

If {\itshape scale} is specified, sets the \href{https://github.com/d3/d3-scale}{\tt scale} and returns the axis. If {\itshape scale} is not specified, returns the current scale.

\label{_axis_ticks}%
\# {\itshape axis}.{\bfseries ticks}({\itshape arguments…}) \href{https://github.com/d3/d3-axis/blob/master/src/axis.js#L124}{\tt $<$$>$} ~\newline
\href{#axis_ticks}{\tt \#} {\itshape axis}.{\bfseries ticks}(\mbox{[}{\itshape count}\mbox{[}, {\itshape specifier}\mbox{]}\mbox{]}) ~\newline
\href{#axis_ticks}{\tt \#} {\itshape axis}.{\bfseries ticks}(\mbox{[}{\itshape interval}\mbox{[}, {\itshape specifier}\mbox{]}\mbox{]})

Sets the {\itshape arguments} that will be passed to \href{https://github.com/d3/d3-scale/blob/master/README.md#continuous_ticks}{\tt {\itshape scale}.ticks} and \href{https://github.com/d3/d3-scale/blob/master/README.md#continuous_tickFormat}{\tt {\itshape scale}.tick\+Format} when the axis is \href{#_axis}{\tt rendered}, and returns the axis generator. The meaning of the {\itshape arguments} depends on the \href{#axis_scale}{\tt axis’ scale} type\+: most commonly, the arguments are a suggested {\itshape count} for the number of ticks (or a \href{https://github.com/d3/d3-time}{\tt time {\itshape interval}} for time scales), and an optional \href{https://github.com/d3/d3-format}{\tt format {\itshape specifier}} to customize how the tick values are formatted.

This method has no effect if the scale does not implement {\itshape scale}.ticks, as with \href{https://github.com/d3/d3-scale/blob/master/README.md#band-scales}{\tt band} and \href{https://github.com/d3/d3-scale/blob/master/README.md#point-scales}{\tt point} scales. To set the tick values explicitly, use \href{#axis_tickValues}{\tt {\itshape axis}.tick\+Values}. To set the tick format explicitly, use \href{#axis_tickFormat}{\tt {\itshape axis}.tick\+Format}.

For example, to generate twenty ticks with S\+I-\/prefix formatting on a linear scale, say\+:


\begin{DoxyCode}
axis.ticks(20, "s");
\end{DoxyCode}


To generate ticks every fifteen minutes with a time scale, say\+:


\begin{DoxyCode}
axis.ticks(d3.timeMinute.every(15));
\end{DoxyCode}


This method is also a convenience function for \href{#axis_tickArguments}{\tt {\itshape axis}.tick\+Arguments}. For example, this\+:


\begin{DoxyCode}
axis.ticks(10);
\end{DoxyCode}


Is equivalent to\+:


\begin{DoxyCode}
axis.tickArguments([10]);
\end{DoxyCode}


\label{_axis_tickArguments}%
\# {\itshape axis}.{\bfseries tick\+Arguments}(\mbox{[}{\itshape arguments}\mbox{]}) \href{https://github.com/d3/d3-axis/blob/master/src/axis.js#L128}{\tt $<$$>$}

If {\itshape arguments} is specified, sets the {\itshape arguments} that will be passed to \href{https://github.com/d3/d3-scale/blob/master/README.md#continuous_ticks}{\tt {\itshape scale}.ticks} and \href{https://github.com/d3/d3-scale/blob/master/README.md#continuous_tickFormat}{\tt {\itshape scale}.tick\+Format} when the axis is \href{#_axis}{\tt rendered}, and returns the axis generator. The meaning of the {\itshape arguments} depends on the \href{#axis_scale}{\tt axis’ scale} type\+: most commonly, the arguments are a suggested {\itshape count} for the number of ticks (or a \href{https://github.com/d3/d3-time}{\tt time {\itshape interval}} for time scales), and an optional \href{https://github.com/d3/d3-format}{\tt format {\itshape specifier}} to customize how the tick values are formatted.

If {\itshape arguments} is specified, this method has no effect if the scale does not implement {\itshape scale}.ticks, as with \href{https://github.com/d3/d3-scale/blob/master/README.md#band-scales}{\tt band} and \href{https://github.com/d3/d3-scale/blob/master/README.md#point-scales}{\tt point} scales. To set the tick values explicitly, use \href{#axis_tickValues}{\tt {\itshape axis}.tick\+Values}. To set the tick format explicitly, use \href{#axis_tickFormat}{\tt {\itshape axis}.tick\+Format}.

If {\itshape arguments} is not specified, returns the current tick arguments, which defaults to the empty array.

For example, to generate twenty ticks with S\+I-\/prefix formatting on a linear scale, say\+:


\begin{DoxyCode}
axis.tickArguments([20, "s"]);
\end{DoxyCode}


To generate ticks every fifteen minutes with a time scale, say\+:


\begin{DoxyCode}
axis.tickArguments([d3.timeMinute.every(15)]);
\end{DoxyCode}


See also \href{#axis_ticks}{\tt {\itshape axis}.ticks}.

\label{_axis_tickValues}%
\# {\itshape axis}.{\bfseries tick\+Values}(\mbox{[}{\itshape values}\mbox{]}) \href{https://github.com/d3/d3-axis/blob/master/src/axis.js#L132}{\tt $<$$>$}

If a {\itshape values} array is specified, the specified values are used for ticks rather than using the scale’s automatic tick generator. If {\itshape values} is null, clears any previously-\/set explicit tick values and reverts back to the scale’s tick generator. If {\itshape values} is not specified, returns the current tick values, which defaults to null. For example, to generate ticks at specific values\+:


\begin{DoxyCode}
var xAxis = d3.axisBottom(x)
    .tickValues([1, 2, 3, 5, 8, 13, 21]);
\end{DoxyCode}


The explicit tick values take precedent over the tick arguments set by \href{#axis_tickArguments}{\tt {\itshape axis}.tick\+Arguments}. However, any tick arguments will still be passed to the scale’s \href{#axis_tickFormat}{\tt tick\+Format} function if a tick format is not also set.

\label{_axis_tickFormat}%
\# {\itshape axis}.{\bfseries tick\+Format}(\mbox{[}{\itshape format}\mbox{]}) \href{https://github.com/d3/d3-axis/blob/master/src/axis.js#L136}{\tt $<$$>$}

If {\itshape format} is specified, sets the tick format function and returns the axis. If {\itshape format} is not specified, returns the current format function, which defaults to null. A null format indicates that the scale’s default formatter should be used, which is generated by calling \href{https://github.com/d3/d3-scale/blob/master/README.md#continuous_tickFormat}{\tt {\itshape scale}.tick\+Format}. In this case, the arguments specified by \href{#axis_tickArguments}{\tt {\itshape axis}.tick\+Arguments} are likewise passed to {\itshape scale}.tick\+Format.

See \href{https://github.com/d3/d3-format}{\tt d3-\/format} and \href{https://github.com/d3/d3-time-format}{\tt d3-\/time-\/format} for help creating formatters. For example, to display integers with comma-\/grouping for thousands\+:


\begin{DoxyCode}
axis.tickFormat(d3.format(",.0f"));
\end{DoxyCode}


More commonly, a format specifier is passed to \href{#axis_ticks}{\tt {\itshape axis}.ticks}\+:


\begin{DoxyCode}
axis.ticks(10, ",f");
\end{DoxyCode}


This has the advantage of setting the format precision automatically based on the tick interval.

\label{_axis_tickSize}%
\# {\itshape axis}.{\bfseries tick\+Size}(\mbox{[}{\itshape size}\mbox{]}) \href{https://github.com/d3/d3-axis/blob/master/src/axis.js#L140}{\tt $<$$>$}

If {\itshape size} is specified, sets the \href{#axis_tickSizeInner}{\tt inner} and \href{#axis_tickSizeOuter}{\tt outer} tick size to the specified value and returns the axis. If {\itshape size} is not specified, returns the current inner tick size, which defaults to 6.

\label{_axis_tickSizeInner}%
\# {\itshape axis}.{\bfseries tick\+Size\+Inner}(\mbox{[}{\itshape size}\mbox{]}) \href{https://github.com/d3/d3-axis/blob/master/src/axis.js#L144}{\tt $<$$>$}

If {\itshape size} is specified, sets the inner tick size to the specified value and returns the axis. If {\itshape size} is not specified, returns the current inner tick size, which defaults to 6. The inner tick size controls the length of the tick lines, offset from the native position of the axis.

\label{_axis_tickSizeOuter}%
\# {\itshape axis}.{\bfseries tick\+Size\+Outer}(\mbox{[}{\itshape size}\mbox{]}) \href{https://github.com/d3/d3-axis/blob/master/src/axis.js#L148}{\tt $<$$>$}

If {\itshape size} is specified, sets the outer tick size to the specified value and returns the axis. If {\itshape size} is not specified, returns the current outer tick size, which defaults to 6. The outer tick size controls the length of the square ends of the domain path, offset from the native position of the axis. Thus, the “outer ticks” are not actually ticks but part of the domain path, and their position is determined by the associated scale’s domain extent. Thus, outer ticks may overlap with the first or last inner tick. An outer tick size of 0 suppresses the square ends of the domain path, instead producing a straight line.

\label{_axis_tickPadding}%
\# {\itshape axis}.{\bfseries tick\+Padding}(\mbox{[}{\itshape padding}\mbox{]}) \href{https://github.com/d3/d3-axis/blob/master/src/axis.js#L152}{\tt $<$$>$}

If {\itshape padding} is specified, sets the padding to the specified value in pixels and returns the axis. If {\itshape padding} is not specified, returns the current padding which defaults to 3 pixels. 