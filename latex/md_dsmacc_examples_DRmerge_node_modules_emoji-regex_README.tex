{\itshape emoji-\/regex} offers a regular expression to match all emoji symbols (including textual representations of emoji) as per the Unicode Standard.

This repository contains a script that generates this regular expression based on \href{https://github.com/mathiasbynens/unicode-tr51}{\tt the data from Unicode Technical Report \#51}. Because of this, the regular expression can easily be updated whenever new emoji are added to the Unicode standard.

\subsection*{Installation}

Via \href{https://www.npmjs.com/}{\tt npm}\+:


\begin{DoxyCode}
npm install emoji-regex
\end{DoxyCode}


In \href{https://nodejs.org/}{\tt Node.\+js}\+:

\`{}\`{}`js const emoji\+Regex = require(\textquotesingle{}emoji-\/regex'); // Note\+: because the regular expression has the global flag set, this module // exports a function that returns the regex rather than exporting the regular // expression itself, to make it impossible to (accidentally) mutate the // original regular expression.

const text = \`{} \{231A\}\+: ⌚ default emoji presentation character (Emoji\+\_\+\+Presentation) \{2194\}\{F\+E0F\}\+: ↔️ default text presentation character rendered as emoji \{1\+F469\}\+: 👩 emoji modifier base (Emoji\+\_\+\+Modifier\+\_\+\+Base) \{1\+F469\}\{1\+F3\+FF\}\+: 👩🏿 emoji modifier base followed by a modifier \`{};

const regex = emoji\+Regex(); let match; while (match = regex.\+exec(text)) \{ const emoji = match\mbox{[}0\mbox{]}; console.\+log({\ttfamily Matched sequence \$\{ emoji \} — code points\+: \$\{ \mbox{[}...emoji\mbox{]}.length \}}); \} 
\begin{DoxyCode}
Console output:
\end{DoxyCode}
 Matched sequence ⌚ — code points\+: 1 Matched sequence ⌚ — code points\+: 1 Matched sequence ↔️ — code points\+: 2 Matched sequence ↔️ — code points\+: 2 Matched sequence 👩 — code points\+: 1 Matched sequence 👩 — code points\+: 1 Matched sequence 👩🏿 — code points\+: 2 Matched sequence 👩🏿 — code points\+: 2 
\begin{DoxyCode}
To match emoji in their textual representation as well (i.e. emoji that are not `Emoji\_Presentation`
       symbols and that aren’t forced to render as emoji by a variation selector), `require` the other regex:

```js
const emojiRegex = require('emoji-regex/text.js');
\end{DoxyCode}


Additionally, in environments which support E\+S2015 Unicode escapes, you may {\ttfamily require} E\+S2015-\/style versions of the regexes\+:


\begin{DoxyCode}
const emojiRegex = require('emoji-regex/es2015/index.js');
const emojiRegexText = require('emoji-regex/es2015/text.js');
\end{DoxyCode}


\subsection*{Author}

\tabulinesep=1mm
\begin{longtabu} spread 0pt [c]{*{1}{|X[-1]}|}
\hline
\rowcolor{\tableheadbgcolor}\textbf{ \mbox{[}!   }\\\cline{1-1}
\endfirsthead
\hline
\endfoot
\hline
\rowcolor{\tableheadbgcolor}\textbf{ \mbox{[}!   }\\\cline{1-1}
\endhead
\href{https://mathiasbynens.be/}{\tt Mathias Bynens}   \\\cline{1-1}
\end{longtabu}


\subsection*{License}

{\itshape emoji-\/regex} is available under the \href{https://mths.be/mit}{\tt M\+IT} license. 