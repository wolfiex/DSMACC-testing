Panning and zooming are popular interaction techniques which let the user focus on a region of interest by restricting the view. It is easy to learn due to direct manipulation\+: click-\/and-\/drag to pan (translate), spin the wheel to zoom (scale), or use touch. Panning and zooming are widely used in web-\/based mapping, but can also be used with visualizations such as time-\/series and scatterplots.

The zoom behavior implemented by d3-\/zoom is a convenient but flexible abstraction for enabling pan-\/and-\/zoom on \href{https://github.com/d3/d3-selection}{\tt selections}. It handles a surprising variety of \href{#api-reference}{\tt input events} and browser quirks. The zoom behavior is agnostic about the D\+OM, so you can use it with S\+VG, H\+T\+ML or Canvas.

\href{https://bl.ocks.org/mbostock/d1f7b58631e71fbf9c568345ee04a60e}{\tt }\href{https://bl.ocks.org/mbostock/4e3925cdc804db257a86fdef3a032a45}{\tt }

The zoom behavior is also designed to work with \href{https://github.com/d3/d3-scale}{\tt d3-\/scale} and \href{https://github.com/d3/d3-axis}{\tt d3-\/axis}; see \href{#transform_rescaleX}{\tt {\itshape transform}.rescaleX} and \href{#transform_rescaleY}{\tt {\itshape transform}.rescaleY}. You can also restrict zooming using \href{#zoom_scaleExtent}{\tt {\itshape zoom}.scale\+Extent} and panning using \href{#zoom_translateExtent}{\tt {\itshape zoom}.translate\+Extent}.

\href{https://bl.ocks.org/mbostock/db6b4335bf1662b413e7968910104f0f}{\tt }

The zoom behavior can be combined with other behaviors, such as \href{https://github.com/d3/d3-drag}{\tt d3-\/drag} for dragging, and \href{https://github.com/d3/d3-brush}{\tt d3-\/brush} for focus + context.

\href{https://bl.ocks.org/mbostock/3127661b6f13f9316be745e77fdfb084}{\tt }\href{https://bl.ocks.org/mbostock/34f08d5e11952a80609169b7917d4172}{\tt }

The zoom behavior can be controlled programmatically using \href{#zoom_transform}{\tt {\itshape zoom}.transform}, allowing you to implement user interface controls which drive the display or to stage animated tours through your data. Smooth zoom transitions are based on \href{http://www.win.tue.nl/~vanwijk/zoompan.pdf}{\tt “\+Smooth and efficient zooming and panning”} by Jarke J. van Wijk and Wim A.\+A. Nuij.

\href{https://bl.ocks.org/mbostock/b783fbb2e673561d214e09c7fb5cedee}{\tt }

See also \href{https://github.com/d3/d3-tile}{\tt d3-\/tile} for examples panning and zooming maps.

\subsection*{Installing}

If you use N\+PM, {\ttfamily npm install d3-\/zoom}. Otherwise, download the \href{https://github.com/d3/d3-zoom/releases/latest}{\tt latest release}. You can also load directly from \href{https://d3js.org}{\tt d3js.\+org}, either as a \href{https://d3js.org/d3-zoom.v1.min.js}{\tt standalone library} or as part of \href{https://github.com/d3/d3}{\tt D3 4.\+0}. A\+MD, Common\+JS, and vanilla environments are supported. In vanilla, a {\ttfamily d3} global is exported\+:


\begin{DoxyCode}
<script src="https://d3js.org/d3-color.v1.min.js"></script>
<script src="https://d3js.org/d3-dispatch.v1.min.js"></script>
<script src="https://d3js.org/d3-ease.v1.min.js"></script>
<script src="https://d3js.org/d3-interpolate.v1.min.js"></script>
<script src="https://d3js.org/d3-selection.v1.min.js"></script>
<script src="https://d3js.org/d3-timer.v1.min.js"></script>
<script src="https://d3js.org/d3-transition.v1.min.js"></script>
<script src="https://d3js.org/d3-drag.v1.min.js"></script>
<script src="https://d3js.org/d3-zoom.v1.min.js"></script>
<script>

var zoom = d3.zoom();

</script>
\end{DoxyCode}


\href{https://tonicdev.com/npm/d3-zoom}{\tt Try d3-\/zoom in your browser.}

\subsection*{A\+PI Reference}

This table describes how the zoom behavior interprets native events\+:

\tabulinesep=1mm
\begin{longtabu} spread 0pt [c]{*{4}{|X[-1]}|}
\hline
\rowcolor{\tableheadbgcolor}\textbf{ Event  }&\textbf{ Listening Element  }&\textbf{ Zoom Event  }&\textbf{ Default Prevented?   }\\\cline{1-4}
\endfirsthead
\hline
\endfoot
\hline
\rowcolor{\tableheadbgcolor}\textbf{ Event  }&\textbf{ Listening Element  }&\textbf{ Zoom Event  }&\textbf{ Default Prevented?   }\\\cline{1-4}
\endhead
mousedown⁵  &selection  &start  &no¹   \\\cline{1-4}
mousemove²  &window¹  &zoom  &yes   \\\cline{1-4}
mouseup²  &window¹  &end  &yes   \\\cline{1-4}
dragstart²  &window  &-\/  &yes   \\\cline{1-4}
selectstart²  &window  &-\/  &yes   \\\cline{1-4}
click³  &window  &-\/  &yes   \\\cline{1-4}
dblclick  &selection  &$\ast$multiple$\ast$⁶  &yes   \\\cline{1-4}
wheel⁸  &selection  &zoom⁷  &yes   \\\cline{1-4}
touchstart  &selection  &$\ast$multiple$\ast$⁶  &no⁴   \\\cline{1-4}
touchmove  &selection  &zoom  &yes   \\\cline{1-4}
touchend  &selection  &end  &no⁴   \\\cline{1-4}
touchcancel  &selection  &end  &no⁴   \\\cline{1-4}
\end{longtabu}


The propagation of all consumed events is \href{https://dom.spec.whatwg.org/#dom-event-stopimmediatepropagation}{\tt immediately stopped}.

¹ Necessary to capture events outside an iframe; see \href{https://github.com/d3/d3-drag/issues/9}{\tt d3-\/drag\#9}. ~\newline
² Only applies during an active, mouse-\/based gesture; see \href{https://github.com/d3/d3-drag/issues/9}{\tt d3-\/drag\#9}. ~\newline
³ Only applies immediately after some mouse-\/based gestures; see \href{#zoom_clickDistance}{\tt {\itshape zoom}.click\+Distance}. ~\newline
⁴ Necessary to allow \href{https://developer.apple.com/library/ios/documentation/AppleApplications/Reference/SafariWebContent/HandlingEvents/HandlingEvents.html#//apple_ref/doc/uid/TP40006511-SW7}{\tt click emulation} on touch input; see \href{https://github.com/d3/d3-drag/issues/9}{\tt d3-\/drag\#9}. ~\newline
⁵ Ignored if within 500ms of a touch gesture ending; assumes \href{https://developer.apple.com/library/ios/documentation/AppleApplications/Reference/SafariWebContent/HandlingEvents/HandlingEvents.html#//apple_ref/doc/uid/TP40006511-SW7}{\tt click emulation}. ~\newline
⁶ Double-\/click and double-\/tap initiate a transition that emits start, zoom and end events. ~\newline
⁷ The first wheel event emits a start event; an end event is emitted when no wheel events are received for 150ms. ~\newline
⁸ Ignored if already at the corresponding limit of the \href{#zoom_scaleExtent}{\tt scale extent}.

\href{#zoom}{\tt \#} d3.{\bfseries zoom}() \href{https://github.com/d3/d3-zoom/blob/master/src/zoom.js}{\tt $<$$>$}

Creates a new zoom behavior. The returned behavior, \href{#_drag}{\tt {\itshape zoom}}, is both an object and a function, and is typically applied to selected elements via \href{https://github.com/d3/d3-selection#selection_call}{\tt {\itshape selection}.call}.

\href{#_zoom}{\tt \#} {\itshape zoom}({\itshape selection}) \href{https://github.com/d3/d3-zoom/blob/master/src/zoom.js#L62}{\tt $<$$>$}

Applies this zoom behavior to the specified \href{https://github.com/d3/d3-selection}{\tt {\itshape selection}}, binding the necessary event listeners to allow panning and zooming, and initializing the \href{#zoom-transforms}{\tt zoom transform} on each selected element to the identity transform if not already defined. This function is typically not invoked directly, and is instead invoked via \href{https://github.com/d3/d3-selection#selection_call}{\tt {\itshape selection}.call}. For example, to instantiate a zoom behavior and apply it to a selection\+:


\begin{DoxyCode}
selection.call(d3.zoom().on("zoom", zoomed));
\end{DoxyCode}


Internally, the zoom behavior uses \href{https://github.com/d3/d3-selection#selection_on}{\tt {\itshape selection}.on} to bind the necessary event listeners for zooming. The listeners use the name {\ttfamily .zoom}, so you can subsequently unbind the zoom behavior as follows\+:


\begin{DoxyCode}
selection.on(".zoom", null);
\end{DoxyCode}


To disable just wheel-\/driven zooming (say to not interfere with native scrolling), you can remove the zoom behavior’s wheel event listener after applying the zoom behavior to the selection\+:


\begin{DoxyCode}
selection
    .call(zoom)
    .on("wheel.zoom", null);
\end{DoxyCode}


Alternatively, use \href{#zoom_filter}{\tt {\itshape zoom}.filter} for greater control over which events can initiate zoom gestures.

Applying the zoom behavior also sets the \href{https://developer.apple.com/library/mac/documentation/AppleApplications/Reference/SafariWebContent/AdjustingtheTextSize/AdjustingtheTextSize.html#//apple_ref/doc/uid/TP40006510-SW5}{\tt -\/webkit-\/tap-\/highlight-\/color} style to transparent, disabling the tap highlight on i\+OS. If you want a different tap highlight color, remove or re-\/apply this style after applying the drag behavior.

\href{#zoom_transform}{\tt \#} {\itshape zoom}.{\bfseries transform}({\itshape selection}, {\itshape transform}) \href{https://github.com/d3/d3-zoom/blob/master/src/zoom.js#L76}{\tt $<$$>$}

If {\itshape selection} is a selection, sets the \href{#zoomTransform}{\tt current zoom transform} of the selected elements to the specified {\itshape transform}, instantaneously emitting start, zoom and end \href{#zoom-events}{\tt events}. If {\itshape selection} is a transition, defines a “zoom” tween to the specified {\itshape transform} using \href{https://github.com/d3/d3-interpolate#interpolateZoom}{\tt d3.\+interpolate\+Zoom}, emitting a start event when the transition starts, zoom events for each tick of the transition, and then an end event when the transition ends (or is interrupted). The {\itshape transform} may be specified either as a \href{#zoom-transforms}{\tt zoom transform} or as a function that returns a zoom transform. If a function, it is invoked for each selected element, being passed the current datum {\ttfamily d} and index {\ttfamily i}, with the {\ttfamily this} context as the current D\+OM element.

This function is typically not invoked directly, and is instead invoked via \href{https://github.com/d3/d3-selection#selection_call}{\tt {\itshape selection}.call} or \href{https://github.com/d3/d3-transition#transition_call}{\tt {\itshape transition}.call}. For example, to reset the zoom transform to the \href{#zoomIdentity}{\tt identity transform} instantaneously\+:


\begin{DoxyCode}
selection.call(zoom.transform, d3.zoomIdentity);
\end{DoxyCode}


To smoothly reset the zoom transform to the identity transform over 750 milliseconds\+:


\begin{DoxyCode}
selection.transition().duration(750).call(zoom.transform, d3.zoomIdentity);
\end{DoxyCode}


This method requires that you specify the new zoom transform completely, and does not enforce the defined \href{#zoom_scaleExtent}{\tt scale extent} and \href{#zoom_translateExtent}{\tt translate extent}, if any. To derive a new transform from the existing transform, and to enforce the scale and translate extents, see the convenience methods \href{#zoom_translateBy}{\tt {\itshape zoom}.translate\+By}, \href{#zoom_scaleBy}{\tt {\itshape zoom}.scale\+By} and \href{#zoom_scaleTo}{\tt {\itshape zoom}.scale\+To}.

\href{#zoom_translateBy}{\tt \#} {\itshape zoom}.{\bfseries translate\+By}({\itshape selection}, {\itshape x}, {\itshape y}) \href{https://github.com/d3/d3-zoom/blob/master/src/zoom.js#L110}{\tt $<$$>$}

If {\itshape selection} is a selection, \href{#transform_translate}{\tt translates} the \href{#zoomTransform}{\tt current zoom transform} of the selected elements by {\itshape x} and {\itshape y}, such that the new {\itshape t\textsubscript{x1}} = {\itshape t\textsubscript{x0}} + {\itshape kx} and {\itshape t\textsubscript{y1}} = {\itshape t\textsubscript{y0}} + {\itshape ky}. If {\itshape selection} is a transition, defines a “zoom” tween translating the current transform. This method is a convenience method for \href{#zoom_transform}{\tt {\itshape zoom}.transform}. The {\itshape x} and {\itshape y} translation amounts may be specified either as numbers or as functions that returns numbers. If a function, it is invoked for each selected element, being passed the current datum {\ttfamily d} and index {\ttfamily i}, with the {\ttfamily this} context as the current D\+OM element.

\href{#zoom_translateTo}{\tt \#} {\itshape zoom}.{\bfseries translate\+To}({\itshape selection}, {\itshape x}, {\itshape y}) \href{https://github.com/d3/d3-zoom/blob/master/src/zoom.js#L119}{\tt $<$$>$}

If {\itshape selection} is a selection, \href{#transform_translate}{\tt translates} the \href{#zoomTransform}{\tt current zoom transform} of the selected elements such that the specified position ⟨$\ast$x$\ast$,{\itshape y$\ast$⟩ appears at the center of the \href{#zoom_extent}{\tt viewport extent}. The new $\ast$t\textsubscript{x}} = {\itshape c\textsubscript{x}} -\/ {\itshape kx} and {\itshape t\textsubscript{y}} = {\itshape c\textsubscript{y}} -\/ {\itshape ky}, where ⟨$\ast$c\textsubscript{x}$\ast$,{\itshape c\textsubscript{y}$\ast$⟩ is the center. If $\ast$selection} is a transition, defines a “zoom” tween translating the current transform. This method is a convenience method for \href{#zoom_transform}{\tt {\itshape zoom}.transform}. The {\itshape x} and {\itshape y} coordinates may be specified either as numbers or as functions that returns numbers. If a function, it is invoked for each selected element, being passed the current datum {\ttfamily d} and index {\ttfamily i}, with the {\ttfamily this} context as the current D\+OM element.

\href{#zoom_scaleBy}{\tt \#} {\itshape zoom}.{\bfseries scale\+By}({\itshape selection}, {\itshape k}) \href{https://github.com/d3/d3-zoom/blob/master/src/zoom.js#L91}{\tt $<$$>$}

If {\itshape selection} is a selection, \href{#transform_scale}{\tt scales} the \href{#zoomTransform}{\tt current zoom transform} of the selected elements by {\itshape k}, such that the new {\itshape k₁} = {\itshape k₀k}. If {\itshape selection} is a transition, defines a “zoom” tween translating the current transform. This method is a convenience method for \href{#zoom_transform}{\tt {\itshape zoom}.transform}. The {\itshape k} scale factor may be specified either as numbers or as functions that returns numbers. If a function, it is invoked for each selected element, being passed the current datum {\ttfamily d} and index {\ttfamily i}, with the {\ttfamily this} context as the current D\+OM element.

\href{#zoom_scaleTo}{\tt \#} {\itshape zoom}.{\bfseries scale\+To}({\itshape selection}, {\itshape k}) \href{https://github.com/d3/d3-zoom/blob/master/src/zoom.js#L99}{\tt $<$$>$}

If {\itshape selection} is a selection, \href{#transform_scale}{\tt scales} the \href{#zoomTransform}{\tt current zoom transform} of the selected elements to {\itshape k}, such that the new {\itshape k₁} = {\itshape k}. If {\itshape selection} is a transition, defines a “zoom” tween translating the current transform. This method is a convenience method for \href{#zoom_transform}{\tt {\itshape zoom}.transform}. The {\itshape k} scale factor may be specified either as numbers or as functions that returns numbers. If a function, it is invoked for each selected element, being passed the current datum {\ttfamily d} and index {\ttfamily i}, with the {\ttfamily this} context as the current D\+OM element.

\href{#zoom_constrain}{\tt \#} {\itshape zoom}.{\bfseries constrain}(\mbox{[}{\itshape constrain}\mbox{]}) \href{https://github.com/d3/d3-zoom/blob/master/src/zoom.js#403}{\tt $<$$>$}

If {\itshape constrain} is specified, sets the transform constraint function to the specified function and returns the zoom behavior. If {\itshape constrain} is not specified, returns the current constraint function, which defaults to\+:


\begin{DoxyCode}
function constrain(transform, extent, translateExtent) \{
  var dx0 = transform.invertX(extent[0][0]) - translateExtent[0][0],
      dx1 = transform.invertX(extent[1][0]) - translateExtent[1][0],
      dy0 = transform.invertY(extent[0][1]) - translateExtent[0][1],
      dy1 = transform.invertY(extent[1][1]) - translateExtent[1][1];
  return transform.translate(
    dx1 > dx0 ? (dx0 + dx1) / 2 : Math.min(0, dx0) || Math.max(0, dx1),
    dy1 > dy0 ? (dy0 + dy1) / 2 : Math.min(0, dy0) || Math.max(0, dy1)
  );
\}
\end{DoxyCode}


The constraint function must return a \href{#zoom-transforms}{\tt {\itshape transform}} given the current {\itshape transform}, \href{#zoom_extent}{\tt viewport extent} and \href{#zoom_translateExtent}{\tt translate extent}. The default implementation attempts to ensure that the viewport extent does not go outside the translate extent.

\href{#zoom_filter}{\tt \#} {\itshape zoom}.{\bfseries filter}(\mbox{[}{\itshape filter}\mbox{]}) \href{https://github.com/d3/d3-zoom/blob/master/src/zoom.js#L386}{\tt $<$$>$}

If {\itshape filter} is specified, sets the filter to the specified function and returns the zoom behavior. If {\itshape filter} is not specified, returns the current filter, which defaults to\+:


\begin{DoxyCode}
function filter() \{
  return !d3.event.button;
\}
\end{DoxyCode}


If the filter returns falsey, the initiating event is ignored and no zoom gestures are started. Thus, the filter determines which input events are ignored. The default filter ignores mousedown events on secondary buttons, since those buttons are typically intended for other purposes, such as the context menu.

\href{#zoom_touchable}{\tt \#} {\itshape zoom}.{\bfseries touchable}(\mbox{[}{\itshape touchable}\mbox{]}) \href{https://github.com/d3/d3-zoom/blob/master/src/zoom.js#L390}{\tt $<$$>$}

If {\itshape touchable} is specified, sets the touch support detector to the specified function and returns the zoom behavior. If {\itshape touchable} is not specified, returns the current touch support detector, which defaults to\+:


\begin{DoxyCode}
function touchable() \{
  return "ontouchstart" in this;
\}
\end{DoxyCode}


Touch event listeners are only registered if the detector returns truthy for the corresponding element when the zoom behavior is \href{#_zoom}{\tt applied}. The default detector works well for most browsers that are capable of touch input, but not all; Chrome’s mobile device emulator, for example, fails detection.

\href{#zoom_wheelDelta}{\tt \#} {\itshape zoom}.{\bfseries wheel\+Delta}(\mbox{[}{\itshape delta}\mbox{]}) \href{https://github.com/d3/d3-zoom/blob/master/src/zoom.js#L382}{\tt $<$$>$}

If {\itshape delta} is specified, sets the wheel delta function to the specified function and returns the zoom behavior. If {\itshape delta} is not specified, returns the current wheel delta function, which defaults to\+:


\begin{DoxyCode}
function wheelDelta() \{
  return -d3.event.deltaY * (d3.event.deltaMode ? 120 : 1) / 500;
\}
\end{DoxyCode}


The value {\itshape Δ} returned by the wheel delta function determines the amount of scaling applied in response to a \href{https://developer.mozilla.org/en-US/docs/Web/API/WheelEvent}{\tt Wheel\+Event}. The scale factor \href{#zoomTransform}{\tt {\itshape transform}.k} is multiplied by 2\textsuperscript{$\ast$Δ$\ast$}; for example, a {\itshape Δ} of +1 doubles the scale factor, {\itshape Δ} of -\/1 halves the scale factor.

\href{#zoom_extent}{\tt \#} {\itshape zoom}.{\bfseries extent}(\mbox{[}{\itshape extent}\mbox{]}) \href{https://github.com/d3/d3-zoom/blob/master/src/zoom.js#L394}{\tt $<$$>$}

If {\itshape extent} is specified, sets the viewport extent to the specified array of points \mbox{[}\mbox{[}{\itshape x0}, {\itshape y0}\mbox{]}, \mbox{[}{\itshape x1}, {\itshape y1}\mbox{]}\mbox{]}, where \mbox{[}{\itshape x0}, {\itshape y0}\mbox{]} is the top-\/left corner of the viewport and \mbox{[}{\itshape x1}, {\itshape y1}\mbox{]} is the bottom-\/right corner of the viewport, and returns this zoom behavior. The {\itshape extent} may also be specified as a function which returns such an array; if a function, it is invoked for each selected element, being passed the current datum {\ttfamily d} and index {\ttfamily i}, with the {\ttfamily this} context as the current D\+OM element.

If {\itshape extent} is not specified, returns the current extent accessor, which defaults to \mbox{[}\mbox{[}0, 0\mbox{]}, \mbox{[}{\itshape width}, {\itshape height}\mbox{]}\mbox{]} where {\itshape width} is the \href{https://developer.mozilla.org/en-US/docs/Web/API/Element/clientWidth}{\tt client width} of the element and {\itshape height} is its \href{https://developer.mozilla.org/en-US/docs/Web/API/Element/clientHeight}{\tt client height}; for S\+VG elements, the nearest ancestor S\+VG element’s \href{https://www.w3.org/TR/SVG/struct.html#SVGElementWidthAttribute}{\tt width} and \href{https://www.w3.org/TR/SVG/struct.html#SVGElementHeightAttribute}{\tt height} is used. In this case, the owner S\+VG element must have defined \href{https://www.w3.org/TR/SVG/struct.html#SVGElementWidthAttribute}{\tt width} and \href{https://www.w3.org/TR/SVG/struct.html#SVGElementHeightAttribute}{\tt height} attributes rather than (for example) relying on C\+SS properties or the view\+Box attribute; S\+VG provides no programmatic method for retrieving the \href{https://www.w3.org/TR/SVG/coords.html#ViewportSpace}{\tt initial viewport size}. Alternatively, consider using \href{https://developer.mozilla.org/en-US/docs/Web/API/Element/getBoundingClientRect}{\tt {\itshape element}.get\+Bounding\+Client\+Rect}. (In Firefox, \href{https://developer.mozilla.org/en-US/docs/Web/API/Element/clientWidth}{\tt {\itshape element}.client\+Width} and \href{https://developer.mozilla.org/en-US/docs/Web/API/Element/clientHeight}{\tt {\itshape element}.client\+Height} is zero for S\+VG elements!)

The viewport extent affects several functions\+: the center of the viewport remains fixed during changes by \href{#zoom_scaleBy}{\tt {\itshape zoom}.scale\+By} and \href{#zoom_scaleTo}{\tt {\itshape zoom}.scale\+To}; the viewport center and dimensions affect the path chosen by \href{https://github.com/d3/d3-interpolate#interpolateZoom}{\tt d3.\+interpolate\+Zoom}; and the viewport extent is needed to enforce the optional \href{#zoom_translateExtent}{\tt translate extent}.

\href{#zoom_scaleExtent}{\tt \#} {\itshape zoom}.{\bfseries scale\+Extent}(\mbox{[}{\itshape extent}\mbox{]}) \href{https://github.com/d3/d3-zoom/blob/master/src/zoom.js#L398}{\tt $<$$>$}

If {\itshape extent} is specified, sets the scale extent to the specified array of numbers \mbox{[}{\itshape k0}, {\itshape k1}\mbox{]} where {\itshape k0} is the minimum allowed scale factor and {\itshape k1} is the maximum allowed scale factor, and returns this zoom behavior. If {\itshape extent} is not specified, returns the current scale extent, which defaults to \mbox{[}0, ∞\mbox{]}. The scale extent restricts zooming in and out. It is enforced on interaction and when using \href{#zoom_scaleBy}{\tt {\itshape zoom}.scale\+By}, \href{#zoom_scaleTo}{\tt {\itshape zoom}.scale\+To} and \href{#zoom_translateBy}{\tt {\itshape zoom}.translate\+By}; however, it is not enforced when using \href{#zoom_transform}{\tt {\itshape zoom}.transform} to set the transform explicitly.

If the user tries to zoom by wheeling when already at the corresponding limit of the scale extent, the wheel events will be ignored and not initiate a zoom gesture. This allows the user to scroll down past a zoomable area after zooming in, or to scroll up after zooming out. If you would prefer to always prevent scrolling on wheel input regardless of the scale extent, register a wheel event listener to prevent the browser default behavior\+:


\begin{DoxyCode}
selection
    .call(zoom)
    .on("wheel", function() \{ d3.event.preventDefault(); \});
\end{DoxyCode}


\href{#zoom_translateExtent}{\tt \#} {\itshape zoom}.{\bfseries translate\+Extent}(\mbox{[}{\itshape extent}\mbox{]}) \href{https://github.com/d3/d3-zoom/blob/master/src/zoom.js#L402}{\tt $<$$>$}

If {\itshape extent} is specified, sets the translate extent to the specified array of points \mbox{[}\mbox{[}{\itshape x0}, {\itshape y0}\mbox{]}, \mbox{[}{\itshape x1}, {\itshape y1}\mbox{]}\mbox{]}, where \mbox{[}{\itshape x0}, {\itshape y0}\mbox{]} is the top-\/left corner of the world and \mbox{[}{\itshape x1}, {\itshape y1}\mbox{]} is the bottom-\/right corner of the world, and returns this zoom behavior. If {\itshape extent} is not specified, returns the current translate extent, which defaults to \mbox{[}\mbox{[}-\/∞, -\/∞\mbox{]}, \mbox{[}+∞, +∞\mbox{]}\mbox{]}. The translate extent restricts panning, and may cause translation on zoom out. It is enforced on interaction and when using \href{#zoom_scaleBy}{\tt {\itshape zoom}.scale\+By}, \href{#zoom_scaleTo}{\tt {\itshape zoom}.scale\+To} and \href{#zoom_translateBy}{\tt {\itshape zoom}.translate\+By}; however, it is not enforced when using \href{#zoom_transform}{\tt {\itshape zoom}.transform} to set the transform explicitly.

\href{#zoom_clickDistance}{\tt \#} {\itshape zoom}.{\bfseries click\+Distance}(\mbox{[}{\itshape distance}\mbox{]}) \href{https://github.com/d3/d3-zoom/blob/master/src/zoom.js#L419}{\tt $<$$>$}

If {\itshape distance} is specified, sets the maximum distance that the mouse can move between mousedown and mouseup that will trigger a subsequent click event. If at any point between mousedown and mouseup the mouse is greater than or equal to {\itshape distance} from its position on mousedown, the click event following mouseup will be suppressed. If {\itshape distance} is not specified, returns the current distance threshold, which defaults to zero. The distance threshold is measured in client coordinates (\href{https://developer.mozilla.org/en-US/docs/Web/API/MouseEvent/clientX}{\tt {\itshape event}.clientX} and \href{https://developer.mozilla.org/en-US/docs/Web/API/MouseEvent/clientY}{\tt {\itshape event}.clientY}).

\href{#zoom_duration}{\tt \#} {\itshape zoom}.{\bfseries duration}(\mbox{[}{\itshape duration}\mbox{]}) \href{https://github.com/d3/d3-zoom/blob/master/src/zoom.js#L406}{\tt $<$$>$}

If {\itshape duration} is specified, sets the duration for zoom transitions on double-\/click and double-\/tap to the specified number of milliseconds and returns the zoom behavior. If {\itshape duration} is not specified, returns the current duration, which defaults to 250 milliseconds. If the duration is not greater than zero, double-\/click and -\/tap trigger instantaneous changes to the zoom transform rather than initiating smooth transitions.

To disable double-\/click and double-\/tap transitions, you can remove the zoom behavior’s dblclick event listener after applying the zoom behavior to the selection\+:


\begin{DoxyCode}
selection
    .call(zoom)
    .on("dblclick.zoom", null);
\end{DoxyCode}


\href{#zoom_interpolate}{\tt \#} {\itshape zoom}.{\bfseries interpolate}(\mbox{[}{\itshape interpolate}\mbox{]}) \href{https://github.com/d3/d3-zoom/blob/master/src/zoom.js#L410}{\tt $<$$>$}

If {\itshape interpolate} is specified, sets the interpolation factory for zoom transitions to the specified function. If {\itshape interpolate} is not specified, returns the current interpolation factory, which defaults to \href{https://github.com/d3/d3-interpolate#interpolateZoom}{\tt d3.\+interpolate\+Zoom} to implement smooth zooming. To apply direct interpolation between two views, try \href{https://github.com/d3/d3-interpolate#interpolate}{\tt d3.\+interpolate} instead.

\href{#zoom_on}{\tt \#} {\itshape zoom}.{\bfseries on}({\itshape typenames}\mbox{[}, {\itshape listener}\mbox{]}) \href{https://github.com/d3/d3-zoom/blob/master/src/zoom.js#L414}{\tt $<$$>$}

If {\itshape listener} is specified, sets the event {\itshape listener} for the specified {\itshape typenames} and returns the zoom behavior. If an event listener was already registered for the same type and name, the existing listener is removed before the new listener is added. If {\itshape listener} is null, removes the current event listeners for the specified {\itshape typenames}, if any. If {\itshape listener} is not specified, returns the first currently-\/assigned listener matching the specified {\itshape typenames}, if any. When a specified event is dispatched, each {\itshape listener} will be invoked with the same context and arguments as \href{https://github.com/d3/d3-selection#selection_on}{\tt {\itshape selection}.on} listeners\+: the current datum {\ttfamily d} and index {\ttfamily i}, with the {\ttfamily this} context as the current D\+OM element.

The {\itshape typenames} is a string containing one or more {\itshape typename} separated by whitespace. Each {\itshape typename} is a {\itshape type}, optionally followed by a period ({\ttfamily .}) and a {\itshape name}, such as {\ttfamily zoom.\+foo} and {\ttfamily zoom.\+bar}; the name allows multiple listeners to be registered for the same {\itshape type}. The {\itshape type} must be one of the following\+:


\begin{DoxyItemize}
\item {\ttfamily start} -\/ after zooming begins (such as on mousedown).
\item {\ttfamily zoom} -\/ after a change to the zoom transform (such as on mousemove).
\item {\ttfamily end} -\/ after zooming ends (such as on mouseup ).
\end{DoxyItemize}

See \href{https://github.com/d3/d3-dispatch#dispatch_on}{\tt {\itshape dispatch}.on} for more.

\subsubsection*{Zoom Events}

When a \href{#zoom_on}{\tt zoom event listener} is invoked, \href{https://github.com/d3/d3-selection#event}{\tt d3.\+event} is set to the current zoom event. The {\itshape event} object exposes several fields\+:


\begin{DoxyItemize}
\item {\itshape event}.target -\/ the associated \href{#zoom}{\tt zoom behavior}.
\item {\itshape event}.type -\/ the string “start”, “zoom” or “end”; see \href{#zoom_on}{\tt {\itshape zoom}.on}.
\item {\itshape event}.transform -\/ the current \href{#zoom-transforms}{\tt zoom transform}.
\item {\itshape event}.source\+Event -\/ the underlying input event, such as mousemove or touchmove.
\end{DoxyItemize}

\subsubsection*{Zoom Transforms}

The zoom behavior stores the zoom state on the element to which the zoom behavior was \href{#_zoom}{\tt applied}, not on the zoom behavior itself. This is because the zoom behavior can be applied to many elements simultaneously, and each element can be zoomed independently. The zoom state can change either on user interaction or programmatically via \href{#zoom_transform}{\tt {\itshape zoom}.transform}.

To retrieve the zoom state, use {\itshape event}.transform on the current \href{#zoom-events}{\tt zoom event} within a zoom event listener (see \href{#zoom_on}{\tt {\itshape zoom}.on}), or use \href{#zoomTransform}{\tt d3.\+zoom\+Transform} for a given node. The latter is particularly useful for modifying the zoom state programmatically, say to implement buttons for zooming in and out.

\href{#zoomTransform}{\tt \#} d3.{\bfseries zoom\+Transform}({\itshape node}) \href{https://github.com/d3/d3-zoom/blob/master/src/transform.js}{\tt $<$$>$}

Returns the current transform for the specified {\itshape node}. Note that {\itshape node} should typically be a D\+OM element, not a {\itshape selection}. (A selection may consist of multiple nodes, in different states, and this function only returns a single transform.) If you have a selection, call \href{https://github.com/d3/d3-selection#selection_node}{\tt {\itshape selection}.node} first\+:


\begin{DoxyCode}
var transform = d3.zoomTransform(selection.node());
\end{DoxyCode}


In the context of an \href{https://github.com/d3/d3-selection#selection_on}{\tt event listener}, the {\itshape node} is typically the element that received the input event (which should be equal to \href{#zoom-events}{\tt {\itshape event}.transform}), {\itshape this}\+:


\begin{DoxyCode}
var transform = d3.zoomTransform(this);
\end{DoxyCode}


Internally, an element’s transform is stored as {\itshape element}.\+\_\+\+\_\+zoom; however, you should use this method rather than accessing it directly. If the given {\itshape node} has no defined transform, returns the \href{#zoomIdentity}{\tt identity transformation}. The returned transform represents a two-\/dimensional \href{https://en.wikipedia.org/wiki/Transformation_matrix#Affine_transformations}{\tt transformation matrix} of the form\+:

{\itshape k} 0 {\itshape t\textsubscript{x}} ~\newline
0 {\itshape k} {\itshape t\textsubscript{y}} ~\newline
0 0 1

(This matrix is capable of representing only scale and translation; a future release may also allow rotation, though this would probably not be a backwards-\/compatible change.) The position ⟨$\ast$x$\ast$,{\itshape y$\ast$⟩ is transformed to ⟨$\ast$xk} + {\itshape t\textsubscript{x}},{\itshape yk} + $\ast$t\textsubscript{y}$\ast$⟩. The transform object exposes the following properties\+:


\begin{DoxyItemize}
\item {\itshape transform}.x -\/ the translation amount {\itshape t\textsubscript{x}} along the {\itshape x}-\/axis.
\item {\itshape transform}.y -\/ the translation amount {\itshape t\textsubscript{y}} along the {\itshape y}-\/axis.
\item {\itshape transform}.k -\/ the scale factor {\itshape k}.
\end{DoxyItemize}

These properties should be considered read-\/only; instead of mutating a transform, use \href{#transform_scale}{\tt {\itshape transform}.scale} and \href{#transform_translate}{\tt {\itshape transform}.translate} to derive a new transform. Also see \href{#zoom_scaleBy}{\tt {\itshape zoom}.scale\+By}, \href{#zoom_scaleTo}{\tt {\itshape zoom}.scale\+To} and \href{#zoom_translateBy}{\tt {\itshape zoom}.translate\+By} for convenience methods on the zoom behavior. To create a transform with a given {\itshape k}, {\itshape t\textsubscript{x}}, and {\itshape t\textsubscript{y}}\+:


\begin{DoxyCode}
var t = d3.zoomIdentity.translate(x, y).scale(k);
\end{DoxyCode}


To apply the transformation to a \href{https://www.w3.org/TR/2dcontext/}{\tt Canvas 2D context}, use \href{https://www.w3.org/TR/2dcontext/#dom-context-2d-translate}{\tt {\itshape context}.translate} followed by \href{https://www.w3.org/TR/2dcontext/#dom-context-2d-scale}{\tt {\itshape context}.scale}\+:


\begin{DoxyCode}
context.translate(transform.x, transform.y);
context.scale(transform.k, transform.k);
\end{DoxyCode}


Similarly, to apply the transformation to H\+T\+ML elements via \href{https://www.w3.org/TR/css-transforms-1/}{\tt C\+SS}\+:


\begin{DoxyCode}
div.style("transform", "translate(" + transform.x + "px," + transform.y + "px) scale(" + transform.k +
       ")");
div.style("transform-origin", "0 0");
\end{DoxyCode}


To apply the transformation to \href{https://www.w3.org/TR/SVG/coords.html#TransformAttribute}{\tt S\+VG}\+:


\begin{DoxyCode}
g.attr("transform", "translate(" + transform.x + "," + transform.y + ") scale(" + transform.k + ")");
\end{DoxyCode}


Or more simply, taking advantage of \href{#transform_toString}{\tt {\itshape transform}.to\+String}\+:


\begin{DoxyCode}
g.attr("transform", transform);
\end{DoxyCode}


Note that the order of transformations matters! The translate must be applied before the scale.

\href{#transform_scale}{\tt \#} {\itshape transform}.{\bfseries scale}({\itshape k}) \href{https://github.com/d3/d3-zoom/blob/master/src/transform.js#L9}{\tt $<$$>$}

Returns a transform whose scale {\itshape k₁} is equal to {\itshape k₀k}, where {\itshape k₀} is this transform’s scale.

\href{#transform_translate}{\tt \#} {\itshape transform}.{\bfseries translate}({\itshape x}, {\itshape y}) \href{https://github.com/d3/d3-zoom/blob/master/src/transform.js#L12}{\tt $<$$>$}

Returns a transform whose translation {\itshape t\textsubscript{x1}} and {\itshape t\textsubscript{y1}} is equal to {\itshape t\textsubscript{x0}} + {\itshape x} and {\itshape t\textsubscript{y0}} + {\itshape y}, where {\itshape t\textsubscript{x0}} and {\itshape t\textsubscript{y0}} is this transform’s translation.

\href{#transform_apply}{\tt \#} {\itshape transform}.{\bfseries apply}({\itshape point}) \href{https://github.com/d3/d3-zoom/blob/master/src/transform.js#L15}{\tt $<$$>$}

Returns the transformation of the specified {\itshape point} which is a two-\/element array of numbers \mbox{[}{\itshape x}, {\itshape y}\mbox{]}. The returned point is equal to \mbox{[}{\itshape xk} + {\itshape t\textsubscript{x}}, {\itshape yk} + {\itshape t\textsubscript{y}}\mbox{]}.

\href{#transform_applyX}{\tt \#} {\itshape transform}.{\bfseries applyX}({\itshape x}) \href{https://github.com/d3/d3-zoom/blob/master/src/transform.js#L18}{\tt $<$$>$}

Returns the transformation of the specified {\itshape x}-\/coordinate, {\itshape xk} + {\itshape t\textsubscript{x}}.

\href{#transform_applyY}{\tt \#} {\itshape transform}.{\bfseries applyY}({\itshape y}) \href{https://github.com/d3/d3-zoom/blob/master/src/transform.js#L21}{\tt $<$$>$}

Returns the transformation of the specified {\itshape y}-\/coordinate, {\itshape yk} + {\itshape t\textsubscript{y}}.

\href{#transform_invert}{\tt \#} {\itshape transform}.{\bfseries invert}({\itshape point}) \href{https://github.com/d3/d3-zoom/blob/master/src/transform.js#L24}{\tt $<$$>$}

Returns the inverse transformation of the specified {\itshape point} which is a two-\/element array of numbers \mbox{[}{\itshape x}, {\itshape y}\mbox{]}. The returned point is equal to \mbox{[}({\itshape x} -\/ {\itshape t\textsubscript{x}}) / {\itshape k}, ({\itshape y} -\/ {\itshape t\textsubscript{y}}) / {\itshape k}\mbox{]}.

\href{#transform_invertX}{\tt \#} {\itshape transform}.{\bfseries invertX}({\itshape x}) \href{https://github.com/d3/d3-zoom/blob/master/src/transform.js#L27}{\tt $<$$>$}

Returns the inverse transformation of the specified {\itshape x}-\/coordinate, ({\itshape x} -\/ {\itshape t\textsubscript{x}}) / {\itshape k}.

\href{#transform_invertY}{\tt \#} {\itshape transform}.{\bfseries invertY}({\itshape y}) \href{https://github.com/d3/d3-zoom/blob/master/src/transform.js#L30}{\tt $<$$>$}

Returns the inverse transformation of the specified {\itshape y}-\/coordinate, ({\itshape y} -\/ {\itshape t\textsubscript{y}}) / {\itshape k}.

\href{#transform_rescaleX}{\tt \#} {\itshape transform}.{\bfseries rescaleX}({\itshape x}) \href{https://github.com/d3/d3-zoom/blob/master/src/transform.js#L33}{\tt $<$$>$}

Returns a \href{https://github.com/d3/d3-scale#continuous_copy}{\tt copy} of the \href{https://github.com/d3/d3-scale#continuous-scales}{\tt continuous scale} {\itshape x} whose \href{https://github.com/d3/d3-scale#continuous_domain}{\tt domain} is transformed. This is implemented by first applying the \href{#transform_invertX}{\tt inverse {\itshape x}-\/transform} on the scale’s \href{https://github.com/d3/d3-scale#continuous_range}{\tt range}, and then applying the \href{https://github.com/d3/d3-scale#continuous_invert}{\tt inverse scale} to compute the corresponding domain\+:


\begin{DoxyCode}
function rescaleX(x) \{
  var range = x.range().map(transform.invertX, transform),
      domain = range.map(x.invert, x);
  return x.copy().domain(domain);
\}
\end{DoxyCode}


The scale {\itshape x} must use \href{https://github.com/d3/d3-interpolate#interpolateNumber}{\tt d3.\+interpolate\+Number}; do not use \href{https://github.com/d3/d3-scale#continuous_rangeRound}{\tt {\itshape continuous}.range\+Round} as this reduces the accuracy of \href{https://github.com/d3/d3-scale#continuous_invert}{\tt {\itshape continuous}.invert} and can lead to an inaccurate rescaled domain. This method does not modify the input scale {\itshape x}; {\itshape x} thus represents the untransformed scale, while the returned scale represents its transformed view.

\href{#transform_rescaleY}{\tt \#} {\itshape transform}.{\bfseries rescaleY}({\itshape y}) \href{https://github.com/d3/d3-zoom/blob/master/src/transform.js#L36}{\tt $<$$>$}

Returns a \href{https://github.com/d3/d3-scale#continuous_copy}{\tt copy} of the \href{https://github.com/d3/d3-scale#continuous-scales}{\tt continuous scale} {\itshape y} whose \href{https://github.com/d3/d3-scale#continuous_domain}{\tt domain} is transformed. This is implemented by first applying the \href{#transform_invertY}{\tt inverse {\itshape y}-\/transform} on the scale’s \href{https://github.com/d3/d3-scale#continuous_range}{\tt range}, and then applying the \href{https://github.com/d3/d3-scale#continuous_invert}{\tt inverse scale} to compute the corresponding domain\+:


\begin{DoxyCode}
function rescaleY(y) \{
  var range = y.range().map(transform.invertY, transform),
      domain = range.map(y.invert, y);
  return y.copy().domain(domain);
\}
\end{DoxyCode}


The scale {\itshape y} must use \href{https://github.com/d3/d3-interpolate#interpolateNumber}{\tt d3.\+interpolate\+Number}; do not use \href{https://github.com/d3/d3-scale#continuous_rangeRound}{\tt {\itshape continuous}.range\+Round} as this reduces the accuracy of \href{https://github.com/d3/d3-scale#continuous_invert}{\tt {\itshape continuous}.invert} and can lead to an inaccurate rescaled domain. This method does not modify the input scale {\itshape y}; {\itshape y} thus represents the untransformed scale, while the returned scale represents its transformed view.

\href{#transform_toString}{\tt \#} {\itshape transform}.{\bfseries to\+String}() \href{https://github.com/d3/d3-zoom/blob/master/src/transform.js#L39}{\tt $<$$>$}

Returns a string representing the \href{https://www.w3.org/TR/SVG/coords.html#TransformAttribute}{\tt S\+VG transform} corresponding to this transform. Implemented as\+:


\begin{DoxyCode}
function toString() \{
  return "translate(" + this.x + "," + this.y + ") scale(" + this.k + ")";
\}
\end{DoxyCode}


\href{#zoomIdentity}{\tt \#} d3.{\bfseries zoom\+Identity} \href{https://github.com/d3/d3-zoom/blob/master/src/transform.js#L44}{\tt $<$$>$}

The identity transform, where {\itshape k} = 1, {\itshape t\textsubscript{x}} = {\itshape t\textsubscript{y}} = 0. 