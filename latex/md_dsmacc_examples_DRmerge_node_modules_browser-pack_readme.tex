pack node-\/style source files from a json stream into a browser bundle

\href{http://travis-ci.org/browserify/browser-pack}{\tt }

\section*{example}

json input\+:


\begin{DoxyCode}
[
  \{
    "id": "a1b5af78",
    "source": "console.log(require('./foo')(5))",
    "deps": \{ "./foo": "b8f69fa5" \},
    "entry": true
  \},
  \{
    "id": "b8f69fa5",
    "source": "module.exports = function (n) \{ return n * 111 \}",
    "deps": \{\}
  \}
]
\end{DoxyCode}


bundle script\+:


\begin{DoxyCode}
var pack = require('browser-pack')();
process.stdin.pipe(pack).pipe(process.stdout);
process.stdin.resume();
\end{DoxyCode}


output\+:


\begin{DoxyCode}
$ browser-pack < input.json
(function(p,c,e)\{function r(n)\{if(!c[n])\{c[n]=\{exports:\{\}\};p[n][0](function(x)\{return
       r(p[n][1][x])\},c[n],c[n].exports);\}return c[n].exports\}for(var i=0;i<e.length;i++)r(e[i]);return
       r\})(\{"a1b5af78":[function(require
      ,module,exports)\{console.log(require('./foo')(5))\},\{"./foo":"b8f69fa5"\}],"b8f69fa5":[function(require,module,exports)\{module.exports = function (n) \{ return n * 111 \}\},\{\}]\},\{\},["a1b5af78","b8f69fa5"])
\end{DoxyCode}


\section*{methods}


\begin{DoxyCode}
var pack = require('browser-pack');
\end{DoxyCode}


\subsection*{pack(opts)}

Return a through stream that takes a stream of json input and produces a stream of javascript output. This module does not export its internal {\ttfamily require()} function but you can prepend `\textquotesingle{}var require='{\ttfamily to the stream contents to get the require function.}require(){\ttfamily will return}undefined\`{} when a module hasn\textquotesingle{}t been defined to support splitting up modules across several bundles with custom fallback logic.

If {\ttfamily opts.\+raw} is given, the writable end of the stream will expect objects to be written to it instead of expecting a stream of json text it will need to parse.

If {\ttfamily opts.\+source\+Map\+Prefix} is given and source maps are computed, the {\ttfamily opts.\+source\+Map\+Prefix} string will be used instead of {\ttfamily //\#}.

If {\ttfamily opts.\+source\+Root} is given and source maps are computed, the root for the output source map will be defined. (default is no root)

Additionally, rows with a truthy {\ttfamily entry} may have an {\ttfamily order} field that determines the numeric index to execute the entries in.

You can specify a custom prelude with {\ttfamily opts.\+prelude} but you should really know what you\textquotesingle{}re doing first. See the {\ttfamily prelude.\+js} file in this repo for the default prelude. If you specify a custom prelude, you must also specify a valid {\ttfamily opts.\+prelude\+Path} to the prelude source file for sourcemaps to work.

{\ttfamily opts.\+standalone} external string name to use for umd

{\ttfamily opts.\+standalone\+Module} sets the internal module name to export for standalone

{\ttfamily opts.\+has\+Exports} whether the bundle should include {\ttfamily require=} (or the {\ttfamily opts.\+external\+Require\+Name}) so that {\ttfamily require()} is available outside the bundle

\section*{install}

With \href{https://npmjs.org}{\tt npm}, to get the library do\+:


\begin{DoxyCode}
npm install browser-pack
\end{DoxyCode}


and to get the command-\/line tool do\+:


\begin{DoxyCode}
npm install -g browser-pack
\end{DoxyCode}


\section*{license}

M\+IT 