Build chainable fluent interfaces the easy way in node.\+js.

With this meta-\/module you can write modules with chainable interfaces. Chainsaw takes care of all of the boring details and makes nested flow control super simple too.

Just call {\ttfamily Chainsaw} with a constructor function like in the examples below. In your methods, just do {\ttfamily saw.\+next()} to move along to the next event and {\ttfamily saw.\+nest()} to create a nested chain.

\section*{Examples }

\subsection*{add\+\_\+do.\+js }

This silly example adds values with a chainsaw. \begin{DoxyVerb}var Chainsaw = require('chainsaw');

function AddDo (sum) {
    return Chainsaw(function (saw) {
        this.add = function (n) {
            sum += n;
            saw.next();
        };

        this.do = function (cb) {
            saw.nest(cb, sum);
        };
    });
}

AddDo(0)
    .add(5)
    .add(10)
    .do(function (sum) {
        if (sum > 12) this.add(-10);
    })
    .do(function (sum) {
        console.log('Sum: ' + sum);
    })
;
\end{DoxyVerb}


Output\+: Sum\+: 5

\subsection*{prompt.\+js }

This example provides a wrapper on top of stdin with the help of \href{https://github.com/pkrumins/node-lazy}{\tt node-\/lazy} for line-\/processing. \begin{DoxyVerb}var Chainsaw = require('chainsaw');
var Lazy = require('lazy');

module.exports = Prompt;
function Prompt (stream) {
    var waiting = [];
    var lines = [];
    var lazy = Lazy(stream).lines.map(String)
        .forEach(function (line) {
            if (waiting.length) {
                var w = waiting.shift();
                w(line);
            }
            else lines.push(line);
        })
    ;

    var vars = {};
    return Chainsaw(function (saw) {
        this.getline = function (f) {
            var g = function (line) {
                saw.nest(f, line, vars);
            };

            if (lines.length) g(lines.shift());
            else waiting.push(g);
        };

        this.do = function (cb) {
            saw.nest(cb, vars);
        };
    });
}
\end{DoxyVerb}


And now for the new Prompt() module in action\+: \begin{DoxyVerb}var util = require('util');
var stdin = process.openStdin();

Prompt(stdin)
    .do(function () {
        util.print('x = ');
    })
    .getline(function (line, vars) {
        vars.x = parseInt(line, 10);
    })
    .do(function () {
        util.print('y = ');
    })
    .getline(function (line, vars) {
        vars.y = parseInt(line, 10);
    })
    .do(function (vars) {
        if (vars.x + vars.y < 10) {
            util.print('z = ');
            this.getline(function (line) {
                vars.z = parseInt(line, 10);
            })
        }
        else {
            vars.z = 0;
        }
    })
    .do(function (vars) {
        console.log('x + y + z = ' + (vars.x + vars.y + vars.z));
        process.exit();
    })
;
\end{DoxyVerb}


\section*{Installation }

With \href{http://github.com/isaacs/npm}{\tt npm}, just do\+: npm install chainsaw

or clone this project on github\+: \begin{DoxyVerb}git clone http://github.com/substack/node-chainsaw.git
\end{DoxyVerb}


To run the tests with \href{http://github.com/visionmedia/expresso}{\tt expresso}, just do\+: \begin{DoxyVerb}expresso
\end{DoxyVerb}


\section*{Light Mode vs Full Mode }

{\ttfamily node-\/chainsaw} supports two different modes. In full mode, every action is recorded, which allows you to replay actions using the {\ttfamily jump()}, {\ttfamily trap()} and {\ttfamily down()} methods.

However, if your chainsaws are long-\/lived, recording every action can consume a tremendous amount of memory, so we also offer a \char`\"{}light\char`\"{} mode where actions are not recorded and the aforementioned methods are disabled.

To enable light mode simply use {\ttfamily Chainsaw.\+light()} to construct your saw, instead of {\ttfamily Chainsaw()}. 