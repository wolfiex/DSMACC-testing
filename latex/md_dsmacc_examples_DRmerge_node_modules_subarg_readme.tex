parse arguments with recursive contexts using \href{https://npmjs.org/package/minimist}{\tt minimist}

\href{https://ci.testling.com/substack/subarg}{\tt }

\href{http://travis-ci.org/substack/subarg}{\tt }

This module is useful if you need to pass arguments into a piece of code without coordinating ahead of time with the main program, like with a plugin system.

\section*{example}


\begin{DoxyCode}
var subarg = require('subarg');
var argv = subarg(process.argv.slice(2));
console.log(argv);
\end{DoxyCode}


Contexts are denoted with square brackets\+:


\begin{DoxyCode}
$ node example/show.js rawr --beep [ boop -a 3 ] -n4 --robots [ -x 8 -y 6 ]
\{ \_: [ 'rawr' ],
  beep: \{ \_: [ 'boop' ], a: 3 \},
  n: 4,
  robots: \{ \_: [], x: 8, y: 6 \} \}
\end{DoxyCode}


\section*{methods}


\begin{DoxyCode}
var subarg = require('subarg')
\end{DoxyCode}


\subsection*{var argv = subarg(args, opts)}

Parse the arguments array {\ttfamily args}, passing {\ttfamily opts} to \href{https://npmjs.org/package/minimist}{\tt minimist}.

An opening {\ttfamily \mbox{[}} in the {\ttfamily args} array creates a new context and a {\ttfamily \mbox{]}} closes a context. Contexts may be nested.

\section*{install}

With \href{https://npmjs.org}{\tt npm} do\+:


\begin{DoxyCode}
npm install subarg
\end{DoxyCode}


\section*{license}

M\+IT 