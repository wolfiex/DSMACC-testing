\href{https://travis-ci.org/RyanZim/universalify}{\tt }   

Make a callback-\/ or promise-\/based function support both promises and callbacks.

Uses the native promise implementation.

\subsection*{Installation}


\begin{DoxyCode}
npm install universalify
\end{DoxyCode}


\subsection*{A\+PI}

\subsubsection*{{\ttfamily universalify.\+from\+Callback(fn)}}

Takes a callback-\/based function to universalify, and returns the universalified function.

Function must take a callback as the last parameter that will be called with the signature {\ttfamily (error, result)}. {\ttfamily universalify} does not support calling the callback with more than three arguments, and does not ensure that the callback is only called once.


\begin{DoxyCode}
function callbackFn (n, cb) \{
  setTimeout(() => cb(null, n), 15)
\}

const fn = universalify.fromCallback(callbackFn)

// Works with Promises:
fn('Hello World!')
.then(result => console.log(result)) // -> Hello World!
.catch(error => console.error(error))

// Works with Callbacks:
fn('Hi!', (error, result) => \{
  if (error) return console.error(error)
  console.log(result)
  // -> Hi!
\})
\end{DoxyCode}


\subsubsection*{{\ttfamily universalify.\+from\+Promise(fn)}}

Takes a promise-\/based function to universalify, and returns the universalified function.

Function must return a valid JS promise. {\ttfamily universalify} does not ensure that a valid promise is returned.


\begin{DoxyCode}
function promiseFn (n) \{
  return new Promise(resolve => \{
    setTimeout(() => resolve(n), 15)
  \})
\}

const fn = universalify.fromPromise(promiseFn)

// Works with Promises:
fn('Hello World!')
.then(result => console.log(result)) // -> Hello World!
.catch(error => console.error(error))

// Works with Callbacks:
fn('Hi!', (error, result) => \{
  if (error) return console.error(error)
  console.log(result)
  // -> Hi!
\})
\end{DoxyCode}


\subsection*{License}

M\+IT 