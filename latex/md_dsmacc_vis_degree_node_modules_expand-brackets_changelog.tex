\subsubsection*{v2.\+0.\+0}

{\bfseries Breaking changes}


\begin{DoxyItemize}
\item The main export now returns the compiled string, instead of the object returned from the compiler
\end{DoxyItemize}

{\bfseries Added features}


\begin{DoxyItemize}
\item Adds a {\ttfamily .create} method to do what the main function did before v2.\+0.\+0
\end{DoxyItemize}

\subsubsection*{v0.\+2.\+0}

In addition to performance and matching improvements, the v0.\+2.\+0 refactor adds complete P\+O\+S\+IX character class support, with the exception of equivalence classes and P\+O\+S\+I\+X.\+2 collating symbols which are not relevant to node.\+js usage.

{\bfseries Added features}


\begin{DoxyItemize}
\item parser is exposed, so that expand-\/brackets parsers can be used by upstream parsers (like \mbox{[}micromatch\mbox{]}\mbox{[}\mbox{]})
\item compiler is exposed, so that expand-\/brackets compilers can be used by upstream compilers
\item source maps
\end{DoxyItemize}

{\bfseries source map example}


\begin{DoxyCode}
var brackets = require('expand-brackets');
var res = brackets('[:alpha:]');
console.log(res.map);

\{ version: 3,
     sources: [ 'brackets' ],
     names: [],
     mappings: 'AAAA,MAAS',
     sourcesContent: [ '[:alpha:]' ] \}
\end{DoxyCode}
 