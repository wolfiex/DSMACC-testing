Short example\+:


\begin{DoxyCode}
var packager = require('electron-packager')
packager(options, function done\_callback (err, appPaths) \{ /* … */ \})
\end{DoxyCode}


\subsection*{{\ttfamily options}}

\subsubsection*{Required}

\paragraph*{{\ttfamily dir}}

{\itshape String}

The source directory.

\subsubsection*{Optional}

\paragraph*{All Platforms}

\subparagraph*{{\ttfamily after\+Copy}}

{\itshape Array of Functions}

An array of functions to be called after your app directory has been copied to a temporary directory. Each function is called with five parameters\+:


\begin{DoxyItemize}
\item {\ttfamily build\+Path} ({\itshape String})\+: The path to the temporary folder where your app has been copied to
\item {\ttfamily electron\+Version} ({\itshape String})\+: The version of electron you are packaging for
\item {\ttfamily platform} ({\itshape String})\+: The target platform you are packaging for
\item {\ttfamily arch} ({\itshape String})\+: The target architecture you are packaging for
\item {\ttfamily callback} ({\itshape Function})\+: Must be called once you have completed your actions
\end{DoxyItemize}

\subparagraph*{{\ttfamily after\+Extract}}

{\itshape Array of Functions}

An array of functions to be called after Electron has been extracted to a temporary directory. Each function is called with five parameters\+:


\begin{DoxyItemize}
\item {\ttfamily build\+Path} ({\itshape String})\+: The path to the temporary folder where Electron has been extracted to
\item {\ttfamily electron\+Version} ({\itshape String})\+: The version of electron you are packaging for
\item {\ttfamily platform} ({\itshape String})\+: The target platform you are packaging for
\item {\ttfamily arch} ({\itshape String})\+: The target architecture you are packaging for
\item {\ttfamily callback} ({\itshape Function})\+: Must be called once you have completed your actions
\end{DoxyItemize}

\subparagraph*{{\ttfamily all}}

{\itshape Boolean}

When {\ttfamily true}, sets both \href{#arch}{\tt {\ttfamily arch}} and \href{#platform}{\tt {\ttfamily platform}} to {\ttfamily all}.

\subparagraph*{{\ttfamily app-\/copyright}}

{\itshape String} ({\bfseries deprecated} and will be removed in a future major version, please use the \href{#appcopyright}{\tt {\ttfamily app\+Copyright}} parameter instead)

\subparagraph*{{\ttfamily app\+Copyright}}

{\itshape String}

The human-\/readable copyright line for the app. Maps to the {\ttfamily Legal\+Copyright} metadata property on Windows, and {\ttfamily N\+S\+Human\+Readable\+Copyright} on OS X.

\subparagraph*{{\ttfamily app-\/version}}

{\itshape String} ({\bfseries deprecated} and will be removed in a future major version, please use the \href{#appversion}{\tt {\ttfamily app\+Version}} parameter instead)

\subparagraph*{{\ttfamily app\+Version}}

{\itshape String}

The release version of the application. By default the {\ttfamily version} property in the {\ttfamily package.\+json} is used but it can be overridden with this argument. If neither are provided, the version of Electron will be used. Maps to the {\ttfamily Product\+Version} metadata property on Windows, and {\ttfamily C\+F\+Bundle\+Short\+Version\+String} on OS X.

\paragraph*{{\ttfamily arch}}

{\itshape String} (default\+: the arch of the host computer running \mbox{\hyperlink{classNode}{Node}})

Allowed values\+: {\ttfamily ia32}, {\ttfamily x64}, {\ttfamily armv7l}, {\ttfamily all}

The target system architecture(s) to build for. Not required if the \href{#all}{\tt {\ttfamily all}} option is set. If {\ttfamily arch} is set to {\ttfamily all}, all supported architectures for the target platforms specified by \href{#platform}{\tt {\ttfamily platform}} will be built. Arbitrary combinations of individual architectures are also supported via a comma-\/delimited string or array of strings. The non-\/{\ttfamily all} values correspond to the architecture names used by \href{https://github.com/electron/electron/releases}{\tt Electron releases}.

\subparagraph*{{\ttfamily asar}}

{\itshape Boolean} or {\itshape Object} (default\+: {\ttfamily false})

Whether to package the application\textquotesingle{}s source code into an archive, using \href{https://github.com/electron/asar}{\tt Electron\textquotesingle{}s archive format}. Reasons why you may want to enable this feature are described in \href{http://electron.atom.io/docs/v0.36.0/tutorial/application-packaging/}{\tt an application packaging tutorial in Electron\textquotesingle{}s documentation}. When the value is {\ttfamily true}, pass default configuration to the {\ttfamily asar} module. The configuration values listed below can be customized when the value is an {\ttfamily Object}. Supported parameters include, but are not limited to\+:
\begin{DoxyItemize}
\item {\ttfamily ordering} ({\itshape String})\+: A path to an ordering file for packing files. An explanation can be found on the \href{https://github.com/atom/atom/issues/10163}{\tt Atom issue tracker}.
\item {\ttfamily unpack} ({\itshape String})\+: A \href{https://github.com/isaacs/minimatch#features}{\tt glob expression}, when specified, unpacks the file with matching names to the {\ttfamily app.\+asar.\+unpacked} directory.
\item {\ttfamily unpack\+Dir} ({\itshape String})\+: Unpacks the dir to the {\ttfamily app.\+asar.\+unpacked} directory whose names exactly or pattern match this string. The {\ttfamily asar.\+unpack\+Dir} is relative to {\ttfamily dir}.

Some examples\+:
\begin{DoxyItemize}
\item `asar.\+unpack\+Dir = \textquotesingle{}sub\+\_\+dir'{\ttfamily will unpack the directory}/$<$dir$>$/sub\+\_\+dir{\ttfamily  -\/}asar.\+unpack\+Dir = \textquotesingle{}$\ast$$\ast$/\{sub\+\_\+dir1/sub\+\_\+sub\+\_\+dir,sub\+\_\+dir2\}/$\ast$\textquotesingle{}{\ttfamily will unpack the directories}/$<$dir$>$/sub\+\_\+dir1/sub\+\_\+sub\+\_\+dir{\ttfamily and}/$<$dir$>$/sub\+\_\+dir2{\ttfamily , but it will not include their subdirectories. -\/}asar.\+unpack\+Dir = \textquotesingle{}$\ast$$\ast$/\{sub\+\_\+dir1/sub\+\_\+sub\+\_\+dir,sub\+\_\+dir2\}/$\ast$$\ast$\textquotesingle{}{\ttfamily will unpack the subdirectories of the directories}/$<$dir$>$/sub\+\_\+dir1/sub\+\_\+sub\+\_\+dir{\ttfamily and}/$<$dir$>$/sub\+\_\+dir2{\ttfamily . -\/}asar.\+unpack\+Dir = \textquotesingle{}$\ast$$\ast$/\{sub\+\_\+dir1/sub\+\_\+sub\+\_\+dir,sub\+\_\+dir2\}/$\ast$$\ast$/$\ast$\textquotesingle{}{\ttfamily will unpack the directories}/$<$dir$>$/sub\+\_\+dir1/sub\+\_\+sub\+\_\+dir{\ttfamily and}/$<$dir$>$/sub\+\_\+dir2\`{} and their subdirectories.
\end{DoxyItemize}
\end{DoxyItemize}

\subparagraph*{{\ttfamily build-\/version}}

{\itshape String} ({\bfseries deprecated} and will be removed in a future major version, please use the \href{#buildversion}{\tt {\ttfamily build\+Version}} parameter instead)

\subparagraph*{{\ttfamily build\+Version}}

{\itshape String}

The build version of the application. Defaults to the value of \href{#appversion}{\tt {\ttfamily app\+Version}}. Maps to the {\ttfamily File\+Version} metadata property on Windows, and {\ttfamily C\+F\+Bundle\+Version} on OS X.

\subparagraph*{{\ttfamily deref\+Symlinks}}

{\itshape Boolean} (default\+: {\ttfamily true})

Whether symlinks should be dereferenced during the copying of the application source.

\subparagraph*{{\ttfamily download}}

{\itshape Object}

If present, passes custom options to \href{https://www.npmjs.com/package/electron-download}{\tt {\ttfamily electron-\/download}} (see the link for more detailed option descriptions and the defaults). Supported parameters include, but are not limited to\+:
\begin{DoxyItemize}
\item {\ttfamily cache} ({\itshape String})\+: The directory where prebuilt, pre-\/packaged Electron downloads are cached.
\item {\ttfamily mirror} ({\itshape String})\+: The U\+RL to override the default Electron download location.
\item {\ttfamily quiet} ({\itshape Boolean} -\/ default\+: {\ttfamily false})\+: Whether to show a progress bar when downloading Electron.
\item {\ttfamily strict\+S\+SL} ({\itshape Boolean} -\/ default\+: {\ttfamily true})\+: Whether S\+SL certificates are required to be valid when downloading Electron.
\end{DoxyItemize}

\subparagraph*{{\ttfamily electron\+Version}}

{\itshape String}

The Electron version with which the app is built (without the leading \textquotesingle{}v\textquotesingle{}) -\/ for example, \href{https://github.com/electron/electron/releases/tag/v1.4.13}{\tt {\ttfamily 1.\+4.\+13}}. See \href{https://github.com/electron/electron/releases}{\tt Electron releases} for valid versions. If omitted, it will use the version of the nearest local installation of {\ttfamily electron}, {\ttfamily electron-\/prebuilt-\/compile}, or {\ttfamily electron-\/prebuilt}, defined in {\ttfamily package.\+json} in either {\ttfamily dependencies} or {\ttfamily dev\+Dependencies}.

\subparagraph*{{\ttfamily icon}}

{\itshape String}

The local path to the icon file, if the target platform supports setting embedding an icon.

Currently you must look for conversion tools in order to supply an icon in the format required by the platform\+:


\begin{DoxyItemize}
\item OS X\+: {\ttfamily .icns}
\item Windows\+: {\ttfamily .ico} (\href{https://github.com/electron-userland/electron-packager#building-windows-apps-from-non-windows-platforms}{\tt See the readme} for details on non-\/\+Windows platforms)
\item Linux\+: this option is not required, as the dock/window list icon is set via \href{http://electron.atom.io/docs/api/browser-window/#new-browserwindowoptions}{\tt the {\ttfamily icon} option in the {\ttfamily Browser\+Window} constructor}. {\itshape Please note that you need to use a P\+NG, and not the OS X or Windows icon formats, in order for it to show up in the dock/window list.} Setting the icon in the file manager is not currently supported.
\end{DoxyItemize}

If the file extension is omitted, it is auto-\/completed to the correct extension based on the platform, including when \href{#platform}{\tt {\ttfamily -\/-\/platform=all}} is in effect.

\subparagraph*{{\ttfamily ignore}}

{\itshape Reg\+Exp}, {\itshape Array} of {\itshape Reg\+Exp$\ast$s, or $\ast$\+Function}

One or more additional \href{https://developer.mozilla.org/en-US/docs/Web/JavaScript/Guide/Regular_Expressions}{\tt regular expression} patterns which specify which files to ignore when copying files to create the app bundle(s). The regular expressions are matched against the absolute path of a given file/directory to be copied.

The following paths are always ignored ({\itshape when you aren\textquotesingle{}t using the predicate function that is described after the list})\+:


\begin{DoxyItemize}
\item the directory specified by the \href{#out}{\tt {\ttfamily out}} parameter
\item the temporary directory used to build the Electron app
\item {\ttfamily node\+\_\+modules/.bin}
\item {\ttfamily node\+\_\+modules/electron}
\item {\ttfamily node\+\_\+modules/electron-\/prebuilt}
\item {\ttfamily node\+\_\+modules/electron-\/prebuilt-\/compile}
\item {\ttfamily node\+\_\+modules/electron-\/packager}
\item {\ttfamily .git}
\item files and folders ending in {\ttfamily .o} and {\ttfamily .obj}
\end{DoxyItemize}

{\bfseries Please note that \href{https://en.wikipedia.org/wiki/Glob_%28programming%29}{\tt glob patterns} will not work.}

{\bfseries Note}\+: If you\textquotesingle{}re trying to ignore \mbox{\hyperlink{classNode}{Node}} modules specified in {\ttfamily dev\+Dependencies}, you may want to use \href{#prune}{\tt {\ttfamily prune}} instead.

Alternatively, this can be a predicate function that, given an absolute file path, returns {\ttfamily true} if the file should be ignored, or {\ttfamily false} if the file should be kept. {\itshape This does not use any of the default ignored directories listed above.}

\subparagraph*{{\ttfamily name}}

{\itshape String}

The application name. If omitted, it will use the {\ttfamily product\+Name} or {\ttfamily name} value from the nearest {\ttfamily package.\+json}.

{\bfseries Regardless of source, characters in the Electron app name which are not allowed in all target platforms\textquotesingle{} filenames (e.\+g., {\ttfamily /}), will be replaced by hyphens ({\ttfamily -\/}).}

\subparagraph*{{\ttfamily out}}

{\itshape String} (default\+: current working directory)

The base directory where the finished package(s) are created.

\subparagraph*{{\ttfamily overwrite}}

{\itshape Boolean} (default\+: {\ttfamily false})

Whether to replace an already existing output directory for a given platform ({\ttfamily true}) or skip recreating it ({\ttfamily false}).

\subparagraph*{{\ttfamily package\+Manager}}

{\itshape String} (default\+: {\ttfamily npm})

The package manager used to \href{#prune}{\tt prune} {\ttfamily dev\+Dependencies} modules from the outputted Electron app. Supported package managers\+:


\begin{DoxyItemize}
\item \href{https://npmjs.com/}{\tt {\ttfamily npm}}
\item \href{https://github.com/cnpm/cnpm}{\tt {\ttfamily cnpm}} (Does not currently work with Windows, see \href{https://github.com/electron-userland/electron-packager/issues/515#issuecomment-297604044}{\tt Git\+Hub issue})
\item \href{https://yarnpkg.com/}{\tt {\ttfamily yarn}}
\end{DoxyItemize}

\subparagraph*{{\ttfamily platform}}

{\itshape String} (default\+: the arch of the host computer running \mbox{\hyperlink{classNode}{Node}})

Allowed values\+: {\ttfamily linux}, {\ttfamily win32}, {\ttfamily darwin}, {\ttfamily mas}, {\ttfamily all}

The target platform(s) to build for. Not required if the \href{#all}{\tt {\ttfamily all}} option is set. If {\ttfamily platform} is set to {\ttfamily all}, all \href{#supported-platforms}{\tt supported target platforms} for the target architectures specified by \href{#arch}{\tt {\ttfamily arch}} will be built. Arbitrary combinations of individual platforms are also supported via a comma-\/delimited string or array of strings. The non-\/{\ttfamily all} values correspond to the platform names used by \href{https://github.com/electron/electron/releases}{\tt Electron releases}.

\subparagraph*{{\ttfamily prune}}

{\itshape Boolean} (default\+: {\ttfamily true})

Runs the \href{#packagemanager}{\tt package manager} command to remove all of the packages specified in the {\ttfamily dev\+Dependencies} section of {\ttfamily package.\+json} from the outputted Electron app.

\subparagraph*{{\ttfamily quiet}}

{\itshape Boolean} (default\+: {\ttfamily false})

If {\ttfamily true}, disables printing informational and warning messages to the console when packaging the application. This does {\itshape not} disable errors.

\subparagraph*{{\ttfamily tmpdir}}

{\itshape String} or $\ast${\ttfamily false}$\ast$ (default\+: system temp directory)

The base directory to use as a temp directory. Set to {\ttfamily false} to disable use of a temporary directory.

\subparagraph*{{\ttfamily version}}

{\itshape String} ({\bfseries deprecated} and will be removed in a future major version, please use the \href{#electronversion}{\tt {\ttfamily electron\+Version}} parameter instead)

\paragraph*{OS X/\+Mac App Store targets only}

\subparagraph*{{\ttfamily app-\/bundle-\/id}}

{\itshape String} ({\bfseries deprecated} and will be removed in a future major version, please use the \href{#appbundleid}{\tt {\ttfamily app\+Bundle\+Id}} parameter instead)

\subparagraph*{{\ttfamily app\+Bundle\+Id}}

{\itshape String}

The bundle identifier to use in the application\textquotesingle{}s plist.

\subparagraph*{{\ttfamily app-\/category-\/type}}

{\itshape String} ({\bfseries deprecated} and will be removed in a future major version, please use the \href{#appcategorytype}{\tt {\ttfamily app\+Category\+Type}} parameter instead)

\subparagraph*{{\ttfamily app\+Category\+Type}}

{\itshape String}

The application category type, as shown in the Finder via {\itshape View → Arrange by Application Category} when viewing the Applications directory.

For example, {\ttfamily app-\/category-\/type=public.\+app-\/category.\+developer-\/tools} will set the application category to {\itshape Developer Tools}.

Valid values are listed in \href{https://developer.apple.com/library/ios/documentation/General/Reference/InfoPlistKeyReference/Articles/LaunchServicesKeys.html#//apple_ref/doc/uid/TP40009250-SW8}{\tt Apple\textquotesingle{}s documentation}.

\subparagraph*{{\ttfamily extend-\/info}}

{\itshape String} ({\bfseries deprecated} and will be removed in a future major version, please use the \href{#extendinfo}{\tt {\ttfamily extend\+Info}} parameter instead)

\subparagraph*{{\ttfamily extend\+Info}}

{\itshape String} or {\itshape Object}

When the value is a {\ttfamily String}, the filename of a plist file. Its contents are added to the app\textquotesingle{}s plist. When the value is an {\ttfamily Object}, an already-\/parsed plist data structure that is merged into the app\textquotesingle{}s plist.

Entries from {\ttfamily extend-\/info} override entries in the base plist file supplied by {\ttfamily electron}, {\ttfamily electron-\/prebuilt-\/compile}, or {\ttfamily electron-\/prebuilt}, but are overridden by other explicit arguments such as \href{#appversion}{\tt {\ttfamily app\+Version}} or \href{#appbundleid}{\tt {\ttfamily app\+Bundle\+Id}}.

\subparagraph*{{\ttfamily extra-\/resource}}

{\itshape String} ({\bfseries deprecated} and will be removed in a future major version, please use the \href{#extraresource}{\tt {\ttfamily extra\+Resource}} parameter instead)

\subparagraph*{{\ttfamily extra\+Resource}}

{\itshape String} or {\itshape Array}

Filename of a file to be copied directly into the app\textquotesingle{}s {\ttfamily Contents/\+Resources} directory.

\subparagraph*{{\ttfamily helper-\/bundle-\/id}}

{\itshape String} ({\bfseries deprecated} and will be removed in a future major version, please use the \href{#helperbundleid}{\tt {\ttfamily helper\+Bundle\+Id}} parameter instead)

\subparagraph*{{\ttfamily helper\+Bundle\+Id}}

{\itshape String}

The bundle identifier to use in the application helper\textquotesingle{}s plist.

\subparagraph*{{\ttfamily osx-\/sign}}

{\itshape String} ({\bfseries deprecated} and will be removed in a future major version, please use the \href{#osxsign}{\tt {\ttfamily osx\+Sign}} parameter instead)

\subparagraph*{{\ttfamily osx\+Sign}}

{\itshape Object} or $\ast${\ttfamily true}$\ast$

If present, signs OS X target apps when the host platform is OS X and X\+Code is installed. When the value is {\ttfamily true}, pass default configuration to the signing module. The configuration values listed below can be customized when the value is an {\ttfamily Object}. See \href{https://www.npmjs.com/package/electron-osx-sign#opts}{\tt electron-\/osx-\/sign} for more detailed option descriptions and the defaults.
\begin{DoxyItemize}
\item {\ttfamily identity} ({\itshape String})\+: The identity used when signing the package via {\ttfamily codesign}.
\item {\ttfamily entitlements} ({\itshape String})\+: The path to the \textquotesingle{}parent\textquotesingle{} entitlements.
\item {\ttfamily entitlements-\/inherit} ({\itshape String})\+: The path to the \textquotesingle{}child\textquotesingle{} entitlements.
\end{DoxyItemize}

\subparagraph*{{\ttfamily protocol}}

{\itshape Array} of $\ast$\+String$\ast$​s

The U\+RL protocol scheme(s) to associate the app with. For example, specifying {\ttfamily myapp} would cause U\+R\+Ls such as {\ttfamily myapp\+://path} to be opened with the app. Maps to the {\ttfamily C\+F\+Bundle\+U\+R\+L\+Schemes} metadata property. This option requires a corresponding {\ttfamily protocol-\/name} option to be specified.

\subparagraph*{{\ttfamily protocol-\/name}}

{\itshape String} ({\bfseries deprecated} and will be removed in a future major version, please use the \href{#protocolname}{\tt {\ttfamily protocol\+Name}} parameter instead)

\subparagraph*{{\ttfamily protocol\+Name}}

{\itshape Array} of $\ast$\+String$\ast$​s

The descriptive name(s) of the U\+RL protocol scheme(s) specified via the {\ttfamily protocol} option. Maps to the {\ttfamily C\+F\+Bundle\+U\+R\+L\+Name} metadata property.

\paragraph*{Windows targets only}

\subparagraph*{{\ttfamily version-\/string}}

{\itshape Object} ({\bfseries deprecated} and will be removed in a future major version, please use the \href{#win32metadata}{\tt {\ttfamily win32metadata}} parameter instead)

\subparagraph*{{\ttfamily win32metadata}}

{\itshape Object}

Object (also known as a \char`\"{}hash\char`\"{}) of application metadata to embed into the executable\+:
\begin{DoxyItemize}
\item {\ttfamily Company\+Name}
\item {\ttfamily File\+Description}
\item {\ttfamily Original\+Filename}
\item {\ttfamily Product\+Name}
\item {\ttfamily Internal\+Name}
\item {\ttfamily requested-\/execution-\/level}
\item {\ttfamily application-\/manifest}
\end{DoxyItemize}

For more information, see the \href{https://github.com/electron/node-rcedit}{\tt node-\/rcedit module}.

\subsection*{callback}

\subsubsection*{{\ttfamily err}}

{\itshape Error} (or {\itshape Array}, in the case of an {\ttfamily copy} error)

Contains errors, if any.

\subsubsection*{{\ttfamily app\+Paths}}

{\itshape Array} of $\ast$\+String$\ast$s

Paths to the newly created application bundles. 