\href{https://en.wikipedia.org/wiki/Drag_and_drop}{\tt Drag-\/and-\/drop} is a popular and easy-\/to-\/learn pointing gesture\+: move the pointer to an object, press and hold to grab it, “drag” the object to a new location, and release to “drop”. D3’s \href{#api-reference}{\tt drag behavior} provides a convenient but flexible abstraction for enabling drag-\/and-\/drop interaction on \href{https://github.com/d3/d3-selection}{\tt selections}. For example, you can use d3-\/drag to facilitate interaction with a \href{https://github.com/d3/d3-force}{\tt force-\/directed graph}, or a simulation of colliding circles\+:

\href{http://bl.ocks.org/mbostock/ad70335eeef6d167bc36fd3c04378048}{\tt }\href{http://bl.ocks.org/mbostock/2990a882e007f8384b04827617752738}{\tt }

You can also use d3-\/drag to implement custom user interface elements, such as a slider. But the drag behavior isn’t just for moving elements around; there are a variety of ways to respond to a drag gesture. For example, you can use it to lasso elements in a scatterplot, or to paint lines on a canvas\+:

\href{http://bl.ocks.org/mbostock/f705fc55e6f26df29354}{\tt }

The drag behavior can be combined with other behaviors, such as \href{https://github.com/d3/d3-zoom}{\tt d3-\/zoom} for zooming.

\href{http://bl.ocks.org/mbostock/3127661b6f13f9316be745e77fdfb084}{\tt }

The drag behavior is agnostic about the D\+OM, so you can use it with S\+VG, H\+T\+ML or even Canvas! And you can extend it with advanced selection techniques, such as a Voronoi overlay or a closest-\/target search\+:

\href{http://bl.ocks.org/mbostock/ec10387f24c1fad2acac3bc11eb218a5}{\tt }\href{http://bl.ocks.org/mbostock/c206c20294258c18832ff80d8fd395c3}{\tt }

Best of all, the drag behavior automatically unifies mouse and touch input, and avoids browser idiosyncrasies. When \href{https://www.w3.org/TR/pointerevents/}{\tt Pointer Events} are more widely available, the drag behavior will support those, too.

\subsection*{Installing}

If you use N\+PM, {\ttfamily npm install d3-\/drag}. Otherwise, download the \href{https://github.com/d3/d3-drag/releases/latest}{\tt latest release}. You can also load directly from \href{https://d3js.org}{\tt d3js.\+org}, either as a \href{https://d3js.org/d3-drag.v1.min.js}{\tt standalone library} or as part of \href{https://github.com/d3/d3}{\tt D3 4.\+0}. A\+MD, Common\+JS, and vanilla environments are supported. In vanilla, a {\ttfamily d3} global is exported\+:


\begin{DoxyCode}
<script src="https://d3js.org/d3-dispatch.v1.min.js"></script>
<script src="https://d3js.org/d3-selection.v1.min.js"></script>
<script src="https://d3js.org/d3-drag.v1.min.js"></script>
<script>

var drag = d3.drag();

</script>
\end{DoxyCode}


\href{https://tonicdev.com/npm/d3-drag}{\tt Try d3-\/drag in your browser.}

\subsection*{A\+PI Reference}

This table describes how the drag behavior interprets native events\+:

\tabulinesep=1mm
\begin{longtabu} spread 0pt [c]{*{4}{|X[-1]}|}
\hline
\rowcolor{\tableheadbgcolor}\textbf{ Event  }&\textbf{ Listening Element  }&\textbf{ Drag Event  }&\textbf{ Default Prevented?   }\\\cline{1-4}
\endfirsthead
\hline
\endfoot
\hline
\rowcolor{\tableheadbgcolor}\textbf{ Event  }&\textbf{ Listening Element  }&\textbf{ Drag Event  }&\textbf{ Default Prevented?   }\\\cline{1-4}
\endhead
mousedown⁵  &selection  &start  &no¹   \\\cline{1-4}
mousemove²  &window¹  &drag  &yes   \\\cline{1-4}
mouseup²  &window¹  &end  &yes   \\\cline{1-4}
dragstart²  &window  &-\/  &yes   \\\cline{1-4}
selectstart²  &window  &-\/  &yes   \\\cline{1-4}
click³  &window  &-\/  &yes   \\\cline{1-4}
touchstart  &selection  &start  &no⁴   \\\cline{1-4}
touchmove  &selection  &drag  &yes   \\\cline{1-4}
touchend  &selection  &end  &no⁴   \\\cline{1-4}
touchcancel  &selection  &end  &no⁴   \\\cline{1-4}
\end{longtabu}


The propagation of all consumed events is \href{https://dom.spec.whatwg.org/#dom-event-stopimmediatepropagation}{\tt immediately stopped}. If you want to prevent some events from initiating a drag gesture, use \href{#drag_filter}{\tt {\itshape drag}.filter}.

¹ Necessary to capture events outside an iframe; see \href{https://github.com/d3/d3-drag/issues/9}{\tt \#9}. ~\newline
² Only applies during an active, mouse-\/based gesture; see \href{https://github.com/d3/d3-drag/issues/9}{\tt \#9}. ~\newline
³ Only applies immediately after some mouse-\/based gestures; see \href{#drag_clickDistance}{\tt {\itshape drag}.click\+Distance}. ~\newline
⁴ Necessary to allow \href{https://developer.apple.com/library/ios/documentation/AppleApplications/Reference/SafariWebContent/HandlingEvents/HandlingEvents.html#//apple_ref/doc/uid/TP40006511-SW7}{\tt click emulation} on touch input; see \href{https://github.com/d3/d3-drag/issues/9}{\tt \#9}. ~\newline
⁵ Ignored if within 500ms of a touch gesture ending; assumes \href{https://developer.apple.com/library/ios/documentation/AppleApplications/Reference/SafariWebContent/HandlingEvents/HandlingEvents.html#//apple_ref/doc/uid/TP40006511-SW7}{\tt click emulation}.

\href{#drag}{\tt \#} d3.{\bfseries drag}() \href{https://github.com/d3/d3-drag/blob/master/src/drag.js}{\tt $<$$>$}

Creates a new drag behavior. The returned behavior, \href{#_drag}{\tt {\itshape drag}}, is both an object and a function, and is typically applied to selected elements via \href{https://github.com/d3/d3-selection#selection_call}{\tt {\itshape selection}.call}.

\href{#_drag}{\tt \#} {\itshape drag}({\itshape selection}) \href{https://github.com/d3/d3-drag/blob/master/src/drag.js#L39}{\tt $<$$>$}

Applies this drag behavior to the specified \href{https://github.com/d3/d3-selection}{\tt {\itshape selection}}. This function is typically not invoked directly, and is instead invoked via \href{https://github.com/d3/d3-selection#selection_call}{\tt {\itshape selection}.call}. For example, to instantiate a drag behavior and apply it to a selection\+:


\begin{DoxyCode}
d3.selectAll(".node").call(d3.drag().on("start", started));
\end{DoxyCode}


Internally, the drag behavior uses \href{https://github.com/d3/d3-selection#selection_on}{\tt {\itshape selection}.on} to bind the necessary event listeners for dragging. The listeners use the name {\ttfamily .drag}, so you can subsequently unbind the drag behavior as follows\+:


\begin{DoxyCode}
selection.on(".drag", null);
\end{DoxyCode}


Applying the drag behavior also sets the \href{https://developer.apple.com/library/mac/documentation/AppleApplications/Reference/SafariWebContent/AdjustingtheTextSize/AdjustingtheTextSize.html#//apple_ref/doc/uid/TP40006510-SW5}{\tt -\/webkit-\/tap-\/highlight-\/color} style to transparent, disabling the tap highlight on i\+OS. If you want a different tap highlight color, remove or re-\/apply this style after applying the drag behavior.

\href{#drag_container}{\tt \#} {\itshape drag}.{\bfseries container}(\mbox{[}{\itshape container}\mbox{]}) \href{https://github.com/d3/d3-drag/blob/master/src/drag.js#L145}{\tt $<$$>$}

If {\itshape container} is specified, sets the container accessor to the specified object or function and returns the drag behavior. If {\itshape container} is not specified, returns the current container accessor, which defaults to\+:


\begin{DoxyCode}
function container() \{
  return this.parentNode;
\}
\end{DoxyCode}


The {\itshape container} of a drag gesture determines the coordinate system of subsequent \href{#drag-events}{\tt drag events}, affecting {\itshape event}.x and {\itshape event}.y. The element returned by the container accessor is subsequently passed to \href{https://github.com/d3/d3-selection#mouse}{\tt d3.\+mouse} or \href{https://github.com/d3/d3-selection#touch}{\tt d3.\+touch}, as appropriate, to determine the local coordinates of the pointer.

The default container accessor returns the parent node of the element in the originating selection (see \href{#_drag}{\tt {\itshape drag}}) that received the initiating input event. This is often appropriate when dragging S\+VG or H\+T\+ML elements, since those elements are typically positioned relative to a parent. For dragging graphical elements with a Canvas, however, you may want to redefine the container as the initiating element itself\+:


\begin{DoxyCode}
function container() \{
  return this;
\}
\end{DoxyCode}


Alternatively, the container may be specified as the element directly, such as {\ttfamily drag.\+container(canvas)}.

\href{#drag_filter}{\tt \#} {\itshape drag}.{\bfseries filter}(\mbox{[}{\itshape filter}\mbox{]}) \href{https://github.com/d3/d3-drag/blob/master/src/drag.js#L141}{\tt $<$$>$}

If {\itshape filter} is specified, sets the filter to the specified function and returns the drag behavior. If {\itshape filter} is not specified, returns the current filter, which defaults to\+:


\begin{DoxyCode}
function filter() \{
  return !d3.event.button;
\}
\end{DoxyCode}


If the filter returns falsey, the initiating event is ignored and no drag gestures are started. Thus, the filter determines which input events are ignored; the default filter ignores mousedown events on secondary buttons, since those buttons are typically intended for other purposes, such as the context menu.

\href{#touchable}{\tt \#} {\itshape drag}.{\bfseries touchable}(\mbox{[}{\itshape touchable}\mbox{]}) \href{https://github.com/d3/d3-drag/blob/master/src/drag.js#L153}{\tt $<$$>$}

If {\itshape touchable} is specified, sets the touch support detector to the specified function and returns the drag behavior. If {\itshape touchable} is not specified, returns the current touch support detector, which defaults to\+:


\begin{DoxyCode}
function touchable() \{
  return "ontouchstart" in this;
\}
\end{DoxyCode}


Touch event listeners are only registered if the detector returns truthy for the corresponding element when the drag behavior is \href{#_drag}{\tt applied}. The default detector works well for most browsers that are capable of touch input, but not all; Chrome’s mobile device emulator, for example, fails detection.

\href{#drag_subject}{\tt \#} {\itshape drag}.{\bfseries subject}(\mbox{[}{\itshape subject}\mbox{]}) \href{https://github.com/d3/d3-drag/blob/master/src/drag.js#L149}{\tt $<$$>$}

If {\itshape subject} is specified, sets the subject accessor to the specified object or function and returns the drag behavior. If {\itshape subject} is not specified, returns the current subject accessor, which defaults to\+:


\begin{DoxyCode}
function subject(d) \{
  return d == null ? \{x: d3.event.x, y: d3.event.y\} : d;
\}
\end{DoxyCode}


The {\itshape subject} of a drag gesture represents {\itshape the thing being dragged}. It is computed when an initiating input event is received, such as a mousedown or touchstart, immediately before the drag gesture starts. The subject is then exposed as {\itshape event}.subject on subsequent \href{#drag-events}{\tt drag events} for this gesture.

The default subject is the \href{https://github.com/d3/d3-selection#selection_datum}{\tt datum} of the element in the originating selection (see \href{#_drag}{\tt {\itshape drag}}) that received the initiating input event; if this datum is undefined, an object representing the coordinates of the pointer is created. When dragging circle elements in S\+VG, the default subject is thus the datum of the circle being dragged. With \href{https://html.spec.whatwg.org/multipage/scripting.html#the-canvas-element}{\tt Canvas}, the default subject is the canvas element’s datum (regardless of where on the canvas you click). In this case, a custom subject accessor would be more appropriate, such as one that picks the closest circle to the mouse within a given search {\itshape radius}\+:


\begin{DoxyCode}
function subject() \{
  var n = circles.length,
      i,
      dx,
      dy,
      d2,
      s2 = radius * radius,
      circle,
      subject;

  for (i = 0; i < n; ++i) \{
    circle = circles[i];
    dx = d3.event.x - circle.x;
    dy = d3.event.y - circle.y;
    d2 = dx * dx + dy * dy;
    if (d2 < s2) subject = circle, s2 = d2;
  \}

  return subject;
\}
\end{DoxyCode}


(If necessary, the above can be accelerated using \href{https://github.com/d3/d3-quadtree#quadtree_find}{\tt {\itshape quadtree}.find}.)

The returned subject should be an object that exposes {\ttfamily x} and {\ttfamily y} properties, so that the relative position of the subject and the pointer can be preserved during the drag gesture. If the subject is null or undefined, no drag gesture is started for this pointer; however, other starting touches may yet start drag gestures. See also \href{#drag_filter}{\tt {\itshape drag}.filter}.

The subject of a drag gesture may not be changed after the gesture starts. The subject accessor is invoked with the same context and arguments as \href{https://github.com/d3/d3-selection#selection_on}{\tt {\itshape selection}.on} listeners\+: the current datum {\ttfamily d} and index {\ttfamily i}, with the {\ttfamily this} context as the current D\+OM element. During the evaluation of the subject accessor, \href{https://github.com/d3/d3-selection#event}{\tt d3.\+event} is a beforestart \href{#drag-events}{\tt drag event}. Use {\itshape event}.source\+Event to access the initiating input event and {\itshape event}.identifier to access the touch identifier. The {\itshape event}.x and {\itshape event}.y are relative to the \href{#drag_container}{\tt container}, and are computed using \href{https://github.com/d3/d3-selection#mouse}{\tt d3.\+mouse} or \href{https://github.com/d3/d3-selection#touch}{\tt d3.\+touch} as appropriate.

\href{#drag_clickDistance}{\tt \#} {\itshape drag}.{\bfseries click\+Distance}(\mbox{[}{\itshape distance}\mbox{]}) \href{https://github.com/d3/d3-drag/blob/master/src/drag.js#L162}{\tt $<$$>$}

If {\itshape distance} is specified, sets the maximum distance that the mouse can move between mousedown and mouseup that will trigger a subsequent click event. If at any point between mousedown and mouseup the mouse is greater than or equal to {\itshape distance} from its position on mousedown, the click event follwing mouseup will be suppressed. If {\itshape distance} is not specified, returns the current distance threshold, which defaults to zero. The distance threshold is measured in client coordinates (\href{https://developer.mozilla.org/en-US/docs/Web/API/MouseEvent/clientX}{\tt {\itshape event}.clientX} and \href{https://developer.mozilla.org/en-US/docs/Web/API/MouseEvent/clientY}{\tt {\itshape event}.clientY}).

\href{#drag_on}{\tt \#} {\itshape drag}.{\bfseries on}({\itshape typenames}, \mbox{[}{\itshape listener}\mbox{]}) \href{https://github.com/d3/d3-drag/blob/master/src/drag.js#L157}{\tt $<$$>$}

If {\itshape listener} is specified, sets the event {\itshape listener} for the specified {\itshape typenames} and returns the drag behavior. If an event listener was already registered for the same type and name, the existing listener is removed before the new listener is added. If {\itshape listener} is null, removes the current event listeners for the specified {\itshape typenames}, if any. If {\itshape listener} is not specified, returns the first currently-\/assigned listener matching the specified {\itshape typenames}, if any. When a specified event is dispatched, each {\itshape listener} will be invoked with the same context and arguments as \href{https://github.com/d3/d3-selection#selection_on}{\tt {\itshape selection}.on} listeners\+: the current datum {\ttfamily d} and index {\ttfamily i}, with the {\ttfamily this} context as the current D\+OM element.

The {\itshape typenames} is a string containing one or more {\itshape typename} separated by whitespace. Each {\itshape typename} is a {\itshape type}, optionally followed by a period ({\ttfamily .}) and a {\itshape name}, such as {\ttfamily drag.\+foo} and {\ttfamily drag.\+bar}; the name allows multiple listeners to be registered for the same {\itshape type}. The {\itshape type} must be one of the following\+:


\begin{DoxyItemize}
\item {\ttfamily start} -\/ after a new pointer becomes active (on mousedown or touchstart).
\item {\ttfamily drag} -\/ after an active pointer moves (on mousemove or touchmove).
\item {\ttfamily end} -\/ after an active pointer becomes inactive (on mouseup, touchend or touchcancel).
\end{DoxyItemize}

See \href{https://github.com/d3/d3-dispatch#dispatch_on}{\tt {\itshape dispatch}.on} for more.

Changes to registered listeners via {\itshape drag}.on during a drag gesture {\itshape do not affect} the current drag gesture. Instead, you must use \href{#event_on}{\tt {\itshape event}.on}, which also allows you to register temporary event listeners for the current drag gesture. {\bfseries Separate events are dispatched for each active pointer} during a drag gesture. For example, if simultaneously dragging multiple subjects with multiple fingers, a start event is dispatched for each finger, even if both fingers start touching simultaneously. See \href{#drag-events}{\tt Drag Events} for more.

\href{#dragDisable}{\tt \#} d3.{\bfseries drag\+Disable}({\itshape window}) \href{https://github.com/d3/d3-drag/blob/master/src/nodrag.js#L4}{\tt $<$$>$}

Prevents native drag-\/and-\/drop and text selection on the specified {\itshape window}. As an alternative to preventing the default action of mousedown events (see \href{https://github.com/d3/d3-drag/issues/9}{\tt \#9}), this method prevents undesirable default actions following mousedown. In supported browsers, this means capturing dragstart and selectstart events, preventing the associated default actions, and immediately stopping their propagation. In browsers that do not support selection events, the user-\/select C\+SS property is set to none on the document element. This method is intended to be called on mousedown, followed by \href{#dragEnable}{\tt d3.\+drag\+Enable} on mouseup.

\href{#dragEnable}{\tt \#} d3.{\bfseries drag\+Enable}({\itshape window}\mbox{[}, {\itshape noclick}\mbox{]}) \href{https://github.com/d3/d3-drag/blob/master/src/nodrag.js#L15}{\tt $<$$>$}

Allows native drag-\/and-\/drop and text selection on the specified {\itshape window}; undoes the effect of \href{#dragDisable}{\tt d3.\+drag\+Disable}. This method is intended to be called on mouseup, preceded by \href{#dragDisable}{\tt d3.\+drag\+Disable} on mousedown. If {\itshape noclick} is true, this method also temporarily suppresses click events. The suppression of click events expires after a zero-\/millisecond timeout, such that it only suppress the click event that would immediately follow the current mouseup event, if any.

\subsubsection*{Drag Events}

When a \href{#drag_on}{\tt drag event listener} is invoked, \href{https://github.com/d3/d3-selection#event}{\tt d3.\+event} is set to the current drag event. The {\itshape event} object exposes several fields\+:


\begin{DoxyItemize}
\item {\ttfamily target} -\/ the associated \href{#drag}{\tt drag behavior}.
\item {\ttfamily type} -\/ the string “start”, “drag” or “end”; see \href{#drag_on}{\tt {\itshape drag}.on}.
\item {\ttfamily subject} -\/ the drag subject, defined by \href{#drag_subject}{\tt {\itshape drag}.subject}.
\item {\ttfamily x} -\/ the new {\itshape x}-\/coordinate of the subject; see \href{#drag_container}{\tt {\itshape drag}.container}.
\item {\ttfamily y} -\/ the new {\itshape y}-\/coordinate of the subject; see \href{#drag_container}{\tt {\itshape drag}.container}.
\item {\ttfamily dx} -\/ the change in {\itshape x}-\/coordinate since the previous drag event.
\item {\ttfamily dy} -\/ the change in {\itshape y}-\/coordinate since the previous drag event.
\item {\ttfamily identifier} -\/ the string “mouse”, or a numeric \href{https://www.w3.org/TR/touch-events/#widl-Touch-identifier}{\tt touch identifier}.
\item {\ttfamily active} -\/ the number of currently active drag gestures (on start and end, not including this one).
\item {\ttfamily source\+Event} -\/ the underlying input event, such as mousemove or touchmove.
\end{DoxyItemize}

The {\itshape event}.active field is useful for detecting the first start event and the last end event in a sequence of concurrent drag gestures\+: it is zero when the first drag gesture starts, and zero when the last drag gesture ends.

The {\itshape event} object also exposes the \href{#event_on}{\tt {\itshape event}.on} method.

\href{#event_on}{\tt \#} {\itshape event}.{\bfseries on}({\itshape typenames}, \mbox{[}{\itshape listener}\mbox{]}) \href{https://github.com/d3/d3-drag/blob/master/src/event.js}{\tt $<$$>$}

Equivalent to \href{#drag_on}{\tt {\itshape drag}.on}, but only applies to the current drag gesture. Before the drag gesture starts, a \href{https://github.com/d3/d3-dispatch#dispatch_copy}{\tt copy} of the current drag \href{#drag_on}{\tt event listeners} is made. This copy is bound to the current drag gesture and modified by {\itshape event}.on. This is useful for temporary listeners that only receive events for the current drag gesture. For example, this start event listener registers temporary drag and end event listeners as closures\+:


\begin{DoxyCode}
function started() \{
  var circle = d3.select(this).classed("dragging", true);

  d3.event.on("drag", dragged).on("end", ended);

  function dragged(d) \{
    circle.raise().attr("cx", d.x = d3.event.x).attr("cy", d.y = d3.event.y);
  \}

  function ended() \{
    circle.classed("dragging", false);
  \}
\}
\end{DoxyCode}
 