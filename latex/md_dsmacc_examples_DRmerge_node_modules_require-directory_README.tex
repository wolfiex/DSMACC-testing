Recursively iterates over specified directory, {\ttfamily require()}\textquotesingle{}ing each file, and returning a nested hash structure containing those modules.

{\bfseries \href{https://twitter.com/intent/user?screen_name=troygoode}{\tt Follow me () on Twitter!}}

\href{https://nodei.co/npm/require-directory/}{\tt }

\href{http://travis-ci.org/troygoode/node-require-directory}{\tt }

\subsection*{How To Use}

\subsubsection*{Installation (via \href{https://npmjs.org/package/require-directory}{\tt npm})}


\begin{DoxyCode}
$ npm install require-directory
\end{DoxyCode}


\subsubsection*{Usage}

A common pattern in node.\+js is to include an index file which creates a hash of the files in its current directory. Given a directory structure like so\+:


\begin{DoxyItemize}
\item app.\+js
\item routes/
\begin{DoxyItemize}
\item index.\+js
\item home.\+js
\item auth/
\begin{DoxyItemize}
\item login.\+js
\item logout.\+js
\item register.\+js
\end{DoxyItemize}
\end{DoxyItemize}
\end{DoxyItemize}

{\ttfamily routes/index.\+js} uses {\ttfamily require-\/directory} to build the hash (rather than doing so manually) like so\+:


\begin{DoxyCode}
var requireDirectory = require('require-directory');
module.exports = requireDirectory(module);
\end{DoxyCode}


{\ttfamily app.\+js} references {\ttfamily routes/index.\+js} like any other module, but it now has a hash/tree of the exports from the {\ttfamily ./routes/} directory\+:


\begin{DoxyCode}
var routes = require('./routes');

// snip

app.get('/', routes.home);
app.get('/register', routes.auth.register);
app.get('/login', routes.auth.login);
app.get('/logout', routes.auth.logout);
\end{DoxyCode}


The {\ttfamily routes} variable above is the equivalent of this\+:


\begin{DoxyCode}
var routes = \{
  home: require('routes/home.js'),
  auth: \{
    login: require('routes/auth/login.js'),
    logout: require('routes/auth/logout.js'),
    register: require('routes/auth/register.js')
  \}
\};
\end{DoxyCode}


{\itshape Note that {\ttfamily routes.\+index} will be {\ttfamily undefined} as you would hope.}

\subsubsection*{Specifying Another Directory}

You can specify which directory you want to build a tree of (if it isn\textquotesingle{}t the current directory for whatever reason) by passing it as the second parameter. Not specifying the path ({\ttfamily require\+Directory(module)}) is the equivelant of {\ttfamily require\+Directory(module, \+\_\+\+\_\+dirname)}\+:


\begin{DoxyCode}
var requireDirectory = require('require-directory');
module.exports = requireDirectory(module, './some/subdirectory');
\end{DoxyCode}


For example, in the \href{#usage}{\tt example in the Usage section} we could have avoided creating {\ttfamily routes/index.\+js} and instead changed the first lines of {\ttfamily app.\+js} to\+:


\begin{DoxyCode}
var requireDirectory = require('require-directory');
var routes = requireDirectory(module, './routes');
\end{DoxyCode}


\subsection*{Options}

You can pass an options hash to {\ttfamily require-\/directory} as the 2nd parameter (or 3rd if you\textquotesingle{}re passing the path to another directory as the 2nd parameter already). Here are the available options\+:

\subsubsection*{Whitelisting}

Whitelisting (either via Reg\+Exp or function) allows you to specify that only certain files be loaded.


\begin{DoxyCode}
var requireDirectory = require('require-directory'),
  whitelist = /onlyinclude.js$/,
  hash = requireDirectory(module, \{include: whitelist\});
\end{DoxyCode}



\begin{DoxyCode}
var requireDirectory = require('require-directory'),
  check = function(path)\{
    if(/onlyinclude.js$/.test(path))\{
      return true; // don't include
    \}else\{
      return false; // go ahead and include
    \}
  \},
  hash = requireDirectory(module, \{include: check\});
\end{DoxyCode}


\subsubsection*{Blacklisting}

Blacklisting (either via Reg\+Exp or function) allows you to specify that all but certain files should be loaded.


\begin{DoxyCode}
var requireDirectory = require('require-directory'),
  blacklist = /dontinclude\(\backslash\).js$/,
  hash = requireDirectory(module, \{exclude: blacklist\});
\end{DoxyCode}



\begin{DoxyCode}
var requireDirectory = require('require-directory'),
  check = function(path)\{
    if(/dontinclude\(\backslash\).js$/.test(path))\{
      return false; // don't include
    \}else\{
      return true; // go ahead and include
    \}
  \},
  hash = requireDirectory(module, \{exclude: check\});
\end{DoxyCode}


\subsubsection*{Visiting Objects As They\textquotesingle{}re Loaded}

{\ttfamily require-\/directory} takes a function as the {\ttfamily visit} option that will be called for each module that is added to module.\+exports.


\begin{DoxyCode}
var requireDirectory = require('require-directory'),
  visitor = function(obj) \{
    console.log(obj); // will be called for every module that is loaded
  \},
  hash = requireDirectory(module, \{visit: visitor\});
\end{DoxyCode}


The visitor can also transform the objects by returning a value\+:


\begin{DoxyCode}
var requireDirectory = require('require-directory'),
  visitor = function(obj) \{
    return obj(new Date());
  \},
  hash = requireDirectory(module, \{visit: visitor\});
\end{DoxyCode}


\subsubsection*{Renaming Keys}


\begin{DoxyCode}
var requireDirectory = require('require-directory'),
  renamer = function(name) \{
    return name.toUpperCase();
  \},
  hash = requireDirectory(module, \{rename: renamer\});
\end{DoxyCode}


\subsubsection*{No Recursion}


\begin{DoxyCode}
var requireDirectory = require('require-directory'),
  hash = requireDirectory(module, \{recurse: false\});
\end{DoxyCode}


\subsection*{Run Unit Tests}


\begin{DoxyCode}
$ npm run lint
$ npm test
\end{DoxyCode}


\subsection*{License}

\href{http://www.opensource.org/licenses/mit-license.php}{\tt M\+IT License}

\subsection*{Author}

\href{https://github.com/TroyGoode}{\tt Troy Goode} (\href{mailto:troygoode@gmail.com}{\tt troygoode@gmail.\+com}) 