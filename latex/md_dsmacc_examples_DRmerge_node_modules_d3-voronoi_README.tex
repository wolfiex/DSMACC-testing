This module implements \mbox{[}Steven J. Fortune’s algorithm\mbox{]}(\href{https://en.wikipedia.org/wiki/Fortune's_algorithm}{\tt https\+://en.\+wikipedia.\+org/wiki/\+Fortune\textquotesingle{}s\+\_\+algorithm}) for computing the \href{https://en.wikipedia.org/wiki/Voronoi_diagram}{\tt Voronoi diagram} or \href{https://en.wikipedia.org/wiki/Delaunay_triangulation}{\tt Delaunay triangulation} of a set of two-\/dimensional points. This implementation is largely based on \href{http://www.raymondhill.net/voronoi/rhill-voronoi.html}{\tt work by Raymond Hill}.

Voronoi diagrams are not only \href{http://bl.ocks.org/mbostock/4360892}{\tt visually} \href{http://bl.ocks.org/mbostock/4636377}{\tt attractive} but practical tools for interaction, such as to increase the target area of points in a scatterplot. See \href{http://www.nytimes.com/interactive/2013/03/29/sports/baseball/Strikeouts-Are-Still-Soaring.html}{\tt “\+Strikeouts on the Rise”} in {\itshape The New York Times} and this \href{http://bl.ocks.org/mbostock/8033015}{\tt multi-\/line chart} for examples; also see Tovi Grossman’s paper on \href{http://www.tovigrossman.com/BubbleCursor}{\tt bubble cursors} for a related technique. Voronoi diagrams can also be used to \href{http://bl.ocks.org/mbostock/6909318}{\tt automate label positioning}, and Delaunay meshes are useful in computing adjacency or grouping of visual elements.

\href{http://bl.ocks.org/mbostock/6675193}{\tt } \href{http://bl.ocks.org/mbostock/4060366}{\tt } \href{http://bl.ocks.org/mbostock/4341156}{\tt } \href{http://bl.ocks.org/mbostock/4360892}{\tt } \href{http://bl.ocks.org/mbostock/7608400}{\tt } \href{http://bl.ocks.org/mbostock/4636377}{\tt } \href{http://bl.ocks.org/mbostock/1073373}{\tt } \href{http://bl.ocks.org/mbostock/8033015}{\tt } \href{http://bl.ocks.org/mbostock/c6966db1fcb0ed2988da}{\tt } \href{http://bl.ocks.org/mbostock/ec10387f24c1fad2acac3bc11eb218a5}{\tt }

\subsection*{Installing}

If you use N\+PM, {\ttfamily npm install d3-\/voronoi}. Otherwise, download the \href{https://github.com/d3/d3-voronoi/releases/latest}{\tt latest release}. You can also load directly from \href{https://d3js.org}{\tt d3js.\+org}, either as a \href{https://d3js.org/d3-voronoi.v1.min.js}{\tt standalone library} or as part of \href{https://github.com/d3/d3}{\tt D3 4.\+0}. A\+MD, Common\+JS, and vanilla environments are supported. In vanilla, a {\ttfamily d3} global is exported\+:


\begin{DoxyCode}
<script src="https://d3js.org/d3-voronoi.v1.min.js"></script>
<script>

var voronoi = d3.voronoi();

</script>
\end{DoxyCode}


\href{https://tonicdev.com/npm/d3-voronoi}{\tt Try d3-\/voronoi in your browser.}

\subsection*{A\+PI Reference}

\label{_voronoi}%
\# d3.{\bfseries voronoi}() \href{https://github.com/d3/d3-voronoi/blob/master/src/voronoi.js}{\tt $<$$>$}

Creates a new Voronoi layout with default \href{#voronoi_x}{\tt {\itshape x}-\/} and \href{#voronoi_y}{\tt {\itshape y}-\/} accessors and a null \href{#voronoi_extent}{\tt extent}.

\label{__voronoi}%
\# {\itshape voronoi}({\itshape data}) \href{https://github.com/d3/d3-voronoi/blob/master/src/voronoi.js#L10}{\tt $<$$>$}

Computes the \href{#voronoi-diagrams}{\tt Voronoi diagram} for the specified {\itshape data} points.

\label{_voronoi_x}%
\# {\itshape voronoi}.{\bfseries x}(\mbox{[}{\itshape x}\mbox{]}) \href{https://github.com/d3/d3-voronoi/blob/master/src/voronoi.js#L31}{\tt $<$$>$}

If {\itshape x} is specified, sets the {\itshape x}-\/coordinate accessor. If {\itshape x} is not specified, returns the current {\itshape x}-\/coordinate accessor, which defaults to\+:


\begin{DoxyCode}
function x(d) \{
  return d[0];
\}
\end{DoxyCode}


\label{_voronoi_y}%
\# {\itshape voronoi}.{\bfseries y}(\mbox{[}{\itshape y}\mbox{]}) \href{https://github.com/d3/d3-voronoi/blob/master/src/voronoi.js#L35}{\tt $<$$>$}

If {\itshape y} is specified, sets the {\itshape y}-\/coordinate accessor. If {\itshape y} is not specified, returns the current {\itshape y}-\/coordinate accessor, which defaults to\+:


\begin{DoxyCode}
function y(d) \{
  return d[1];
\}
\end{DoxyCode}


\label{_voronoi_extent}%
\# {\itshape voronoi}.{\bfseries extent}(\mbox{[}{\itshape extent}\mbox{]}) \href{https://github.com/d3/d3-voronoi/blob/master/src/voronoi.js#L39}{\tt $<$$>$}

If {\itshape extent} is specified, sets the clip extent of the Voronoi layout to the specified bounds and returns the layout. The {\itshape extent} bounds are specified as an array \mbox{[}\mbox{[}{\itshape x0}, {\itshape y0}\mbox{]}, \mbox{[}{\itshape x1}, {\itshape y1}\mbox{]}\mbox{]}, where {\itshape x0} is the left side of the extent, {\itshape y0} is the top, {\itshape x1} is the right and {\itshape y1} is the bottom. If {\itshape extent} is not specified, returns the current clip extent which defaults to null. A clip extent is required when using \href{#voronoi_polygons}{\tt {\itshape voronoi}.polygons}.

\label{_voronoi_size}%
\# {\itshape voronoi}.{\bfseries size}(\mbox{[}{\itshape size}\mbox{]}) \href{https://github.com/d3/d3-voronoi/blob/master/src/voronoi.js#L43}{\tt $<$$>$}

An alias for \href{#voronoi_extent}{\tt {\itshape voronoi}.extent} where the minimum {\itshape x} and {\itshape y} of the extent are ⟨0,0⟩. Equivalent to\+:


\begin{DoxyCode}
voronoi.extent([[0, 0], size]);
\end{DoxyCode}


\label{_voronoi_polygons}%
\# {\itshape voronoi}.{\bfseries polygons}({\itshape data}) \href{https://github.com/d3/d3-voronoi/blob/master/src/voronoi.js#L19}{\tt $<$$>$}

Returns a sparse array of polygons, one for each unique input point in the specified {\itshape data} points, corresponding to the cells in the computed Voronoi diagram. Equivalent to\+:


\begin{DoxyCode}
voronoi(data).polygons();
\end{DoxyCode}


See \href{#diagram_polygons}{\tt {\itshape diagram}.polygons} for more detail. Note\+: an \href{#voronoi_extent}{\tt extent} is required.

\label{_voronoi_triangles}%
\# {\itshape voronoi}.{\bfseries triangles}({\itshape data}) \href{https://github.com/d3/d3-voronoi/blob/master/src/voronoi.js#L27}{\tt $<$$>$}

Returns the Delaunay triangulation of the specified {\itshape data} array as an array of triangles. Each triangle is a three-\/element array of elements from {\itshape data}. Equivalent to\+:


\begin{DoxyCode}
voronoi(data).triangles();
\end{DoxyCode}


See \href{#diagram_triangles}{\tt {\itshape diagram}.triangles} for more detail.

\label{_voronoi_links}%
\# {\itshape voronoi}.{\bfseries links}({\itshape data}) \href{https://github.com/d3/d3-voronoi/blob/master/src/voronoi.js#L23}{\tt $<$$>$}

Returns the Delaunay triangulation of the specified {\itshape data} array as an array of links. Each link has {\ttfamily source} and {\ttfamily target} attributes referring to elements in {\itshape data}. Equivalent to\+:


\begin{DoxyCode}
voronoi(data).links();
\end{DoxyCode}


See \href{#diagram_links}{\tt {\itshape diagram}.links} for more detail.

\subsubsection*{Voronoi Diagrams}

\label{_diagram}%
\# {\itshape diagram} \href{https://github.com/d3/d3-voronoi/blob/master/src/Diagram.js}{\tt $<$$>$}

The computed Voronoi diagram returned by \href{#_voronoi}{\tt {\itshape voronoi}} has the following properties\+:


\begin{DoxyItemize}
\item {\ttfamily edges} -\/ an array of \href{#diagram_edge}{\tt edges}.
\item {\ttfamily cells} -\/ a sparse array of \href{#diagram_cell}{\tt cells}, one for each unique input point.
\end{DoxyItemize}

For each set of coincident input points, one of the points is chosen arbitrarily and assigned the associated cell; the other coincident input points’ entries are missing from the returned sparse array.

\label{_diagram_polygons}%
\# {\itshape diagram}.{\bfseries polygons}() \href{https://github.com/d3/d3-voronoi/blob/master/src/Diagram.js#L72}{\tt $<$$>$}

Returns a sparse array of polygons clipped to the \href{#voronoi_extent}{\tt {\itshape extent}}, one for each cell (each unique input point) in the diagram. Each polygon is represented as an array of points \mbox{[}{\itshape x}, {\itshape y}\mbox{]} where {\itshape x} and {\itshape y} are the point coordinates, and a {\ttfamily data} field that refers to the corresponding element in {\itshape data}. Polygons are open\+: they do not contain a closing point that duplicates the first point; a triangle, for example, is an array of three points. Polygons are also counterclockwise, assuming the origin ⟨0,0⟩ is in the top-\/left corner.

For each set of coincident input points, one of the points is chosen arbitrarily and assigned the associated polygon; the other coincident input points’ entries are missing from the returned sparse array.

\label{_diagram_triangles}%
\# {\itshape diagram}.{\bfseries triangles}() \href{https://github.com/d3/d3-voronoi/blob/master/src/Diagram.js#L82}{\tt $<$$>$}

Returns the Delaunay triangulation of the specified {\itshape data} array as an array of triangles. Each triangle is a three-\/element array of elements from {\itshape data}. Since the triangulation is computed as the dual of the Voronoi diagram, and the Voronoi diagram is clipped by the \href{#voronoi_extent}{\tt extent}, a subset of the Delaunay triangulation is returned.

\label{_diagram_links}%
\# {\itshape diagram}.{\bfseries links}() \href{https://github.com/d3/d3-voronoi/blob/master/src/Diagram.js#L108}{\tt $<$$>$}

Returns the Delaunay triangulation of the specified {\itshape data} array as an array of links, one for each edge in the mesh. Each link has the following attributes\+:


\begin{DoxyItemize}
\item {\ttfamily source} -\/ the source node, an element in {\itshape data}.
\item {\ttfamily target} -\/ the target node, an element in {\itshape data}.
\end{DoxyItemize}

Since the triangulation is computed as the dual of the Voronoi diagram, and the Voronoi diagram is clipped by the \href{#voronoi_extent}{\tt extent}, a subset of the Delaunay links is returned.

\label{_diagram_find}%
\# {\itshape diagram}.{\bfseries find}({\itshape x}, {\itshape y}\mbox{[}, {\itshape radius}\mbox{]}) \href{https://github.com/d3/d3-voronoi/blob/master/src/Diagram.js#L119}{\tt $<$$>$}

Returns the nearest site to point \mbox{[}{\itshape x}, {\itshape y}\mbox{]}. If {\itshape radius} is specified, only sites within {\itshape radius} distance are considered.

See Philippe Rivière’s \href{http://bl.ocks.org/Fil/1b7ddbcd71454d685d1259781968aefc}{\tt bl.\+ocks.\+org/1b7ddbcd71454d685d1259781968aefc} for an example.

\label{_cell}%
\# {\itshape cell}

Each cell in the diagram is an object with the following properties\+:


\begin{DoxyItemize}
\item {\ttfamily site} -\/ the \href{#site}{\tt site} of the cell’s associated input point.
\item {\ttfamily halfedges} -\/ an array of indexes into \href{#diagram}{\tt {\itshape diagram}.edges} representing the cell’s polygon.
\end{DoxyItemize}

\label{_site}%
\# {\itshape site}

Each site in the diagram is an array \mbox{[}{\itshape x}, {\itshape y}\mbox{]} with two additional properties\+:


\begin{DoxyItemize}
\item {\ttfamily index} -\/ the site’s index, corresponding to the associated input point.
\item {\ttfamily data} -\/ the input data corresponding to this site.
\end{DoxyItemize}

\label{_edge}%
\# {\itshape edge}

Each edge in the diagram is an array \mbox{[}\mbox{[}{\itshape x0}, {\itshape y0}\mbox{]}, \mbox{[}{\itshape x1}, {\itshape y1}\mbox{]}\mbox{]} with two additional properties\+:


\begin{DoxyItemize}
\item {\ttfamily left} -\/ the \href{#site}{\tt site} on the left side of the edge.
\item {\ttfamily right} -\/ the \href{#site}{\tt site} on the right side of the edge; null for a clipped border edge. 
\end{DoxyItemize}