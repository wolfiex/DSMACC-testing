A \href{https://en.wikipedia.org/wiki/Quadtree}{\tt quadtree} recursively partitions two-\/dimensional space into squares, dividing each square into four equally-\/sized squares. Each distinct point exists in a unique leaf \href{#nodes}{\tt node}; coincident points are represented by a linked list. Quadtrees can accelerate various spatial operations, such as the \href{https://en.wikipedia.org/wiki/Barnes–Hut_simulation}{\tt Barnes–\+Hut approximation} for computing many-\/body forces, collision detection, and searching for nearby points.

\href{http://bl.ocks.org/mbostock/9078690}{\tt } \href{http://bl.ocks.org/mbostock/4343214}{\tt }

\subsection*{Installing}

If you use N\+PM, {\ttfamily npm install d3-\/quadtree}. Otherwise, download the \href{https://github.com/d3/d3-quadtree/releases/latest}{\tt latest release}. You can also load directly from \href{https://d3js.org}{\tt d3js.\+org}, either as a \href{https://d3js.org/d3-quadtree.v1.min.js}{\tt standalone library} or as part of \href{https://github.com/d3/d3}{\tt D3 4.\+0}. A\+MD, Common\+JS, and vanilla environments are supported. In vanilla, a {\ttfamily d3} global is exported\+:


\begin{DoxyCode}
<script src="https://d3js.org/d3-quadtree.v1.min.js"></script>
<script>

var quadtree = d3.quadtree();

</script>
\end{DoxyCode}


\href{https://tonicdev.com/npm/d3-quadtree}{\tt Try d3-\/quadtree in your browser.}

\subsection*{A\+PI Reference}

\label{_quadtree}%
\# d3.{\bfseries quadtree}(\mbox{[}{\itshape data}\mbox{[}, {\itshape x}, {\itshape y}\mbox{]}\mbox{]}) \href{https://github.com/d3/d3-quadtree/blob/master/src/quadtree.js#L14}{\tt $<$$>$}

Creates a new, empty quadtree with an empty \href{#quadtree_extent}{\tt extent} and the default \href{#quadtree_x}{\tt {\itshape x}-\/} and \href{#quadtree_y}{\tt {\itshape y}-\/}accessors. If {\itshape data} is specified, \href{#quadtree_addAll}{\tt adds} the specified array of data to the quadtree. This is equivalent to\+:


\begin{DoxyCode}
var tree = d3.quadtree()
    .addAll(data);
\end{DoxyCode}


If {\itshape x} and {\itshape y} are also specified, sets the \href{#quadtree_x}{\tt {\itshape x}-\/} and \href{#quadtree_y}{\tt {\itshape y}-\/} accessors to the specified functions before adding the specified array of data to the quadtree, equivalent to\+:


\begin{DoxyCode}
var tree = d3.quadtree()
    .x(x)
    .y(y)
    .addAll(data);
\end{DoxyCode}


\label{_quadtree_x}%
\# {\itshape quadtree}.{\bfseries x}(\mbox{[}{\itshape x}\mbox{]}) \href{https://github.com/d3/d3-quadtree/blob/master/src/x.js}{\tt $<$$>$}

If {\itshape x} is specified, sets the current {\itshape x}-\/coordinate accessor and returns the quadtree. If {\itshape x} is not specified, returns the current {\itshape x}-\/accessor, which defaults to\+:


\begin{DoxyCode}
function x(d) \{
  return d[0];
\}
\end{DoxyCode}


The {\itshape x}-\/acccessor is used to derive the {\itshape x}-\/coordinate of data when \href{#quadtree_add}{\tt adding} to and \href{#quadtree_remove}{\tt removing} from the tree. It is also used when \href{#quadtree_find}{\tt finding} to re-\/access the coordinates of data previously added to the tree; therefore, the {\itshape x}-\/ and {\itshape y}-\/accessors must be consistent, returning the same value given the same input.

\label{_quadtree_y}%
\# {\itshape quadtree}.{\bfseries y}(\mbox{[}{\itshape y}\mbox{]}) \href{https://github.com/d3/d3-quadtree/blob/master/src/y.js}{\tt $<$$>$}

If {\itshape y} is specified, sets the current {\itshape y}-\/coordinate accessor and returns the quadtree. If {\itshape y} is not specified, returns the current {\itshape y}-\/accessor, which defaults to\+:


\begin{DoxyCode}
function y(d) \{
  return d[1];
\}
\end{DoxyCode}


The {\itshape y}-\/acccessor is used to derive the {\itshape y}-\/coordinate of data when \href{#quadtree_add}{\tt adding} to and \href{#quadtree_remove}{\tt removing} from the tree. It is also used when \href{#quadtree_find}{\tt finding} to re-\/access the coordinates of data previously added to the tree; therefore, the {\itshape x}-\/ and {\itshape y}-\/accessors must be consistent, returning the same value given the same input.

\label{_quadtree_extent}%
\# {\itshape quadtree}.{\bfseries extent}(\mbox{[}{\itshape extent}\mbox{]}) \href{https://github.com/d3/d3-quadtree/blob/master/src/extent.js}{\tt $<$$>$}

If {\itshape extent} is specified, expands the quadtree to \href{#quadtree_cover}{\tt cover} the specified points \mbox{[}\mbox{[}{\itshape x0}, {\itshape y0}\mbox{]}, \mbox{[}{\itshape x1}, {\itshape y1}\mbox{]}\mbox{]} and returns the quadtree. If {\itshape extent} is not specified, returns the quadtree’s current extent \mbox{[}\mbox{[}{\itshape x0}, {\itshape y0}\mbox{]}, \mbox{[}{\itshape x1}, {\itshape y1}\mbox{]}\mbox{]}, where {\itshape x0} and {\itshape y0} are the inclusive lower bounds and {\itshape x1} and {\itshape y1} are the inclusive upper bounds, or undefined if the quadtree has no extent. The extent may also be expanded by calling \href{#quadtree_cover}{\tt {\itshape quadtree}.cover} or \href{#quadtree_add}{\tt {\itshape quadtree}.add}.

\label{_quadtree_cover}%
\# {\itshape quadtree}.{\bfseries cover}({\itshape x}, {\itshape y}) \href{https://github.com/d3/d3-quadtree/blob/master/src/cover.js}{\tt $<$$>$}

Expands the quadtree to cover the specified point ⟨$\ast$x$\ast$,$\ast$y$\ast$⟩, and returns the quadtree. If the quadtree’s extent already covers the specified point, this method does nothing. If the quadtree has an extent, the extent is repeatedly doubled to cover the specified point, wrapping the \href{#quadtree_root}{\tt root} \href{#nodes}{\tt node} as necessary; if the quadtree is empty, the extent is initialized to the extent \mbox{[}\mbox{[}⌊$\ast$x$\ast$⌋, ⌊$\ast$y$\ast$⌋\mbox{]}, \mbox{[}⌈$\ast$x$\ast$⌉, ⌈$\ast$y$\ast$⌉\mbox{]}\mbox{]}. (Rounding is necessary such that if the extent is later doubled, the boundaries of existing quadrants do not change due to floating point error.)

\label{_quadtree_add}%
\# {\itshape quadtree}.{\bfseries add}({\itshape datum}) \href{https://github.com/d3/d3-quadtree/blob/master/src/add.js}{\tt $<$$>$}

Adds the specified {\itshape datum} to the quadtree, deriving its coordinates ⟨$\ast$x$\ast$,{\itshape y$\ast$⟩ using the current \mbox{[}$\ast$x}-\/\mbox{]}(\#quadtree\+\_\+x) and \href{#quadtree_y}{\tt {\itshape y}-\/}accessors, and returns the quadtree. If the new point is outside the current \href{#quadtree_extent}{\tt extent} of the quadtree, the quadtree is automatically expanded to \href{#quadtree_cover}{\tt cover} the new point.

\label{_quadtree_addAll}%
\# {\itshape quadtree}.{\bfseries add\+All}({\itshape data}) \href{https://github.com/d3/d3-quadtree/blob/master/src/add.js#L50}{\tt $<$$>$}

Adds the specified array of {\itshape data} to the quadtree, deriving each element’s coordinates ⟨$\ast$x$\ast$,{\itshape y$\ast$⟩ using the current \mbox{[}$\ast$x}-\/\mbox{]}(\#quadtree\+\_\+x) and \href{#quadtree_y}{\tt {\itshape y}-\/}accessors, and return this quadtree. This is approximately equivalent to calling \href{#quadtree_add}{\tt {\itshape quadtree}.add} repeatedly\+:


\begin{DoxyCode}
for (var i = 0, n = data.length; i < n; ++i) \{
  quadtree.add(data[i]);
\}
\end{DoxyCode}


However, this method results in a more compact quadtree because the extent of the {\itshape data} is computed first before adding the data.

\label{_quadtree_remove}%
\# {\itshape quadtree}.{\bfseries remove}({\itshape datum}) \href{https://github.com/d3/d3-quadtree/blob/master/src/remove.js}{\tt $<$$>$}

Removes the specified {\itshape datum} to the quadtree, deriving its coordinates ⟨$\ast$x$\ast$,{\itshape y$\ast$⟩ using the current \mbox{[}$\ast$x}-\/\mbox{]}(\#quadtree\+\_\+x) and \href{#quadtree_y}{\tt {\itshape y}-\/}accessors, and returns the quadtree. If the specified {\itshape datum} does not exist in this quadtree, this method does nothing.

\label{_quadtree_removeAll}%
\# {\itshape quadtree}.{\bfseries remove\+All}({\itshape data}) \href{https://github.com/d3/d3-quadtree/blob/master/src/remove.js#L59}{\tt $<$$>$}

…

\label{_quadtree_copy}%
\# {\itshape quadtree}.{\bfseries copy}()

Returns a copy of the quadtree. All \href{#nodes}{\tt nodes} in the returned quadtree are identical copies of the corresponding node in the quadtree; however, any data in the quadtree is shared by reference and not copied.

\label{_quadtree_root}%
\# {\itshape quadtree}.{\bfseries root}() \href{https://github.com/d3/d3-quadtree/blob/master/src/root.js}{\tt $<$$>$}

Returns the root \href{#nodes}{\tt node} of the quadtree.

\label{_quadtree_data}%
\# {\itshape quadtree}.{\bfseries data}() \href{https://github.com/d3/d3-quadtree/blob/master/src/data.js}{\tt $<$$>$}

Returns an array of all data in the quadtree.

\label{_quadtree_size}%
\# {\itshape quadtree}.{\bfseries size}() \href{https://github.com/d3/d3-quadtree/blob/master/src/size.js}{\tt $<$$>$}

Returns the total number of data in the quadtree.

\label{_quadtree_find}%
\# {\itshape quadtree}.{\bfseries find}({\itshape x}, {\itshape y}\mbox{[}, {\itshape radius}\mbox{]}) \href{https://github.com/d3/d3-quadtree/blob/master/src/find.js}{\tt $<$$>$}

Returns the datum closest to the position ⟨$\ast$x$\ast$,{\itshape y$\ast$⟩ with the given search $\ast$radius}. If {\itshape radius} is not specified, it defaults to infinity. If there is no datum within the search area, returns undefined.

\label{_quadtree_visit}%
\# {\itshape quadtree}.{\bfseries visit}({\itshape callback}) \href{https://github.com/d3/d3-quadtree/blob/master/src/visit.js}{\tt $<$$>$}

Visits each \href{#nodes}{\tt node} in the quadtree in pre-\/order traversal, invoking the specified {\itshape callback} with arguments {\itshape node}, {\itshape x0}, {\itshape y0}, {\itshape x1}, {\itshape y1} for each node, where {\itshape node} is the node being visited, ⟨$\ast$x0$\ast$, {\itshape y0$\ast$⟩ are the lower bounds of the node, and ⟨$\ast$x1}, {\itshape y1$\ast$⟩ are the upper bounds, and returns the quadtree. (Assuming that positive $\ast$x} is right and positive {\itshape y} is down, as is typically the case in Canvas and S\+VG, ⟨$\ast$x0$\ast$, {\itshape y0$\ast$⟩ is the top-\/left corner and ⟨$\ast$x1}, {\itshape y1$\ast$⟩ is the lower-\/right corner; however, the coordinate system is arbitrary, so more formally $\ast$x0} $<$= {\itshape x1} and {\itshape y0} $<$= {\itshape y1}.)

If the {\itshape callback} returns true for a given node, then the children of that node are not visited; otherwise, all child nodes are visited. This can be used to quickly visit only parts of the tree, for example when using the \href{https://en.wikipedia.org/wiki/Barnes–Hut_simulation}{\tt Barnes–\+Hut approximation}. Note, however, that child quadrants are always visited in sibling order\+: top-\/left, top-\/right, bottom-\/left, bottom-\/right. In cases such as \href{#quadtree_find}{\tt search}, visiting siblings in a specific order may be faster.

\label{_quadtree_visitAfter}%
\# {\itshape quadtree}.{\bfseries visit\+After}({\itshape callback}) \href{https://github.com/d3/d3-quadtree/blob/master/src/visitAfter.js}{\tt $<$$>$}

Visits each \href{#nodes}{\tt node} in the quadtree in post-\/order traversal, invoking the specified {\itshape callback} with arguments {\itshape node}, {\itshape x0}, {\itshape y0}, {\itshape x1}, {\itshape y1} for each node, where {\itshape node} is the node being visited, ⟨$\ast$x0$\ast$, {\itshape y0$\ast$⟩ are the lower bounds of the node, and ⟨$\ast$x1}, {\itshape y1$\ast$⟩ are the upper bounds, and returns the quadtree. (Assuming that positive $\ast$x} is right and positive {\itshape y} is down, as is typically the case in Canvas and S\+VG, ⟨$\ast$x0$\ast$, {\itshape y0$\ast$⟩ is the top-\/left corner and ⟨$\ast$x1}, {\itshape y1$\ast$⟩ is the lower-\/right corner; however, the coordinate system is arbitrary, so more formally $\ast$x0} $<$= {\itshape x1} and {\itshape y0} $<$= {\itshape y1}.) Returns {\itshape root}.

\subsubsection*{Nodes}

Internal nodes of the quadtree are represented as four-\/element arrays in left-\/to-\/right, top-\/to-\/bottom order\+:


\begin{DoxyItemize}
\item {\ttfamily 0} -\/ the top-\/left quadrant, if any.
\item {\ttfamily 1} -\/ the top-\/right quadrant, if any.
\item {\ttfamily 2} -\/ the bottom-\/left quadrant, if any.
\item {\ttfamily 3} -\/ the bottom-\/right quadrant, if any.
\end{DoxyItemize}

A child quadrant may be undefined if it is empty.

Leaf nodes are represented as objects with the following properties\+:


\begin{DoxyItemize}
\item {\ttfamily data} -\/ the data associated with this point, as passed to \href{#quadtree_add}{\tt {\itshape quadtree}.add}.
\item {\ttfamily next} -\/ the next datum in this leaf, if any.
\end{DoxyItemize}

The {\ttfamily length} property may be used to distinguish leaf nodes from internal nodes\+: it is undefined for leaf nodes, and 4 for internal nodes. For example, to iterate over all data in a leaf node\+:


\begin{DoxyCode}
if (!node.length) do console.log(node.data); while (node = node.next);
\end{DoxyCode}


The point’s {\itshape x}-\/ and {\itshape y}-\/coordinates {\bfseries must not be modified} while the point is in the quadtree. To update a point’s position, \href{#quadtree_remove}{\tt remove} the point and then re-\/\href{#quadtree_add}{\tt add} it to the quadtree at the new position. Alternatively, you may discard the existing quadtree entirely and create a new one from scratch; this may be more efficient if many of the points have moved. 