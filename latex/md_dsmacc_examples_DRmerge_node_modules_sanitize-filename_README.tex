Sanitize a string to be safe for use as a filename by removing directory paths and invalid characters.

\subsection*{Install}

\href{https://www.npmjs.com/package/sanitize-filename}{\tt npm\+: {\itshape sanitize-\/filename}}


\begin{DoxyCode}
npm install sanitize-filename
\end{DoxyCode}


\subsection*{Example}


\begin{DoxyCode}
var sanitize = require("sanitize-filename");

// Some string that may be unsafe or invalid as a filename
var UNSAFE\_USER\_INPUT = "~/.\(\backslash\)u0000ssh/authorized\_keys";

// Sanitize the string to be safe for use as a filename.
var filename = sanitize(UNSAFE\_USER\_INPUT);
// -> "~.sshauthorized\_keys"
\end{DoxyCode}


\subsection*{Details}

{\itshape sanitize-\/filename} removes the following\+:


\begin{DoxyItemize}
\item \href{https://en.wikipedia.org/wiki/C0_and_C1_control_codes}{\tt Control characters} ({\ttfamily 0x00}–{\ttfamily 0x1f} and {\ttfamily 0x80}–{\ttfamily 0x9f})
\item \href{https://kb.acronis.com/content/39790}{\tt Reserved characters} ({\ttfamily /}, {\ttfamily ?}, {\ttfamily $<$}, {\ttfamily $>$}, {\ttfamily \textbackslash{}}, {\ttfamily \+:}, {\ttfamily $\ast$}, {\ttfamily $\vert$}, and {\ttfamily "})
\item Unix reserved filenames ({\ttfamily .} and {\ttfamily ..})
\item Trailing periods and spaces (\href{https://msdn.microsoft.com/en-us/library/aa365247(v=vs.85).aspx#Naming_Conventions}{\tt for Windows})
\item Windows reserved filenames ({\ttfamily C\+ON}, {\ttfamily P\+RN}, {\ttfamily A\+UX}, {\ttfamily N\+UL}, {\ttfamily C\+O\+M1}, {\ttfamily C\+O\+M2}, {\ttfamily C\+O\+M3}, {\ttfamily C\+O\+M4}, {\ttfamily C\+O\+M5}, {\ttfamily C\+O\+M6}, {\ttfamily C\+O\+M7}, {\ttfamily C\+O\+M8}, {\ttfamily C\+O\+M9}, {\ttfamily L\+P\+T1}, {\ttfamily L\+P\+T2}, {\ttfamily L\+P\+T3}, {\ttfamily L\+P\+T4}, {\ttfamily L\+P\+T5}, {\ttfamily L\+P\+T6}, {\ttfamily L\+P\+T7}, {\ttfamily L\+P\+T8}, and {\ttfamily L\+P\+T9})
\end{DoxyItemize}

The resulting string is truncated to \href{http://unix.stackexchange.com/questions/32795/what-is-the-maximum-allowed-filename-and-folder-size-with-ecryptfs}{\tt 255 bytes in length}. The string will not contain any directory paths and will be safe to use as a filename.

\subsubsection*{Empty String {\ttfamily \char`\"{}\char`\"{}} Result}

An empty string {\ttfamily \char`\"{}\char`\"{}} can be returned. For example\+:


\begin{DoxyCode}
var sanitize = require("sanitize-filename");
sanitize("..")
// -> ""
\end{DoxyCode}


\subsubsection*{Non-\/unique Filenames}

Two different inputs can return the same value. For example\+:


\begin{DoxyCode}
var sanitize = require("sanitize-filename");
sanitize("file?")
// -> "file"
sanitize ("*file*")
// -> "file"
\end{DoxyCode}


\subsubsection*{File Systems}

Sanitized filenames will be safe for use on modern Windows, OS X, and Unix file systems ({\ttfamily N\+T\+FS}, {\ttfamily ext}, etc.).

\href{https://en.wikipedia.org/wiki/8.3_filename}{\tt {\ttfamily F\+AT} 8.\+3 filenames} are not supported.

\paragraph*{Test Your File System}

The test program will use various strings (including the \href{https://github.com/minimaxir/big-list-of-naughty-strings}{\tt Big List of Naughty Strings}) to create files in the working directory. Run {\ttfamily npm test} to run tests against your file system.

\subsection*{A\+PI}

\subsubsection*{{\ttfamily sanitize(input\+String, \mbox{[}options\mbox{]})}}

Sanitize {\ttfamily input\+String} by removing or replacing invalid characters.

Options\+:


\begin{DoxyItemize}
\item {\ttfamily options.\+replacement}\+: {\itshape optional, string/function, default\+: {\ttfamily \char`\"{}\char`\"{}}}. If passed as a string, it\textquotesingle{}s used as the replacement for invalid characters. If passed as a function, the function will be called with the invalid characters and it\textquotesingle{}s return value will be used as the replacement. See \href{https://developer.mozilla.org/en-US/docs/Web/JavaScript/Reference/Global_Objects/String/replace}{\tt {\ttfamily String.\+prototype.\+replace}} for more info. 
\end{DoxyItemize}