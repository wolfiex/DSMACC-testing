\href{https://nodei.co/npm/readdirp/}{\tt }

Recursive version of \href{http://nodejs.org/docs/latest/api/fs.html#fs_fs_readdir_path_callback}{\tt fs.\+readdir}. Exposes a {\bfseries stream api}.


\begin{DoxyCode}
var readdirp = require('readdirp')
  , path = require('path')
  , es = require('event-stream');

// print out all JavaScript files along with their size

var stream = readdirp(\{ root: path.join(\_\_dirname), fileFilter: '*.js' \});
stream
  .on('warn', function (err) \{
    console.error('non-fatal error', err);
    // optionally call stream.destroy() here in order to abort and cause 'close' to be emitted
  \})
  .on('error', function (err) \{ console.error('fatal error', err); \})
  .pipe(es.mapSync(function (entry) \{
    return \{ path: entry.path, size: entry.stat.size \};
  \}))
  .pipe(es.stringify())
  .pipe(process.stdout);
\end{DoxyCode}


Meant to be one of the recursive versions of \href{http://nodejs.org/docs/latest/api/fs.html}{\tt fs} functions, e.\+g., like \href{https://github.com/substack/node-mkdirp}{\tt mkdirp}.

{\bfseries Table of Contents} {\itshape generated with \href{http://doctoc.herokuapp.com/}{\tt Doc\+Toc}}


\begin{DoxyItemize}
\item \href{#installation}{\tt Installation}
\item \href{#api}{\tt A\+PI}
\begin{DoxyItemize}
\item \href{#entry-stream}{\tt entry stream}
\item \href{#options}{\tt options}
\item \href{#entry-info}{\tt entry info}
\item \href{#filters}{\tt Filters}
\item \href{#callback-api}{\tt Callback A\+PI}
\begin{DoxyItemize}
\item \href{#allprocessed}{\tt all\+Processed}
\item \href{#fileprocessed}{\tt file\+Processed}
\end{DoxyItemize}
\end{DoxyItemize}
\item \href{#more-examples}{\tt More Examples}
\begin{DoxyItemize}
\item \href{#stream-api}{\tt stream api}
\item \href{#stream-api-pipe}{\tt stream api pipe}
\item \href{#grep}{\tt grep}
\item \href{#using-callback-api}{\tt using callback api}
\item \href{#tests}{\tt tests}
\end{DoxyItemize}
\end{DoxyItemize}

\section*{Installation}

\begin{DoxyVerb}npm install readdirp
\end{DoxyVerb}


\section*{A\+PI}

{\itshape {\bfseries var entry\+Stream = readdirp (options)}}

Reads given root recursively and returns a {\ttfamily stream} of \href{#entry-info}{\tt entry info}s.

\subsection*{entry stream}

Behaves as follows\+:


\begin{DoxyItemize}
\item `emit(\textquotesingle{}data'){\ttfamily passes an \mbox{[}entry info\mbox{]}(\#entry-\/info) whenever one is found -\/}emit(\textquotesingle{}warn\textquotesingle{}){\ttfamily passes a non-\/fatal}Error\`{} that prevents a file/directory from being processed (i.\+e., if it is inaccessible to the user)
\item `emit(\textquotesingle{}error'){\ttfamily passes a fatal}Error{\ttfamily which also ends the stream (i.\+e., when illegal options where passed) -\/}emit(\textquotesingle{}end\textquotesingle{}){\ttfamily called when all entries were found and no more will be emitted (i.\+e., we are done) -\/}emit(\textquotesingle{}close\textquotesingle{}){\ttfamily called when the stream is destroyed via}stream.\+destroy(){\ttfamily (which could be useful if you want to manually abort even on a non fatal error) -\/ at that point the stream is no longer}readable\`{} and no more entries, warning or errors are emitted
\item to learn more about streams, consult the very detailed \href{http://nodejs.org/api/stream.html}{\tt nodejs streams documentation} or the \href{https://github.com/substack/stream-handbook}{\tt stream-\/handbook}
\end{DoxyItemize}

\subsection*{options}


\begin{DoxyItemize}
\item {\bfseries root}\+: path in which to start reading and recursing into subdirectories
\item {\bfseries file\+Filter}\+: filter to include/exclude files found (see \href{#filters}{\tt Filters} for more)
\item {\bfseries directory\+Filter}\+: filter to include/exclude directories found and to recurse into (see \href{#filters}{\tt Filters} for more)
\item {\bfseries depth}\+: depth at which to stop recursing even if more subdirectories are found
\item {\bfseries entry\+Type}\+: determines if data events on the stream should be emitted for `\textquotesingle{}files'{\ttfamily ,}\textquotesingle{}directories\textquotesingle{}{\ttfamily ,}\textquotesingle{}both\textquotesingle{}{\ttfamily , or}\textquotesingle{}all\textquotesingle{}{\ttfamily . Setting to}\textquotesingle{}all\textquotesingle{}{\ttfamily will also include entries for other types of file descriptors like character devices, unix sockets and named pipes. Defaults to}\textquotesingle{}files\textquotesingle{}\`{}.
\item {\bfseries lstat}\+: if {\ttfamily true}, readdirp uses {\ttfamily fs.\+lstat} instead of {\ttfamily fs.\+stat} in order to stat files and includes symlink entries in the stream along with files.
\end{DoxyItemize}

\subsection*{entry info}

Has the following properties\+:


\begin{DoxyItemize}
\item {\bfseries parent\+Dir} \+: directory in which entry was found (relative to given root)
\item {\bfseries full\+Parent\+Dir} \+: full path to parent directory
\item {\bfseries name} \+: name of the file/directory
\item {\bfseries path} \+: path to the file/directory (relative to given root)
\item {\bfseries full\+Path} \+: full path to the file/directory found
\item {\bfseries stat} \+: built in \href{http://nodejs.org/docs/v0.4.9/api/fs.html#fs.Stats}{\tt stat object}
\item {\bfseries Example}\+: (assuming root was {\ttfamily /\+User/dev/readdirp}) \begin{DoxyVerb}  parentDir     :  'test/bed/root_dir1',
  fullParentDir :  '/User/dev/readdirp/test/bed/root_dir1',
  name          :  'root_dir1_subdir1',
  path          :  'test/bed/root_dir1/root_dir1_subdir1',
  fullPath      :  '/User/dev/readdirp/test/bed/root_dir1/root_dir1_subdir1',
  stat          :  [ ... ]
\end{DoxyVerb}

\end{DoxyItemize}

\subsection*{Filters}

There are three different ways to specify filters for files and directories respectively.


\begin{DoxyItemize}
\item {\bfseries function}\+: a function that takes an entry info as a parameter and returns true to include or false to exclude the entry
\item {\bfseries glob string}\+: a string (e.\+g., {\ttfamily $\ast$.js}) which is matched using \href{https://github.com/isaacs/minimatch}{\tt minimatch}, so go there for more information.

Globstars ({\ttfamily $\ast$$\ast$}) are not supported since specifying a recursive pattern for an already recursive function doesn\textquotesingle{}t make sense.

Negated globs (as explained in the minimatch documentation) are allowed, e.\+g., {\ttfamily !$\ast$.txt} matches everything but text files.
\item {\bfseries array of glob strings}\+: either need to be all inclusive or all exclusive (negated) patterns otherwise an error is thrown.

`\mbox{[} '$\ast$.json\textquotesingle{}, \textquotesingle{}$\ast$.js\textquotesingle{} \mbox{]}\`{} includes all Java\+Script and Json files.
\end{DoxyItemize}

\begin{DoxyVerb}`[ '!.git', '!node_modules' ]` includes all directories except the '.git' and 'node_modules'.
\end{DoxyVerb}


Directories that do not pass a filter will not be recursed into.

\subsection*{Callback A\+PI}

Although the stream api is recommended, readdirp also exposes a callback based api.

{\itshape {\bfseries readdirp (options, callback1 \mbox{[}, callback2\mbox{]})}}

If callback2 is given, callback1 functions as the {\bfseries file\+Processed} callback, and callback2 as the {\bfseries all\+Processed} callback.

If only callback1 is given, it functions as the {\bfseries all\+Processed} callback.

\subsubsection*{all\+Processed}


\begin{DoxyItemize}
\item function with err and res parameters, e.\+g., {\ttfamily function (err, res) \{ ... \}}
\item {\bfseries err}\+: array of errors that occurred during the operation, {\bfseries res may still be present, even if errors occurred}
\item {\bfseries res}\+: collection of file/directory \href{#entry-info}{\tt entry infos}
\end{DoxyItemize}

\subsubsection*{file\+Processed}


\begin{DoxyItemize}
\item function with \href{#entry-info}{\tt entry info} parameter e.\+g., {\ttfamily function (entry\+Info) \{ ... \}}
\end{DoxyItemize}

\section*{More Examples}

`on(\textquotesingle{}error', ..){\ttfamily ,}on(\textquotesingle{}warn\textquotesingle{}, ..){\ttfamily and}on(\textquotesingle{}end\textquotesingle{}, ..)\`{} handling omitted for brevity


\begin{DoxyCode}
var readdirp = require('readdirp');

// Glob file filter
readdirp(\{ root: './test/bed', fileFilter: '*.js' \})
  .on('data', function (entry) \{
    // do something with each JavaScript file entry
  \});

// Combined glob file filters
readdirp(\{ root: './test/bed', fileFilter: [ '*.js', '*.json' ] \})
  .on('data', function (entry) \{
    // do something with each JavaScript and Json file entry
  \});

// Combined negated directory filters
readdirp(\{ root: './test/bed', directoryFilter: [ '!.git', '!*modules' ] \})
  .on('data', function (entry) \{
    // do something with each file entry found outside '.git' or any modules directory
  \});

// Function directory filter
readdirp(\{ root: './test/bed', directoryFilter: function (di) \{ return di.name.length === 9; \} \})
  .on('data', function (entry) \{
    // do something with each file entry found inside directories whose name has length 9
  \});

// Limiting depth
readdirp(\{ root: './test/bed', depth: 1 \})
  .on('data', function (entry) \{
    // do something with each file entry found up to 1 subdirectory deep
  \});

// callback api
readdirp(\{ root: '.' \}, function(fileInfo) \{
   // do something with file entry here
  \}, function (err, res) \{
    // all done, move on or do final step for all file entries here
\});
\end{DoxyCode}


Try more examples by following https\+://github.com/paulmillr/readdirp/blob/master/examples/\+Readme.\+md \char`\"{}instructions\char`\"{} on how to get going.

\subsection*{tests}

The \href{https://github.com/paulmillr/readdirp/blob/master/test/readdirp.js}{\tt readdirp tests} also will give you a good idea on how things work. 