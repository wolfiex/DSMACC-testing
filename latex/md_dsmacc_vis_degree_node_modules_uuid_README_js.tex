
\begin{DoxyCode}
runmd.onRequire = path => path.replace(/^uuid/, './');
\end{DoxyCode}


\section*{uuid \href{http://travis-ci.org/kelektiv/node-uuid}{\tt }}

Simple, fast generation of \href{http://www.ietf.org/rfc/rfc4122.txt}{\tt R\+F\+C4122} U\+U\+I\+DS.

Features\+:


\begin{DoxyItemize}
\item Support for version 1, 3, 4 and 5 U\+U\+I\+Ds
\item Cross-\/platform
\item Uses cryptographically-\/strong random number A\+P\+Is (when available)
\item Zero-\/dependency, small footprint (... but not \href{https://gist.github.com/982883}{\tt this small})
\end{DoxyItemize}

\mbox{[}{\bfseries Deprecation warning}\+: The use of `require(\textquotesingle{}uuid'){\ttfamily is deprecated and will not be supported after version 3.\+x of this module. Instead, use}require(\textquotesingle{}uuid/\mbox{[}v1$\vert$v3$\vert$v4$\vert$v5\mbox{]}\textquotesingle{})\`{} as shown in the examples below.\mbox{]}

\subsection*{Quickstart -\/ Common\+JS (Recommended)}


\begin{DoxyCode}
npm install uuid
\end{DoxyCode}


Then generate your uuid version of choice ...

Version 1 (timestamp)\+:


\begin{DoxyCode}
const uuidv1 = require('uuid/v1');
uuidv1(); // RESULT
\end{DoxyCode}


Version 3 (namespace)\+:


\begin{DoxyCode}
const uuidv3 = require('uuid/v3');

// ... using predefined DNS namespace (for domain names)
uuidv3('hello.example.com', uuidv3.DNS); // RESULT

// ... using predefined URL namespace (for, well, URLs)
uuidv3('http://example.com/hello', uuidv3.URL); // RESULT

// ... using a custom namespace
//
// Note: Custom namespaces should be a UUID string specific to your application!
// E.g. the one here was generated using this modules `uuid` CLI.
const MY\_NAMESPACE = '1b671a64-40d5-491e-99b0-da01ff1f3341';
uuidv3('Hello, World!', MY\_NAMESPACE); // RESULT
\end{DoxyCode}


Version 4 (random)\+:


\begin{DoxyCode}
const uuidv4 = require('uuid/v4');
uuidv4(); // RESULT
\end{DoxyCode}


Version 5 (namespace)\+:


\begin{DoxyCode}
const uuidv5 = require('uuid/v5');

// ... using predefined DNS namespace (for domain names)
uuidv5('hello.example.com', uuidv5.DNS); // RESULT

// ... using predefined URL namespace (for, well, URLs)
uuidv5('http://example.com/hello', uuidv5.URL); // RESULT

// ... using a custom namespace
//
// Note: Custom namespaces should be a UUID string specific to your application!
// E.g. the one here was generated using this modules `uuid` CLI.
const MY\_NAMESPACE = '1b671a64-40d5-491e-99b0-da01ff1f3341';
uuidv5('Hello, World!', MY\_NAMESPACE); // RESULT
\end{DoxyCode}


\subsection*{Quickstart -\/ Browser-\/ready Versions}

Browser-\/ready versions of this module are available via \href{https://github.com/jfhbrook/wzrd.in}{\tt wzrd.\+in}.

For version 1 uuids\+:


\begin{DoxyCode}
<script src="http://wzrd.in/standalone/uuid%2Fv1@latest"></script>
<script>
uuidv1(); // -> v1 UUID
</script>
\end{DoxyCode}


For version 3 uuids\+:


\begin{DoxyCode}
<script src="http://wzrd.in/standalone/uuid%2Fv3@latest"></script>
<script>
uuidv3('http://example.com/hello', uuidv3.URL); // -> v3 UUID
</script>
\end{DoxyCode}


For version 4 uuids\+:


\begin{DoxyCode}
<script src="http://wzrd.in/standalone/uuid%2Fv4@latest"></script>
<script>
uuidv4(); // -> v4 UUID
</script>
\end{DoxyCode}


For version 5 uuids\+:


\begin{DoxyCode}
<script src="http://wzrd.in/standalone/uuid%2Fv5@latest"></script>
<script>
uuidv5('http://example.com/hello', uuidv5.URL); // -> v5 UUID
</script>
\end{DoxyCode}


\subsection*{A\+PI}

\subsubsection*{Version 1}


\begin{DoxyCode}
const uuidv1 = require('uuid/v1');

// Incantations
uuidv1();
uuidv1(options);
uuidv1(options, buffer, offset);
\end{DoxyCode}


Generate and return a R\+F\+C4122 v1 (timestamp-\/based) U\+U\+ID.


\begin{DoxyItemize}
\item {\ttfamily options} -\/ (Object) Optional uuid state to apply. Properties may include\+:
\begin{DoxyItemize}
\item {\ttfamily node} -\/ (Array) \mbox{\hyperlink{classNode}{Node}} id as Array of 6 bytes (per 4.\+1.\+6). Default\+: Randomly generated ID. See note 1.
\item {\ttfamily clockseq} -\/ (Number between 0 -\/ 0x3fff) R\+FC clock sequence. Default\+: An internally maintained clockseq is used.
\item {\ttfamily msecs} -\/ (Number) Time in milliseconds since unix Epoch. Default\+: The current time is used.
\item {\ttfamily nsecs} -\/ (Number between 0-\/9999) additional time, in 100-\/nanosecond units. Ignored if {\ttfamily msecs} is unspecified. Default\+: internal uuid counter is used, as per 4.\+2.\+1.\+2.
\end{DoxyItemize}
\item {\ttfamily buffer} -\/ (Array $\vert$ Buffer) Array or buffer where U\+U\+ID bytes are to be written.
\item {\ttfamily offset} -\/ (Number) Starting index in {\ttfamily buffer} at which to begin writing.
\end{DoxyItemize}

Returns {\ttfamily buffer}, if specified, otherwise the string form of the U\+U\+ID

Note\+: The $<$node$>$ id is generated guaranteed to stay constant for the lifetime of the current JS runtime. (Future versions of this module may use persistent storage mechanisms to extend this guarantee.)

Example\+: Generate string U\+U\+ID with fully-\/specified options


\begin{DoxyCode}
const v1options = \{
  node: [0x01, 0x23, 0x45, 0x67, 0x89, 0xab],
  clockseq: 0x1234,
  msecs: new Date('2011-11-01').getTime(),
  nsecs: 5678
\};
uuidv1(v1options); // RESULT
\end{DoxyCode}


Example\+: In-\/place generation of two binary I\+Ds


\begin{DoxyCode}
// Generate two ids in an array
const arr = new Array();
uuidv1(null, arr, 0);  // RESULT
uuidv1(null, arr, 16); // RESULT
\end{DoxyCode}


\subsubsection*{Version 3}


\begin{DoxyCode}
const uuidv3 = require('uuid/v3');

// Incantations
uuidv3(name, namespace);
uuidv3(name, namespace, buffer);
uuidv3(name, namespace, buffer, offset);
\end{DoxyCode}


Generate and return a R\+F\+C4122 v3 U\+U\+ID.


\begin{DoxyItemize}
\item {\ttfamily name} -\/ (String $\vert$ Array\mbox{[}\mbox{]}) \char`\"{}name\char`\"{} to create U\+U\+ID with
\item {\ttfamily namespace} -\/ (String $\vert$ Array\mbox{[}\mbox{]}) \char`\"{}namespace\char`\"{} U\+U\+ID either as a String or Array\mbox{[}16\mbox{]} of byte values
\item {\ttfamily buffer} -\/ (Array $\vert$ Buffer) Array or buffer where U\+U\+ID bytes are to be written.
\item {\ttfamily offset} -\/ (Number) Starting index in {\ttfamily buffer} at which to begin writing. Default = 0
\end{DoxyItemize}

Returns {\ttfamily buffer}, if specified, otherwise the string form of the U\+U\+ID

Example\+:


\begin{DoxyCode}
uuidv3('hello world', MY\_NAMESPACE);  // RESULT
\end{DoxyCode}


\subsubsection*{Version 4}


\begin{DoxyCode}
const uuidv4 = require('uuid/v4')

// Incantations
uuidv4();
uuidv4(options);
uuidv4(options, buffer, offset);
\end{DoxyCode}


Generate and return a R\+F\+C4122 v4 U\+U\+ID.


\begin{DoxyItemize}
\item {\ttfamily options} -\/ (Object) Optional uuid state to apply. Properties may include\+:
\begin{DoxyItemize}
\item {\ttfamily random} -\/ (Number\mbox{[}16\mbox{]}) Array of 16 numbers (0-\/255) to use in place of randomly generated values
\item {\ttfamily rng} -\/ (Function) Random \# generator function that returns an Array\mbox{[}16\mbox{]} of byte values (0-\/255)
\end{DoxyItemize}
\item {\ttfamily buffer} -\/ (Array $\vert$ Buffer) Array or buffer where U\+U\+ID bytes are to be written.
\item {\ttfamily offset} -\/ (Number) Starting index in {\ttfamily buffer} at which to begin writing.
\end{DoxyItemize}

Returns {\ttfamily buffer}, if specified, otherwise the string form of the U\+U\+ID

Example\+: Generate string U\+U\+ID with predefined {\ttfamily random} values


\begin{DoxyCode}
const v4options = \{
  random: [
    0x10, 0x91, 0x56, 0xbe, 0xc4, 0xfb, 0xc1, 0xea,
    0x71, 0xb4, 0xef, 0xe1, 0x67, 0x1c, 0x58, 0x36
  ]
\};
uuidv4(v4options); // RESULT
\end{DoxyCode}


Example\+: Generate two I\+Ds in a single buffer


\begin{DoxyCode}
const buffer = new Array();
uuidv4(null, buffer, 0);  // RESULT
uuidv4(null, buffer, 16); // RESULT
\end{DoxyCode}


\subsubsection*{Version 5}


\begin{DoxyCode}
const uuidv5 = require('uuid/v5');

// Incantations
uuidv5(name, namespace);
uuidv5(name, namespace, buffer);
uuidv5(name, namespace, buffer, offset);
\end{DoxyCode}


Generate and return a R\+F\+C4122 v5 U\+U\+ID.


\begin{DoxyItemize}
\item {\ttfamily name} -\/ (String $\vert$ Array\mbox{[}\mbox{]}) \char`\"{}name\char`\"{} to create U\+U\+ID with
\item {\ttfamily namespace} -\/ (String $\vert$ Array\mbox{[}\mbox{]}) \char`\"{}namespace\char`\"{} U\+U\+ID either as a String or Array\mbox{[}16\mbox{]} of byte values
\item {\ttfamily buffer} -\/ (Array $\vert$ Buffer) Array or buffer where U\+U\+ID bytes are to be written.
\item {\ttfamily offset} -\/ (Number) Starting index in {\ttfamily buffer} at which to begin writing. Default = 0
\end{DoxyItemize}

Returns {\ttfamily buffer}, if specified, otherwise the string form of the U\+U\+ID

Example\+:


\begin{DoxyCode}
uuidv5('hello world', MY\_NAMESPACE);  // RESULT
\end{DoxyCode}


\subsection*{Command Line}

U\+U\+I\+Ds can be generated from the command line with the {\ttfamily uuid} command.


\begin{DoxyCode}
$ uuid
ddeb27fb-d9a0-4624-be4d-4615062daed4

$ uuid v1
02d37060-d446-11e7-a9fa-7bdae751ebe1
\end{DoxyCode}


Type {\ttfamily uuid -\/-\/help} for usage details

\subsection*{Testing}


\begin{DoxyCode}
npm test
\end{DoxyCode}
 