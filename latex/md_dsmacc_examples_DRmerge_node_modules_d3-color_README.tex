Even though your browser understands a lot about colors, it doesn’t offer much help in manipulating colors through Java\+Script. The d3-\/color module therefore provides representations for various color spaces, allowing specification, conversion and manipulation. (Also see \href{https://github.com/d3/d3-interpolate}{\tt d3-\/interpolate} for color interpolation.)

For example, take the color named “steelblue”\+:


\begin{DoxyCode}
var c = d3.color("steelblue"); // \{r: 70, g: 130, b: 180, opacity: 1\}
\end{DoxyCode}


Let’s try converting it to H\+SL\+:


\begin{DoxyCode}
var c = d3.hsl("steelblue"); // \{h: 207.27…, s: 0.44, l: 0.4902…, opacity: 1\}
\end{DoxyCode}


Now rotate the hue by 90°, bump up the saturation, and format as a string for C\+SS\+:


\begin{DoxyCode}
c.h += 90;
c.s += 0.2;
c + ""; // rgb(198, 45, 205)
\end{DoxyCode}


To fade the color slightly\+:


\begin{DoxyCode}
c.opacity = 0.8;
c + ""; // rgba(198, 45, 205, 0.8)
\end{DoxyCode}


In addition to the ubiquitous and machine-\/friendly \href{#rgb}{\tt R\+GB} and \href{#hsl}{\tt H\+SL} color space, d3-\/color supports two color spaces that are designed for humans\+:


\begin{DoxyItemize}
\item Dave Green’s \href{#cubehelix}{\tt Cubehelix}
\item \href{#lab}{\tt Lab (C\+I\+E\+L\+AB)} and \href{#hcl}{\tt H\+CL (C\+I\+E\+L\+CH)}
\end{DoxyItemize}

Cubehelix features monotonic lightness, while Lab and H\+CL are perceptually uniform. Note that H\+CL is the cylindrical form of Lab, similar to how H\+SL is the cylindrical form of R\+GB.

\subsection*{Installing}

If you use N\+PM, {\ttfamily npm install d3-\/color}. Otherwise, download the \href{https://github.com/d3/d3-color/releases/latest}{\tt latest release}. You can also load directly from \href{https://d3js.org}{\tt d3js.\+org}, either as a \href{https://d3js.org/d3-color.v1.min.js}{\tt standalone library} or as part of \href{https://github.com/d3/d3}{\tt D3 4.\+0}. A\+MD, Common\+JS, and vanilla environments are supported. In vanilla, a {\ttfamily d3} global is exported\+:


\begin{DoxyCode}
<script src="https://d3js.org/d3-color.v1.min.js"></script>
<script>

var steelblue = d3.rgb("steelblue");

</script>
\end{DoxyCode}


\href{https://tonicdev.com/npm/d3-color}{\tt Try d3-\/color in your browser.}

\subsection*{A\+PI Reference}

\label{_color}%
\# d3.{\bfseries color}({\itshape specifier}) \href{https://github.com/d3/d3-color/blob/master/src/color.js}{\tt $<$$>$}

Parses the specified \href{http://www.w3.org/TR/css3-color/#colorunits}{\tt C\+SS Color Module Level 3} {\itshape specifier} string, returning an \href{#rgb}{\tt R\+GB} or \href{#hsl}{\tt H\+SL} color. If the specifier was not valid, null is returned. Some examples\+:


\begin{DoxyItemize}
\item {\ttfamily rgb(255, 255, 255)}
\item {\ttfamily rgb(10\%, 20\%, 30\%)}
\item {\ttfamily rgba(255, 255, 255, 0.\+4)}
\item {\ttfamily rgba(10\%, 20\%, 30\%, 0.\+4)}
\item {\ttfamily hsl(120, 50\%, 20\%)}
\item {\ttfamily hsla(120, 50\%, 20\%, 0.\+4)}
\item {\ttfamily \#ffeeaa}
\item {\ttfamily \#fea}
\item {\ttfamily steelblue}
\end{DoxyItemize}

The list of supported \href{http://www.w3.org/TR/SVG/types.html#ColorKeywords}{\tt named colors} is specified by C\+SS.

Note\+: this function may also be used with {\ttfamily instanceof} to test if an object is a color instance. The same is true of color subclasses, allowing you to test whether a color is in a particular color space.

\label{_color_opacity}%
\# {\itshape color}.{\bfseries opacity}

This color’s opacity, typically in the range \mbox{[}0, 1\mbox{]}.

\label{_color_rgb}%
\# {\itshape color}.{\bfseries rgb}() \href{https://github.com/d3/d3-color/blob/master/src/color.js#L209}{\tt $<$$>$}

Returns the \href{#rgb}{\tt R\+GB equivalent} of this color. For R\+GB colors, that’s {\ttfamily this}.

\label{_color_brighter}%
\# {\itshape color}.{\bfseries brighter}(\mbox{[}{\itshape k}\mbox{]}) \href{https://github.com/d3/d3-color/blob/master/src/color.js#L221}{\tt $<$$>$}

Returns a brighter copy of this color. If {\itshape k} is specified, it controls how much brighter the returned color should be. If {\itshape k} is not specified, it defaults to 1. The behavior of this method is dependent on the implementing color space.

\label{_color_darker}%
\# {\itshape color}.{\bfseries darker}(\mbox{[}{\itshape k}\mbox{]}) \href{https://github.com/d3/d3-color/blob/master/src/color.js#L225}{\tt $<$$>$}

Returns a darker copy of this color. If {\itshape k} is specified, it controls how much darker the returned color should be. If {\itshape k} is not specified, it defaults to 1. The behavior of this method is dependent on the implementing color space.

\label{_color_displayable}%
\# {\itshape color}.{\bfseries displayable}() \href{https://github.com/d3/d3-color/blob/master/src/color.js#L169}{\tt $<$$>$}

Returns true if and only if the color is displayable on standard hardware. For example, this returns false for an R\+GB color if any channel value is less than zero or greater than 255, or if the opacity is not in the range \mbox{[}0, 1\mbox{]}.

\label{_color_toString}%
\# {\itshape color}.{\bfseries to\+String}() \href{https://github.com/d3/d3-color/blob/master/src/color.js#L172}{\tt $<$$>$}

Returns a string representing this color according to the \href{https://drafts.csswg.org/cssom/#serialize-a-css-component-value}{\tt C\+SS Object Model specification}, such as {\ttfamily rgb(247, 234, 186)}. If this color is not displayable, a suitable displayable color is returned instead. For example, R\+GB channel values greater than 255 are clamped to 255.

\label{_rgb}%
\# d3.{\bfseries rgb}({\itshape r}, {\itshape g}, {\itshape b}\mbox{[}, {\itshape opacity}\mbox{]}) \href{https://github.com/d3/d3-color/blob/master/src/color.js#L209}{\tt $<$$>$}~\newline
 \href{#rgb}{\tt \#} d3.{\bfseries rgb}({\itshape specifier})~\newline
 \href{#rgb}{\tt \#} d3.{\bfseries rgb}({\itshape color})~\newline


Constructs a new \href{https://en.wikipedia.org/wiki/RGB_color_model}{\tt R\+GB} color. The channel values are exposed as {\ttfamily r}, {\ttfamily g} and {\ttfamily b} properties on the returned instance. Use the \href{http://bl.ocks.org/mbostock/78d64ca7ef013b4dcf8f}{\tt R\+GB color picker} to explore this color space.

If {\itshape r}, {\itshape g} and {\itshape b} are specified, these represent the channel values of the returned color; an {\itshape opacity} may also be specified. If a C\+SS Color Module Level 3 {\itshape specifier} string is specified, it is parsed and then converted to the R\+GB color space. See \href{#color}{\tt color} for examples. If a \href{#color}{\tt {\itshape color}} instance is specified, it is converted to the R\+GB color space using \href{#color_rgb}{\tt {\itshape color}.rgb}. Note that unlike \href{#color_rgb}{\tt {\itshape color}.rgb} this method {\itshape always} returns a new instance, even if {\itshape color} is already an R\+GB color.

\label{_hsl}%
\# d3.{\bfseries hsl}({\itshape h}, {\itshape s}, {\itshape l}\mbox{[}, {\itshape opacity}\mbox{]}) \href{https://github.com/d3/d3-color/blob/master/src/color.js#L281}{\tt $<$$>$}~\newline
 \href{#hsl}{\tt \#} d3.{\bfseries hsl}({\itshape specifier})~\newline
 \href{#hsl}{\tt \#} d3.{\bfseries hsl}({\itshape color})~\newline


Constructs a new \href{https://en.wikipedia.org/wiki/HSL_and_HSV}{\tt H\+SL} color. The channel values are exposed as {\ttfamily h}, {\ttfamily s} and {\ttfamily l} properties on the returned instance. Use the \href{http://bl.ocks.org/mbostock/debaad4fcce9bcee14cf}{\tt H\+SL color picker} to explore this color space.

If {\itshape h}, {\itshape s} and {\itshape l} are specified, these represent the channel values of the returned color; an {\itshape opacity} may also be specified. If a C\+SS Color Module Level 3 {\itshape specifier} string is specified, it is parsed and then converted to the H\+SL color space. See \href{#color}{\tt color} for examples. If a \href{#color}{\tt {\itshape color}} instance is specified, it is converted to the R\+GB color space using \href{#color_rgb}{\tt {\itshape color}.rgb} and then converted to H\+SL. (Colors already in the H\+SL color space skip the conversion to R\+GB.)

\label{_lab}%
\# d3.{\bfseries lab}({\itshape l}, {\itshape a}, {\itshape b}\mbox{[}, {\itshape opacity}\mbox{]}) \href{https://github.com/d3/d3-color/blob/master/src/lab.js#L30}{\tt $<$$>$}~\newline
 \href{#lab}{\tt \#} d3.{\bfseries lab}({\itshape specifier})~\newline
 \href{#lab}{\tt \#} d3.{\bfseries lab}({\itshape color})~\newline


Constructs a new \href{https://en.wikipedia.org/wiki/Lab_color_space#CIELAB}{\tt Lab} color. The channel values are exposed as {\ttfamily l}, {\ttfamily a} and {\ttfamily b} properties on the returned instance. Use the \href{http://bl.ocks.org/mbostock/9f37cc207c0cb166921b}{\tt Lab color picker} to explore this color space.

If {\itshape l}, {\itshape a} and {\itshape b} are specified, these represent the channel values of the returned color; an {\itshape opacity} may also be specified. If a C\+SS Color Module Level 3 {\itshape specifier} string is specified, it is parsed and then converted to the Lab color space. See \href{#color}{\tt color} for examples. If a \href{#color}{\tt {\itshape color}} instance is specified, it is converted to the R\+GB color space using \href{#color_rgb}{\tt {\itshape color}.rgb} and then converted to Lab. (Colors already in the Lab color space skip the conversion to R\+GB, and colors in the H\+CL color space are converted directly to Lab.)

\label{_hcl}%
\# d3.{\bfseries hcl}({\itshape h}, {\itshape c}, {\itshape l}\mbox{[}, {\itshape opacity}\mbox{]}) \href{https://github.com/d3/d3-color/blob/master/src/lab.js#L87}{\tt $<$$>$}~\newline
 \href{#hcl}{\tt \#} d3.{\bfseries hcl}({\itshape specifier})~\newline
 \href{#hcl}{\tt \#} d3.{\bfseries hcl}({\itshape color})~\newline


Constructs a new \href{https://en.wikipedia.org/wiki/Lab_color_space#CIELAB}{\tt H\+CL} color. The channel values are exposed as {\ttfamily h}, {\ttfamily c} and {\ttfamily l} properties on the returned instance. Use the \href{http://bl.ocks.org/mbostock/3e115519a1b495e0bd95}{\tt H\+CL color picker} to explore this color space.

If {\itshape h}, {\itshape c} and {\itshape l} are specified, these represent the channel values of the returned color; an {\itshape opacity} may also be specified. If a C\+SS Color Module Level 3 {\itshape specifier} string is specified, it is parsed and then converted to the H\+CL color space. See \href{#color}{\tt color} for examples. If a \href{#color}{\tt {\itshape color}} instance is specified, it is converted to the R\+GB color space using \href{#color_rgb}{\tt {\itshape color}.rgb} and then converted to H\+CL. (Colors already in the H\+CL color space skip the conversion to R\+GB, and colors in the Lab color space are converted directly to H\+CL.)

\label{_cubehelix}%
\# d3.{\bfseries cubehelix}({\itshape h}, {\itshape s}, {\itshape l}\mbox{[}, {\itshape opacity}\mbox{]}) \href{https://github.com/d3/d3-color/blob/master/src/cubehelix.js#L32}{\tt $<$$>$}~\newline
 \href{#cubehelix}{\tt \#} d3.{\bfseries cubehelix}({\itshape specifier})~\newline
 \href{#cubehelix}{\tt \#} d3.{\bfseries cubehelix}({\itshape color})~\newline


Constructs a new \href{https://www.mrao.cam.ac.uk/~dag/CUBEHELIX/}{\tt Cubehelix} color. The channel values are exposed as {\ttfamily h}, {\ttfamily s} and {\ttfamily l} properties on the returned instance. Use the \href{http://bl.ocks.org/mbostock/ba8d75e45794c27168b5}{\tt Cubehelix color picker} to explore this color space.

If {\itshape h}, {\itshape s} and {\itshape l} are specified, these represent the channel values of the returned color; an {\itshape opacity} may also be specified. If a C\+SS Color Module Level 3 {\itshape specifier} string is specified, it is parsed and then converted to the Cubehelix color space. See \href{#color}{\tt color} for examples. If a \href{#color}{\tt {\itshape color}} instance is specified, it is converted to the R\+GB color space using \href{#color_rgb}{\tt {\itshape color}.rgb} and then converted to Cubehelix. (Colors already in the Cubehelix color space skip the conversion to R\+GB.) 