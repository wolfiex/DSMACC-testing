Port of \href{http://tweetnacl.cr.yp.to}{\tt Tweet\+Na\+Cl} / \href{http://nacl.cr.yp.to/}{\tt Na\+Cl} to Java\+Script for modern browsers and Node.\+js. Public domain.

\href{https://travis-ci.org/dchest/tweetnacl-js}{\tt }

Demo\+: \href{https://tweetnacl.js.org}{\tt https\+://tweetnacl.\+js.\+org}

$\ast$$\ast$\+:warning\+: The library is stable and A\+PI is frozen, however it has not been independently reviewed. If you can help reviewing it, please \href{mailto:dmitry@codingrobots.com}{\tt contact me}.$\ast$$\ast$

\section*{Documentation }


\begin{DoxyItemize}
\item \href{#overview}{\tt Overview}
\item \href{#installation}{\tt Installation}
\item \href{#usage}{\tt Usage}
\begin{DoxyItemize}
\item \href{#public-key-authenticated-encryption-box}{\tt Public-\/key authenticated encryption (box)}
\item \href{#secret-key-authenticated-encryption-secretbox}{\tt Secret-\/key authenticated encryption (secretbox)}
\item \href{#scalar-multiplication}{\tt Scalar multiplication}
\item \href{#signatures}{\tt Signatures}
\item \href{#hashing}{\tt Hashing}
\item \href{#random-bytes-generation}{\tt Random bytes generation}
\item \href{#constant-time-comparison}{\tt Constant-\/time comparison}
\end{DoxyItemize}
\item \href{#system-requirements}{\tt System requirements}
\item \href{#development-and-testing}{\tt Development and testing}
\item \href{#benchmarks}{\tt Benchmarks}
\item \href{#contributors}{\tt Contributors}
\item \href{#who-uses-it}{\tt Who uses it}
\end{DoxyItemize}

\subsection*{Overview }

The primary goal of this project is to produce a translation of Tweet\+Na\+Cl to Java\+Script which is as close as possible to the original C implementation, plus a thin layer of idiomatic high-\/level A\+PI on top of it.

There are two versions, you can use either of them\+:


\begin{DoxyItemize}
\item {\ttfamily nacl.\+js} is the port of Tweet\+Na\+Cl with minimum differences from the original + high-\/level A\+PI.
\item {\ttfamily nacl-\/fast.\+js} is like {\ttfamily nacl.\+js}, but with some functions replaced with faster versions.
\end{DoxyItemize}

\subsection*{Installation }

You can install Tweet\+Na\+Cl.\+js via a package manager\+:

\href{http://bower.io}{\tt Bower}\+: \begin{DoxyVerb}$ bower install tweetnacl
\end{DoxyVerb}


\href{https://www.npmjs.org/}{\tt N\+PM}\+: \begin{DoxyVerb}$ npm install tweetnacl
\end{DoxyVerb}


or \href{https://github.com/dchest/tweetnacl-js/releases}{\tt download source code}.

\subsection*{Usage }

All A\+PI functions accept and return bytes as {\ttfamily Uint8\+Array}s. If you need to encode or decode strings, use functions from \href{https://github.com/dchest/tweetnacl-util-js}{\tt https\+://github.\+com/dchest/tweetnacl-\/util-\/js} or one of the more robust codec packages.

In Node.\+js v4 and later {\ttfamily Buffer} objects are backed by {\ttfamily Uint8\+Array}s, so you can freely pass them to Tweet\+Na\+Cl.\+js functions as arguments. The returned objects are still {\ttfamily Uint8\+Array}s, so if you need {\ttfamily Buffer}s, you\textquotesingle{}ll have to convert them manually; make sure to convert using copying\+: {\ttfamily new Buffer(array)}, instead of sharing\+: {\ttfamily new Buffer(array.\+buffer)}, because some functions return subarrays of their buffers.

\subsubsection*{Public-\/key authenticated encryption (box)}

Implements {\itshape curve25519-\/xsalsa20-\/poly1305}.

\paragraph*{nacl.\+box.\+key\+Pair()}

Generates a new random key pair for box and returns it as an object with {\ttfamily public\+Key} and {\ttfamily secret\+Key} members\+: \begin{DoxyVerb}{
   publicKey: ...,  // Uint8Array with 32-byte public key
   secretKey: ...   // Uint8Array with 32-byte secret key
}
\end{DoxyVerb}


\paragraph*{nacl.\+box.\+key\+Pair.\+from\+Secret\+Key(secret\+Key)}

Returns a key pair for box with public key corresponding to the given secret key.

\paragraph*{nacl.\+box(message, nonce, their\+Public\+Key, my\+Secret\+Key)}

Encrypt and authenticates message using peer\textquotesingle{}s public key, our secret key, and the given nonce, which must be unique for each distinct message for a key pair.

Returns an encrypted and authenticated message, which is {\ttfamily nacl.\+box.\+overhead\+Length} longer than the original message.

\paragraph*{nacl.\+box.\+open(box, nonce, their\+Public\+Key, my\+Secret\+Key)}

Authenticates and decrypts the given box with peer\textquotesingle{}s public key, our secret key, and the given nonce.

Returns the original message, or {\ttfamily false} if authentication fails.

\paragraph*{nacl.\+box.\+before(their\+Public\+Key, my\+Secret\+Key)}

Returns a precomputed shared key which can be used in {\ttfamily nacl.\+box.\+after} and {\ttfamily nacl.\+box.\+open.\+after}.

\paragraph*{nacl.\+box.\+after(message, nonce, shared\+Key)}

Same as {\ttfamily nacl.\+box}, but uses a shared key precomputed with {\ttfamily nacl.\+box.\+before}.

\paragraph*{nacl.\+box.\+open.\+after(box, nonce, shared\+Key)}

Same as {\ttfamily nacl.\+box.\+open}, but uses a shared key precomputed with {\ttfamily nacl.\+box.\+before}.

\paragraph*{nacl.\+box.\+public\+Key\+Length = 32}

Length of public key in bytes.

\paragraph*{nacl.\+box.\+secret\+Key\+Length = 32}

Length of secret key in bytes.

\paragraph*{nacl.\+box.\+shared\+Key\+Length = 32}

Length of precomputed shared key in bytes.

\paragraph*{nacl.\+box.\+nonce\+Length = 24}

Length of nonce in bytes.

\paragraph*{nacl.\+box.\+overhead\+Length = 16}

Length of overhead added to box compared to original message.

\subsubsection*{Secret-\/key authenticated encryption (secretbox)}

Implements {\itshape xsalsa20-\/poly1305}.

\paragraph*{nacl.\+secretbox(message, nonce, key)}

Encrypt and authenticates message using the key and the nonce. The nonce must be unique for each distinct message for this key.

Returns an encrypted and authenticated message, which is {\ttfamily nacl.\+secretbox.\+overhead\+Length} longer than the original message.

\paragraph*{nacl.\+secretbox.\+open(box, nonce, key)}

Authenticates and decrypts the given secret box using the key and the nonce.

Returns the original message, or {\ttfamily false} if authentication fails.

\paragraph*{nacl.\+secretbox.\+key\+Length = 32}

Length of key in bytes.

\paragraph*{nacl.\+secretbox.\+nonce\+Length = 24}

Length of nonce in bytes.

\paragraph*{nacl.\+secretbox.\+overhead\+Length = 16}

Length of overhead added to secret box compared to original message.

\subsubsection*{Scalar multiplication}

Implements {\itshape curve25519}.

\paragraph*{nacl.\+scalar\+Mult(n, p)}

Multiplies an integer {\ttfamily n} by a group element {\ttfamily p} and returns the resulting group element.

\paragraph*{nacl.\+scalar\+Mult.\+base(n)}

Multiplies an integer {\ttfamily n} by a standard group element and returns the resulting group element.

\paragraph*{nacl.\+scalar\+Mult.\+scalar\+Length = 32}

Length of scalar in bytes.

\paragraph*{nacl.\+scalar\+Mult.\+group\+Element\+Length = 32}

Length of group element in bytes.

\subsubsection*{Signatures}

Implements \href{http://ed25519.cr.yp.to}{\tt ed25519}.

\paragraph*{nacl.\+sign.\+key\+Pair()}

Generates new random key pair for signing and returns it as an object with {\ttfamily public\+Key} and {\ttfamily secret\+Key} members\+: \begin{DoxyVerb}{
   publicKey: ...,  // Uint8Array with 32-byte public key
   secretKey: ...   // Uint8Array with 64-byte secret key
}
\end{DoxyVerb}


\paragraph*{nacl.\+sign.\+key\+Pair.\+from\+Secret\+Key(secret\+Key)}

Returns a signing key pair with public key corresponding to the given 64-\/byte secret key. The secret key must have been generated by {\ttfamily nacl.\+sign.\+key\+Pair} or {\ttfamily nacl.\+sign.\+key\+Pair.\+from\+Seed}.

\paragraph*{nacl.\+sign.\+key\+Pair.\+from\+Seed(seed)}

Returns a new signing key pair generated deterministically from a 32-\/byte seed. The seed must contain enough entropy to be secure. This method is not recommended for general use\+: instead, use {\ttfamily nacl.\+sign.\+key\+Pair} to generate a new key pair from a random seed.

\paragraph*{nacl.\+sign(message, secret\+Key)}

Signs the message using the secret key and returns a signed message.

\paragraph*{nacl.\+sign.\+open(signed\+Message, public\+Key)}

Verifies the signed message and returns the message without signature.

Returns {\ttfamily null} if verification failed.

\paragraph*{nacl.\+sign.\+detached(message, secret\+Key)}

Signs the message using the secret key and returns a signature.

\paragraph*{nacl.\+sign.\+detached.\+verify(message, signature, public\+Key)}

Verifies the signature for the message and returns {\ttfamily true} if verification succeeded or {\ttfamily false} if it failed.

\paragraph*{nacl.\+sign.\+public\+Key\+Length = 32}

Length of signing public key in bytes.

\paragraph*{nacl.\+sign.\+secret\+Key\+Length = 64}

Length of signing secret key in bytes.

\paragraph*{nacl.\+sign.\+seed\+Length = 32}

Length of seed for {\ttfamily nacl.\+sign.\+key\+Pair.\+from\+Seed} in bytes.

\paragraph*{nacl.\+sign.\+signature\+Length = 64}

Length of signature in bytes.

\subsubsection*{Hashing}

Implements {\itshape S\+H\+A-\/512}.

\paragraph*{nacl.\+hash(message)}

Returns S\+H\+A-\/512 hash of the message.

\paragraph*{nacl.\+hash.\+hash\+Length = 64}

Length of hash in bytes.

\subsubsection*{Random bytes generation}

\paragraph*{nacl.\+random\+Bytes(length)}

Returns a {\ttfamily Uint8\+Array} of the given length containing random bytes of cryptographic quality.

{\bfseries Implementation note}

Tweet\+Na\+Cl.\+js uses the following methods to generate random bytes, depending on the platform it runs on\+:


\begin{DoxyItemize}
\item {\ttfamily window.\+crypto.\+get\+Random\+Values} (Web\+Crypto standard)
\item {\ttfamily window.\+ms\+Crypto.\+get\+Random\+Values} (Internet Explorer 11)
\item {\ttfamily crypto.\+random\+Bytes} (Node.\+js)
\end{DoxyItemize}

If the platform doesn\textquotesingle{}t provide a suitable P\+R\+NG, the following functions, which require random numbers, will throw exception\+:


\begin{DoxyItemize}
\item {\ttfamily nacl.\+random\+Bytes}
\item {\ttfamily nacl.\+box.\+key\+Pair}
\item {\ttfamily nacl.\+sign.\+key\+Pair}
\end{DoxyItemize}

Other functions are deterministic and will continue working.

If a platform you are targeting doesn\textquotesingle{}t implement secure random number generator, but you somehow have a cryptographically-\/strong source of entropy (not {\ttfamily Math.\+random}!), and you know what you are doing, you can plug it into Tweet\+Na\+Cl.\+js like this\+: \begin{DoxyVerb}nacl.setPRNG(function(x, n) {
  // ... copy n random bytes into x ...
});
\end{DoxyVerb}


Note that {\ttfamily nacl.\+set\+P\+R\+NG} {\itshape completely replaces} internal random byte generator with the one provided.

\subsubsection*{Constant-\/time comparison}

\paragraph*{nacl.\+verify(x, y)}

Compares {\ttfamily x} and {\ttfamily y} in constant time and returns {\ttfamily true} if their lengths are non-\/zero and equal, and their contents are equal.

Returns {\ttfamily false} if either of the arguments has zero length, or arguments have different lengths, or their contents differ.

\subsection*{System requirements }

Tweet\+Na\+Cl.\+js supports modern browsers that have a cryptographically secure pseudorandom number generator and typed arrays, including the latest versions of\+:


\begin{DoxyItemize}
\item Chrome
\item Firefox
\item Safari (Mac, i\+OS)
\item Internet Explorer 11
\end{DoxyItemize}

Other systems\+:


\begin{DoxyItemize}
\item Node.\+js
\end{DoxyItemize}

\subsection*{Development and testing }

Install N\+PM modules needed for development\+: \begin{DoxyVerb}$ npm install
\end{DoxyVerb}


To build minified versions\+: \begin{DoxyVerb}$ npm run build
\end{DoxyVerb}


Tests use minified version, so make sure to rebuild it every time you change {\ttfamily nacl.\+js} or {\ttfamily nacl-\/fast.\+js}.

\subsubsection*{Testing}

To run tests in Node.\+js\+: \begin{DoxyVerb}$ npm run test-node
\end{DoxyVerb}


By default all tests described here work on {\ttfamily nacl.\+min.\+js}. To test other versions, set environment variable {\ttfamily N\+A\+C\+L\+\_\+\+S\+RC} to the file name you want to test. For example, the following command will test fast minified version\+: \begin{DoxyVerb}$ NACL_SRC=nacl-fast.min.js npm run test-node
\end{DoxyVerb}


To run full suite of tests in Node.\+js, including comparing outputs of Java\+Script port to outputs of the original C version\+: \begin{DoxyVerb}$ npm run test-node-all
\end{DoxyVerb}


To prepare tests for browsers\+: \begin{DoxyVerb}$ npm run build-test-browser
\end{DoxyVerb}


and then open {\ttfamily test/browser/test.\+html} (or {\ttfamily test/browser/test-\/fast.\+html}) to run them.

To run headless browser tests with {\ttfamily tape-\/run} (powered by Electron)\+: \begin{DoxyVerb}$ npm run test-browser
\end{DoxyVerb}


(If you get {\ttfamily Error\+: spawn E\+N\+O\+E\+NT}, install {\itshape xvfb}\+: {\ttfamily sudo apt-\/get install xvfb}.)

To run tests in both \mbox{\hyperlink{classNode}{Node}} and Electron\+: \begin{DoxyVerb}$ npm test
\end{DoxyVerb}


\subsubsection*{Benchmarking}

To run benchmarks in Node.\+js\+: \begin{DoxyVerb}$ npm run bench
$ NACL_SRC=nacl-fast.min.js npm run bench
\end{DoxyVerb}


To run benchmarks in a browser, open {\ttfamily test/benchmark/bench.\+html} (or {\ttfamily test/benchmark/bench-\/fast.\+html}).

\subsection*{Benchmarks }

For reference, here are benchmarks from Mac\+Book Pro (Retina, 13-\/inch, Mid 2014) laptop with 2.\+6 G\+Hz Intel Core i5 C\+PU (Intel) in Chrome 53/\+OS X and Xiaomi Redmi Note 3 smartphone with 1.\+8 G\+Hz Qualcomm Snapdragon 650 64-\/bit C\+PU (A\+RM) in Chrome 52/\+Android\+:

\tabulinesep=1mm
\begin{longtabu} spread 0pt [c]{*{5}{|X[-1]}|}
\hline
\rowcolor{\tableheadbgcolor}\textbf{ }&\textbf{ nacl.\+js Intel  }&\textbf{ nacl-\/fast.\+js Intel  }&\textbf{ nacl.\+js A\+RM  }&\textbf{ nacl-\/fast.\+js A\+RM   }\\\cline{1-5}
\endfirsthead
\hline
\endfoot
\hline
\rowcolor{\tableheadbgcolor}\textbf{ }&\textbf{ nacl.\+js Intel  }&\textbf{ nacl-\/fast.\+js Intel  }&\textbf{ nacl.\+js A\+RM  }&\textbf{ nacl-\/fast.\+js A\+RM   }\\\cline{1-5}
\endhead
salsa20  &1.\+3 M\+B/s  &128 M\+B/s  &0.\+4 M\+B/s  &43 M\+B/s   \\\cline{1-5}
poly1305  &13 M\+B/s  &171 M\+B/s  &4 M\+B/s  &52 M\+B/s   \\\cline{1-5}
hash  &4 M\+B/s  &34 M\+B/s  &0.\+9 M\+B/s  &12 M\+B/s   \\\cline{1-5}
secretbox 1K  &1113 op/s  &57583 op/s  &334 op/s  &14227 op/s   \\\cline{1-5}
box 1K  &145 op/s  &718 op/s  &37 op/s  &368 op/s   \\\cline{1-5}
scalar\+Mult  &171 op/s  &733 op/s  &56 op/s  &380 op/s   \\\cline{1-5}
sign  &77 op/s  &200 op/s  &20 op/s  &61 op/s   \\\cline{1-5}
sign.\+open  &39 op/s  &102 op/s  &11 op/s  &31 op/s   \\\cline{1-5}
\end{longtabu}


(You can run benchmarks on your devices by clicking on the links at the bottom of the \href{https://tweetnacl.js.org}{\tt home page}).

In short, with {\itshape nacl-\/fast.\+js} and 1024-\/byte messages you can expect to encrypt and authenticate more than 57000 messages per second on a typical laptop or more than 14000 messages per second on a \$170 smartphone, sign about 200 and verify 100 messages per second on a laptop or 60 and 30 messages per second on a smartphone, per C\+PU core (with Web Workers you can do these operations in parallel), which is good enough for most applications.

\subsection*{Contributors }

See A\+U\+T\+H\+O\+R\+S.\+md file.

\subsection*{Third-\/party libraries based on Tweet\+Na\+Cl.\+js }


\begin{DoxyItemize}
\item \href{https://github.com/alax/forward-secrecy}{\tt forward-\/secrecy} — Axolotl ratchet implementation
\item \href{https://github.com/dchest/nacl-stream-js}{\tt nacl-\/stream} -\/ streaming encryption
\item \href{https://github.com/dchest/tweetnacl-auth-js}{\tt tweetnacl-\/auth-\/js} — implementation of \href{http://nacl.cr.yp.to/auth.html}{\tt {\ttfamily crypto\+\_\+auth}}
\item \href{https://github.com/dominictarr/chloride}{\tt chloride} -\/ unified A\+PI for various Na\+Cl modules
\end{DoxyItemize}

\subsection*{Who uses it }

Some notable users of Tweet\+Na\+Cl.\+js\+:


\begin{DoxyItemize}
\item \href{http://minilock.io/}{\tt mini\+Lock}
\item \href{https://www.stellar.org/}{\tt Stellar} 
\end{DoxyItemize}