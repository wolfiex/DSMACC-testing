A transition is a \href{https://github.com/d3/d3-selection}{\tt selection}-\/like interface for animating changes to the D\+OM. Instead of applying changes instantaneously, transitions smoothly interpolate the D\+OM from its current state to the desired target state over a given duration.

To apply a transition, select elements, call \href{#selection_transition}{\tt {\itshape selection}.transition}, and then make the desired changes. For example\+:


\begin{DoxyCode}
d3.select("body")
  .transition()
    .style("background-color", "red");
\end{DoxyCode}


Transitions support most selection methods (such as \href{#transition_attr}{\tt {\itshape transition}.attr} and \href{#transition_style}{\tt {\itshape transition}.style} in place of \href{https://github.com/d3/d3-selection#selection_attr}{\tt {\itshape selection}.attr} and \href{https://github.com/d3/d3-selection#selection_style}{\tt {\itshape selection}.style}), but not all methods are supported; for example, you must \href{https://github.com/d3/d3-selection#selection_append}{\tt append} elements or \href{https://github.com/d3/d3-selection#joining-data}{\tt bind data} before a transition starts. A \href{#transition_remove}{\tt {\itshape transition}.remove} operator is provided for convenient removal of elements when the transition ends.

To compute intermediate state, transitions leverage a variety of \href{https://github.com/d3/d3-interpolate}{\tt built-\/in interpolators}. \href{https://github.com/d3/d3-interpolate#interpolateRgb}{\tt Colors}, \href{https://github.com/d3/d3-interpolate#interpolateNumber}{\tt numbers}, and \href{https://github.com/d3/d3-interpolate#interpolateTransform}{\tt transforms} are automatically detected. \href{https://github.com/d3/d3-interpolate#interpolateString}{\tt Strings} with embedded numbers are also detected, as is common with many styles (such as padding or font sizes) and paths. To specify a custom interpolator, use \href{#transition_attrTween}{\tt {\itshape transition}.attr\+Tween}, \href{#transition_styleTween}{\tt {\itshape transition}.style\+Tween} or \href{#transition_tween}{\tt {\itshape transition}.tween}.

\subsection*{Installing}

If you use N\+PM, {\ttfamily npm install d3-\/transition}. Otherwise, download the \href{https://github.com/d3/d3-transition/releases/latest}{\tt latest release}. You can also load directly from \href{https://d3js.org}{\tt d3js.\+org}, either as a \href{https://d3js.org/d3-transition.v1.min.js}{\tt standalone library} or as part of \href{https://github.com/d3/d3}{\tt D3 4.\+0}. A\+MD, Common\+JS, and vanilla environments are supported. In vanilla, a {\ttfamily d3} global is exported\+:


\begin{DoxyCode}
<script src="https://d3js.org/d3-color.v1.min.js"></script>
<script src="https://d3js.org/d3-dispatch.v1.min.js"></script>
<script src="https://d3js.org/d3-ease.v1.min.js"></script>
<script src="https://d3js.org/d3-interpolate.v1.min.js"></script>
<script src="https://d3js.org/d3-selection.v1.min.js"></script>
<script src="https://d3js.org/d3-timer.v1.min.js"></script>
<script src="https://d3js.org/d3-transition.v1.min.js"></script>
<script>

var transition = d3.transition();

</script>
\end{DoxyCode}


\href{https://tonicdev.com/npm/d3-transition}{\tt Try d3-\/transition in your browser.}

\subsection*{A\+PI Reference}


\begin{DoxyItemize}
\item \href{#selecting-elements}{\tt Selecting Elements}
\item \href{#modifying-elements}{\tt Modifying Elements}
\item \href{#timing}{\tt Timing}
\item \href{#control-flow}{\tt Control Flow}
\item \href{#the-life-of-a-transition}{\tt The Life of a Transition}
\end{DoxyItemize}

\subsubsection*{Selecting Elements}

Transitions are derived from \href{https://github.com/d3/d3-selection}{\tt selections} via \href{#selection_transition}{\tt {\itshape selection}.transition}. You can also create a transition on the document root element using \href{#transition}{\tt d3.\+transition}.

\label{_selection_transition}%
\# {\itshape selection}.{\bfseries transition}(\mbox{[}{\itshape name}\mbox{]}) \href{https://github.com/d3/d3-transition/blob/master/src/selection/transition.js}{\tt $<$$>$}

Returns a new transition on the given {\itshape selection} with the specified {\itshape name}. If a {\itshape name} is not specified, null is used. The new transition is only exclusive with other transitions of the same name.

If the {\itshape name} is a \href{#transition}{\tt transition} instance, the returned transition has the same id and name as the specified transition. If a transition with the same id already exists on a selected element, the existing transition is returned for that element. Otherwise, the timing of the returned transition is inherited from the existing transition of the same id on the nearest ancestor of each selected element. Thus, this method can be used to synchronize a transition across multiple selections, or to re-\/select a transition for specific elements and modify its configuration. For example\+:


\begin{DoxyCode}
var t = d3.transition()
    .duration(750)
    .ease(d3.easeLinear);

d3.selectAll(".apple").transition(t)
    .style("fill", "red");

d3.selectAll(".orange").transition(t)
    .style("fill", "orange");
\end{DoxyCode}


If the specified {\itshape transition} is not found on a selected node or its ancestors (such as if the transition \href{#the-life-of-a-transition}{\tt already ended}), the default timing parameters are used; however, in a future release, this will likely be changed to throw an error. See \href{https://github.com/d3/d3-transition/issues/59}{\tt \#59}.

\label{_selection_interrupt}%
\# {\itshape selection}.{\bfseries interrupt}(\mbox{[}{\itshape name}\mbox{]}) \href{https://github.com/d3/d3-transition/blob/master/src/selection/interrupt.js}{\tt $<$$>$}

Interrupts the active transition of the specified {\itshape name} on the selected elements, and cancels any pending transitions with the specified {\itshape name}, if any. If a name is not specified, null is used.

Interrupting a transition on an element has no effect on any transitions on any descendant elements. For example, an \href{https://github.com/d3/d3-axis}{\tt axis transition} consists of multiple independent, synchronized transitions on the descendants of the axis \href{https://www.w3.org/TR/SVG/struct.html#Groups}{\tt G element} (the tick lines, the tick labels, the domain path, {\itshape etc.}). To interrupt the axis transition, you must therefore interrupt the descendants\+:


\begin{DoxyCode}
selection.selectAll("*").interrupt();
\end{DoxyCode}


The \href{https://developer.mozilla.org/en-US/docs/Web/CSS/Universal_selectors}{\tt universal selector}, {\ttfamily $\ast$}, selects all descendant elements. If you also want to interrupt the G element itself\+:


\begin{DoxyCode}
selection.interrupt().selectAll("*").interrupt();
\end{DoxyCode}


\label{_interrupt}%
\# d3.{\bfseries interrupt}({\itshape node}\mbox{[}, {\itshape name}\mbox{]}) \href{https://github.com/d3/d3-transition/blob/master/src/interrupt.js}{\tt $<$$>$}

Interrupts the active transition of the specified {\itshape name} on the specified {\itshape node}, and cancels any pending transitions with the specified {\itshape name}, if any. If a name is not specified, null is used. See also \href{#selection_interrupt}{\tt {\itshape selection}.interrupt}.

\label{_transition}%
\# d3.{\bfseries transition}(\mbox{[}{\itshape name}\mbox{]}) \href{https://github.com/d3/d3-transition/blob/master/src/transition/index.js#L29}{\tt $<$$>$}

Returns a new transition on the root element, {\ttfamily document.\+document\+Element}, with the specified {\itshape name}. If a {\itshape name} is not specified, null is used. The new transition is only exclusive with other transitions of the same name. The {\itshape name} may also be a \href{#transition}{\tt transition} instance; see \href{#selection_transition}{\tt {\itshape selection}.transition}. This method is equivalent to\+:


\begin{DoxyCode}
d3.selection()
  .transition(name)
\end{DoxyCode}


This function can also be used to test for transitions ({\ttfamily instanceof d3.\+transition}) or to extend the transition prototype.

\label{_transition_select}%
\# {\itshape transition}.{\bfseries select}({\itshape selector}) \href{https://github.com/d3/d3-transition/blob/master/src/transition/select.js}{\tt $<$$>$}

For each selected element, selects the first descendant element that matches the specified {\itshape selector} string, if any, and returns a transition on the resulting selection. The {\itshape selector} may be specified either as a selector string or a function. If a function, it is evaluated for each selected element, in order, being passed the current datum {\ttfamily d} and index {\ttfamily i}, with the {\ttfamily this} context as the current D\+OM element. The new transition has the same id, name and timing as this transition; however, if a transition with the same id already exists on a selected element, the existing transition is returned for that element.

This method is equivalent to deriving the selection for this transition via \href{#transition_selection}{\tt {\itshape transition}.selection}, creating a subselection via \href{https://github.com/d3/d3-selection#selection_select}{\tt {\itshape selection}.select}, and then creating a new transition via \href{#selection_transition}{\tt {\itshape selection}.transition}\+:


\begin{DoxyCode}
transition
  .selection()
  .select(selector)
  .transition(transition)
\end{DoxyCode}


\label{_transition_selectAll}%
\# {\itshape transition}.{\bfseries select\+All}({\itshape selector}) \href{https://github.com/d3/d3-transition/blob/master/src/transition/selectAll.js}{\tt $<$$>$}

For each selected element, selects all descendant elements that match the specified {\itshape selector} string, if any, and returns a transition on the resulting selection. The {\itshape selector} may be specified either as a selector string or a function. If a function, it is evaluated for each selected element, in order, being passed the current datum {\ttfamily d} and index {\ttfamily i}, with the {\ttfamily this} context as the current D\+OM element. The new transition has the same id, name and timing as this transition; however, if a transition with the same id already exists on a selected element, the existing transition is returned for that element.

This method is equivalent to deriving the selection for this transition via \href{#transition_selection}{\tt {\itshape transition}.selection}, creating a subselection via \href{https://github.com/d3/d3-selection#selection_selectAll}{\tt {\itshape selection}.select\+All}, and then creating a new transition via \href{#selection_transition}{\tt {\itshape selection}.transition}\+:


\begin{DoxyCode}
transition
  .selection()
  .selectAll(selector)
  .transition(transition)
\end{DoxyCode}


\label{_transition_filter}%
\# {\itshape transition}.{\bfseries filter}({\itshape filter}) \href{https://github.com/d3/d3-transition/blob/master/src/transition/filter.js}{\tt $<$$>$}

For each selected element, selects only the elements that match the specified {\itshape filter}, and returns a transition on the resulting selection. The {\itshape filter} may be specified either as a selector string or a function. If a function, it is evaluated for each selected element, in order, being passed the current datum {\ttfamily d} and index {\ttfamily i}, with the {\ttfamily this} context as the current D\+OM element. The new transition has the same id, name and timing as this transition; however, if a transition with the same id already exists on a selected element, the existing transition is returned for that element.

This method is equivalent to deriving the selection for this transition via \href{#transition_selection}{\tt {\itshape transition}.selection}, creating a subselection via \href{https://github.com/d3/d3-selection#selection_filter}{\tt {\itshape selection}.filter}, and then creating a new transition via \href{#selection_transition}{\tt {\itshape selection}.transition}\+:


\begin{DoxyCode}
transition
  .selection()
  .filter(filter)
  .transition(transition)
\end{DoxyCode}


\label{_transition_merge}%
\# {\itshape transition}.{\bfseries merge}({\itshape other}) \href{https://github.com/d3/d3-transition/blob/master/src/transition/merge.js}{\tt $<$$>$}

Returns a new transition merging this transition with the specified {\itshape other} transition, which must have the same id as this transition. The returned transition has the same number of groups, the same parents, the same name and the same id as this transition. Any missing (null) elements in this transition are filled with the corresponding element, if present (not null), from the {\itshape other} transition.

This method is equivalent to deriving the selection for this transition via \href{#transition_selection}{\tt {\itshape transition}.selection}, merging with the selection likewise derived from the {\itshape other} transition via \href{https://github.com/d3/d3-selection#selection_merge}{\tt {\itshape selection}.merge}, and then creating a new transition via \href{#selection_transition}{\tt {\itshape selection}.transition}\+:


\begin{DoxyCode}
transition
  .selection()
  .merge(other.selection())
  .transition(transition)
\end{DoxyCode}


\label{_transition_transition}%
\# {\itshape transition}.{\bfseries transition}() \href{https://github.com/d3/d3-transition/blob/master/src/transition/transition.js}{\tt $<$$>$}

Returns a new transition on the same selected elements as this transition, scheduled to start when this transition ends. The new transition inherits a reference time equal to this transition’s time plus its \href{#transition_delay}{\tt delay} and \href{#transition_duration}{\tt duration}. The new transition also inherits this transition’s name, duration, and \href{#transition_ease}{\tt easing}. This method can be used to schedule a sequence of chained transitions. For example\+:


\begin{DoxyCode}
d3.selectAll(".apple")
  .transition() // First fade to green.
    .style("fill", "green")
  .transition() // Then red.
    .style("fill", "red")
  .transition() // Wait one second. Then brown, and remove.
    .delay(1000)
    .style("fill", "brown")
    .remove();
\end{DoxyCode}


The delay for each transition is relative to its previous transition. Thus, in the above example, apples will stay red for one second before the last transition to brown starts.

\label{_transition_selection}%
\# {\itshape transition}.{\bfseries selection}() \href{https://github.com/d3/d3-transition/blob/master/src/transition/selection.js}{\tt $<$$>$}

Returns the \href{https://github.com/d3/d3-selection#selection}{\tt selection} corresponding to this transition.

\label{_active}%
\# d3.{\bfseries active}({\itshape node}\mbox{[}, {\itshape name}\mbox{]}) \href{https://github.com/d3/d3-transition/blob/master/src/active.js}{\tt $<$$>$}

Returns the active transition on the specified {\itshape node} with the specified {\itshape name}, if any. If no {\itshape name} is specified, null is used. Returns null if there is no such active transition on the specified node. This method is useful for creating chained transitions. For example, to initiate disco mode\+:


\begin{DoxyCode}
d3.selectAll("circle").transition()
    .delay(function(d, i) \{ return i * 50; \})
    .on("start", function repeat() \{
        d3.active(this)
            .style("fill", "red")
          .transition()
            .style("fill", "green")
          .transition()
            .style("fill", "blue")
          .transition()
            .on("start", repeat);
      \});
\end{DoxyCode}


See \href{http://bl.ocks.org/mbostock/70d5541b547cc222aa02}{\tt chained transitions} for an example.

\subsubsection*{Modifying Elements}

After selecting elements and creating a transition with \href{#selection_transition}{\tt {\itshape selection}.transition}, use the transition’s transformation methods to affect document content.

\label{_transition_attr}%
\# {\itshape transition}.{\bfseries attr}({\itshape name}, {\itshape value}) \href{https://github.com/d3/d3-transition/blob/master/src/transition/attr.js}{\tt $<$$>$}

For each selected element, assigns the \href{#transition_attrTween}{\tt attribute tween} for the attribute with the specified {\itshape name} to the specified target {\itshape value}. The starting value of the tween is the attribute’s value when the transition starts. The target {\itshape value} may be specified either as a constant or a function. If a function, it is immediately evaluated for each selected element, in order, being passed the current datum {\ttfamily d} and index {\ttfamily i}, with the {\ttfamily this} context as the current D\+OM element.

If the target value is null, the attribute is removed when the transition starts. Otherwise, an interpolator is chosen based on the type of the target value, using the following algorithm\+:


\begin{DoxyEnumerate}
\item If {\itshape value} is a number, use \href{https://github.com/d3/d3-interpolate#interpolateNumber}{\tt interpolate\+Number}.
\item If {\itshape value} is a \href{https://github.com/d3/d3-color#color}{\tt color} or a string coercible to a color, use \href{https://github.com/d3/d3-interpolate#interpolateRgb}{\tt interpolate\+Rgb}.
\item Use \href{https://github.com/d3/d3-interpolate#interpolateString}{\tt interpolate\+String}.
\end{DoxyEnumerate}

To apply a different interpolator, use \href{#transition_attrTween}{\tt {\itshape transition}.attr\+Tween}.

\label{_transition_attrTween}%
\# {\itshape transition}.{\bfseries attr\+Tween}({\itshape name}\mbox{[}, {\itshape factory}\mbox{]}) \href{https://github.com/d3/d3-transition/blob/master/src/transition/attrTween.js}{\tt $<$$>$}

If {\itshape factory} is specified and not null, assigns the attribute \href{#transition_tween}{\tt tween} for the attribute with the specified {\itshape name} to the specified interpolator {\itshape factory}. An interpolator factory is a function that returns an \href{https://github.com/d3/d3-interpolate}{\tt interpolator}; when the transition starts, the {\itshape factory} is evaluated for each selected element, in order, being passed the current datum {\ttfamily d} and index {\ttfamily i}, with the {\ttfamily this} context as the current D\+OM element. The returned interpolator will then be invoked for each frame of the transition, in order, being passed the \href{#transition_ease}{\tt eased} time {\itshape t}, typically in the range \mbox{[}0, 1\mbox{]}. Lastly, the return value of the interpolator will be used to set the attribute value. The interpolator must return a string. (To remove an attribute at the start of a transition, use \href{#transition_attr}{\tt {\itshape transition}.attr}; to remove an attribute at the end of a transition, use \href{#transition_on}{\tt {\itshape transition}.on} to listen for the {\itshape end} event.)

If the specified {\itshape factory} is null, removes the previously-\/assigned attribute tween of the specified {\itshape name}, if any. If {\itshape factory} is not specified, returns the current interpolator factory for attribute with the specified {\itshape name}, or undefined if no such tween exists.

For example, to interpolate the fill attribute from red to blue\+:


\begin{DoxyCode}
transition.attrTween("fill", function() \{
  return d3.interpolateRgb("red", "blue");
\});
\end{DoxyCode}


Or to interpolate from the current fill to blue, like \href{#transition_attr}{\tt {\itshape transition}.attr}\+:


\begin{DoxyCode}
transition.attrTween("fill", function() \{
  return d3.interpolateRgb(this.getAttribute("fill"), "blue");
\});
\end{DoxyCode}


Or to apply a custom rainbow interpolator\+:


\begin{DoxyCode}
transition.attrTween("fill", function() \{
  return function(t) \{
    return "hsl(" + t * 360 + ",100%,50%)";
  \};
\});
\end{DoxyCode}


This method is useful to specify a custom interpolator, such as one that understands \href{http://bl.ocks.org/mbostock/3916621}{\tt S\+VG paths}. A useful technique is {\itshape data interpolation}, where \href{https://github.com/d3/d3-interpolate#interpolateObject}{\tt d3.\+interpolate\+Object} is used to interpolate two data values, and the resulting value is then used (say, with a \href{https://github.com/d3/d3-shape}{\tt shape}) to compute the new attribute value.

\label{_transition_style}%
\# {\itshape transition}.{\bfseries style}({\itshape name}, {\itshape value}\mbox{[}, {\itshape priority}\mbox{]}) \href{https://github.com/d3/d3-transition/blob/master/src/transition/style.js}{\tt $<$$>$}

For each selected element, assigns the \href{#transition_styleTween}{\tt style tween} for the style with the specified {\itshape name} to the specified target {\itshape value} with the specified {\itshape priority}. The starting value of the tween is the style’s inline value if present, and otherwise its computed value, when the transition starts. The target {\itshape value} may be specified either as a constant or a function. If a function, it is immediately evaluated for each selected element, in order, being passed the current datum {\ttfamily d} and index {\ttfamily i}, with the {\ttfamily this} context as the current D\+OM element.

If the target value is null, the style is removed when the transition starts. Otherwise, an interpolator is chosen based on the type of the target value, using the following algorithm\+:


\begin{DoxyEnumerate}
\item If {\itshape value} is a number, use \href{https://github.com/d3/d3-interpolate#interpolateNumber}{\tt interpolate\+Number}.
\item If {\itshape value} is a \href{https://github.com/d3/d3-color#color}{\tt color} or a string coercible to a color, use \href{https://github.com/d3/d3-interpolate#interpolateRgb}{\tt interpolate\+Rgb}.
\item Use \href{https://github.com/d3/d3-interpolate#interpolateString}{\tt interpolate\+String}.
\end{DoxyEnumerate}

To apply a different interpolator, use \href{#transition_styleTween}{\tt {\itshape transition}.style\+Tween}.

\label{_transition_styleTween}%
\# {\itshape transition}.{\bfseries style\+Tween}({\itshape name}\mbox{[}, {\itshape factory}\mbox{[}, {\itshape priority}\mbox{]}\mbox{]})) \href{https://github.com/d3/d3-transition/blob/master/src/transition/styleTween.js}{\tt $<$$>$}

If {\itshape factory} is specified and not null, assigns the style \href{#transition_tween}{\tt tween} for the style with the specified {\itshape name} to the specified interpolator {\itshape factory}. An interpolator factory is a function that returns an \href{https://github.com/d3/d3-interpolate}{\tt interpolator}; when the transition starts, the {\itshape factory} is evaluated for each selected element, in order, being passed the current datum {\ttfamily d} and index {\ttfamily i}, with the {\ttfamily this} context as the current D\+OM element. The returned interpolator will then be invoked for each frame of the transition, in order, being passed the \href{#transition_ease}{\tt eased} time {\itshape t}, typically in the range \mbox{[}0, 1\mbox{]}. Lastly, the return value of the interpolator will be used to set the style value with the specified {\itshape priority}. The interpolator must return a string. (To remove an style at the start of a transition, use \href{#transition_style}{\tt {\itshape transition}.style}; to remove an style at the end of a transition, use \href{#transition_on}{\tt {\itshape transition}.on} to listen for the {\itshape end} event.)

If the specified {\itshape factory} is null, removes the previously-\/assigned style tween of the specified {\itshape name}, if any. If {\itshape factory} is not specified, returns the current interpolator factory for style with the specified {\itshape name}, or undefined if no such tween exists.

For example, to interpolate the fill style from red to blue\+:


\begin{DoxyCode}
transition.styleTween("fill", function() \{
  return d3.interpolateRgb("red", "blue");
\});
\end{DoxyCode}


Or to interpolate from the current fill to blue, like \href{#transition_style}{\tt {\itshape transition}.style}\+:


\begin{DoxyCode}
transition.styleTween("fill", function() \{
  return d3.interpolateRgb(this.style.fill, "blue");
\});
\end{DoxyCode}


Or to apply a custom rainbow interpolator\+:


\begin{DoxyCode}
transition.styleTween("fill", function() \{
  return function(t) \{
    return "hsl(" + t * 360 + ",100%,50%)";
  \};
\});
\end{DoxyCode}


This method is useful to specify a custom interpolator, such as with {\itshape data interpolation}, where \href{https://github.com/d3/d3-interpolate#interpolateObject}{\tt d3.\+interpolate\+Object} is used to interpolate two data values, and the resulting value is then used to compute the new style value.

\label{_transition_text}%
\# {\itshape transition}.{\bfseries text}({\itshape value}) \href{https://github.com/d3/d3-transition/blob/master/src/transition/text.js}{\tt $<$$>$}

For each selected element, sets the \href{http://www.w3.org/TR/DOM-Level-3-Core/core.html#Node3-textContent}{\tt text content} to the specified target {\itshape value} when the transition starts. The {\itshape value} may be specified either as a constant or a function. If a function, it is immediately evaluated for each selected element, in order, being passed the current datum {\ttfamily d} and index {\ttfamily i}, with the {\ttfamily this} context as the current D\+OM element. The function’s return value is then used to set each element’s text content. A null value will clear the content.

To interpolate text rather than to set it on start, use \href{#transition_tween}{\tt {\itshape transition}.tween} (\href{http://bl.ocks.org/mbostock/7004f92cac972edef365}{\tt for example}) or append a replacement element and cross-\/fade opacity (\href{http://bl.ocks.org/mbostock/f7dcecb19c4af317e464}{\tt for example}). Text is not interpolated by default because it is usually undesirable.

\label{_transition_remove}%
\# {\itshape transition}.{\bfseries remove}() \href{https://github.com/d3/d3-transition/blob/master/src/transition/remove.js}{\tt $<$$>$}

For each selected element, \href{https://github.com/d3/d3-selection#selection_remove}{\tt removes} the element when the transition ends, as long as the element has no other active or pending transitions. If the element has other active or pending transitions, does nothing.

\label{_transition_tween}%
\# {\itshape transition}.{\bfseries tween}({\itshape name}\mbox{[}, {\itshape value}\mbox{]}) \href{https://github.com/d3/d3-transition/blob/master/src/transition/tween.js}{\tt $<$$>$}

For each selected element, assigns the tween with the specified {\itshape name} with the specified {\itshape value} function. The {\itshape value} must be specified as a function that returns a function. When the transition starts, the {\itshape value} function is evaluated for each selected element, in order, being passed the current datum {\ttfamily d} and index {\ttfamily i}, with the {\ttfamily this} context as the current D\+OM element. The returned function is then invoked for each frame of the transition, in order, being passed the \href{#transition_ease}{\tt eased} time {\itshape t}, typically in the range \mbox{[}0, 1\mbox{]}. If the specified {\itshape value} is null, removes the previously-\/assigned tween of the specified {\itshape name}, if any.

For example, to interpolate the fill attribute to blue, like \href{#transition_attr}{\tt {\itshape transition}.attr}\+:


\begin{DoxyCode}
transition.tween("attr.fill", function() \{
  var node = this, i = d3.interpolateRgb(node.getAttribute("fill"), "blue");
  return function(t) \{
    node.setAttribute("fill", i(t));
  \};
\});
\end{DoxyCode}


This method is useful to specify a custom interpolator, or to perform side-\/effects, say to animate the \href{http://bl.ocks.org/mbostock/1649463}{\tt scroll offset}.

\subsubsection*{Timing}

The \href{#transition_ease}{\tt easing}, \href{#transition_delay}{\tt delay} and \href{#transition_duration}{\tt duration} of a transition is configurable. For example, a per-\/element delay can be used to \href{http://bl.ocks.org/mbostock/3885705}{\tt stagger the reordering} of elements, improving perception. See \href{http://vis.berkeley.edu/papers/animated_transitions/}{\tt Animated Transitions in Statistical Data Graphics} for recommendations.

\label{_transition_delay}%
\# {\itshape transition}.{\bfseries delay}(\mbox{[}{\itshape value}\mbox{]}) \href{https://github.com/d3/d3-transition/blob/master/src/transition/delay.js}{\tt $<$$>$}

For each selected element, sets the transition delay to the specified {\itshape value} in milliseconds. The {\itshape value} may be specified either as a constant or a function. If a function, it is immediately evaluated for each selected element, in order, being passed the current datum {\ttfamily d} and index {\ttfamily i}, with the {\ttfamily this} context as the current D\+OM element. The function’s return value is then used to set each element’s transition delay. If a delay is not specified, it defaults to zero.

If a {\itshape value} is not specified, returns the current value of the delay for the first (non-\/null) element in the transition. This is generally useful only if you know that the transition contains exactly one element.

Setting the delay to a multiple of the index {\ttfamily i} is a convenient way to stagger transitions across a set of elements. For example\+:


\begin{DoxyCode}
transition.delay(function(d, i) \{ return i * 10; \});
\end{DoxyCode}


Of course, you can also compute the delay as a function of the data, or \href{https://github.com/d3/d3-selection#selection_sort}{\tt sort the selection} before computed an index-\/based delay.

\label{_transition_duration}%
\# {\itshape transition}.{\bfseries duration}(\mbox{[}{\itshape value}\mbox{]}) \href{https://github.com/d3/d3-transition/blob/master/src/transition/duration.js}{\tt $<$$>$}

For each selected element, sets the transition duration to the specified {\itshape value} in milliseconds. The {\itshape value} may be specified either as a constant or a function. If a function, it is immediately evaluated for each selected element, in order, being passed the current datum {\ttfamily d} and index {\ttfamily i}, with the {\ttfamily this} context as the current D\+OM element. The function’s return value is then used to set each element’s transition duration. If a duration is not specified, it defaults to 250ms.

If a {\itshape value} is not specified, returns the current value of the duration for the first (non-\/null) element in the transition. This is generally useful only if you know that the transition contains exactly one element.

\label{_transition_ease}%
\# {\itshape transition}.{\bfseries ease}(\mbox{[}{\itshape value}\mbox{]}) \href{https://github.com/d3/d3-transition/blob/master/src/transition/ease.js}{\tt $<$$>$}

Specifies the transition \href{https://github.com/d3/d3-ease}{\tt easing function} for all selected elements. The {\itshape value} must be specified as a function. The easing function is invoked for each frame of the animation, being passed the normalized time {\itshape t} in the range \mbox{[}0, 1\mbox{]}; it must then return the eased time {\itshape tʹ} which is typically also in the range \mbox{[}0, 1\mbox{]}. A good easing function should return 0 if {\itshape t} = 0 and 1 if {\itshape t} = 1. If an easing function is not specified, it defaults to \href{https://github.com/d3/d3-ease#easeCubic}{\tt d3.\+ease\+Cubic}.

If a {\itshape value} is not specified, returns the current easing function for the first (non-\/null) element in the transition. This is generally useful only if you know that the transition contains exactly one element.

\subsubsection*{Control Flow}

For advanced usage, transitions provide methods for custom control flow.

\label{_transition_on}%
\# {\itshape transition}.{\bfseries on}({\itshape typenames}\mbox{[}, {\itshape listener}\mbox{]}) \href{https://github.com/d3/d3-transition/blob/master/src/transition/on.js}{\tt $<$$>$}

Adds or removes a {\itshape listener} to each selected element for the specified event {\itshape typenames}. The {\itshape typenames} is one of the following string event types\+:


\begin{DoxyItemize}
\item {\ttfamily start} -\/ when the transition starts.
\item {\ttfamily end} -\/ when the transition ends.
\item {\ttfamily interrupt} -\/ when the transition is interrupted.
\end{DoxyItemize}

See \href{#the-life-of-a-transition}{\tt The Life of a Transition} for more. Note that these are {\itshape not} native D\+OM events as implemented by \href{https://github.com/d3/d3-selection#selection_on}{\tt {\itshape selection}.on} and \href{https://github.com/d3/d3-selection#selection_dispatch}{\tt {\itshape selection}.dispatch}, but transition events!

The type may be optionally followed by a period ({\ttfamily .}) and a name; the optional name allows multiple callbacks to be registered to receive events of the same type, such as {\ttfamily start.\+foo} and {\ttfamily start.\+bar}. To specify multiple typenames, separate typenames with spaces, such as {\ttfamily interrupt end} or {\ttfamily start.\+foo start.\+bar}.

When a specified transition event is dispatched on a selected node, the specified {\itshape listener} will be invoked for the transitioning element, being passed the current datum {\ttfamily d} and index {\ttfamily i}, with the {\ttfamily this} context as the current D\+OM element. Listeners always see the latest datum for their element, but the index is a property of the selection and is fixed when the listener is assigned; to update the index, re-\/assign the listener.

If an event listener was previously registered for the same {\itshape typename} on a selected element, the old listener is removed before the new listener is added. To remove a listener, pass null as the {\itshape listener}. To remove all listeners for a given name, pass null as the {\itshape listener} and {\ttfamily .foo} as the {\itshape typename}, where {\ttfamily foo} is the name; to remove all listeners with no name, specify {\ttfamily .} as the {\itshape typename}.

If a {\itshape listener} is not specified, returns the currently-\/assigned listener for the specified event {\itshape typename} on the first (non-\/null) selected element, if any. If multiple typenames are specified, the first matching listener is returned.

\label{_transition_each}%
\# {\itshape transition}.{\bfseries each}({\itshape function}) \href{https://github.com/d3/d3-selection/blob/master/src/selection/each.js}{\tt $<$$>$}

Invokes the specified {\itshape function} for each selected element, passing in the current datum {\ttfamily d} and index {\ttfamily i}, with the {\ttfamily this} context of the current D\+OM element. This method can be used to invoke arbitrary code for each selected element, and is useful for creating a context to access parent and child data simultaneously. Equivalent to \href{https://github.com/d3/d3-selection#selection_each}{\tt {\itshape selection}.each}.

\label{_transition_call}%
\# {\itshape transition}.{\bfseries call}({\itshape function}\mbox{[}, {\itshape arguments…}\mbox{]}) \href{https://github.com/d3/d3-selection/blob/master/src/selection/call.js}{\tt $<$$>$}

Invokes the specified {\itshape function} exactly once, passing in this transition along with any optional {\itshape arguments}. Returns this transition. This is equivalent to invoking the function by hand but facilitates method chaining. For example, to set several attributes in a reusable function\+:


\begin{DoxyCode}
function color(transition, fill, stroke) \{
  transition
      .style("fill", fill)
      .style("stroke", stroke);
\}
\end{DoxyCode}


Now say\+:


\begin{DoxyCode}
d3.selectAll("div").transition().call(color, "red", "blue");
\end{DoxyCode}


This is equivalent to\+:


\begin{DoxyCode}
color(d3.selectAll("div").transition(), "red", "blue");
\end{DoxyCode}


Equivalent to \href{https://github.com/d3/d3-selection#selection_call}{\tt {\itshape selection}.call}.

\label{_transition_empty}%
\# {\itshape transition}.{\bfseries empty}() \href{https://github.com/d3/d3-selection/blob/master/src/selection/empty.js}{\tt $<$$>$}

Returns true if this transition contains no (non-\/null) elements. Equivalent to \href{https://github.com/d3/d3-selection#selection_empty}{\tt {\itshape selection}.empty}.

\label{_transition_nodes}%
\# {\itshape transition}.{\bfseries nodes}() \href{https://github.com/d3/d3-selection/blob/master/src/selection/nodes.js}{\tt $<$$>$}

Returns an array of all (non-\/null) elements in this transition. Equivalent to \href{https://github.com/d3/d3-selection#selection_nodes}{\tt {\itshape selection}.nodes}.

\label{_transition_node}%
\# {\itshape transition}.{\bfseries node}() \href{https://github.com/d3/d3-selection/blob/master/src/selection/node.js}{\tt $<$$>$}

Returns the first (non-\/null) element in this transition. If the transition is empty, returns null. Equivalent to \href{https://github.com/d3/d3-selection#selection_node}{\tt {\itshape selection}.node}.

\label{_transition_size}%
\# {\itshape transition}.{\bfseries size}() \href{https://github.com/d3/d3-selection/blob/master/src/selection/size.js}{\tt $<$$>$}

Returns the total number of elements in this transition. Equivalent to \href{https://github.com/d3/d3-selection#selection_size}{\tt {\itshape selection}.size}.

\subsubsection*{The Life of a Transition}

Immediately after creating a transition, such as by \href{#selection_transition}{\tt {\itshape selection}.transition} or \href{#transition_transition}{\tt {\itshape transition}.transition}, you may configure the transition using methods such as \href{#transition_delay}{\tt {\itshape transition}.delay}, \href{#transition_duration}{\tt {\itshape transition}.duration}, \href{#transition_attr}{\tt {\itshape transition}.attr} and \href{#transition_style}{\tt {\itshape transition}.style}. Methods that specify target values (such as {\itshape transition}.attr) are evaluated synchronously; however, methods that require the starting value for interpolation, such as \href{#transition_attrTween}{\tt {\itshape transition}.attr\+Tween} and \href{#transition_styleTween}{\tt {\itshape transition}.style\+Tween}, must be deferred until the transition starts.

Shortly after creation, either at the end of the current frame or during the next frame, the transition is scheduled. At this point, the delay and {\ttfamily start} event listeners may no longer be changed; attempting to do so throws an error with the message “too late\+: already scheduled” (or if the transition has ended, “transition not found”).

When the transition subsequently starts, it interrupts the active transition of the same name on the same element, if any, dispatching an {\ttfamily interrupt} event to registered listeners. (Note that interrupts happen on start, not creation, and thus even a zero-\/delay transition will not immediately interrupt the active transition\+: the old transition is given a final frame. Use \href{#selection_interrupt}{\tt {\itshape selection}.interrupt} to interrupt immediately.) The starting transition also cancels any pending transitions of the same name on the same element that were created before the starting transition. The transition then dispatches a {\ttfamily start} event to registered listeners. This is the last moment at which the transition may be modified\+: after starting, the transition’s timing, tweens, and listeners may no longer be changed; attempting to do so throws an error with the message “too late\+: already started” (or if the transition has ended, “transition not found”). The transition initializes its tweens immediately after starting.

During the frame the transition starts, but {\itshape after} all transitions starting this frame have been started, the transition invokes its tweens for the first time. Batching tween initialization, which typically involves reading from the D\+OM, improves performance by avoiding interleaved D\+OM reads and writes.

For each frame that a transition is active, it invokes its tweens with an \href{#transition_ease}{\tt eased} {\itshape t}-\/value ranging from 0 to 1. Within each frame, the transition invokes its tweens in the order they were registered.

When a transition ends, it invokes its tweens a final time with a (non-\/eased) {\itshape t}-\/value of 1. It then dispatches an {\ttfamily end} event to registered listeners. This is the last moment at which the transition may be inspected\+: after ending, the transition is deleted from the element, and its configuration is destroyed. (A transition’s configuration is also destroyed on interrupt or cancel.) Attempting to inspect a transition after it is destroyed throws an error with the message “transition not found”. 