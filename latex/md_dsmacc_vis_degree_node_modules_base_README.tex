\href{https://github.com/node-base/base}{\tt } 

\section*{base \href{https://www.npmjs.com/package/base}{\tt } \href{https://npmjs.org/package/base}{\tt } \href{https://npmjs.org/package/base}{\tt } \href{https://travis-ci.org/node-base/base}{\tt }}

\begin{quote}
base is the foundation for creating modular, unit testable and highly pluggable node.\+js applications, starting with a handful of common methods, like {\ttfamily set}, {\ttfamily get}, {\ttfamily del} and {\ttfamily use}. \end{quote}


\subsection*{Install}

Install with \href{https://www.npmjs.com/}{\tt npm}\+:


\begin{DoxyCode}
$ npm install --save base
\end{DoxyCode}


\subsection*{What is Base?}

Base is a framework for rapidly creating high quality node.\+js applications, using plugins like building blocks.

\subsubsection*{Guiding principles}

The core team follows these principles to help guide A\+PI decisions\+:


\begin{DoxyItemize}
\item {\bfseries Compact A\+PI surface}\+: The smaller the A\+PI surface, the easier the library will be to learn and use.
\item {\bfseries Easy to extend}\+: Implementors can use any npm package, and write plugins in pure Java\+Script. If you\textquotesingle{}re building complex apps, Base simplifies inheritance.
\item {\bfseries Easy to test}\+: No special setup should be required to unit test {\ttfamily Base} or base plugins
\end{DoxyItemize}

\subsubsection*{Minimal A\+PI surface}

\href{#api}{\tt The A\+PI} was designed to provide only the minimum necessary functionality for creating a useful application, with or without \href{#plugins}{\tt plugins}.

{\bfseries Base core}

Base itself ships with only a handful of \href{#api}{\tt useful methods}, such as\+:


\begin{DoxyItemize}
\item {\ttfamily .set}\+: for setting values on the instance
\item {\ttfamily .get}\+: for getting values from the instance
\item {\ttfamily .has}\+: to check if a property exists on the instance
\item {\ttfamily .define}\+: for setting non-\/enumerable values on the instance
\item {\ttfamily .use}\+: for adding plugins
\end{DoxyItemize}

{\bfseries Be generic}

When deciding on method to add or remove, we try to answer these questions\+:


\begin{DoxyEnumerate}
\item Will all or most Base applications need this method?
\item Will this method encourage practices or enforce conventions that are beneficial to implementors?
\item Can or should this be done in a plugin instead?
\end{DoxyEnumerate}

\subsubsection*{Composability}

{\bfseries Plugin system}

It couldn\textquotesingle{}t be easier to extend Base with any features or custom functionality you can think of.

Base plugins are just functions that take an instance of {\ttfamily Base}\+:


\begin{DoxyCode}
var base = new Base();

function plugin(base) \{
  // do plugin stuff, in pure JavaScript
\}
// use the plugin
base.use(plugin);
\end{DoxyCode}


{\bfseries Inheritance}

Easily inherit Base using {\ttfamily .extend}\+:


\begin{DoxyCode}
var Base = require('base');

function MyApp() \{
  Base.call(this);
\}
Base.extend(MyApp);

var app = new MyApp();
app.set('a', 'b');
app.get('a');
//=> 'b';
\end{DoxyCode}


{\bfseries Inherit or instantiate with a namespace}

By default, the {\ttfamily .get}, {\ttfamily .set} and {\ttfamily .has} methods set and get values from the root of the {\ttfamily base} instance. You can customize this using the {\ttfamily .namespace} method exposed on the exported function. For example\+:


\begin{DoxyCode}
var Base = require('base');
// get and set values on the `base.cache` object
var base = Base.namespace('cache');

var app = base();
app.set('foo', 'bar');
console.log(app.cache.foo);
//=> 'bar'
\end{DoxyCode}


\subsection*{A\+PI}

{\bfseries Usage}


\begin{DoxyCode}
var Base = require('base');
var app = new Base();
app.set('foo', 'bar');
console.log(app.foo);
//=> 'bar'
\end{DoxyCode}


\subsubsection*{\href{index.js#L44}{\tt Base}}

Create an instance of {\ttfamily Base} with the given {\ttfamily config} and {\ttfamily options}.

{\bfseries Params}


\begin{DoxyItemize}
\item {\ttfamily config} $\ast$$\ast$\{Object\}$\ast$$\ast$\+: If supplied, this object is passed to \href{https://github.com/jonschlinkert/cache-base}{\tt cache-\/base} to merge onto the the instance upon instantiation.
\item {\ttfamily options} $\ast$$\ast$\{Object\}$\ast$$\ast$\+: If supplied, this object is used to initialize the {\ttfamily base.\+options} object.
\end{DoxyItemize}

{\bfseries Example}

\`{}\`{}{\ttfamily js // initialize with}config{\ttfamily and}options\`{} var app = new Base(\{is\+App\+: true\}, \{abc\+: true\}); app.\+set(\textquotesingle{}foo\textquotesingle{}, \textquotesingle{}bar\textquotesingle{});

// values defined with the given {\ttfamily config} object will be on the root of the instance console.\+log(app.\+baz); //=$>$ undefined console.\+log(app.\+foo); //=$>$ \textquotesingle{}bar\textquotesingle{} // or use {\ttfamily .get} console.\+log(app.\+get(\textquotesingle{}is\+App\textquotesingle{})); //=$>$ true console.\+log(app.\+get(\textquotesingle{}foo\textquotesingle{})); //=$>$ \textquotesingle{}bar\textquotesingle{}

// values defined with the given {\ttfamily options} object will be on \`{}app.options console.\+log(app.\+options.\+abc); //=$>$ true 
\begin{DoxyCode}
### [.is](index.js#L107)

Set the given `name` on `app.\_name` and `app.is*` properties. Used for doing lookups in plugins.

**Params**

* `name` **\{String\}**
* `returns` **\{Boolean\}**

**Example**

```js
app.is('foo');
console.log(app.\_name);
//=> 'foo'
console.log(app.isFoo);
//=> true
app.is('bar');
console.log(app.isFoo);
//=> true
console.log(app.isBar);
//=> true
console.log(app.\_name);
//=> 'bar'
\end{DoxyCode}


\subsubsection*{\href{index.js#L145}{\tt .is\+Registered}}

Returns true if a plugin has already been registered on an instance.

Plugin implementors are encouraged to use this first thing in a plugin to prevent the plugin from being called more than once on the same instance.

{\bfseries Params}


\begin{DoxyItemize}
\item {\ttfamily name} $\ast$$\ast$\{String\}$\ast$$\ast$\+: The plugin name.
\item {\ttfamily register} $\ast$$\ast$\{Boolean\}$\ast$$\ast$\+: If the plugin if not already registered, to record it as being registered pass {\ttfamily true} as the second argument.
\item {\ttfamily returns} $\ast$$\ast$\{Boolean\}$\ast$$\ast$\+: Returns true if a plugin is already registered.
\end{DoxyItemize}

{\bfseries Events}


\begin{DoxyItemize}
\item {\ttfamily emits}\+: {\ttfamily plugin} Emits the name of the plugin being registered. Useful for unit tests, to ensure plugins are only registered once.
\end{DoxyItemize}

{\bfseries Example}

\`{}\`{}\`{}js var base = new Base(); base.\+use(function(app) \{ if (app.\+is\+Registered(\textquotesingle{}my\+Plugin\textquotesingle{})) return; // do stuff to {\ttfamily app} \});

// to also record the plugin as being registered base.\+use(function(app) \{ if (app.\+is\+Registered(\textquotesingle{}my\+Plugin\textquotesingle{}, true)) return; // do stuff to {\ttfamily app} \}); \`{}\`{}\`{}

\subsubsection*{\href{index.js#L175}{\tt .use}}

Define a plugin function to be called immediately upon init. Plugins are chainable and expose the following arguments to the plugin function\+:


\begin{DoxyItemize}
\item {\ttfamily app}\+: the current instance of {\ttfamily Base}
\item {\ttfamily base}\+: the \href{#base}{\tt first ancestor instance} of {\ttfamily Base}
\end{DoxyItemize}

{\bfseries Params}


\begin{DoxyItemize}
\item {\ttfamily fn} $\ast$$\ast$\{Function\}$\ast$$\ast$\+: plugin function to call
\item {\ttfamily returns} $\ast$$\ast$\{Object\}$\ast$$\ast$\+: Returns the item instance for chaining.
\end{DoxyItemize}

{\bfseries Example}


\begin{DoxyCode}
var app = new Base()
  .use(foo)
  .use(bar)
  .use(baz)
\end{DoxyCode}


\subsubsection*{\href{index.js#L197}{\tt .define}}

The {\ttfamily .define} method is used for adding non-\/enumerable property on the instance. Dot-\/notation is {\bfseries not supported} with {\ttfamily define}.

{\bfseries Params}


\begin{DoxyItemize}
\item {\ttfamily key} $\ast$$\ast$\{String\}$\ast$$\ast$\+: The name of the property to define.
\item {\ttfamily value} $\ast$$\ast$\{any\}$\ast$$\ast$
\item {\ttfamily returns} $\ast$$\ast$\{Object\}$\ast$$\ast$\+: Returns the instance for chaining.
\end{DoxyItemize}

{\bfseries Example}

\`{}\`{}{\ttfamily js // arbitrary}render{\ttfamily function using lodash}template` app.\+define(\textquotesingle{}render', function(str, locals) \{ return \+\_\+.\+template(str)(locals); \}); 
\begin{DoxyCode}
### [.mixin](index.js#L222)

Mix property `key` onto the Base prototype. If base is inherited using `Base.extend` this method will be
       overridden by a new `mixin` method that will only add properties to the prototype of the inheriting
       application.

**Params**

* `key` **\{String\}**
* `val` **\{Object|Array\}**
* `returns` **\{Object\}**: Returns the `base` instance for chaining.

**Example**

```js
app.mixin('foo', function() \{
  // do stuff
\});
\end{DoxyCode}


\subsubsection*{\href{index.js#L268}{\tt .base}}

Getter/setter used when creating nested instances of {\ttfamily Base}, for storing a reference to the first ancestor instance. This works by setting an instance of {\ttfamily Base} on the {\ttfamily parent} property of a \char`\"{}child\char`\"{} instance. The {\ttfamily base} property defaults to the current instance if no {\ttfamily parent} property is defined.

{\bfseries Example}


\begin{DoxyCode}
// create an instance of `Base`, this is our first ("base") instance
var first = new Base();
first.foo = 'bar'; // arbitrary property, to make it easier to see what's happening later

// create another instance
var second = new Base();
// create a reference to the first instance (`first`)
second.parent = first;

// create another instance
var third = new Base();
// create a reference to the previous instance (`second`)
// repeat this pattern every time a "child" instance is created
third.parent = second;

// we can always access the first instance using the `base` property
console.log(first.base.foo);
//=> 'bar'
console.log(second.base.foo);
//=> 'bar'
console.log(third.base.foo);
//=> 'bar'
// and now you know how to get to third base ;)
\end{DoxyCode}


\subsubsection*{\href{index.js#L293}{\tt \#use}}

Static method for adding global plugin functions that will be added to an instance when created.

{\bfseries Params}


\begin{DoxyItemize}
\item {\ttfamily fn} $\ast$$\ast$\{Function\}$\ast$$\ast$\+: Plugin function to use on each instance.
\item {\ttfamily returns} $\ast$$\ast$\{Object\}$\ast$$\ast$\+: Returns the {\ttfamily Base} constructor for chaining
\end{DoxyItemize}

{\bfseries Example}


\begin{DoxyCode}
Base.use(function(app) \{
  app.foo = 'bar';
\});
var app = new Base();
console.log(app.foo);
//=> 'bar'
\end{DoxyCode}


\subsubsection*{\href{index.js#L337}{\tt \#extend}}

Static method for inheriting the prototype and static methods of the {\ttfamily Base} class. This method greatly simplifies the process of creating inheritance-\/based applications. See \href{https://github.com/jonschlinkert/static-extend}{\tt static-\/extend} for more details.

{\bfseries Params}


\begin{DoxyItemize}
\item {\ttfamily Ctor} $\ast$$\ast$\{Function\}$\ast$$\ast$\+: constructor to extend
\item {\ttfamily methods} $\ast$$\ast$\{Object\}$\ast$$\ast$\+: Optional prototype properties to mix in.
\item {\ttfamily returns} $\ast$$\ast$\{Object\}$\ast$$\ast$\+: Returns the {\ttfamily Base} constructor for chaining
\end{DoxyItemize}

{\bfseries Example}


\begin{DoxyCode}
var extend = cu.extend(Parent);
Parent.extend(Child);

// optional methods
Parent.extend(Child, \{
  foo: function() \{\},
  bar: function() \{\}
\});
\end{DoxyCode}


\subsubsection*{\href{index.js#L379}{\tt \#mixin}}

Used for adding methods to the {\ttfamily Base} prototype, and/or to the prototype of child instances. When a mixin function returns a function, the returned function is pushed onto the {\ttfamily .mixins} array, making it available to be used on inheriting classes whenever {\ttfamily Base.\+mixins()} is called (e.\+g. {\ttfamily Base.\+mixins(\+Child)}).

{\bfseries Params}


\begin{DoxyItemize}
\item {\ttfamily fn} $\ast$$\ast$\{Function\}$\ast$$\ast$\+: Function to call
\item {\ttfamily returns} $\ast$$\ast$\{Object\}$\ast$$\ast$\+: Returns the {\ttfamily Base} constructor for chaining
\end{DoxyItemize}

{\bfseries Example}


\begin{DoxyCode}
Base.mixin(function(proto) \{
  proto.foo = function(msg) \{
    return 'foo ' + msg;
  \};
\});
\end{DoxyCode}


\subsubsection*{\href{index.js#L401}{\tt \#mixins}}

Static method for running global mixin functions against a child constructor. Mixins must be registered before calling this method.

{\bfseries Params}


\begin{DoxyItemize}
\item {\ttfamily Child} $\ast$$\ast$\{Function\}$\ast$$\ast$\+: Constructor function of a child class
\item {\ttfamily returns} $\ast$$\ast$\{Object\}$\ast$$\ast$\+: Returns the {\ttfamily Base} constructor for chaining
\end{DoxyItemize}

{\bfseries Example}


\begin{DoxyCode}
Base.extend(Child);
Base.mixins(Child);
\end{DoxyCode}


\subsubsection*{\href{index.js#L420}{\tt \#inherit}}

Similar to {\ttfamily util.\+inherit}, but copies all static properties, prototype properties, and getters/setters from {\ttfamily Provider} to {\ttfamily Receiver}. See \href{https://github.com/jonschlinkert/class-utils#inherit}{\tt class-\/utils} for more details.

{\bfseries Params}


\begin{DoxyItemize}
\item {\ttfamily Receiver} $\ast$$\ast$\{Function\}$\ast$$\ast$\+: Receiving (child) constructor
\item {\ttfamily Provider} $\ast$$\ast$\{Function\}$\ast$$\ast$\+: Providing (parent) constructor
\item {\ttfamily returns} $\ast$$\ast$\{Object\}$\ast$$\ast$\+: Returns the {\ttfamily Base} constructor for chaining
\end{DoxyItemize}

{\bfseries Example}


\begin{DoxyCode}
Base.inherit(Foo, Bar);
\end{DoxyCode}


\subsection*{In the wild}

The following node.\+js applications were built with {\ttfamily Base}\+:


\begin{DoxyItemize}
\item \href{https://github.com/assemble/assemble}{\tt assemble}
\item \href{https://github.com/verbose/verb}{\tt verb}
\item \href{https://github.com/generate/generate}{\tt generate}
\item \href{https://github.com/jonschlinkert/scaffold}{\tt scaffold}
\item \href{https://github.com/jonschlinkert/boilerplate}{\tt boilerplate}
\end{DoxyItemize}

\subsection*{Test coverage}


\begin{DoxyCode}
Statements   : 98.91% ( 91/92 )
Branches     : 92.86% ( 26/28 )
Functions    : 100% ( 17/17 )
Lines        : 98.9% ( 90/91 )
\end{DoxyCode}


\subsection*{History}

\subsubsection*{v0.\+11.\+2}


\begin{DoxyItemize}
\item fixes \href{https://github.com/micromatch/micromatch/issues/99}{\tt https\+://github.\+com/micromatch/micromatch/issues/99}
\end{DoxyItemize}

\subsubsection*{v0.\+11.\+0}

{\bfseries Breaking changes}


\begin{DoxyItemize}
\item Static {\ttfamily .use} and {\ttfamily .run} methods are now non-\/enumerable
\end{DoxyItemize}

\subsubsection*{v0.\+9.\+0}

{\bfseries Breaking changes}


\begin{DoxyItemize}
\item {\ttfamily .is} no longer takes a function, a string must be passed
\item all remaining {\ttfamily .debug} code has been removed
\item {\ttfamily app.\+\_\+namespace} was removed (related to {\ttfamily debug})
\item {\ttfamily .plugin}, {\ttfamily .use}, and {\ttfamily .define} no longer emit events
\item {\ttfamily .assert\+Plugin} was removed
\item {\ttfamily .lazy} was removed
\end{DoxyItemize}

\subsection*{About}

\subsubsection*{Related projects}


\begin{DoxyItemize}
\item \href{https://www.npmjs.com/package/base-cwd}{\tt base-\/cwd}\+: Base plugin that adds a getter/setter for the current working directory. $\vert$ \href{https://github.com/node-base/base-cwd}{\tt homepage}
\item \href{https://www.npmjs.com/package/base-data}{\tt base-\/data}\+: adds a {\ttfamily data} method to base-\/methods. $\vert$ \href{https://github.com/node-base/base-data}{\tt homepage}
\item \href{https://www.npmjs.com/package/base-fs}{\tt base-\/fs}\+: base-\/methods plugin that adds vinyl-\/fs methods to your \textquotesingle{}base\textquotesingle{} application for working with the file… \href{https://github.com/node-base/base-fs}{\tt more} $\vert$ \href{https://github.com/node-base/base-fs}{\tt homepage}
\item \href{https://www.npmjs.com/package/base-generators}{\tt base-\/generators}\+: Adds project-\/generator support to your {\ttfamily base} application. $\vert$ \href{https://github.com/node-base/base-generators}{\tt homepage}
\item \href{https://www.npmjs.com/package/base-option}{\tt base-\/option}\+: Adds a few options methods to base, like {\ttfamily option}, {\ttfamily enable} and {\ttfamily disable}. See the readme… \href{https://github.com/node-base/base-option}{\tt more} $\vert$ \href{https://github.com/node-base/base-option}{\tt homepage}
\item \href{https://www.npmjs.com/package/base-pipeline}{\tt base-\/pipeline}\+: base-\/methods plugin that adds pipeline and plugin methods for dynamically composing streaming plugin pipelines. $\vert$ \href{https://github.com/node-base/base-pipeline}{\tt homepage}
\item \href{https://www.npmjs.com/package/base-pkg}{\tt base-\/pkg}\+: Plugin for adding a {\ttfamily pkg} method that exposes pkg-\/store to your base application. $\vert$ \href{https://github.com/node-base/base-pkg}{\tt homepage}
\item \href{https://www.npmjs.com/package/base-plugins}{\tt base-\/plugins}\+: Adds \textquotesingle{}smart plugin\textquotesingle{} support to your base application. $\vert$ \href{https://github.com/node-base/base-plugins}{\tt homepage}
\item \href{https://www.npmjs.com/package/base-questions}{\tt base-\/questions}\+: Plugin for base-\/methods that adds methods for prompting the user and storing the answers on… \href{https://github.com/node-base/base-questions}{\tt more} $\vert$ \href{https://github.com/node-base/base-questions}{\tt homepage}
\item \href{https://www.npmjs.com/package/base-store}{\tt base-\/store}\+: Plugin for getting and persisting config values with your base-\/methods application. Adds a \textquotesingle{}store\textquotesingle{} object… \href{https://github.com/node-base/base-store}{\tt more} $\vert$ \href{https://github.com/node-base/base-store}{\tt homepage}
\item \href{https://www.npmjs.com/package/base-task}{\tt base-\/task}\+: base plugin that provides a very thin wrapper around \href{https://github.com/doowb/composer}{\tt https\+://github.\+com/doowb/composer} for adding task methods to… \href{https://github.com/node-base/base-task}{\tt more} $\vert$ \href{https://github.com/node-base/base-task}{\tt homepage}
\end{DoxyItemize}

\subsubsection*{Contributing}

Pull requests and stars are always welcome. For bugs and feature requests, \href{../../issues/new}{\tt please create an issue}.

\subsubsection*{Contributors}

\tabulinesep=1mm
\begin{longtabu} spread 0pt [c]{*{2}{|X[-1]}|}
\hline
\rowcolor{\tableheadbgcolor}\multicolumn{2}{|p{(\linewidth-\tabcolsep*2-\arrayrulewidth*1)*2/2}|}{\cellcolor{\tableheadbgcolor}\textbf{ $\ast$$\ast$\+Commits$\ast$   }}\\\cline{1-2}
\endfirsthead
\hline
\endfoot
\hline
\rowcolor{\tableheadbgcolor}\multicolumn{2}{|p{(\linewidth-\tabcolsep*2-\arrayrulewidth*1)*2/2}|}{\cellcolor{\tableheadbgcolor}\textbf{ $\ast$$\ast$\+Commits$\ast$   }}\\\cline{1-2}
\endhead
141  &\href{https://github.com/jonschlinkert}{\tt jonschlinkert}   \\\cline{1-2}
30  &\href{https://github.com/doowb}{\tt doowb}   \\\cline{1-2}
3  &\href{https://github.com/charlike}{\tt charlike}   \\\cline{1-2}
1  &\href{https://github.com/criticalmash}{\tt criticalmash}   \\\cline{1-2}
1  &\href{https://github.com/wtgtybhertgeghgtwtg}{\tt wtgtybhertgeghgtwtg}   \\\cline{1-2}
\end{longtabu}


\subsubsection*{Building docs}

\+\_\+(This project\textquotesingle{}s readme.\+md is generated by \href{https://github.com/verbose/verb-generate-readme}{\tt verb}, please don\textquotesingle{}t edit the readme directly. Any changes to the readme must be made in the .verb.\+md \char`\"{}.\+verb.\+md\char`\"{} readme template.)\+\_\+

To generate the readme, run the following command\+:


\begin{DoxyCode}
$ npm install -g verbose/verb#dev verb-generate-readme && verb
\end{DoxyCode}


\subsubsection*{Running tests}

Running and reviewing unit tests is a great way to get familiarized with a library and its A\+PI. You can install dependencies and run tests with the following command\+:


\begin{DoxyCode}
$ npm install && npm test
\end{DoxyCode}


\subsubsection*{Author}

{\bfseries Jon Schlinkert}


\begin{DoxyItemize}
\item \href{https://github.com/jonschlinkert}{\tt github/jonschlinkert}
\item \href{https://twitter.com/jonschlinkert}{\tt twitter/jonschlinkert}
\end{DoxyItemize}

\subsubsection*{License}

Copyright © 2017, \href{https://github.com/jonschlinkert}{\tt Jon Schlinkert}. Released under the \mbox{[}M\+IT License\mbox{]}(L\+I\+C\+E\+N\+SE).





{\itshape This file was generated by \href{https://github.com/verbose/verb-generate-readme}{\tt verb-\/generate-\/readme}, v0.\+6.\+0, on September 07, 2017.} 