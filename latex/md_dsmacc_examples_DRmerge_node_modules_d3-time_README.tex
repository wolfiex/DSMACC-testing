When visualizing time series data, analyzing temporal patterns, or working with time in general, the irregularities of conventional time units quickly become apparent. In the \href{https://en.wikipedia.org/wiki/Gregorian_calendar}{\tt Gregorian calendar}, for example, most months have 31 days but some have 28, 29 or 30; most years have 365 days but \href{https://en.wikipedia.org/wiki/Leap_year}{\tt leap years} have 366; and with \href{https://en.wikipedia.org/wiki/Daylight_saving_time}{\tt daylight saving}, most days have 24 hours but some have 23 or 25. Adding to complexity, daylight saving conventions vary around the world.

As a result of these temporal peculiarities, it can be difficult to perform seemingly-\/trivial tasks. For example, if you want to compute the number of days that have passed between two dates, you can’t simply subtract and divide by 24 hours (86,400,000 ms)\+:


\begin{DoxyCode}
var start = new Date(2015, 02, 01), // Sun Mar 01 2015 00:00:00 GMT-0800 (PST)
    end = new Date(2015, 03, 01); // Wed Apr 01 2015 00:00:00 GMT-0700 (PDT)
(end - start) / 864e5; // 30.958333333333332, oops!
\end{DoxyCode}


You can, however, use \href{#timeDay}{\tt d3.\+time\+Day}.\href{#interval_count}{\tt count}\+:


\begin{DoxyCode}
d3.timeDay.count(start, end); // 31
\end{DoxyCode}


The \href{#day}{\tt day} \href{#api-reference}{\tt interval} is one of several provided by d3-\/time. Each interval represents a conventional unit of time—\href{#timeHour}{\tt hours}, \href{#timeWeek}{\tt weeks}, \href{#timeMonth}{\tt months}, $\ast$etc.$\ast$—and has methods to calculate boundary dates. For example, \href{#timeDay}{\tt d3.\+time\+Day} computes midnight (typically 12\+:00 AM local time) of the corresponding day. In addition to \href{#interval_round}{\tt rounding} and \href{#interval_count}{\tt counting}, intervals can also be used to generate arrays of boundary dates. For example, to compute each Sunday in the current month\+:


\begin{DoxyCode}
var now = new Date;
d3.timeWeek.range(d3.timeMonth.floor(now), d3.timeMonth.ceil(now));
// [Sun Jun 07 2015 00:00:00 GMT-0700 (PDT),
//  Sun Jun 14 2015 00:00:00 GMT-0700 (PDT),
//  Sun Jun 21 2015 00:00:00 GMT-0700 (PDT),
//  Sun Jun 28 2015 00:00:00 GMT-0700 (PDT)]
\end{DoxyCode}


The d3-\/time module does not implement its own calendaring system; it merely implements a convenient A\+PI for calendar math on top of E\+C\+M\+A\+Script \href{https://developer.mozilla.org/en-US/docs/Web/JavaScript/Reference/Global_Objects/Date}{\tt Date}. Thus, it ignores leap seconds and can only work with the local time zone and \href{https://en.wikipedia.org/wiki/Coordinated_Universal_Time}{\tt Coordinated Universal Time} (U\+TC).

This module is used by D3’s time scales to generate sensible ticks, by D3’s time format, and can also be used directly to do things like \href{http://bl.ocks.org/mbostock/4063318}{\tt calendar layouts}.

\subsection*{Installing}

If you use N\+PM, {\ttfamily npm install d3-\/time}. Otherwise, download the \href{https://github.com/d3/d3-time/releases/latest}{\tt latest release}. You can also load directly from \href{https://d3js.org}{\tt d3js.\+org}, either as a \href{https://d3js.org/d3-time.v1.min.js}{\tt standalone library} or as part of \href{https://github.com/d3/d3}{\tt D3 4.\+0}. A\+MD, Common\+JS, and vanilla environments are supported. In vanilla, a {\ttfamily d3} global is exported\+:


\begin{DoxyCode}
<script src="https://d3js.org/d3-time.v1.min.js"></script>
<script>

var day = d3.timeDay(new Date);

</script>
\end{DoxyCode}


\href{https://tonicdev.com/npm/d3-time}{\tt Try d3-\/time in your browser.}

\subsection*{A\+PI Reference}

\label{__interval}%
\# {\itshape interval}({\itshape date}) \href{https://github.com/d3/d3-time/blob/master/src/interval.js#L6}{\tt $<$$>$}

Alias for \href{#interval_floor}{\tt {\itshape interval}.floor}. For example, \href{#timeYear}{\tt d3.\+time\+Year}({\itshape date}) and d3.\+time\+Year.\+floor({\itshape date}) are equivalent.

\label{_interval_floor}%
\# {\itshape interval}.{\bfseries floor}({\itshape date}) \href{https://github.com/d3/d3-time/blob/master/src/interval.js#L10}{\tt $<$$>$}

Returns a new date representing the latest interval boundary date before or equal to {\itshape date}. For example, \href{#timeDay}{\tt d3.\+time\+Day}.floor({\itshape date}) typically returns 12\+:00 AM local time on the given {\itshape date}.

This method is idempotent\+: if the specified {\itshape date} is already floored to the current interval, a new date with an identical time is returned. Furthermore, the returned date is the minimum expressible value of the associated interval, such that {\itshape interval}.floor({\itshape interval}.floor({\itshape date}) -\/ 1) returns the preceeding interval boundary date.

Note that the {\ttfamily ==} and {\ttfamily ===} operators do not compare by value with \href{https://developer.mozilla.org/en-US/docs/Web/JavaScript/Reference/Global_Objects/Date}{\tt Date} objects, and thus you cannot use them to tell whether the specified {\itshape date} has already been floored. Instead, coerce to a number and then compare\+:


\begin{DoxyCode}
// Returns true if the specified date is a day boundary.
function isDay(date) \{
  return +d3.timeDay.floor(date) === +date;
\}
\end{DoxyCode}


This is more reliable than testing whether the time is 12\+:00 AM, as in some time zones midnight may not exist due to daylight saving.

\label{_interval_round}%
\# {\itshape interval}.{\bfseries round}({\itshape date}) \href{https://github.com/d3/d3-time/blob/master/src/interval.js#L16}{\tt $<$$>$}

Returns a new date representing the closest interval boundary date to {\itshape date}. For example, \href{#timeDay}{\tt d3.\+time\+Day}.round({\itshape date}) typically returns 12\+:00 AM local time on the given {\itshape date} if it is on or before noon, and 12\+:00 AM of the following day if it is after noon.

This method is idempotent\+: if the specified {\itshape date} is already rounded to the current interval, a new date with an identical time is returned.

\label{_interval_ceil}%
\# {\itshape interval}.{\bfseries ceil}({\itshape date}) \href{https://github.com/d3/d3-time/blob/master/src/interval.js#L12}{\tt $<$$>$}

Returns a new date representing the earliest interval boundary date after or equal to {\itshape date}. For example, \href{#timeDay}{\tt d3.\+time\+Day}.ceil({\itshape date}) typically returns 12\+:00 AM local time on the date following the given {\itshape date}.

This method is idempotent\+: if the specified {\itshape date} is already ceilinged to the current interval, a new date with an identical time is returned. Furthermore, the returned date is the maximum expressible value of the associated interval, such that {\itshape interval}.ceil({\itshape interval}.ceil({\itshape date}) + 1) returns the following interval boundary date.

\label{_interval_offset}%
\# {\itshape interval}.{\bfseries offset}({\itshape date}\mbox{[}, {\itshape step}\mbox{]}) \href{https://github.com/d3/d3-time/blob/master/src/interval.js#L22}{\tt $<$$>$}

Returns a new date equal to {\itshape date} plus {\itshape step} intervals. If {\itshape step} is not specified it defaults to 1. If {\itshape step} is negative, then the returned date will be before the specified {\itshape date}; if {\itshape step} is zero, then a copy of the specified {\itshape date} is returned; if {\itshape step} is not an integer, it is \href{https://developer.mozilla.org/en-US/docs/Web/JavaScript/Reference/Global_Objects/Math/floor}{\tt floored}. This method does not round the specified {\itshape date} to the interval. For example, if {\itshape date} is today at 5\+:34 PM, then \href{#timeDay}{\tt d3.\+time\+Day}.offset({\itshape date}, 1) returns 5\+:34 PM tomorrow (even if daylight saving changes!).

\label{_interval_range}%
\# {\itshape interval}.{\bfseries range}({\itshape start}, {\itshape stop}\mbox{[}, {\itshape step}\mbox{]}) \href{https://github.com/d3/d3-time/blob/master/src/interval.js#L26}{\tt $<$$>$}

Returns every an array of dates representing every interval boundary after or equal to {\itshape start} (inclusive) and before {\itshape stop} (exclusive). If {\itshape step} is specified, then every {\itshape step$\ast$th boundary will be returned; for example, for the \href{#timeDay}{\tt d3.\+time\+Day} interval a $\ast$step} of 2 will return every other day. If {\itshape step} is not an integer, it is \href{https://developer.mozilla.org/en-US/docs/Web/JavaScript/Reference/Global_Objects/Math/floor}{\tt floored}.

The first date in the returned array is the earliest boundary after or equal to {\itshape start}; subsequent dates are \href{#interval_offset}{\tt offset} by {\itshape step} intervals and \href{#interval_floor}{\tt floored}. Thus, two overlapping ranges may be consistent. For example, this range contains odd days\+:


\begin{DoxyCode}
d3.timeDay.range(new Date(2015, 0, 1), new Date(2015, 0, 7), 2);
// [Thu Jan 01 2015 00:00:00 GMT-0800 (PST),
//  Sat Jan 03 2015 00:00:00 GMT-0800 (PST),
//  Mon Jan 05 2015 00:00:00 GMT-0800 (PST)]
\end{DoxyCode}


While this contains even days\+:


\begin{DoxyCode}
d3.timeDay.range(new Date(2015, 0, 2), new Date(2015, 0, 8), 2);
// [Fri Jan 02 2015 00:00:00 GMT-0800 (PST),
//  Sun Jan 04 2015 00:00:00 GMT-0800 (PST),
//  Tue Jan 06 2015 00:00:00 GMT-0800 (PST)]
\end{DoxyCode}


To make ranges consistent when a {\itshape step} is specified, use \href{#interval_every}{\tt {\itshape interval}.every} instead.

\label{_interval_filter}%
\# {\itshape interval}.{\bfseries filter}({\itshape test}) \href{https://github.com/d3/d3-time/blob/master/src/interval.js#L35}{\tt $<$$>$}

Returns a new interval that is a filtered subset of this interval using the specified {\itshape test} function. The {\itshape test} function is passed a date and should return true if and only if the specified date should be considered part of the interval. For example, to create an interval that returns the 1st, 11th, 21th and 31th (if it exists) of each month\+:


\begin{DoxyCode}
var i = d3.timeDay.filter(function(d) \{ return (d.getDate() - 1) % 10 === 0; \});
\end{DoxyCode}


The returned filtered interval does not support \href{#interval_count}{\tt {\itshape interval}.count}. See also \href{#interval_every}{\tt {\itshape interval}.every}.

\label{_interval_every}%
\# {\itshape interval}.{\bfseries every}({\itshape step}) \href{https://github.com/d3/d3-time/blob/master/src/interval.js#L50}{\tt $<$$>$}

Returns a \href{#interval_filter}{\tt filtered} view of this interval representing every {\itshape step$\ast$th date. The meaning of $\ast$step} is dependent on this interval’s parent interval as defined by the field function. For example, \href{#timeMinute}{\tt d3.\+time\+Minute}.every(15) returns an interval representing every fifteen minutes, starting on the hour\+: \+:00, \+:15, \+:30, \+:45, {\itshape etc.} Note that for some intervals, the resulting dates may not be uniformly-\/spaced; \href{#timeDay}{\tt d3.\+time\+Day}’s parent interval is \href{#timeMonth}{\tt d3.\+time\+Month}, and thus the interval number resets at the start of each month. If {\itshape step} is not valid, returns null. If {\itshape step} is one, returns this interval.

This method can be used in conjunction with \href{#interval_range}{\tt {\itshape interval}.range} to ensure that two overlapping ranges are consistent. For example, this range contains odd days\+:


\begin{DoxyCode}
d3.timeDay.every(2).range(new Date(2015, 0, 1), new Date(2015, 0, 7));
// [Thu Jan 01 2015 00:00:00 GMT-0800 (PST),
//  Sat Jan 03 2015 00:00:00 GMT-0800 (PST),
//  Mon Jan 05 2015 00:00:00 GMT-0800 (PST)]
\end{DoxyCode}


As does this one\+:


\begin{DoxyCode}
d3.timeDay.every(2).range(new Date(2015, 0, 2), new Date(2015, 0, 8));
// [Sat Jan 03 2015 00:00:00 GMT-0800 (PST),
//  Mon Jan 05 2015 00:00:00 GMT-0800 (PST),
//  Wed Jan 07 2015 00:00:00 GMT-0800 (PST)]
\end{DoxyCode}


The returned filtered interval does not support \href{#interval_count}{\tt {\itshape interval}.count}. See also \href{#interval_filter}{\tt {\itshape interval}.filter}.

\label{_interval_count}%
\# {\itshape interval}.{\bfseries count}({\itshape start}, {\itshape end}) \href{https://github.com/d3/d3-time/blob/master/src/interval.js#L44}{\tt $<$$>$}

Returns the number of interval boundaries after {\itshape start} (exclusive) and before or equal to {\itshape end} (inclusive). Note that this behavior is slightly different than \href{#interval_range}{\tt {\itshape interval}.range} because its purpose is to return the zero-\/based number of the specified {\itshape end} date relative to the specified {\itshape start} date. For example, to compute the current zero-\/based day-\/of-\/year number\+:


\begin{DoxyCode}
var now = new Date;
d3.timeDay.count(d3.timeYear(now), now); // 177
\end{DoxyCode}


Likewise, to compute the current zero-\/based week-\/of-\/year number for weeks that start on Sunday\+:


\begin{DoxyCode}
d3.timeSunday.count(d3.timeYear(now), now); // 25
\end{DoxyCode}


\label{_timeInterval}%
\# d3.{\bfseries time\+Interval}({\itshape floor}, {\itshape offset}\mbox{[}, {\itshape count}\mbox{[}, {\itshape field}\mbox{]}\mbox{]}) \href{https://github.com/d3/d3-time/blob/master/src/interval.js#L4}{\tt $<$$>$}

Constructs a new custom interval given the specified {\itshape floor} and {\itshape offset} functions and an optional {\itshape count} function.

The {\itshape floor} function takes a single date as an argument and rounds it down to the nearest interval boundary.

The {\itshape offset} function takes a date and an integer step as arguments and advances the specified date by the specified number of boundaries; the step may be positive, negative or zero.

The optional {\itshape count} function takes a start date and an end date, already floored to the current interval, and returns the number of boundaries between the start (exclusive) and end (inclusive). If a {\itshape count} function is not specified, the returned interval does not expose \href{#interval_count}{\tt {\itshape interval}.count} or \href{#interval_every}{\tt {\itshape interval}.every} methods. Note\+: due to an internal optimization, the specified {\itshape count} function must not invoke {\itshape interval}.count on other time intervals.

The optional {\itshape field} function takes a date, already floored to the current interval, and returns the field value of the specified date, corresponding to the number of boundaries between this date (exclusive) and the latest previous parent boundary. For example, for the \href{#timeDay}{\tt d3.\+time\+Day} interval, this returns the number of days since the start of the month. If a {\itshape field} function is not specified, it defaults to counting the number of interval boundaries since the U\+N\+IX epoch of January 1, 1970 U\+TC. The {\itshape field} function defines the behavior of \href{#interval_every}{\tt {\itshape interval}.every}.

\subsubsection*{Intervals}

The following intervals are provided\+:

\label{_timeMillisecond}%
\# d3.{\bfseries time\+Millisecond} \href{https://github.com/d3/d3-time/blob/master/src/millisecond.js}{\tt $<$$>$} ~\newline
\href{#timeMillisecond}{\tt \#} d3.{\bfseries utc\+Millisecond}

Milliseconds; the shortest available time unit.

\label{_timeSecond}%
\# d3.{\bfseries time\+Second} \href{https://github.com/d3/d3-time/blob/master/src/second.js}{\tt $<$$>$} ~\newline
\href{#timeSecond}{\tt \#} d3.{\bfseries utc\+Second}

Seconds (e.\+g., 01\+:23\+:45.\+0000 AM); 1,000 milliseconds.

\label{_timeMinute}%
\# d3.{\bfseries time\+Minute} \href{https://github.com/d3/d3-time/blob/master/src/minute.js}{\tt $<$$>$} ~\newline
\href{#timeMinute}{\tt \#} d3.{\bfseries utc\+Minute} \href{https://github.com/d3/d3-time/blob/master/src/utcMinute.js}{\tt $<$$>$}

Minutes (e.\+g., 01\+:02\+:00 AM); 60 seconds. Note that E\+C\+M\+A\+Script \href{http://www.ecma-international.org/ecma-262/5.1/#sec-15.9.1.1}{\tt ignores leap seconds}.

\label{_timeHour}%
\# d3.{\bfseries time\+Hour} \href{https://github.com/d3/d3-time/blob/master/src/hour.js}{\tt $<$$>$} ~\newline
\href{#timeHour}{\tt \#} d3.{\bfseries utc\+Hour} \href{https://github.com/d3/d3-time/blob/master/src/utcHour.js}{\tt $<$$>$}

Hours (e.\+g., 01\+:00 AM); 60 minutes. Note that advancing time by one hour in local time can return the same hour or skip an hour due to daylight saving.

\label{_timeDay}%
\# d3.{\bfseries time\+Day} \href{https://github.com/d3/d3-time/blob/master/src/day.js}{\tt $<$$>$} ~\newline
\href{#timeDay}{\tt \#} d3.{\bfseries utc\+Day} \href{https://github.com/d3/d3-time/blob/master/src/utcDay.js}{\tt $<$$>$}

Days (e.\+g., February 7, 2012 at 12\+:00 AM); typically 24 hours. Days in local time may range from 23 to 25 hours due to daylight saving.

\label{_timeWeek}%
\# d3.{\bfseries time\+Week} \href{https://github.com/d3/d3-time/blob/master/src/week.js}{\tt $<$$>$} ~\newline
\href{#timeWeek}{\tt \#} d3.{\bfseries utc\+Week} \href{https://github.com/d3/d3-time/blob/master/src/utcWeek.js}{\tt $<$$>$}

Alias for \href{#timeSunday}{\tt d3.\+time\+Sunday}; 7 days and typically 168 hours. Weeks in local time may range from 167 to 169 hours due on daylight saving.

\label{_timeSunday}%
\# d3.{\bfseries time\+Sunday} \href{https://github.com/d3/d3-time/blob/master/src/week.js#L15}{\tt $<$$>$} ~\newline
\href{#timeSunday}{\tt \#} d3.{\bfseries utc\+Sunday} \href{https://github.com/d3/d3-time/blob/master/src/utcWeek.js#L15}{\tt $<$$>$}

Sunday-\/based weeks (e.\+g., February 5, 2012 at 12\+:00 AM).

\label{_timeMonday}%
\# d3.{\bfseries time\+Monday} \href{https://github.com/d3/d3-time/blob/master/src/week.js#L16}{\tt $<$$>$} ~\newline
\href{#timeMonday}{\tt \#} d3.{\bfseries utc\+Monday} \href{https://github.com/d3/d3-time/blob/master/src/utcWeek.js#L16}{\tt $<$$>$}

Monday-\/based weeks (e.\+g., February 6, 2012 at 12\+:00 AM).

\label{_timeTuesday}%
\# d3.{\bfseries time\+Tuesday} \href{https://github.com/d3/d3-time/blob/master/src/week.js#L17}{\tt $<$$>$} ~\newline
\href{#timeTuesday}{\tt \#} d3.{\bfseries utc\+Tuesday} \href{https://github.com/d3/d3-time/blob/master/src/utcWeek.js#L17}{\tt $<$$>$}

Tuesday-\/based weeks (e.\+g., February 7, 2012 at 12\+:00 AM).

\label{_timeWednesday}%
\# d3.{\bfseries time\+Wednesday} \href{https://github.com/d3/d3-time/blob/master/src/week.js#L18}{\tt $<$$>$} ~\newline
\href{#timeWednesday}{\tt \#} d3.{\bfseries utc\+Wednesday} \href{https://github.com/d3/d3-time/blob/master/src/utcWeek.js#L18}{\tt $<$$>$}

Wednesday-\/based weeks (e.\+g., February 8, 2012 at 12\+:00 AM).

\label{_timeThursday}%
\# d3.{\bfseries time\+Thursday} \href{https://github.com/d3/d3-time/blob/master/src/week.js#L19}{\tt $<$$>$} ~\newline
\href{#timeThursday}{\tt \#} d3.{\bfseries utc\+Thursday} \href{https://github.com/d3/d3-time/blob/master/src/utcWeek.js#L19}{\tt $<$$>$}

Thursday-\/based weeks (e.\+g., February 9, 2012 at 12\+:00 AM).

\label{_timeFriday}%
\# d3.{\bfseries time\+Friday} \href{https://github.com/d3/d3-time/blob/master/src/week.js#L20}{\tt $<$$>$} ~\newline
\href{#timeFriday}{\tt \#} d3.{\bfseries utc\+Friday} \href{https://github.com/d3/d3-time/blob/master/src/utcWeek.js#L20}{\tt $<$$>$}

Friday-\/based weeks (e.\+g., February 10, 2012 at 12\+:00 AM).

\label{_timeSaturday}%
\# d3.{\bfseries time\+Saturday} \href{https://github.com/d3/d3-time/blob/master/src/week.js#L21}{\tt $<$$>$} ~\newline
\href{#timeSaturday}{\tt \#} d3.{\bfseries utc\+Saturday} \href{https://github.com/d3/d3-time/blob/master/src/utcWeek.js#L21}{\tt $<$$>$}

Saturday-\/based weeks (e.\+g., February 11, 2012 at 12\+:00 AM).

\label{_timeMonth}%
\# d3.{\bfseries time\+Month} \href{https://github.com/d3/d3-time/blob/master/src/month.js}{\tt $<$$>$} ~\newline
\href{#timeMonth}{\tt \#} d3.{\bfseries utc\+Month} \href{https://github.com/d3/d3-time/blob/master/src/utcMonth.js}{\tt $<$$>$}

Months (e.\+g., February 1, 2012 at 12\+:00 AM); ranges from 28 to 31 days.

\label{_timeYear}%
\# d3.{\bfseries time\+Year} \href{https://github.com/d3/d3-time/blob/master/src/year.js}{\tt $<$$>$} ~\newline
\href{#timeYear}{\tt \#} d3.{\bfseries utc\+Year} \href{https://github.com/d3/d3-time/blob/master/src/utcYear.js}{\tt $<$$>$}

Years (e.\+g., January 1, 2012 at 12\+:00 AM); ranges from 365 to 366 days.

\subsubsection*{Ranges}

For convenience, aliases for \href{#interval_range}{\tt {\itshape interval}.range} are also provided as plural forms of the corresponding interval.

\label{_timeMilliseconds}%
\# d3.{\bfseries time\+Milliseconds}({\itshape start}, {\itshape stop}\mbox{[}, {\itshape step}\mbox{]}) \href{https://github.com/d3/d3-time/blob/master/src/millisecond.js#L26}{\tt $<$$>$} ~\newline
\href{#timeMilliseconds}{\tt \#} d3.{\bfseries utc\+Milliseconds}({\itshape start}, {\itshape stop}\mbox{[}, {\itshape step}\mbox{]})

Aliases for \href{#timeMillisecond}{\tt d3.\+time\+Millisecond}.\href{#interval_range}{\tt range} and \href{#timeMillisecond}{\tt d3.\+utc\+Millisecond}.\href{#interval_range}{\tt range}.

\label{_timeSeconds}%
\# d3.{\bfseries time\+Seconds}({\itshape start}, {\itshape stop}\mbox{[}, {\itshape step}\mbox{]}) \href{https://github.com/d3/d3-time/blob/master/src/second.js#L15}{\tt $<$$>$} ~\newline
\href{#timeSeconds}{\tt \#} d3.{\bfseries utc\+Seconds}({\itshape start}, {\itshape stop}\mbox{[}, {\itshape step}\mbox{]})

Aliases for \href{#timeSecond}{\tt d3.\+time\+Second}.\href{#interval_range}{\tt range} and \href{#timeSecond}{\tt d3.\+utc\+Second}.\href{#interval_range}{\tt range}.

\label{_timeMinutes}%
\# d3.{\bfseries time\+Minutes}({\itshape start}, {\itshape stop}\mbox{[}, {\itshape step}\mbox{]}) \href{https://github.com/d3/d3-time/blob/master/src/minute.js#L15}{\tt $<$$>$} ~\newline
\href{#timeMinutes}{\tt \#} d3.{\bfseries utc\+Minutes}({\itshape start}, {\itshape stop}\mbox{[}, {\itshape step}\mbox{]}) \href{https://github.com/d3/d3-time/blob/master/src/utcMinute.js#L15}{\tt $<$$>$}

Aliases for \href{#timeMinute}{\tt d3.\+time\+Minute}.\href{#interval_range}{\tt range} and \href{#timeMinute}{\tt d3.\+utc\+Minute}.\href{#interval_range}{\tt range}.

\label{_timeHours}%
\# d3.{\bfseries time\+Hours}({\itshape start}, {\itshape stop}\mbox{[}, {\itshape step}\mbox{]}) \href{https://github.com/d3/d3-time/blob/master/src/hour.js#L17}{\tt $<$$>$} ~\newline
\href{#timeHours}{\tt \#} d3.{\bfseries utc\+Hours}({\itshape start}, {\itshape stop}\mbox{[}, {\itshape step}\mbox{]}) \href{https://github.com/d3/d3-time/blob/master/src/utcHour.js#L15}{\tt $<$$>$}

Aliases for \href{#timeHour}{\tt d3.\+time\+Hour}.\href{#interval_range}{\tt range} and \href{#timeHour}{\tt d3.\+utc\+Hour}.\href{#interval_range}{\tt range}.

\label{_timeDays}%
\# d3.{\bfseries time\+Days}({\itshape start}, {\itshape stop}\mbox{[}, {\itshape step}\mbox{]}) \href{https://github.com/d3/d3-time/blob/master/src/day.js#L15}{\tt $<$$>$} ~\newline
\href{#timeDays}{\tt \#} d3.{\bfseries utc\+Days}({\itshape start}, {\itshape stop}\mbox{[}, {\itshape step}\mbox{]}) \href{https://github.com/d3/d3-time/blob/master/src/utcDay.js#L15}{\tt $<$$>$}

Aliases for \href{#timeDay}{\tt d3.\+time\+Day}.\href{#interval_range}{\tt range} and \href{#timeDay}{\tt d3.\+utc\+Day}.\href{#interval_range}{\tt range}.

\label{_timeWeeks}%
\# d3.{\bfseries time\+Weeks}({\itshape start}, {\itshape stop}\mbox{[}, {\itshape step}\mbox{]}) ~\newline
\href{#timeWeeks}{\tt \#} d3.{\bfseries utc\+Weeks}({\itshape start}, {\itshape stop}\mbox{[}, {\itshape step}\mbox{]})

Aliases for \href{#timeWeek}{\tt d3.\+time\+Week}.\href{#interval_range}{\tt range} and \href{#timeWeek}{\tt d3.\+utc\+Week}.\href{#interval_range}{\tt range}.

\label{_timeSundays}%
\# d3.{\bfseries time\+Sundays}({\itshape start}, {\itshape stop}\mbox{[}, {\itshape step}\mbox{]}) \href{https://github.com/d3/d3-time/blob/master/src/week.js#L23}{\tt $<$$>$} ~\newline
\href{#timeSundays}{\tt \#} d3.{\bfseries utc\+Sundays}({\itshape start}, {\itshape stop}\mbox{[}, {\itshape step}\mbox{]}) \href{https://github.com/d3/d3-time/blob/master/src/utcWeek.js#L23}{\tt $<$$>$}

Aliases for \href{#timeSunday}{\tt d3.\+time\+Sunday}.\href{#interval_range}{\tt range} and \href{#timeSunday}{\tt d3.\+utc\+Sunday}.\href{#interval_range}{\tt range}.

\label{_timeMondays}%
\# d3.{\bfseries time\+Mondays}({\itshape start}, {\itshape stop}\mbox{[}, {\itshape step}\mbox{]}) \href{https://github.com/d3/d3-time/blob/master/src/week.js#L24}{\tt $<$$>$} ~\newline
\href{#timeMondays}{\tt \#} d3.{\bfseries utc\+Mondays}({\itshape start}, {\itshape stop}\mbox{[}, {\itshape step}\mbox{]}) \href{https://github.com/d3/d3-time/blob/master/src/utcWeek.js#L24}{\tt $<$$>$}

Aliases for \href{#timeMonday}{\tt d3.\+time\+Monday}.\href{#interval_range}{\tt range} and \href{#timeMonday}{\tt d3.\+utc\+Monday}.\href{#interval_range}{\tt range}.

\label{_timeTuesdays}%
\# d3.{\bfseries time\+Tuesdays}({\itshape start}, {\itshape stop}\mbox{[}, {\itshape step}\mbox{]}) \href{https://github.com/d3/d3-time/blob/master/src/week.js#L25}{\tt $<$$>$} ~\newline
\href{#timeTuesdays}{\tt \#} d3.{\bfseries utc\+Tuesdays}({\itshape start}, {\itshape stop}\mbox{[}, {\itshape step}\mbox{]}) \href{https://github.com/d3/d3-time/blob/master/src/utcWeek.js#L25}{\tt $<$$>$}

Aliases for \href{#timeTuesday}{\tt d3.\+time\+Tuesday}.\href{#interval_range}{\tt range} and \href{#timeTuesday}{\tt d3.\+utc\+Tuesday}.\href{#interval_range}{\tt range}.

\label{_timeWednesdays}%
\# d3.{\bfseries time\+Wednesdays}({\itshape start}, {\itshape stop}\mbox{[}, {\itshape step}\mbox{]}) \href{https://github.com/d3/d3-time/blob/master/src/week.js#L26}{\tt $<$$>$} ~\newline
\href{#timeWednesdays}{\tt \#} d3.{\bfseries utc\+Wednesdays}({\itshape start}, {\itshape stop}\mbox{[}, {\itshape step}\mbox{]}) \href{https://github.com/d3/d3-time/blob/master/src/utcWeek.js#L26}{\tt $<$$>$}

Aliases for \href{#timeWednesday}{\tt d3.\+time\+Wednesday}.\href{#interval_range}{\tt range} and \href{#timeWednesday}{\tt d3.\+utc\+Wednesday}.\href{#interval_range}{\tt range}.

\label{_timeThursdays}%
\# d3.{\bfseries time\+Thursdays}({\itshape start}, {\itshape stop}\mbox{[}, {\itshape step}\mbox{]}) \href{https://github.com/d3/d3-time/blob/master/src/week.js#L27}{\tt $<$$>$} ~\newline
\href{#timeThursdays}{\tt \#} d3.{\bfseries utc\+Thursdays}({\itshape start}, {\itshape stop}\mbox{[}, {\itshape step}\mbox{]}) \href{https://github.com/d3/d3-time/blob/master/src/utcWeek.js#L27}{\tt $<$$>$}

Aliases for \href{#timeThursday}{\tt d3.\+time\+Thursday}.\href{#interval_range}{\tt range} and \href{#timeThursday}{\tt d3.\+utc\+Thursday}.\href{#interval_range}{\tt range}.

\label{_timeFridays}%
\# d3.{\bfseries time\+Fridays}({\itshape start}, {\itshape stop}\mbox{[}, {\itshape step}\mbox{]}) \href{https://github.com/d3/d3-time/blob/master/src/week.js#L28}{\tt $<$$>$} ~\newline
\href{#timeFridays}{\tt \#} d3.{\bfseries utc\+Fridays}({\itshape start}, {\itshape stop}\mbox{[}, {\itshape step}\mbox{]}) \href{https://github.com/d3/d3-time/blob/master/src/utcWeek.js#L28}{\tt $<$$>$}

Aliases for \href{#timeFriday}{\tt d3.\+time\+Friday}.\href{#interval_range}{\tt range} and \href{#timeFriday}{\tt d3.\+utc\+Friday}.\href{#interval_range}{\tt range}.

\label{_timeSaturdays}%
\# d3.{\bfseries time\+Saturdays}({\itshape start}, {\itshape stop}\mbox{[}, {\itshape step}\mbox{]}) \href{https://github.com/d3/d3-time/blob/master/src/week.js#L29}{\tt $<$$>$} ~\newline
\href{#timeSaturdays}{\tt \#} d3.{\bfseries utc\+Saturdays}({\itshape start}, {\itshape stop}\mbox{[}, {\itshape step}\mbox{]}) \href{https://github.com/d3/d3-time/blob/master/src/utcWeek.js#L29}{\tt $<$$>$}

Aliases for \href{#timeSaturday}{\tt d3.\+time\+Saturday}.\href{#interval_range}{\tt range} and \href{#timeSaturday}{\tt d3.\+utc\+Saturday}.\href{#interval_range}{\tt range}.

\label{_timeMonths}%
\# d3.{\bfseries time\+Months}({\itshape start}, {\itshape stop}\mbox{[}, {\itshape step}\mbox{]}) \href{https://github.com/d3/d3-time/blob/master/src/month.js#L15}{\tt $<$$>$} ~\newline
\href{#timeMonths}{\tt \#} d3.{\bfseries utc\+Months}({\itshape start}, {\itshape stop}\mbox{[}, {\itshape step}\mbox{]}) \href{https://github.com/d3/d3-time/blob/master/src/utcMonth.js#L15}{\tt $<$$>$}

Aliases for \href{#timeMonth}{\tt d3.\+time\+Month}.\href{#interval_range}{\tt range} and \href{#timeMonth}{\tt d3.\+utc\+Month}.\href{#interval_range}{\tt range}.

\label{_timeYears}%
\# d3.{\bfseries time\+Years}({\itshape start}, {\itshape stop}\mbox{[}, {\itshape step}\mbox{]}) \href{https://github.com/d3/d3-time/blob/master/src/year.js#L26}{\tt $<$$>$} ~\newline
\href{#timeYears}{\tt \#} d3.{\bfseries utc\+Years}({\itshape start}, {\itshape stop}\mbox{[}, {\itshape step}\mbox{]}) \href{https://github.com/d3/d3-time/blob/master/src/utcYear.js#L26}{\tt $<$$>$}

Aliases for \href{#timeYear}{\tt d3.\+time\+Year}.\href{#interval_range}{\tt range} and \href{#timeYear}{\tt d3.\+utc\+Year}.\href{#interval_range}{\tt range}. 