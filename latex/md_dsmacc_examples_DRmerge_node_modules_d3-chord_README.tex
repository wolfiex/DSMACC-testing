Visualize relationships or network flow with an aesthetically-\/pleasing circular layout.

\href{http://bl.ocks.org/mbostock/4062006}{\tt }

\subsection*{Installing}

If you use N\+PM, {\ttfamily npm install d3-\/chord}. Otherwise, download the \href{https://github.com/d3/d3-chord/releases/latest}{\tt latest release}. You can also load directly from \href{https://d3js.org}{\tt d3js.\+org}, either as a \href{https://d3js.org/d3-chord.v1.min.js}{\tt standalone library} or as part of \href{https://github.com/d3/d3}{\tt D3 4.\+0}. A\+MD, Common\+JS, and vanilla environments are supported. In vanilla, a {\ttfamily d3} global is exported\+:


\begin{DoxyCode}
<script src="https://d3js.org/d3-array.v1.min.js"></script>
<script src="https://d3js.org/d3-path.v1.min.js"></script>
<script src="https://d3js.org/d3-chord.v1.min.js"></script>
<script>

var chord = d3.chord();

</script>
\end{DoxyCode}


\href{https://tonicdev.com/npm/d3-chord}{\tt Try d3-\/chord in your browser.}

\subsection*{A\+PI Reference}

\href{#chord}{\tt \#} d3.{\bfseries chord}() \href{https://github.com/d3/d3-chord/blob/master/src/chord.js}{\tt $<$$>$}

Constructs a new chord layout with the default settings.

\href{#_chord}{\tt \#} {\itshape chord}({\itshape matrix}) \href{https://github.com/d3/d3-chord/blob/master/src/chord.js#L19}{\tt $<$$>$}

Computes the chord layout for the specified square {\itshape matrix} of size {\itshape n$\ast$×$\ast$n}, where the {\itshape matrix} represents the directed flow amongst a network (a complete digraph) of {\itshape n} nodes. The given {\itshape matrix} must be an array of length {\itshape n}, where each element {\itshape matrix}\mbox{[}{\itshape i}\mbox{]} is an array of {\itshape n} numbers, where each {\itshape matrix}\mbox{[}{\itshape i}\mbox{]}\mbox{[}{\itshape j}\mbox{]} represents the flow from the {\itshape i$\ast$th node in the network to the $\ast$j$\ast$th node. Each number $\ast$matrix}\mbox{[}{\itshape i}\mbox{]}\mbox{[}{\itshape j}\mbox{]} must be nonnegative, though it can be zero if there is no flow from node {\itshape i} to node {\itshape j}. From the \href{http://mkweb.bcgsc.ca/circos/guide/tables/}{\tt Circos tableviewer example}\+:


\begin{DoxyCode}
var matrix = [
  [11975,  5871, 8916, 2868],
  [ 1951, 10048, 2060, 6171],
  [ 8010, 16145, 8090, 8045],
  [ 1013,   990,  940, 6907]
];
\end{DoxyCode}


The return value of {\itshape chord}({\itshape matrix}) is an array of {\itshape chords}, where each chord represents the combined bidirectional flow between two nodes {\itshape i} and {\itshape j} (where {\itshape i} may be equal to {\itshape j}) and is an object with the following properties\+:


\begin{DoxyItemize}
\item {\ttfamily source} -\/ the source subgroup
\item {\ttfamily target} -\/ the target subgroup
\end{DoxyItemize}

Each source and target subgroup is also an object with the following properties\+:


\begin{DoxyItemize}
\item {\ttfamily start\+Angle} -\/ the start angle in radians
\item {\ttfamily end\+Angle} -\/ the end angle in radians
\item {\ttfamily value} -\/ the flow value {\itshape matrix}\mbox{[}{\itshape i}\mbox{]}\mbox{[}{\itshape j}\mbox{]}
\item {\ttfamily index} -\/ the node index {\itshape i}
\item {\ttfamily subindex} -\/ the node index {\itshape j}
\end{DoxyItemize}

The chords are typically passed to \href{#ribbon}{\tt d3.\+ribbon} to display the network relationships. The returned array includes only chord objects for which the value {\itshape matrix}\mbox{[}{\itshape i}\mbox{]}\mbox{[}{\itshape j}\mbox{]} or {\itshape matrix}\mbox{[}{\itshape j}\mbox{]}\mbox{[}{\itshape i}\mbox{]} is non-\/zero. Furthermore, the returned array only contains unique chords\+: a given chord {\itshape ij} represents the bidirectional flow from {\itshape i} to {\itshape j} {\itshape and} from {\itshape j} to {\itshape i}, and does not contain a duplicate chord {\itshape ji}; {\itshape i} and {\itshape j} are chosen such that the chord’s source always represents the larger of {\itshape matrix}\mbox{[}{\itshape i}\mbox{]}\mbox{[}{\itshape j}\mbox{]} and {\itshape matrix}\mbox{[}{\itshape j}\mbox{]}\mbox{[}{\itshape i}\mbox{]}. In other words, {\itshape chord}.source.\+index equals {\itshape chord}.target.\+subindex, {\itshape chord}.source.\+subindex equals {\itshape chord}.target.\+index, {\itshape chord}.source.\+value is greater than or equal to {\itshape chord}.target.\+value, and {\itshape chord}.source.\+value is always greater than zero.

The {\itshape chords} array also defines a secondary array of length {\itshape n}, {\itshape chords}.groups, where each group represents the combined outflow for node {\itshape i}, corresponding to the elements {\itshape matrix}\mbox{[}{\itshape i}\mbox{]}\mbox{[}0 … {\itshape n} -\/ 1\mbox{]}, and is an object with the following properties\+:


\begin{DoxyItemize}
\item {\ttfamily start\+Angle} -\/ the start angle in radians
\item {\ttfamily end\+Angle} -\/ the end angle in radians
\item {\ttfamily value} -\/ the total outgoing flow value for node {\itshape i}
\item {\ttfamily index} -\/ the node index {\itshape i}
\end{DoxyItemize}

The groups are typically passed to \href{https://github.com/d3/d3-shape#arc}{\tt d3.\+arc} to produce a donut chart around the circumference of the chord layout.

\href{#chord_padAngle}{\tt \#} {\itshape chord}.{\bfseries pad\+Angle}(\mbox{[}{\itshape angle}\mbox{]}) \href{https://github.com/d3/d3-chord/blob/master/src/chord.js#L104}{\tt $<$$>$}

If {\itshape angle} is specified, sets the pad angle between adjacent groups to the specified number in radians and returns this chord layout. If {\itshape angle} is not specified, returns the current pad angle, which defaults to zero.

\href{#chord_sortGroups}{\tt \#} {\itshape chord}.{\bfseries sort\+Groups}(\mbox{[}{\itshape compare}\mbox{]}) \href{https://github.com/d3/d3-chord/blob/master/src/chord.js#L108}{\tt $<$$>$}

If {\itshape compare} is specified, sets the group comparator to the specified function or null and returns this chord layout. If {\itshape compare} is not specified, returns the current group comparator, which defaults to null. If the group comparator is non-\/null, it is used to sort the groups by their total outflow. See also \href{https://github.com/d3/d3-array#ascending}{\tt d3.\+ascending} and \href{https://github.com/d3/d3-array#descending}{\tt d3.\+descending}.

\href{#chord_sortSubgroups}{\tt \#} {\itshape chord}.{\bfseries sort\+Subgroups}(\mbox{[}{\itshape compare}\mbox{]}) \href{https://github.com/d3/d3-chord/blob/master/src/chord.js#L112}{\tt $<$$>$}

If {\itshape compare} is specified, sets the subgroup comparator to the specified function or null and returns this chord layout. If {\itshape compare} is not specified, returns the current subgroup comparator, which defaults to null. If the subgroup comparator is non-\/null, it is used to sort the subgroups corresponding to {\itshape matrix}\mbox{[}{\itshape i}\mbox{]}\mbox{[}0 … {\itshape n} -\/ 1\mbox{]} for a given group {\itshape i} by their total outflow. See also \href{https://github.com/d3/d3-array#ascending}{\tt d3.\+ascending} and \href{https://github.com/d3/d3-array#descending}{\tt d3.\+descending}.

\href{#chord_sortChords}{\tt \#} {\itshape chord}.{\bfseries sort\+Chords}(\mbox{[}{\itshape compare}\mbox{]}) \href{https://github.com/d3/d3-chord/blob/master/src/chord.js#L116}{\tt $<$$>$}

If {\itshape compare} is specified, sets the chord comparator to the specified function or null and returns this chord layout. If {\itshape compare} is not specified, returns the current chord comparator, which defaults to null. If the chord comparator is non-\/null, it is used to sort the \href{#_chord}{\tt chords} by their combined flow; this only affects the {\itshape z}-\/order of the chords. See also \href{https://github.com/d3/d3-array#ascending}{\tt d3.\+ascending} and \href{https://github.com/d3/d3-array#descending}{\tt d3.\+descending}.

\href{#ribbon}{\tt \#} d3.{\bfseries ribbon}() \href{https://github.com/d3/d3-chord/blob/master/src/ribbon.js}{\tt $<$$>$}

Creates a new ribbon generator with the default settings.

\href{#_ribbon}{\tt \#} {\itshape ribbon}({\itshape arguments…}) \href{https://github.com/d3/d3-chord/blob/master/src/ribbon.js#L34}{\tt $<$$>$}

Generates a ribbon for the given {\itshape arguments}. The {\itshape arguments} are arbitrary; they are simply propagated to the ribbon generator’s accessor functions along with the {\ttfamily this} object. For example, with the default settings, a \href{#_chord}{\tt chord object} expected\+:


\begin{DoxyCode}
var ribbon = d3.ribbon();

ribbon(\{
  source: \{startAngle: 0.7524114, endAngle: 1.1212972, radius: 240\},
  target: \{startAngle: 1.8617078, endAngle: 1.9842927, radius: 240\}
\}); //
       "M164.0162810494058,-175.21032946354026A240,240,0,0,1,216.1595644740915,-104.28347273835429Q0,0,229.9
      158815306728,68.8381247563705A240,240,0,0,1,219.77316791012538,96.43523560788266Q0,0,164.0162810494058,-175.21032946354026Z"
\end{DoxyCode}


Or equivalently if the radius is instead defined as a constant\+:


\begin{DoxyCode}
var ribbon = d3.ribbon()
    .radius(240);

ribbon(\{
  source: \{startAngle: 0.7524114, endAngle: 1.1212972\},
  target: \{startAngle: 1.8617078, endAngle: 1.9842927\}
\}); //
       "M164.0162810494058,-175.21032946354026A240,240,0,0,1,216.1595644740915,-104.28347273835429Q0,0,229.9
      158815306728,68.8381247563705A240,240,0,0,1,219.77316791012538,96.43523560788266Q0,0,164.0162810494058,-175.21032946354026Z"
\end{DoxyCode}


If the ribbon generator has a context, then the ribbon is rendered to this context as a sequence of path method calls and this function returns void. Otherwise, a path data string is returned.

\href{#ribbon_source}{\tt \#} {\itshape ribbon}.{\bfseries source}(\mbox{[}{\itshape source}\mbox{]}) \href{https://github.com/d3/d3-chord/blob/master/src/ribbon.js#L74}{\tt $<$$>$}

If {\itshape source} is specified, sets the source accessor to the specified function and returns this ribbon generator. If {\itshape source} is not specified, returns the current source accessor, which defaults to\+:


\begin{DoxyCode}
function source(d) \{
  return d.source;
\}
\end{DoxyCode}


\href{#ribbon_target}{\tt \#} {\itshape ribbon}.{\bfseries target}(\mbox{[}{\itshape target}\mbox{]}) \href{https://github.com/d3/d3-chord/blob/master/src/ribbon.js#L78}{\tt $<$$>$}

If {\itshape target} is specified, sets the target accessor to the specified function and returns this ribbon generator. If {\itshape target} is not specified, returns the current target accessor, which defaults to\+:


\begin{DoxyCode}
function target(d) \{
  return d.target;
\}
\end{DoxyCode}


\href{#ribbon_radius}{\tt \#} {\itshape ribbon}.{\bfseries radius}(\mbox{[}{\itshape radius}\mbox{]}) \href{https://github.com/d3/d3-chord/blob/master/src/ribbon.js#L62}{\tt $<$$>$}

If {\itshape radius} is specified, sets the radius accessor to the specified function and returns this ribbon generator. If {\itshape radius} is not specified, returns the current radius accessor, which defaults to\+:


\begin{DoxyCode}
function radius(d) \{
  return d.radius;
\}
\end{DoxyCode}


\href{#ribbon_startAngle}{\tt \#} {\itshape ribbon}.{\bfseries start\+Angle}(\mbox{[}{\itshape angle}\mbox{]}) \href{https://github.com/d3/d3-chord/blob/master/src/ribbon.js#L66}{\tt $<$$>$}

If {\itshape angle} is specified, sets the start angle accessor to the specified function and returns this ribbon generator. If {\itshape angle} is not specified, returns the current start angle accessor, which defaults to\+:


\begin{DoxyCode}
function startAngle(d) \{
  return d.startAngle;
\}
\end{DoxyCode}


The {\itshape angle} is specified in radians, with 0 at -\/$\ast$y$\ast$ (12 o’clock) and positive angles proceeding clockwise.

\href{#ribbon_endAngle}{\tt \#} {\itshape ribbon}.{\bfseries end\+Angle}(\mbox{[}{\itshape angle}\mbox{]}) \href{https://github.com/d3/d3-chord/blob/master/src/ribbon.js#L70}{\tt $<$$>$}

If {\itshape angle} is specified, sets the end angle accessor to the specified function and returns this ribbon generator. If {\itshape angle} is not specified, returns the current end angle accessor, which defaults to\+:


\begin{DoxyCode}
function endAngle(d) \{
  return d.endAngle;
\}
\end{DoxyCode}


The {\itshape angle} is specified in radians, with 0 at -\/$\ast$y$\ast$ (12 o’clock) and positive angles proceeding clockwise.

\href{#ribbon_context}{\tt \#} {\itshape ribbon}.{\bfseries context}(\mbox{[}{\itshape context}\mbox{]}) \href{https://github.com/d3/d3-chord/blob/master/src/ribbon.js#L82}{\tt $<$$>$}

If {\itshape context} is specified, sets the context and returns this ribbon generator. If {\itshape context} is not specified, returns the current context, which defaults to null. If the context is not null, then the \href{#_ribbon}{\tt generated ribbon} is rendered to this context as a sequence of \href{http://www.w3.org/TR/2dcontext/#canvaspathmethods}{\tt path method} calls. Otherwise, a \href{http://www.w3.org/TR/SVG/paths.html#PathData}{\tt path data} string representing the generated ribbon is returned. See also \href{https://github.com/d3/d3-path}{\tt d3-\/path}. 