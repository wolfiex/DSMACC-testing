{\bfseries What does this do?}

\begin{quote}
Pass two numbers, get a regex-\/compatible source string for matching ranges. Validated against more than 2.\+78 million test assertions. \end{quote}


\subsection*{Install}

Install with \href{https://www.npmjs.com/}{\tt npm}\+:


\begin{DoxyCode}
$ npm install --save to-regex-range
\end{DoxyCode}


Install with \href{https://yarnpkg.com}{\tt yarn}\+:


\begin{DoxyCode}
$ yarn add to-regex-range
\end{DoxyCode}


$<$details$>$

~\newline


This libary generates the {\ttfamily source} string to be passed to {\ttfamily new Reg\+Exp()} for matching a range of numbers.

{\bfseries Example}


\begin{DoxyCode}
var toRegexRange = require('to-regex-range');
var regex = new RegExp(toRegexRange('15', '95'));
\end{DoxyCode}


A string is returned so that you can do whatever you need with it before passing it to {\ttfamily new Reg\+Exp()} (like adding {\ttfamily $^\wedge$} or {\ttfamily \$} boundaries, defining flags, or combining it another string).

~\newline


$<$/details$>$

$<$details$>$ 

{\bfseries Why use this library?}

~\newline


\subsubsection*{Convenience}

Creating regular expressions for matching numbers gets deceptively complicated pretty fast.

For example, let\textquotesingle{}s say you need a validation regex for matching part of a user-\/id, postal code, social security number, tax id, etc\+:


\begin{DoxyItemize}
\item regex for matching {\ttfamily 1} =$>$ {\ttfamily /1/} (easy enough)
\item regex for matching {\ttfamily 1} through {\ttfamily 5} =$>$ {\ttfamily /\mbox{[}1-\/5\mbox{]}/} (not bad...)
\item regex for matching {\ttfamily 1} or {\ttfamily 5} =$>$ {\ttfamily /(1$\vert$5)/} (still easy...)
\item regex for matching {\ttfamily 1} through {\ttfamily 50} =$>$ {\ttfamily /(\mbox{[}1-\/9\mbox{]}$\vert$\mbox{[}1-\/4\mbox{]}\mbox{[}0-\/9\mbox{]}$\vert$50)/} (uh-\/oh...)
\item regex for matching {\ttfamily 1} through {\ttfamily 55} =$>$ {\ttfamily /(\mbox{[}1-\/9\mbox{]}$\vert$\mbox{[}1-\/4\mbox{]}\mbox{[}0-\/9\mbox{]}$\vert$5\mbox{[}0-\/5\mbox{]})/} (no prob, I can do this...)
\item regex for matching {\ttfamily 1} through {\ttfamily 555} =$>$ {\ttfamily /(\mbox{[}1-\/9\mbox{]}$\vert$\mbox{[}1-\/9\mbox{]}\mbox{[}0-\/9\mbox{]}$\vert$\mbox{[}1-\/4\mbox{]}\mbox{[}0-\/9\mbox{]}\{2\}$\vert$5\mbox{[}0-\/4\mbox{]}\mbox{[}0-\/9\mbox{]}$\vert$55\mbox{[}0-\/5\mbox{]})/} (maybe not...)
\item regex for matching {\ttfamily 0001} through {\ttfamily 5555} =$>$ {\ttfamily /(0\{3\}\mbox{[}1-\/9\mbox{]}$\vert$0\{2\}\mbox{[}1-\/9\mbox{]}\mbox{[}0-\/9\mbox{]}$\vert$0\mbox{[}1-\/9\mbox{]}\mbox{[}0-\/9\mbox{]}\{2\}$\vert$\mbox{[}1-\/4\mbox{]}\mbox{[}0-\/9\mbox{]}\{3\}$\vert$5\mbox{[}0-\/4\mbox{]}\mbox{[}0-\/9\mbox{]}\{2\}$\vert$55\mbox{[}0-\/4\mbox{]}\mbox{[}0-\/9\mbox{]}$\vert$555\mbox{[}0-\/5\mbox{]})/} (okay, I get the point!)
\end{DoxyItemize}

The numbers are contrived, but they\textquotesingle{}re also really basic. In the real world you might need to generate a regex on-\/the-\/fly for validation.

{\bfseries Learn more}

If you\textquotesingle{}re interested in learning more about \href{http://www.regular-expressions.info/charclass.html}{\tt character classes} and other regex features, I personally have always found \href{http://www.regular-expressions.info/charclass.html}{\tt regular-\/expressions.\+info} to be pretty useful.

\subsubsection*{Heavily tested}

As of April 27, 2017, this library runs \href{./test/test.js}{\tt 2,783,483 test assertions} against generated regex-\/ranges to provide brute-\/force verification that results are indeed correct.

Tests run in $\sim$870ms on my Mac\+Book Pro, 2.\+5 G\+Hz Intel Core i7.

\subsubsection*{Highly optimized}

Generated regular expressions are highly optimized\+:


\begin{DoxyItemize}
\item duplicate sequences and character classes are reduced using quantifiers
\item smart enough to use {\ttfamily ?} conditionals when number(s) or range(s) can be positive or negative
\item uses fragment caching to avoid processing the same exact string more than once
\end{DoxyItemize}

~\newline


$<$/details$>$

\subsection*{Usage}

Add this library to your javascript application with the following line of code


\begin{DoxyCode}
var toRegexRange = require('to-regex-range');
\end{DoxyCode}


The main export is a function that takes two integers\+: the {\ttfamily min} value and {\ttfamily max} value (formatted as strings or numbers).


\begin{DoxyCode}
var source = toRegexRange('15', '95');
//=> 1[5-9]|[2-8][0-9]|9[0-5]

var re = new RegExp('^' + source + '$');
console.log(re.test('14')); //=> false
console.log(re.test('50')); //=> true
console.log(re.test('94')); //=> true
console.log(re.test('96')); //=> false
\end{DoxyCode}


\subsection*{Options}

\subsubsection*{options.\+capture}

{\bfseries Type}\+: {\ttfamily boolean}

{\bfseries Deafault}\+: {\ttfamily undefined}

Wrap the returned value in parentheses when there is more than one regex condition. Useful when you\textquotesingle{}re dynamically generating ranges.


\begin{DoxyCode}
console.log(toRegexRange('-10', '10'));
//=> -[1-9]|-?10|[0-9]

console.log(toRegexRange('-10', '10', \{capture: true\}));
//=> (-[1-9]|-?10|[0-9])
\end{DoxyCode}


\subsubsection*{options.\+shorthand}

{\bfseries Type}\+: {\ttfamily boolean}

{\bfseries Deafault}\+: {\ttfamily undefined}

Use the regex shorthand for {\ttfamily \mbox{[}0-\/9\mbox{]}}\+:


\begin{DoxyCode}
console.log(toRegexRange('0', '999999'));
//=> [0-9]|[1-9][0-9]\{1,5\}

console.log(toRegexRange('0', '999999', \{shorthand: true\}));
//=> \(\backslash\)d|[1-9]\(\backslash\)d\{1,5\}
\end{DoxyCode}


\subsubsection*{options.\+relax\+Zeros}

{\bfseries Type}\+: {\ttfamily boolean}

{\bfseries Default}\+: {\ttfamily true}

This option only applies to {\bfseries negative zero-\/padded ranges}. By default, when a negative zero-\/padded range is defined, the number of leading zeros is relaxed using {\ttfamily -\/0$\ast$}.


\begin{DoxyCode}
console.log(toRegexRange('-001', '100'));
//=> -0*1|0\{2\}[0-9]|0[1-9][0-9]|100

console.log(toRegexRange('-001', '100', \{relaxZeros: false\}));
//=> -0\{2\}1|0\{2\}[0-9]|0[1-9][0-9]|100
\end{DoxyCode}


$<$details$>$ 

{\bfseries Why are zeros relaxed for negative zero-\/padded ranges by default?}

Consider the following.


\begin{DoxyCode}
var regex = toRegexRange('-001', '100');
\end{DoxyCode}


{\itshape Note that {\ttfamily -\/001} and {\ttfamily 100} are both three digits long}.

In most zero-\/padding implementations, only a single leading zero is enough to indicate that zero-\/padding should be applied. Thus, the leading zeros would be \char`\"{}corrected\char`\"{} on the negative range in the example to {\ttfamily -\/01}, instead of {\ttfamily -\/001}, to make total length of each string no greater than the length of the largest number in the range (in other words, {\ttfamily -\/001} is 4 digits, but {\ttfamily 100} is only three digits).

If zeros were not relaxed by default, you might expect the resulting regex of the above pattern to match {\ttfamily -\/001} -\/ given that it\textquotesingle{}s defined that way in the arguments -\/ {\itshape but it wouldn\textquotesingle{}t}. It would, however, match {\ttfamily -\/01}. This gets even more ambiguous with large ranges, like {\ttfamily -\/01} to {\ttfamily 1000000}.

Thus, we relax zeros by default to provide a more predictable experience for users.

$<$/details$>$

\subsection*{Examples}

\tabulinesep=1mm
\begin{longtabu} spread 0pt [c]{*{3}{|X[-1]}|}
\hline
\rowcolor{\tableheadbgcolor}\textbf{ {\bfseries Range}  }&\multicolumn{2}{p{(\linewidth-\tabcolsep*3-\arrayrulewidth*2)*2/3}|}{\cellcolor{\tableheadbgcolor}\textbf{ $\ast$$\ast$\+Re   }}\\\cline{1-3}
\endfirsthead
\hline
\endfoot
\hline
\rowcolor{\tableheadbgcolor}\textbf{ {\bfseries Range}  }&\multicolumn{2}{p{(\linewidth-\tabcolsep*3-\arrayrulewidth*2)*2/3}|}{\cellcolor{\tableheadbgcolor}\textbf{ $\ast$$\ast$\+Re   }}\\\cline{1-3}
\endhead
`to\+Regex\+Range(\textquotesingle{}5, 5'){\ttfamily $<$/td$>$ $<$td class=\char`\"{}markdown\+Table\+Body\+None\char`\"{}$>$}5\`{}  &{\itshape 33μs}   \\\cline{1-3}
`to\+Regex\+Range(\textquotesingle{}5, 6'){\ttfamily $<$/td$>$ $<$td class=\char`\"{}markdown\+Table\+Body\+None\char`\"{}$>$}5$|$6\`{}  &{\itshape 53μs}   \\\cline{1-3}
`to\+Regex\+Range(\textquotesingle{}29, 51'){\ttfamily $<$/td$>$ $<$td class=\char`\"{}markdown\+Table\+Body\+None\char`\"{}$>$}29$|$\mbox{[}34\mbox{]}\mbox{[}0-\/9\mbox{]}$|$5\mbox{[}01\mbox{]}\`{}  &{\itshape 699μs}   \\\cline{1-3}
`to\+Regex\+Range(\textquotesingle{}31, 877'){\ttfamily $<$/td$>$ $<$td class=\char`\"{}markdown\+Table\+Body\+None\char`\"{}$>$}3\mbox{[}1-\/9\mbox{]}$|$\mbox{[}4-\/9\mbox{]}\mbox{[}0-\/9\mbox{]}$|$\mbox{[}1-\/7\mbox{]}\mbox{[}0-\/9\mbox{]}\{2\}$|$8\mbox{[}0-\/6\mbox{]}\mbox{[}0-\/9\mbox{]}$|$87\mbox{[}0-\/7\mbox{]}\`{}  &{\itshape 711μs}   \\\cline{1-3}
`to\+Regex\+Range(\textquotesingle{}111, 555'){\ttfamily $<$/td$>$ $<$td class=\char`\"{}markdown\+Table\+Body\+None\char`\"{}$>$}11\mbox{[}1-\/9\mbox{]}$|$1\mbox{[}2-\/9\mbox{]}\mbox{[}0-\/9\mbox{]}$|$\mbox{[}2-\/4\mbox{]}\mbox{[}0-\/9\mbox{]}\{2\}$|$5\mbox{[}0-\/4\mbox{]}\mbox{[}0-\/9\mbox{]}$|$55\mbox{[}0-\/5\mbox{]}\`{}  &{\itshape 62μs}   \\\cline{1-3}
`to\+Regex\+Range('-\/10, 10\textquotesingle{}){\ttfamily $<$/td$>$ $<$td class=\char`\"{}markdown\+Table\+Body\+None\char`\"{}$>$}-\/\mbox{[}1-\/9\mbox{]}$|$-\/?10$|$\mbox{[}0-\/9\mbox{]}\`{}  &{\itshape 74μs}   \\\cline{1-3}
`to\+Regex\+Range('-\/100, -\/10\textquotesingle{}){\ttfamily $<$/td$>$ $<$td class=\char`\"{}markdown\+Table\+Body\+None\char`\"{}$>$}-\/1\mbox{[}0-\/9\mbox{]}$|$-\/\mbox{[}2-\/9\mbox{]}\mbox{[}0-\/9\mbox{]}$|$-\/100\`{}  &{\itshape 49μs}   \\\cline{1-3}
`to\+Regex\+Range('-\/100, 100\textquotesingle{}){\ttfamily $<$/td$>$ $<$td class=\char`\"{}markdown\+Table\+Body\+None\char`\"{}$>$}-\/\mbox{[}1-\/9\mbox{]}$|$-\/?\mbox{[}1-\/9\mbox{]}\mbox{[}0-\/9\mbox{]}$|$-\/?100$|$\mbox{[}0-\/9\mbox{]}\`{}  &{\itshape 45μs}   \\\cline{1-3}
`to\+Regex\+Range(\textquotesingle{}001, 100'){\ttfamily $<$/td$>$ $<$td class=\char`\"{}markdown\+Table\+Body\+None\char`\"{}$>$}0\{2\}\mbox{[}1-\/9\mbox{]}$|$0\mbox{[}1-\/9\mbox{]}\mbox{[}0-\/9\mbox{]}$|$100\`{}  &{\itshape 158μs}   \\\cline{1-3}
`to\+Regex\+Range(\textquotesingle{}0010, 1000'){\ttfamily $<$/td$>$ $<$td class=\char`\"{}markdown\+Table\+Body\+None\char`\"{}$>$}0\{2\}1\mbox{[}0-\/9\mbox{]}$|$0\{2\}\mbox{[}2-\/9\mbox{]}\mbox{[}0-\/9\mbox{]}$|$0\mbox{[}1-\/9\mbox{]}\mbox{[}0-\/9\mbox{]}\{2\}$|$1000\`{}  &{\itshape 61μs}   \\\cline{1-3}
`to\+Regex\+Range(\textquotesingle{}1, 2'){\ttfamily $<$/td$>$ $<$td class=\char`\"{}markdown\+Table\+Body\+None\char`\"{}$>$}1$|$2\`{}  &{\itshape 10μs}   \\\cline{1-3}
`to\+Regex\+Range(\textquotesingle{}1, 5'){\ttfamily $<$/td$>$ $<$td class=\char`\"{}markdown\+Table\+Body\+None\char`\"{}$>$}\mbox{[}1-\/5\mbox{]}\`{}  &{\itshape 24μs}   \\\cline{1-3}
`to\+Regex\+Range(\textquotesingle{}1, 10'){\ttfamily $<$/td$>$ $<$td class=\char`\"{}markdown\+Table\+Body\+None\char`\"{}$>$}\mbox{[}1-\/9\mbox{]}$|$10\`{}  &{\itshape 23μs}   \\\cline{1-3}
`to\+Regex\+Range(\textquotesingle{}1, 100'){\ttfamily $<$/td$>$ $<$td class=\char`\"{}markdown\+Table\+Body\+None\char`\"{}$>$}\mbox{[}1-\/9\mbox{]}$|$\mbox{[}1-\/9\mbox{]}\mbox{[}0-\/9\mbox{]}$|$100\`{}  &{\itshape 30μs}   \\\cline{1-3}
`to\+Regex\+Range(\textquotesingle{}1, 1000'){\ttfamily $<$/td$>$ $<$td class=\char`\"{}markdown\+Table\+Body\+None\char`\"{}$>$}\mbox{[}1-\/9\mbox{]}$|$\mbox{[}1-\/9\mbox{]}\mbox{[}0-\/9\mbox{]}\{1,2\}$|$1000\`{}  &{\itshape 52μs}   \\\cline{1-3}
`to\+Regex\+Range(\textquotesingle{}1, 10000'){\ttfamily $<$/td$>$ $<$td class=\char`\"{}markdown\+Table\+Body\+None\char`\"{}$>$}\mbox{[}1-\/9\mbox{]}$|$\mbox{[}1-\/9\mbox{]}\mbox{[}0-\/9\mbox{]}\{1,3\}$|$10000\`{}  &{\itshape 47μs}   \\\cline{1-3}
`to\+Regex\+Range(\textquotesingle{}1, 100000'){\ttfamily $<$/td$>$ $<$td class=\char`\"{}markdown\+Table\+Body\+None\char`\"{}$>$}\mbox{[}1-\/9\mbox{]}$|$\mbox{[}1-\/9\mbox{]}\mbox{[}0-\/9\mbox{]}\{1,4\}$|$100000\`{}  &{\itshape 44μs}   \\\cline{1-3}
`to\+Regex\+Range(\textquotesingle{}1, 1000000'){\ttfamily $<$/td$>$ $<$td class=\char`\"{}markdown\+Table\+Body\+None\char`\"{}$>$}\mbox{[}1-\/9\mbox{]}$|$\mbox{[}1-\/9\mbox{]}\mbox{[}0-\/9\mbox{]}\{1,5\}$|$1000000\`{}  &{\itshape 49μs}   \\\cline{1-3}
`to\+Regex\+Range(\textquotesingle{}1, 10000000'){\ttfamily $<$/td$>$ $<$td class=\char`\"{}markdown\+Table\+Body\+None\char`\"{}$>$}\mbox{[}1-\/9\mbox{]}$|$\mbox{[}1-\/9\mbox{]}\mbox{[}0-\/9\mbox{]}\{1,6\}$|$10000000\`{}  &{\itshape 63μs}   \\\cline{1-3}
\end{longtabu}


\subsection*{Heads up!}

{\bfseries Order of arguments}

When the {\ttfamily min} is larger than the {\ttfamily max}, values will be flipped to create a valid range\+:


\begin{DoxyCode}
toRegexRange('51', '29');
\end{DoxyCode}


Is effectively flipped to\+:


\begin{DoxyCode}
toRegexRange('29', '51');
//=> 29|[3-4][0-9]|5[0-1]
\end{DoxyCode}


{\bfseries Steps / increments}

This library does not support steps (increments). A pr to add support would be welcome.

\subsection*{History}

\subsubsection*{v2.\+0.\+0 -\/ 2017-\/04-\/21}

{\bfseries New features}

Adds support for zero-\/padding!

\subsubsection*{v1.\+0.\+0}

{\bfseries Optimizations}

Repeating ranges are now grouped using quantifiers. rocessing time is roughly the same, but the generated regex is much smaller, which should result in faster matching.

\subsection*{Attribution}

Inspired by the python library \href{https://github.com/dimka665/range-regex}{\tt range-\/regex}.

\subsection*{About}

\subsubsection*{Related projects}


\begin{DoxyItemize}
\item \href{https://www.npmjs.com/package/expand-range}{\tt expand-\/range}\+: Fast, bash-\/like range expansion. Expand a range of numbers or letters, uppercase or lowercase. See… \href{https://github.com/jonschlinkert/expand-range}{\tt more} $\vert$ \href{https://github.com/jonschlinkert/expand-range}{\tt homepage}
\item \href{https://www.npmjs.com/package/fill-range}{\tt fill-\/range}\+: Fill in a range of numbers or letters, optionally passing an increment or {\ttfamily step} to… \href{https://github.com/jonschlinkert/fill-range}{\tt more} $\vert$ \href{https://github.com/jonschlinkert/fill-range}{\tt homepage}
\item \href{https://www.npmjs.com/package/micromatch}{\tt micromatch}\+: Glob matching for javascript/node.\+js. A drop-\/in replacement and faster alternative to minimatch and multimatch. $\vert$ \href{https://github.com/jonschlinkert/micromatch}{\tt homepage}
\item \href{https://www.npmjs.com/package/repeat-element}{\tt repeat-\/element}\+: Create an array by repeating the given value n times. $\vert$ \href{https://github.com/jonschlinkert/repeat-element}{\tt homepage}
\item \href{https://www.npmjs.com/package/repeat-string}{\tt repeat-\/string}\+: Repeat the given string n times. Fastest implementation for repeating a string. $\vert$ \href{https://github.com/jonschlinkert/repeat-string}{\tt homepage}
\end{DoxyItemize}

\subsubsection*{Contributing}

Pull requests and stars are always welcome. For bugs and feature requests, \href{../../issues/new}{\tt please create an issue}.

\subsubsection*{Building docs}

\+\_\+(This project\textquotesingle{}s readme.\+md is generated by \href{https://github.com/verbose/verb-generate-readme}{\tt verb}, please don\textquotesingle{}t edit the readme directly. Any changes to the readme must be made in the .verb.\+md \char`\"{}.\+verb.\+md\char`\"{} readme template.)\+\_\+

To generate the readme, run the following command\+:


\begin{DoxyCode}
$ npm install -g verbose/verb#dev verb-generate-readme && verb
\end{DoxyCode}


\subsubsection*{Running tests}

Running and reviewing unit tests is a great way to get familiarized with a library and its A\+PI. You can install dependencies and run tests with the following command\+:


\begin{DoxyCode}
$ npm install && npm test
\end{DoxyCode}


\subsubsection*{Author}

{\bfseries Jon Schlinkert}


\begin{DoxyItemize}
\item \href{https://github.com/jonschlinkert}{\tt github/jonschlinkert}
\item \href{https://twitter.com/jonschlinkert}{\tt twitter/jonschlinkert}
\end{DoxyItemize}

\subsubsection*{License}

Copyright © 2017, \href{https://github.com/jonschlinkert}{\tt Jon Schlinkert}. Released under the \mbox{[}M\+IT License\mbox{]}(L\+I\+C\+E\+N\+SE).





{\itshape This file was generated by \href{https://github.com/verbose/verb-generate-readme}{\tt verb-\/generate-\/readme}, v0.\+6.\+0, on April 27, 2017.} 