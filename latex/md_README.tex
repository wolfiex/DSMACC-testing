The heavily edited and improved version of the Dynamically Simple Model for Atmospheric Chemical Complexity (D\+S\+M\+A\+CC), as used in the Thesis\+:

Understanding the Atmosphere using graph theory, visualisation and machine learning

by Daniel Ellis.

\subsection*{Cite}

\paragraph*{If using any of the code from this repository, please cite the codebase as follows\+:}

$<$pending final=\char`\"{}\char`\"{} changes=\char`\"{}\char`\"{} to=\char`\"{}\char`\"{} be=\char`\"{}\char`\"{} made$>$=\char`\"{}\char`\"{}$>$

\paragraph*{Additionally the original work for the D\+S\+M\+A\+CC model was accomplished by Emmerson and Evans, which should also be cited\+:}

Emmerson, KM; Evans, MJ (2009) Comparison of tropospheric gas-\/phase chemistry schemes for use within global models, {\itshape A\+T\+M\+OS C\+H\+EM P\+H\+YS}, {\bfseries 9(5)}, pp1831-\/1845 \href{http://dx.doi.org/10.5194/acp-9-1831-2009}{\tt doi\+: 10.\+5194/acp-\/9-\/1831-\/2009} .

\subsection*{Install}

To install we may use conda coupled with the (Yet Another Markup Language) file.

$\ast$-- This will soon be changed into a setup.\+py file --$\ast$


\begin{DoxyCode}
export MPICC=\(\backslash\)`which mpicc\(\backslash\)` &&
export CC=mpicc &&
conda-env create -f py3.yaml
\end{DoxyCode}
 And to use this, we run (or add within our .bashrc)\+: {\ttfamily source activate dsmacc-\/env}

\subsection*{testing}

To test run {\ttfamily make test} or {\ttfamily pytest dsmacc/test/}.

\subsection*{Check Version, Species or Equations}

To check the version, species or equations in a model you can run {\ttfamily ./model 0 0} with the parameters {\ttfamily -\/-\/species}, {\ttfamily -\/-\/equations} or {\ttfamily -\/-\/version} -\/ see example below.


\begin{DoxyCode}
./model 0 0 --species
 NA             SA             SO3            O1D            CL             
 CH4            H2O2           HSO3           H2             N2O5           
 CH3O2NO2       HONO           CH3OH          CO             SO2            
 HO2NO2         CH3O           CH3OOH         CH3NO3         HNO3           
 HCHO           CH3O2          HO2            O3             OH             
 NO3            O              NO             NO2            EMISS          
 R              DUMMY     
\end{DoxyCode}


\subsection*{S\+P\+I\+N\+UP}

There are two methods of spinning up a model-\/ these are with or without constraint to observations. To activate this the python -\/run library must be run with the spinup flag. {\itshape N\+O\+TE\+: using a negative time no longer does anything}

\subsubsection*{Without observations}

{\ttfamily python -\/m dsmacc.\+run -\/s -\/c -\/r} (spinup, create\+\_\+new, run)

This runs an iterative reset of the diurnal cycle until the avarage difference between {\ttfamily sum(old-\/new)/new} concentrations for each species is less than 1e-\/3. N\+O\+TE -\/ this can potentially lead to an infinitely long simulation if the model does not converge on a steady state simulation.

On each restart, the concentrations from the initial conditions file are reset.

\subsubsection*{With observations}

Set up the observations file as before. {\ttfamily python -\/m dsmacc.\+run -\/s -\/o -\/c -\/r} (spinup, observations, create\+\_\+new, run)

\subsubsection*{setting the \textquotesingle{}S\+P\+I\+N\+UP\textquotesingle{} variable in the initial conditions file.}

In running using the spinup flag, this is automatically reset to 0, and the diurnal cycle used. In the case that observations are used without a diurnal spinup flag, this variable can be used to determine how long the simulation is constrained to observational amounts before letting the model run free. See the observations section. \mbox{[}Note this really should be renamed within the initial conditions file.\mbox{]}

\subsection*{Initial conditions qwerks}


\begin{DoxyItemize}
\item Species with 0 or negative concentration values are ignored.
\item Species starting with X are ignored
\item D\+E\+P\+OS is a multiplier flag to determine rate of deposition (0 to disable)
\item E\+M\+I\+SS is a multiplier flag to determine rate of emission (0 to disable)
\item For multiple runs, a description helps with identification.
\item (disabled) save parameter in description name to access archived (compiled) model setups. This allows a single initial conditions file to run multiple mechanisms for comparison.
\end{DoxyItemize}

\subsection*{Model debugging}

f90 model output is presented in the temp.\+txt file. This should be your first point of call for problems with no visible output.

It is then important to check that an initial conditions file {\ttfamily Init\+\_\+cons.\+dat} has been created, and that the model has been compiled. Try {\ttfamily ./model 0 0 -\/-\/version} and {\ttfamily ./model 0 0} to run the first set of initial conditions.

\subsubsection*{Compiler Notes}

The intel compiler is preferable, although the makefile has been rewritten to fall back to gfortran should this not be available. In the rare case where ifort is installed, but not functional, you may have to either comment {\ttfamily \#intel \+:= \$(shell command -\/v ifort 2$>$ /dev/null)} within the Makefile (which disables the switch) or uninstall it for gfortran to be used.

\#-\/-\/-\/-\/-\/-\/-\/-\/-\/-\/-\/-\/-\/-\/-\/-\/-\/-\/-\/-\/-\/-\/-\/-\/-\/-\/-\/-\/-\/-\/-\/-\/-\/-\/-\/-\/-\/-\/-\/-\/-\/-\/---

\section*{dsmacc python library}


\begin{DoxyItemize}
\item python 3 hassle
\item test scripts
\item used to run parallel instances
\item create constraints to observations
\item preparse kpp mechanisms
\item diagnostics and read tools
\end{DoxyItemize}

\subsection*{reformat kppfiles}


\begin{DoxyItemize}
\item {\ttfamily make kpp} or -\/{\ttfamily python -\/m dsmacc.\+parsekpp.\+reformat.\+py} (then use the ncurses interface -\/ arrow keys, space and enter)
\end{DoxyItemize}

\subsection*{Create a new run -\/ and execute}

-\/{\ttfamily python -\/m dsmacc.\+run -\/r -\/c} (run, create)

\subsection*{Run last modified intial conditions file (useful for testing)}

-\/{\ttfamily python -\/m dsmacc.\+run -\/r -\/c -\/l} (run, create, last)

\subsection*{Spinup until steady state}


\begin{DoxyItemize}
\item set spinup time in ics
\item {\ttfamily python -\/m dsmacc.\+run -\/r -\/c -\/s}
\end{DoxyItemize}

\subsection*{Observation constrains}


\begin{DoxyItemize}
\item create the required files in format....
\item {\ttfamily python -\/m \mbox{\hyperlink{namespacedsmacc_1_1observations_1_1constrain}{dsmacc.\+observations.\+constrain}} $<$csvfilenamewithdata$>$}
\item {\ttfamily python -\/m dsmacc.\+run -\/r -\/c -\/o}
\end{DoxyItemize}

Try the wiki -\/ also in progress but contains some debug tips.

T\+UV repository updated with thanks to 

\subsection*{Setting up files}


\begin{DoxyEnumerate}
\item Download organic mechanism from mcm.\+york.\+ac.\+uk.
\item Place file in mechanisms folder (and optionally add a version name\+: `V\+E\+RS=\textquotesingle{}Tropospheric\+Chemistry'{\ttfamily }
\item {\ttfamily Reformat this to keep K\+PP happy. Use}make reformat{\ttfamily or}python -\/m dsmacc.\+parsekpp.\+reformat$<$tt$>$for a quick format with additional deposition rates of 1/day.
\item runmake kpp{\ttfamily }
\item {\ttfamily run}make\`{} to compile.
\end{DoxyEnumerate}

\subsection*{Running a model}


\begin{DoxyEnumerate}
\item Set up the initial conditions csv file
\item To quickrun the model type {\ttfamily make run} or {\ttfamily python -\/m dsmacc.\+run -\/c -\/r}
\end{DoxyEnumerate}

\subsection*{Install}

\paragraph*{General}

To install we may use the (Yet Another Markup Language) file.


\begin{DoxyCode}
export MPICC=\(\backslash\)`which mpicc\(\backslash\)` &&
export CC=mpicc &&
conda-env create -f meta.yaml
\end{DoxyCode}


And to use this, we run (or add within our .bashrc)\+:

{\ttfamily source activate dsmacc-\/env}

\subsubsection*{Parallel Libraries}

If parallel installs fail, remove the conda installs, then follow the instructions below.

\paragraph*{mpi4py}

First we make sure the correct modules are loaded\+: {\ttfamily module load intel-\/mpi/intel/....}

Set the loaded version of M\+PI to be used with mpi4py {\ttfamily export M\+P\+I\+CC=\textbackslash{}}which mpicc\`{}\`{}

Then run {\ttfamily pip install mpi4py}

\subsubsection*{Parallel h5py -\/ not enabled as none of the york hpc clusters have been configured to do this}

1 Build hdf5 library with the following flags (note many clusters dont seem to do this for some reason) {\ttfamily \$./configure -\/-\/enable-\/parallel -\/-\/enable-\/shared} Note that --enable-\/shared is required.


\begin{DoxyCode}
$ h5cc -showconfig
\end{DoxyCode}



\begin{DoxyCode}
$ export CC=mpicc
$ python setup.py configure --mpi [--hdf5=/path/to/parallel/hdf5]
$ python setup.py build
\end{DoxyCode}


Notes \+:
\begin{DoxyItemize}
\item Cannot constrain to 0 due to spinup conditions, either use giant sink or F\+IX species \mbox{[}util.\+inc\mbox{]}
\end{DoxyItemize}

If mpirun failes with \mbox{[}\mbox{]} then run has failed.

\section*{Updates to run procedures}

Compile and prep as before, these changes only affect the running. Filenames may have to be manually changed for the time being, .... sorry.

To create ics\+: {\ttfamily python run.\+py -\/c}

This makes a hdf5 file containing all your information.

To run\+: {\ttfamily python run.\+py -\/s}

If the env variable N\+C\+P\+US is set, it uses this for an mpi run of the model, else a serial run is set. On earth, each queue automatically sets the N\+C\+P\+US environment variable.

To read\+: in ipython `run zhdf; a = new(\textquotesingle{}yourfilename.\+hdf'); a.\+specs / rates / flux\`{}

Custom mydepos definition file in src -\/ change depos without having to run kpp, just make

\section*{Install and run kpp as before! Run / read model using above}

\subsection*{New user}

Run {\ttfamily make new} to clean everything, update latest T\+UV, and download K\+PP. {\bfseries In order to initialise all submodules correctly, you need to have a clean repository.}

\subsection*{Updating the submodule (T\+UV and K\+PP)}

This needs to be done to include contents here. Can be accomplished through {\ttfamily git submodule init; git submodule update} or typing {\ttfamily make update\+\_\+submodule}

\subsection*{Makefile}

Type {\ttfamily make man} to see a description of available functions. 