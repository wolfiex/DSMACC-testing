Codesign Electron mac\+OS apps

\subsection*{About}

\href{https://github.com/electron-userland/electron-osx-sign}{\tt {\ttfamily electron-\/osx-\/sign}} minimizes the extra work needed to eventually prepare your apps for shipping, providing the most basic tools and assets. Note that the bare necessities here are sufficient for enabling app sandbox, yet other configurations for network access etc. require additional work.

Check out \href{https://mintkit.net/electron-userland/electron-osx-sign/guide/}{\tt {\ttfamily electron-\/osx-\/sign} guide} for suggestions on setting up your environment and workflow for distribution or development.

Please visit our \href{https://github.com/electron-userland/electron-osx-sign/wiki}{\tt wiki} for walk-\/throughs, notes and \href{https://github.com/electron-userland/electron-osx-sign/wiki/FAQ}{\tt frequently asked questions} from past projects shipped with \href{https://github.com/electron-userland/electron-packager}{\tt {\ttfamily electron-\/packager}} and \href{https://github.com/electron-userland/electron-osx-sign}{\tt {\ttfamily electron-\/osx-\/sign}}.

$\ast$\+NB\+: Since \href{https://github.com/electron-userland/electron-osx-sign}{\tt {\ttfamily electron-\/osx-\/sign}} injects the entry {\ttfamily com.\+apple.\+security.\+application-\/groups} into the entitlements file as part of the pre-\/signing process, this would reportedly limit app transfer on i\+Tunes Connect (see \href{https://github.com/electron-userland/electron-osx-sign/issues/150}{\tt \#150}). However, opting out entitlements automation `opts\mbox{[}\textquotesingle{}pre-\/auto-\/entitlements'\mbox{]} === false\`{} may result in worse graphics performance.$\ast$

{\itshape The signing procedure implemented in this package is based on what described in https\+://github.com/atom/electron/blob/master/docs/tutorial/mac-\/app-\/store-\/submission-\/guide.\+md \char`\"{}\+Mac App Store Submission Guide\char`\"{}.}

\subsubsection*{\href{https://github.com/electron/electron}{\tt Electron}}

It is worth noting as well that starting from \href{https://github.com/electron/electron}{\tt Electron} v1.\+1.\+1, a new mechanism was introduced to allow I\+PC in App Sandbox (see \href{https://github.com/electron/electron/pull/5601}{\tt electron\#5601}); wish to have full support of legacy Electron versions, please utilize {\ttfamily opts.\+version}, which option brings less hassle with making default settings among Electron builds.

We are trying to keep updated to the latest \href{https://github.com/electron/electron}{\tt Electron} specs; please \href{https://github.com/electron-userland/electron-osx-sign/issues/new}{\tt file us an issue} if having any suggestions or experiencing difficulties code signing your products.

\subsubsection*{An \href{http://openopensource.org/}{\tt O\+P\+EN Open Source Project}}

Individuals making significant and valuable contributions are given commit-\/access to the project to contribute as they see fit. This project is more like an open wiki than a standard guarded open source project.

\subsubsection*{Collaborators}

Thanks to \href{https://github.com/seanchas116}{\tt seanchas116}, \href{https://github.com/jasonhinkle}{\tt jasonhinkle}, and \href{https://github.com/develar}{\tt develar} for improving the usability of this project implementation.

\subsection*{Installation}


\begin{DoxyCode}
# For use in npm scripts
npm install --save electron-osx-sign
\end{DoxyCode}



\begin{DoxyCode}
# For use from CLI
npm install -g electron-osx-sign
\end{DoxyCode}


{\itshape Note\+: {\ttfamily electron-\/osx-\/sign} is a dependency of {\ttfamily electron-\/packager} as of 6.\+0.\+0 for signing apps on mac\+OS. However, feel free to install this package globally for more customization beyond specifying identity and entitlements.}

\subsection*{Usage}

\subsubsection*{electron-\/osx-\/sign}

\paragraph*{From the Command Line}


\begin{DoxyCode}
electron-osx-sign app [embedded-binary ...] [options ...]
\end{DoxyCode}


\subparagraph*{Examples}

Since {\ttfamily electron-\/osx-\/sign} adds the entry {\ttfamily com.\+apple.\+developer.\+team-\/identifier} to a temporary copy of the specified entitlements file (with the default option {\ttfamily -\/-\/pre-\/auto-\/entitlements}) distribution builds can no longer be run directly. To run the app codesigned for distribution locally after codesigning, you may manually add {\ttfamily Electron\+Team\+ID} in your {\ttfamily Info.\+plist} and {\ttfamily com.\+apple.\+security.\+application-\/groups} in the entitlements file, and provide the flag {\ttfamily -\/-\/no-\/pre-\/auto-\/entitlements} for {\ttfamily electron-\/osx-\/sign} to avoid this extra bit. Note that \char`\"{}certain features are only allowed across apps whose team-\/identifier value match\char`\"{} (\href{https://developer.apple.com/library/content/technotes/tn2415/_index.html#//apple_ref/doc/uid/DTS40016427-CH1-ENTITLEMENTSLIST}{\tt Technical Note T\+N2415}).

The examples below assume that {\ttfamily -\/-\/pre-\/auto-\/entitlements} is enabled.


\begin{DoxyItemize}
\item To sign a distribution version by default\+: 
\begin{DoxyCode}
electron-osx-sign path/to/my.app
\end{DoxyCode}
 For distribution in the Mac App Store\+: Have the provisioning profile for distribution placed in the current working directory and the signing identity installed in the default keychain. {\itshape The app is not expected to run after codesigning since there is no provisioned device, and it is intended only for submission to i\+Tunes Connect.} For distribution outside the Mac App Store\+: Have the signing identity for distribution installed in the default keychain and optionally place the provisioning profile in the current working directory. By default App Sandbox is not enabled. {\itshape The app should run on all devices.}
\item To sign development version\+: 
\begin{DoxyCode}
electron-osx-sign path/to/my.app --type=development
\end{DoxyCode}
 For testing Mac App Store builds\+: Have the provisioning profile for development placed in the current working directory and the signing identity installed in the default keychain. {\itshape The app will only run on provisioned devices.} For testing apps for distribution outside the Mac App Store, have the signing identity for development installed in the default keychain and optionally the provisioning profile placed in the current working directory. {\itshape The app will only run on provisioned devices.} However, you may prefer to just go with signing a distribution version because the app is expected to launch properly after codesigned.
\item It is recommended to place the provisioning profile(s) under the working directory for {\ttfamily electron-\/osx-\/sign} to pick up automatically; however, to specify provisioning profile to be embedded explicitly\+: 
\begin{DoxyCode}
electron-osx-sign path/to/my.app --provisioning-profile=path/to/my.provisionprofile
\end{DoxyCode}

\item To specify the entitlements file\+: 
\begin{DoxyCode}
electron-osx-sign path/to/my.app --entitlements=path/to/my.entitlements
\end{DoxyCode}

\item It is recommended to make use of {\ttfamily -\/-\/version} while signing legacy versions of Electron\+: 
\begin{DoxyCode}
electron-osx-sign path/to/my.app --version=0.34.0
\end{DoxyCode}

\end{DoxyItemize}

Run {\ttfamily electron-\/osx-\/sign -\/-\/help} or see \href{https://github.com/electron-userland/electron-osx-sign/blob/master/bin/electron-osx-sign-usage.txt}{\tt electron-\/osx-\/sign-\/usage.\+txt} for C\+L\+I-\/specific options.

\paragraph*{From the A\+PI}


\begin{DoxyCode}
var sign = require('electron-osx-sign')
sign(opts[, function done (err) \{\}])
\end{DoxyCode}


Example\+:


\begin{DoxyCode}
var sign = require('electron-osx-sign')
sign(\{
  app: 'path/to/my.app'
\}, function done (err) \{
  if (err) \{
    // Handle the error
    return;
  \}
  // Application signed
\})
\end{DoxyCode}


From release v0.\+4.\+0-\/beta, \href{https://github.com/petkaantonov/bluebird}{\tt Bluebird} promises are introduced for better async method calls; the following is also available for use.


\begin{DoxyCode}
var signAsync = require('electron-osx-sign').signAsync
signAsync(opts)
  [.then(function () \{\})]
  [.catch(function (err) \{\})]
\end{DoxyCode}


Example\+:


\begin{DoxyCode}
var signAsync = require('electron-osx-sign').signAsync
signAsync(\{
  app: 'path/to/my.app'
\})
  .then(function () \{
    // Application signed
  \})
  .catch(function (err) \{
    // Handle the error
  \})
\end{DoxyCode}


\subparagraph*{opts -\/ Options}

{\bfseries Required}

{\ttfamily app} -\/ {\itshape String}

Path to the application package. Needs file extension {\ttfamily .app}.

{\bfseries Optional}

{\ttfamily binaries} -\/ {\itshape Array}

Path to additional binaries that will be signed along with built-\/ins of Electron. Default to {\ttfamily undefined}.

{\ttfamily entitlements} -\/ {\itshape String}

Path to entitlements file for signing the app. Default to built-\/in entitlements file, Sandbox enabled for Mac App Store platform. See \href{https://github.com/electron-userland/electron-osx-sign/blob/master/default.entitlements.mas.plist}{\tt default.\+entitlements.\+mas.\+plist} or \href{https://github.com/electron-userland/electron-osx-sign/blob/master/default.entitlements.darwin.plist}{\tt default.\+entitlements.\+darwin.\+plist} with respect to your platform.

{\ttfamily entitlements-\/inherit} -\/ {\itshape String}

Path to child entitlements which inherit the security settings for signing frameworks and bundles of a distribution. {\itshape This option only applies when signing with entitlements.} See \href{https://github.com/electron-userland/electron-osx-sign/blob/master/default.entitlements.mas.inherit.plist}{\tt default.\+entitlements.\+mas.\+inherit.\+plist} or \href{https://github.com/electron-userland/electron-osx-sign/blob/master/default.entitlements.darwin.inherit.plist}{\tt default.\+entitlements.\+darwin.\+inherit.\+plist} with respect to your platform.

{\ttfamily gatekeeper-\/assess} -\/ {\itshape Boolean}

Flag to enable/disable Gatekeeper assessment after signing the app. Disabling it is useful for signing with self-\/signed certificates. Gatekeeper assessment is enabled by default on {\ttfamily darwin} platform. Default to {\ttfamily true}.

{\ttfamily hardened\+Runtime} or {\ttfamily hardened-\/runtime} -\/ {\itshape Boolean}

Flag to enable the Mojave hardened runtime when signing the app. Disabled by default, requires Xcode $>$= 10 and mac\+OS $>$= 10.\+13.\+6.

{\ttfamily identity} -\/ {\itshape String}

Name of certificate to use when signing. Default to be selected with respect to {\ttfamily provisioning-\/profile} and {\ttfamily platform} from {\ttfamily keychain} or keychain by system default.

Signing platform {\ttfamily mas} will look for {\ttfamily 3rd Party Mac Developer Application\+: $\ast$ ($\ast$)}, and platform {\ttfamily darwin} will look for {\ttfamily Developer ID Application\+: $\ast$ ($\ast$)} by default.

{\ttfamily identity-\/validation} -\/ {\itshape Boolean}

Flag to enable/disable validation for the signing identity. If enabled, the {\ttfamily identity} provided will be validated in the {\ttfamily keychain} specified. Default to {\ttfamily true}.

{\ttfamily keychain} -\/ {\itshape String}

The keychain name. Default to system default keychain.

{\ttfamily ignore} -\/ {\itshape Reg\+Exp$\vert$\+Function$\vert$\+Array.$<$(Reg\+Exp$\vert$\+Function)$>$}

Regex, function or an array of regex\textquotesingle{}s and functions that signal skipping signing a file. Elements of other types are treated as {\ttfamily Reg\+Exp}. Default to {\ttfamily undefined}.

{\ttfamily platform} -\/ {\itshape String}

Build platform of Electron. Allowed values\+: {\ttfamily darwin}, {\ttfamily mas}. Default to auto detect by presence of {\ttfamily Squirrel.\+framework} within the application bundle.

{\ttfamily pre-\/auto-\/entitlements} -\/ {\itshape Boolean}

Flag to enable/disable automation of {\ttfamily com.\+apple.\+security.\+application-\/groups} in entitlements file and update {\ttfamily Info.\+plist} with {\ttfamily Electron\+Team\+ID}. Default to {\ttfamily true}.

{\ttfamily pre-\/embed-\/provisioning-\/profile} -\/ {\itshape Boolean}

Flag to enable/disable embedding of provisioning profile in the current working directory. Default to {\ttfamily true}.

{\ttfamily provisioning-\/profile} -\/ {\itshape String}

Path to provisioning profile.

{\ttfamily requirements} -\/ {\itshape String}

Specify the criteria that you recommend to be used to evaluate the code signature. See more info from \href{https://developer.apple.com/library/mac/documentation/Security/Conceptual/CodeSigningGuide/RequirementLang/RequirementLang.html}{\tt https\+://developer.\+apple.\+com/library/mac/documentation/\+Security/\+Conceptual/\+Code\+Signing\+Guide/\+Requirement\+Lang/\+Requirement\+Lang.\+html} Default to {\ttfamily undefined}.

{\ttfamily restrict} -\/ {\itshape Boolean}

{\bfseries To be deprecated, see {\ttfamily signature-\/flags}.} Restrict dyld loading. See doc about this \href{https://developer.apple.com/documentation/security/seccodesignatureflags/kseccodesignaturerestrict?language=objc}{\tt code signature flag} for more details. Disabled by default.

{\ttfamily signature-\/flags} -\/ {\itshape String} Comma separated string or array for \href{https://developer.apple.com/documentation/security/seccodesignatureflags?language=objc}{\tt code signature flag}. Default to {\ttfamily undefined}.

{\ttfamily strict-\/verify} -\/ {\itshape Boolean$\vert$\+String$\vert$\+Array.$<$\+String$>$}

Flag to enable/disable {\ttfamily -\/-\/strict} flag when verifying the signed application bundle. If provided as a string, each component should be separated with comma ({\ttfamily ,}). If provided as an array, each item should be a string corresponding to a component. Default to {\ttfamily true}.

{\ttfamily timestamp} -\/ {\itshape String}

Specify the U\+RL of the timestamp authority server, default to server provided by Apple. Please note that this default server may not support signatures not furnished by Apple. Disable the timestamp service with {\ttfamily none}.

{\ttfamily type} -\/ {\itshape String}

Specify whether to sign app for development or for distribution. Allowed values\+: {\ttfamily development}, {\ttfamily distribution}. Default to {\ttfamily distribution}.

{\ttfamily version} -\/ {\itshape String}

Build version of Electron. Values may be like\+: {\ttfamily 1.\+1.\+1}, {\ttfamily 1.\+2.\+0}. Default to latest Electron version.

It is recommended to utilize this option for best support of specific Electron versions. This may trigger pre/post operations for signing\+: For example, automation of setting {\ttfamily com.\+apple.\+security.\+application-\/groups} in entitlements file and of updating {\ttfamily Info.\+plist} with {\ttfamily Electron\+Team\+ID} is enabled for all versions starting from {\ttfamily 1.\+1.\+1}; set {\ttfamily pre-\/auto-\/entitlements} option to {\ttfamily false} to disable this feature.

\subparagraph*{cb -\/ Callback}

{\ttfamily err} -\/ {\itshape Error}

\subsubsection*{electron-\/osx-\/flat}

\paragraph*{From the Command Line}


\begin{DoxyCode}
electron-osx-flat app [options ...]
\end{DoxyCode}


Example\+:


\begin{DoxyCode}
electron-osx-flat path/to/my.app
\end{DoxyCode}


Run {\ttfamily electron-\/osx-\/flat -\/-\/help} or see \href{https://github.com/electron-userland/electron-osx-sign/blob/master/bin/electron-osx-flat-usage.txt}{\tt electron-\/osx-\/flat-\/usage.\+txt} for C\+L\+I-\/specific options.

\paragraph*{From the A\+PI}


\begin{DoxyCode}
var flat = require('electron-osx-sign').flat
flat(opts[, function done (err) \{\}])
\end{DoxyCode}


Example\+:


\begin{DoxyCode}
var flat = require('electron-osx-sign').flat
flat(\{
  app: 'path/to/my.app'
\}, function done (err) \{
  if (err) \{
    // Handle the error
    return;
  \}
  // Application flattened
\})
\end{DoxyCode}


From release v0.\+4.\+0-\/beta, \href{https://github.com/petkaantonov/bluebird}{\tt Bluebird} promises are introduced for better async method calls; the following is also available for use.


\begin{DoxyCode}
var flatAsync = require('electron-osx-sign').flatAsync
flatAsync(opts)
  [.then(function () \{\})]
  [.catch(function (err) \{\})]
\end{DoxyCode}


Example\+:


\begin{DoxyCode}
var flatAsync = require('electron-osx-sign').flatAsync
flatAsync(\{
  app: 'path/to/my.app'
\})
  .then(function () \{
    // Application flattened
  \})
  .catch(function (err) \{
    // Handle the error
  \})
\end{DoxyCode}


\subparagraph*{opts -\/ Options}

{\bfseries Required}

{\ttfamily app} -\/ {\itshape String}

Path to the application bundle. Needs file extension {\ttfamily .app}.

{\bfseries Optional}

{\ttfamily identity} -\/ {\itshape String}

Name of certificate to use when signing. Default to be selected with respect to {\ttfamily platform} from {\ttfamily keychain} or keychain by system default.

Flattening platform {\ttfamily mas} will look for {\ttfamily 3rd Party Mac Developer Installer\+: $\ast$ ($\ast$)}, and platform {\ttfamily darwin} will look for {\ttfamily Developer ID Installer\+: $\ast$ ($\ast$)} by default.

{\ttfamily identity-\/validation} -\/ {\itshape Boolean}

Flag to enable/disable validation for signing identity. If enabled, the {\ttfamily identity} provided will be validated in the {\ttfamily keychain} specified. Default to {\ttfamily true}.

{\ttfamily install} -\/ {\itshape String}

Path to install the bundle. Default to {\ttfamily /\+Applications}.

{\ttfamily keychain} -\/ {\itshape String}

The keychain name. Default to system default keychain.

{\ttfamily platform} -\/ {\itshape String}

Build platform of Electron. Allowed values\+: {\ttfamily darwin}, {\ttfamily mas}. Default to auto detect by presence of {\ttfamily Squirrel.\+framework} within the application bundle.

{\ttfamily pkg} -\/ {\itshape String}

Path to the output the flattened package. Needs file extension {\ttfamily .pkg}.

{\ttfamily scripts} -\/ {\itshape String} Path to a directory containing pre and/or post install scripts.

\subparagraph*{cb -\/ Callback}

{\ttfamily err} -\/ {\itshape Error}

\subsection*{Debug}

As of release v0.\+3.\+1, external module {\ttfamily debug} is used to display logs and messages; remember to {\ttfamily export D\+E\+B\+UG=electron-\/osx-\/sign$\ast$} when necessary.

\subsection*{Test}

The project\textquotesingle{}s configured to run automated tests on Circle\+CI.

If you wish to manually test the module, first comment out {\ttfamily opts.\+identity} in {\ttfamily test/basic.\+js} to enable auto discovery. Then run the command {\ttfamily npm test} from the dev directory.

When this command is run for the first time\+: {\ttfamily electron-\/download} will download mac\+OS Electron releases defined in {\ttfamily test/config.\+json}, and save to {\ttfamily $\sim$/.electron/}, which might take up less than 1\+GB of disk space.

A successful testing should look something like\+:


\begin{DoxyCode}
$ npm test

> electron-osx-sign@0.4.14 pretest electron-osx-sign
> rimraf test/work

> electron-osx-sign@0.4.14 test electron-osx-sign
> standard && tape test

Calling electron-download before running tests...
Running tests...
TAP version 13
# setup
# defaults-test:v7.0.0-beta.3-darwin-x64
ok 1 app signed
# defaults-test:v7.0.0-beta.3-mas-x64
ok 2 app signed
# defaults-test:v6.0.3-darwin-x64
ok 3 app signed
# defaults-test:v6.0.3-mas-x64
ok 4 app signed
# defaults-test:v5.0.10-darwin-x64
ok 5 app signed
# defaults-test:v5.0.10-mas-x64
ok 6 app signed
# defaults-test:v4.2.9-darwin-x64
ok 7 app signed
# defaults-test:v4.2.9-mas-x64
ok 8 app signed
# defaults-test:v3.1.2-darwin-x64
ok 9 app signed
# defaults-test:v3.1.2-mas-x64
ok 10 app signed
# teardown

1..10
# tests 10
# pass  10

# ok
\end{DoxyCode}


\subsection*{Related}


\begin{DoxyItemize}
\item \href{https://github.com/electron-userland/electron-packager}{\tt electron-\/packager} -\/ Package your electron app in OS executables (.app, .exe, etc) via JS or C\+LI
\item \href{https://github.com/electron-userland/electron-builder}{\tt electron-\/builder} -\/ A complete solution to package and build a ready for distribution Electron app with “auto update” support out of the box 
\end{DoxyItemize}