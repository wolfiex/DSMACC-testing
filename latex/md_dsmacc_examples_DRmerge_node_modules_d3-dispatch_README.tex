Dispatching is a convenient mechanism for separating concerns with loosely-\/coupled code\+: register named callbacks and then call them with arbitrary arguments. A variety of D3 components, such as \href{https://github.com/d3/d3-request}{\tt d3-\/request}, use this mechanism to emit events to listeners. Think of this like Node’s \href{https://nodejs.org/api/events.html}{\tt Event\+Emitter}, except every listener has a well-\/defined name so it’s easy to remove or replace them.

For example, to create a dispatch for {\itshape start} and {\itshape end} events\+:


\begin{DoxyCode}
var dispatch = d3.dispatch("start", "end");
\end{DoxyCode}


You can then register callbacks for these events using \href{#dispatch_on}{\tt {\itshape dispatch}.on}\+:


\begin{DoxyCode}
dispatch.on("start", callback1);
dispatch.on("start.foo", callback2);
dispatch.on("end", callback3);
\end{DoxyCode}


Then, you can invoke all the {\itshape start} callbacks using \href{#dispatch_call}{\tt {\itshape dispatch}.call} or \href{#dispatch_apply}{\tt {\itshape dispatch}.apply}\+:


\begin{DoxyCode}
dispatch.call("start");
\end{DoxyCode}


Like {\itshape function}.call, you may also specify the {\ttfamily this} context and any arguments\+:


\begin{DoxyCode}
dispatch.call("start", \{about: "I am a context object"\}, "I am an argument");
\end{DoxyCode}


Want a more involved example? See how to use \href{http://bl.ocks.org/mbostock/5872848}{\tt d3-\/dispatch for coordinated views}.

\subsection*{Installing}

If you use N\+PM, {\ttfamily npm install d3-\/dispatch}. Otherwise, download the \href{https://github.com/d3/d3-dispatch/releases/latest}{\tt latest release}. You can also load directly from \href{https://d3js.org}{\tt d3js.\+org}, either as a \href{https://d3js.org/d3-dispatch.v1.min.js}{\tt standalone library} or as part of \href{https://github.com/d3/d3}{\tt D3 4.\+0}. A\+MD, Common\+JS, and vanilla environments are supported. In vanilla, a {\ttfamily d3} global is exported\+:


\begin{DoxyCode}
<script src="https://d3js.org/d3-dispatch.v1.min.js"></script>
<script>

var dispatch = d3.dispatch("start", "end");

</script>
\end{DoxyCode}


\href{https://tonicdev.com/npm/d3-dispatch}{\tt Try d3-\/dispatch in your browser.}

\subsection*{A\+PI Reference}

\label{_dispatch}%
\# d3.{\bfseries dispatch}({\itshape types…}) \href{https://github.com/d3/d3-dispatch/blob/master/src/dispatch.js}{\tt $<$$>$}

Creates a new dispatch for the specified event {\itshape types}. Each {\itshape type} is a string, such as {\ttfamily \char`\"{}start\char`\"{}} or {\ttfamily \char`\"{}end\char`\"{}}.

\label{_dispatch_on}%
\# {\itshape dispatch}.{\bfseries on}({\itshape typenames}\mbox{[}, {\itshape callback}\mbox{]}) \href{https://github.com/d3/d3-dispatch/blob/master/src/dispatch.js#L26}{\tt $<$$>$}

Adds, removes or gets the {\itshape callback} for the specified {\itshape typenames}. If a {\itshape callback} function is specified, it is registered for the specified (fully-\/qualified) {\itshape typenames}. If a callback was already registered for the given {\itshape typenames}, the existing callback is removed before the new callback is added.

The specified {\itshape typenames} is a string, such as {\ttfamily start} or {\ttfamily end.\+foo}. The type may be optionally followed by a period ({\ttfamily .}) and a name; the optional name allows multiple callbacks to be registered to receive events of the same type, such as {\ttfamily start.\+foo} and {\ttfamily start.\+bar}. To specify multiple typenames, separate typenames with spaces, such as {\ttfamily start end} or {\ttfamily start.\+foo start.\+bar}.

To remove all callbacks for a given name {\ttfamily foo}, say {\ttfamily dispatch.\+on(\char`\"{}.\+foo\char`\"{}, null)}.

If {\itshape callback} is not specified, returns the current callback for the specified {\itshape typenames}, if any. If multiple typenames are specified, the first matching callback is returned.

\label{_dispatch_copy}%
\# {\itshape dispatch}.{\bfseries copy}() \href{https://github.com/d3/d3-dispatch/blob/master/src/dispatch.js#L49}{\tt $<$$>$}

Returns a copy of this dispatch object. Changes to this dispatch do not affect the returned copy and {\itshape vice versa}.

\label{_dispatch_call}%
\# {\itshape dispatch}.{\bfseries call}({\itshape type}\mbox{[}, {\itshape that}\mbox{[}, {\itshape arguments…}\mbox{]}\mbox{]}) \href{https://github.com/d3/d3-dispatch/blob/master/src/dispatch.js#L54}{\tt $<$$>$}

Like \href{https://developer.mozilla.org/en-US/docs/Web/JavaScript/Reference/Global_Objects/Function/call}{\tt {\itshape function}.call}, invokes each registered callback for the specified {\itshape type}, passing the callback the specified {\itshape arguments}, with {\itshape that} as the {\ttfamily this} context. See \href{#dispatch_apply}{\tt {\itshape dispatch}.apply} for more information.

\label{_dispatch_apply}%
\# {\itshape dispatch}.{\bfseries apply}({\itshape type}\mbox{[}, {\itshape that}\mbox{[}, {\itshape arguments}\mbox{]}\mbox{]}) \href{https://github.com/d3/d3-dispatch/blob/master/src/dispatch.js#L59}{\tt $<$$>$}

Like \href{https://developer.mozilla.org/en-US/docs/Web/JavaScript/Reference/Global_Objects/Function/call}{\tt {\itshape function}.apply}, invokes each registered callback for the specified {\itshape type}, passing the callback the specified {\itshape arguments}, with {\itshape that} as the {\ttfamily this} context. For example, if you wanted to dispatch your {\itshape custom} callbacks after handling a native {\itshape click} event, while preserving the current {\ttfamily this} context and arguments, you could say\+:


\begin{DoxyCode}
selection.on("click", function() \{
  dispatch.apply("custom", this, arguments);
\});
\end{DoxyCode}


You can pass whatever arguments you want to callbacks; most commonly, you might create an object that represents an event, or pass the current datum ({\itshape d}) and index ({\itshape i}). See \href{https://developer.mozilla.org/en/JavaScript/Reference/Global_Objects/Function/Call}{\tt function.\+call} and \href{https://developer.mozilla.org/en/JavaScript/Reference/Global_Objects/Function/Apply}{\tt function.\+apply} for further information. 