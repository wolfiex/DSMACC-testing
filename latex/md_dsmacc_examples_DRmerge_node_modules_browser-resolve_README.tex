node.\+js resolve algorithm with \href{https://github.com/defunctzombie/package-browser-field-spec}{\tt browser field} support.

\subsection*{api}

\subsubsection*{resolve(id, opts=\{\}, cb)}

Resolve a module path and call {\ttfamily cb(err, path \mbox{[}, pkg\mbox{]})}

Options\+:


\begin{DoxyItemize}
\item {\ttfamily basedir} -\/ directory to begin resolving from
\item {\ttfamily browser} -\/ the \textquotesingle{}browser\textquotesingle{} property to use from package.\+json (defaults to \textquotesingle{}browser\textquotesingle{})
\item {\ttfamily filename} -\/ the calling filename where the {\ttfamily require()} call originated (in the source)
\item {\ttfamily modules} -\/ object with module id/name -\/$>$ path mappings to consult before doing manual resolution (use to provide core modules)
\item {\ttfamily package\+Filter} -\/ transform the parsed {\ttfamily package.\+json} contents before looking at the {\ttfamily main} field
\item {\ttfamily paths} -\/ {\ttfamily require.\+paths} array to use if nothing is found on the normal {\ttfamily node\+\_\+modules} recursive walk
\end{DoxyItemize}

Options supported by \href{https://github.com/substack/node-resolve#resolveid-opts-cb}{\tt node-\/resolve} can be used.

\subsubsection*{resolve.\+sync(id, opts=\{\})}

Same as the async resolve, just uses sync methods.

Options supported by \href{https://github.com/substack/node-resolve#resolvesyncid-opts}{\tt node-\/resolve} {\ttfamily sync} can be used.

\subsection*{basic usage}

you can resolve files like {\ttfamily require.\+resolve()}\+: 
\begin{DoxyCode}
var resolve = require('browser-resolve');
resolve('../', \{ filename: \_\_filename \}, function(err, path) \{
    console.log(path);
\});
\end{DoxyCode}



\begin{DoxyCode}
$ node example/resolve.js
/home/substack/projects/node-browser-resolve/index.js
\end{DoxyCode}


\subsection*{core modules}

By default, core modules (http, dgram, etc) will return their same name as the path. If you want to have specific paths returned, specify a {\ttfamily modules} property in the options object.


\begin{DoxyCode}
var shims = \{
    http: '/your/path/to/http.js'
\};

var resolve = require('browser-resolve');
resolve('fs', \{ modules: shims \}, function(err, path) \{
    console.log(path);
\});
\end{DoxyCode}



\begin{DoxyCode}
$ node example/builtin.js
/home/substack/projects/node-browser-resolve/builtin/fs.js
\end{DoxyCode}


\subsection*{browser field}

browser-\/specific versions of modules


\begin{DoxyCode}
\{
  "name": "custom",
  "version": "0.0.0",
  "browser": \{
    "./main.js": "custom.js"
  \},
  "chromeapp": \{
    "./main.js": "custom-chromeapp.js"
  \}
\}
\end{DoxyCode}



\begin{DoxyCode}
var resolve = require('browser-resolve');
var parent = \{ filename: \_\_dirname + '/custom/file.js' /*, browser: 'chromeapp' */ \};
resolve('./main.js', parent, function(err, path) \{
    console.log(path);
\});
\end{DoxyCode}



\begin{DoxyCode}
$ node example/custom.js
/home/substack/projects/node-browser-resolve/example/custom/custom.js
\end{DoxyCode}


\subsection*{skip}

You can skip over dependencies by setting a \href{https://gist.github.com/defunctzombie/4339901}{\tt browser field} value to {\ttfamily false}\+:


\begin{DoxyCode}
\{
  "name": "skip",
  "version": "0.0.0",
  "browser": \{
    "tar": false
  \}
\}
\end{DoxyCode}


This is handy if you have code like\+:


\begin{DoxyCode}
var tar = require('tar');

exports.add = function (a, b) \{
    return a + b;
\};

exports.parse = function () \{
    return tar.Parse();
\};
\end{DoxyCode}


so that `require(\textquotesingle{}tar'){\ttfamily will just return}\{\}{\ttfamily in the browser because you don\textquotesingle{}t intend to support the}.parse()\`{} export in a browser environment.


\begin{DoxyCode}
var resolve = require('browser-resolve');
var parent = \{ filename: \_\_dirname + '/skip/main.js' \};
resolve('tar', parent, function(err, path) \{
    console.log(path);
\});
\end{DoxyCode}



\begin{DoxyCode}
$ node example/skip.js
/home/substack/projects/node-browser-resolve/empty.js
\end{DoxyCode}


\section*{license}

M\+IT

\section*{upgrade notes}

Prior to v1.\+x this library provided shims for node core modules. These have since been removed. If you want to have alternative core modules provided, use the {\ttfamily modules} option when calling resolve.

This was done to allow package managers to choose which shims they want to use without browser-\/resolve being the central point of update. 