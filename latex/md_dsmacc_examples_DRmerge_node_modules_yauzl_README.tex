\href{https://travis-ci.org/thejoshwolfe/yauzl}{\tt } \href{https://coveralls.io/r/thejoshwolfe/yauzl}{\tt }

yet another unzip library for node. For zipping, see \href{https://github.com/thejoshwolfe/yazl}{\tt yazl}.

Design principles\+:


\begin{DoxyItemize}
\item Follow the spec. Don\textquotesingle{}t scan for local file headers. Read the central directory for file metadata. (see \href{#no-streaming-unzip-api}{\tt No Streaming Unzip A\+PI}).
\item Don\textquotesingle{}t block the Java\+Script thread. Use and provide async A\+P\+Is.
\item Keep memory usage under control. Don\textquotesingle{}t attempt to buffer entire files in R\+AM at once.
\item Never crash (if used properly). Don\textquotesingle{}t let malformed zip files bring down client applications who are trying to catch errors.
\item Catch unsafe filenames entries. A zip file entry throws an error if its file name starts with {\ttfamily \char`\"{}/\char`\"{}} or {\ttfamily /\mbox{[}A-\/\+Za-\/z\mbox{]}\+:\textbackslash{}//} or if it contains {\ttfamily \char`\"{}..\char`\"{}} path segments or {\ttfamily \char`\"{}\textbackslash{}\textbackslash{}\char`\"{}} (per the spec).
\end{DoxyItemize}

\subsection*{Usage}


\begin{DoxyCode}
var yauzl = require("yauzl");
var fs = require("fs");
var path = require("path");
var mkdirp = require("mkdirp"); // or similar

yauzl.open("path/to/file.zip", \{lazyEntries: true\}, function(err, zipfile) \{
  if (err) throw err;
  zipfile.readEntry();
  zipfile.on("entry", function(entry) \{
    if (/\(\backslash\)/$/.test(entry.fileName)) \{
      // directory file names end with '/'
      mkdirp(entry.fileName, function(err) \{
        if (err) throw err;
        zipfile.readEntry();
      \});
    \} else \{
      // file entry
      zipfile.openReadStream(entry, function(err, readStream) \{
        if (err) throw err;
        // ensure parent directory exists
        mkdirp(path.dirname(entry.fileName), function(err) \{
          if (err) throw err;
          readStream.pipe(fs.createWriteStream(entry.fileName));
          readStream.on("end", function() \{
            zipfile.readEntry();
          \});
        \});
      \});
    \}
  \});
\});
\end{DoxyCode}


\subsection*{A\+PI}

The default for every optional {\ttfamily callback} parameter is\+:


\begin{DoxyCode}
function defaultCallback(err) \{
  if (err) throw err;
\}
\end{DoxyCode}


\subsubsection*{open(path, \mbox{[}options\mbox{]}, \mbox{[}callback\mbox{]})}

Calls {\ttfamily fs.\+open(path, \char`\"{}r\char`\"{})} and gives the {\ttfamily fd}, {\ttfamily options}, and {\ttfamily callback} to {\ttfamily from\+Fd()} below.

{\ttfamily options} may be omitted or {\ttfamily null}. The defaults are {\ttfamily \{auto\+Close\+: true, lazy\+Entries\+: false\}}.

{\ttfamily auto\+Close} is effectively equivalent to\+:


\begin{DoxyCode}
zipfile.once("end", function() \{
  zipfile.close();
\});
\end{DoxyCode}


{\ttfamily lazy\+Entries} indicates that entries should be read only when {\ttfamily read\+Entry()} is called. If {\ttfamily lazy\+Entries} is {\ttfamily false}, {\ttfamily entry} events will be emitted as fast as possible to allow {\ttfamily pipe()}ing file data from all entries in parallel. This is not recommended, as it can lead to out of control memory usage for zip files with many entries. See \href{https://github.com/thejoshwolfe/yauzl/issues/22}{\tt issue \#22}. If {\ttfamily lazy\+Entries} is {\ttfamily true}, an {\ttfamily entry} or {\ttfamily end} event will be emitted in response to each call to {\ttfamily read\+Entry()}. This allows processing of one entry at a time, and will keep memory usage under control for zip files with many entries.

\subsubsection*{from\+Fd(fd, \mbox{[}options\mbox{]}, \mbox{[}callback\mbox{]})}

Reads from the fd, which is presumed to be an open .zip file. Note that random access is required by the zip file specification, so the fd cannot be an open socket or any other fd that does not support random access.

The {\ttfamily callback} is given the arguments {\ttfamily (err, zipfile)}. An {\ttfamily err} is provided if the End of Central Directory Record Signature cannot be found in the file, which indicates that the fd is not a zip file. {\ttfamily zipfile} is an instance of {\ttfamily Zip\+File}.

{\ttfamily options} may be omitted or {\ttfamily null}. The defaults are {\ttfamily \{auto\+Close\+: false, lazy\+Entries\+: false\}}. See {\ttfamily open()} for the meaning of the options.

\subsubsection*{from\+Buffer(buffer, \mbox{[}options\mbox{]}, \mbox{[}callback\mbox{]})}

Like {\ttfamily from\+Fd()}, but reads from a R\+AM buffer instead of an open file. {\ttfamily buffer} is a {\ttfamily Buffer}. {\ttfamily callback} is effectively passed directly to {\ttfamily from\+Fd()}.

If a {\ttfamily Zip\+File} is acquired from this method, it will never emit the {\ttfamily close} event, and calling {\ttfamily close()} is not necessary.

{\ttfamily options} may be omitted or {\ttfamily null}. The defaults are {\ttfamily \{lazy\+Entries\+: false\}}. See {\ttfamily open()} for the meaning of the options. The {\ttfamily auto\+Close} option is ignored for this method.

\subsubsection*{from\+Random\+Access\+Reader(reader, total\+Size, \mbox{[}options\mbox{]}, \mbox{[}callback\mbox{]})}

This method of creating a zip file allows clients to implement their own back-\/end file system. For example, a client might translate read calls into network requests.

The {\ttfamily reader} parameter must be of a type that is a subclass of \href{#class-randomaccessreader}{\tt Random\+Access\+Reader} that implements the required methods. The {\ttfamily total\+Size} is a Number and indicates the total file size of the zip file.

{\ttfamily options} may be omitted or {\ttfamily null}. The defaults are {\ttfamily \{auto\+Close\+: true, lazy\+Entries\+: false\}}. See {\ttfamily open()} for the meaning of the options.

\subsubsection*{dos\+Date\+Time\+To\+Date(date, time)}

Converts M\+S-\/\+D\+OS {\ttfamily date} and {\ttfamily time} data into a Java\+Script {\ttfamily \mbox{\hyperlink{classDate}{Date}}} object. Each parameter is a {\ttfamily Number} treated as an unsigned 16-\/bit integer. Note that this format does not support timezones, so the returned object will use the local timezone.

\subsubsection*{Class\+: Zip\+File}

The constructor for the class is not part of the public A\+PI. Use {\ttfamily open()}, {\ttfamily from\+Fd()}, {\ttfamily from\+Buffer()}, or {\ttfamily from\+Random\+Access\+Reader()} instead.

\paragraph*{Event\+: \char`\"{}entry\char`\"{}}

Callback gets {\ttfamily (entry)}, which is an {\ttfamily Entry}. See {\ttfamily open()} and {\ttfamily read\+Entry()} for when this event is emitted.

\paragraph*{Event\+: \char`\"{}end\char`\"{}}

Emitted after the last {\ttfamily entry} event has been emitted. See {\ttfamily open()} and {\ttfamily read\+Entry()} for more info on when this event is emitted.

\paragraph*{Event\+: \char`\"{}close\char`\"{}}

Emitted after the fd is actually closed. This is after calling {\ttfamily close()} (or after the {\ttfamily end} event when {\ttfamily auto\+Close} is {\ttfamily true}), and after all stream pipelines created from {\ttfamily open\+Read\+Stream()} have finished reading data from the fd.

If this {\ttfamily Zip\+File} was acquired from {\ttfamily from\+Random\+Access\+Reader()}, the \char`\"{}fd\char`\"{} in the previous paragraph refers to the {\ttfamily Random\+Access\+Reader} implemented by the client.

If this {\ttfamily Zip\+File} was acquired from {\ttfamily from\+Buffer()}, this event is never emitted.

\paragraph*{Event\+: \char`\"{}error\char`\"{}}

Emitted in the case of errors with reading the zip file. (Note that other errors can be emitted from the streams created from {\ttfamily open\+Read\+Stream()} as well.) After this event has been emitted, no further {\ttfamily entry}, {\ttfamily end}, or {\ttfamily error} events will be emitted, but the {\ttfamily close} event may still be emitted.

\paragraph*{read\+Entry()}

Causes this {\ttfamily Zip\+File} to emit an {\ttfamily entry} or {\ttfamily end} event (or an {\ttfamily error} event). This method must only be called when this {\ttfamily Zip\+File} was created with the {\ttfamily lazy\+Entries} option set to {\ttfamily true} (see {\ttfamily open()}). When this {\ttfamily Zip\+File} was created with the {\ttfamily lazy\+Entries} option set to {\ttfamily true}, {\ttfamily entry} and {\ttfamily end} events are only ever emitted in response to this method call.

The event that is emitted in response to this method will not be emitted until after this method has returned, so it is safe to call this method before attaching event listeners.

After calling this method, calling this method again before the response event has been emitted will cause undefined behavior. Calling this method after the {\ttfamily end} event has been emitted will cause undefined behavior. Calling this method after calling {\ttfamily close()} will cause undefined behavior.

\paragraph*{open\+Read\+Stream(entry, callback)}

{\ttfamily entry} must be an {\ttfamily Entry} object from this {\ttfamily Zip\+File}. {\ttfamily callback} gets {\ttfamily (err, read\+Stream)}, where {\ttfamily read\+Stream} is a {\ttfamily Readable Stream}. If the entry is compressed (with a supported compression method), the read stream provides the decompressed data. If this zipfile is already closed (see {\ttfamily close()}), the {\ttfamily callback} will receive an {\ttfamily err}.

It\textquotesingle{}s possible for the {\ttfamily read\+Stream} it to emit errors for several reasons. For example, if zlib cannot decompress the data, the zlib error will be emitted from the {\ttfamily read\+Stream}. Two more error cases are if the decompressed data has too many or too few actual bytes compared to the reported byte count from the entry\textquotesingle{}s {\ttfamily uncompressed\+Size} field. yauzl notices this false information and emits an error from the {\ttfamily read\+Stream} after some number of bytes have already been piped through the stream.

Because of this check, clients can always trust the {\ttfamily uncompressed\+Size} field in {\ttfamily Entry} objects. Guarding against \href{http://en.wikipedia.org/wiki/Zip_bomb}{\tt zip bomb} attacks can be accomplished by doing some heuristic checks on the size metadata and then watching out for the above errors. Such heuristics are outside the scope of this library, but enforcing the {\ttfamily uncompressed\+Size} is implemented here as a security feature.

It is possible to destroy the {\ttfamily read\+Stream} before it has piped all of its data. To do this, call {\ttfamily read\+Stream.\+destroy()}. You must {\ttfamily unpipe()} the {\ttfamily read\+Stream} from any destination before calling {\ttfamily read\+Stream.\+destroy()}. If this zipfile was created using {\ttfamily from\+Random\+Access\+Reader()}, the {\ttfamily Random\+Access\+Reader} implementation must provide readable streams that implement a {\ttfamily .destroy()} method (see {\ttfamily random\+Access\+Reader.\+\_\+read\+Stream\+For\+Range()}) in order for calls to {\ttfamily read\+Stream.\+destroy()} to work in this context.

\paragraph*{close()}

Causes all future calls to {\ttfamily open\+Read\+Stream()} to fail, and closes the fd after all streams created by {\ttfamily open\+Read\+Stream()} have emitted their {\ttfamily end} events.

If the {\ttfamily auto\+Close} option is set to {\ttfamily true} (see {\ttfamily open()}), this function will be called automatically effectively in response to this object\textquotesingle{}s {\ttfamily end} event.

If the {\ttfamily lazy\+Entries} option is set to {\ttfamily false} (see {\ttfamily open()}) and this object\textquotesingle{}s {\ttfamily end} event has not been emitted yet, this function causes undefined behavior. If the {\ttfamily lazy\+Entries} option is set to {\ttfamily true}, you can call this function instead of calling {\ttfamily read\+Entry()} to abort reading the entries of a zipfile.

It is safe to call this function multiple times; after the first call, successive calls have no effect. This includes situations where the {\ttfamily auto\+Close} option effectively calls this function for you.

\paragraph*{is\+Open}

{\ttfamily Boolean}. {\ttfamily true} until {\ttfamily close()} is called; then it\textquotesingle{}s {\ttfamily false}.

\paragraph*{entry\+Count}

{\ttfamily Number}. Total number of central directory records.

\paragraph*{comment}

{\ttfamily String}. Always decoded with {\ttfamily C\+P437} per the spec.

\subsubsection*{Class\+: Entry}

Objects of this class represent Central Directory Records. Refer to the zipfile specification for more details about these fields.

These fields are of type {\ttfamily Number}\+:


\begin{DoxyItemize}
\item {\ttfamily version\+Made\+By}
\item {\ttfamily version\+Needed\+To\+Extract}
\item {\ttfamily general\+Purpose\+Bit\+Flag}
\item {\ttfamily compression\+Method}
\item {\ttfamily last\+Mod\+File\+Time} (M\+S-\/\+D\+OS format, see {\ttfamily get\+Last\+Mod\+Date\+Time})
\item {\ttfamily last\+Mod\+File\+Date} (M\+S-\/\+D\+OS format, see {\ttfamily get\+Last\+Mod\+Date\+Time})
\item {\ttfamily crc32}
\item {\ttfamily compressed\+Size}
\item {\ttfamily uncompressed\+Size}
\item {\ttfamily file\+Name\+Length} (bytes)
\item {\ttfamily extra\+Field\+Length} (bytes)
\item {\ttfamily file\+Comment\+Length} (bytes)
\item {\ttfamily internal\+File\+Attributes}
\item {\ttfamily external\+File\+Attributes}
\item {\ttfamily relative\+Offset\+Of\+Local\+Header}
\end{DoxyItemize}

\paragraph*{file\+Name}

{\ttfamily String}. Following the spec, the bytes for the file name are decoded with {\ttfamily U\+T\+F-\/8} if {\ttfamily general\+Purpose\+Bit\+Flag \& 0x800}, otherwise with {\ttfamily C\+P437}.

If {\ttfamily file\+Name} would contain unsafe characters, such as an absolute path or a relative directory, yauzl emits an error instead of an entry.

\paragraph*{extra\+Fields}

{\ttfamily Array} with each entry in the form {\ttfamily \{id\+: id, data\+: data\}}, where {\ttfamily id} is a {\ttfamily Number} and {\ttfamily data} is a {\ttfamily Buffer}. This library looks for and reads the Z\+I\+P64 Extended Information Extra Field (0x0001) in order to support Z\+I\+P64 format zip files. None of the other fields are considered significant by this library.

\paragraph*{comment}

{\ttfamily String} decoded with the same charset as used for {\ttfamily file\+Name}.

\paragraph*{get\+Last\+Mod\+Date()}

Effectively implemented as\+:


\begin{DoxyCode}
return dosDateTimeToDate(this.lastModFileDate, this.lastModFileTime);
\end{DoxyCode}


\subsubsection*{Class\+: Random\+Access\+Reader}

This class is meant to be subclassed by clients and instantiated for the {\ttfamily from\+Random\+Access\+Reader()} function.

An example implementation can be found in {\ttfamily test/test.\+js}.

\paragraph*{random\+Access\+Reader.\+\_\+read\+Stream\+For\+Range(start, end)}

Subclasses {\itshape must} implement this method.

{\ttfamily start} and {\ttfamily end} are Numbers and indicate byte offsets from the start of the file. {\ttfamily end} is exclusive, so {\ttfamily \+\_\+read\+Stream\+For\+Range(0x1000, 0x2000)} would indicate to read {\ttfamily 0x1000} bytes. {\ttfamily end -\/ start} will always be at least {\ttfamily 1}.

This method should return a readable stream which will be {\ttfamily pipe()}ed into another stream. It is expected that the readable stream will provide data in several chunks if necessary. If the readable stream provides too many or too few bytes, an error will be emitted. Any errors emitted on the readable stream will be handled and re-\/emitted on the client-\/visible stream (returned from {\ttfamily zipfile.\+open\+Read\+Stream()}) or provided as the {\ttfamily err} argument to the appropriate callback (for example, for {\ttfamily from\+Random\+Access\+Reader()}).

The returned stream {\itshape must} implement a method {\ttfamily .destroy()} if you call {\ttfamily read\+Stream.\+destroy()} on streams you get from {\ttfamily open\+Read\+Stream()}. If you never call {\ttfamily read\+Stream.\+destroy()}, then streams returned from this method do not need to implement a method {\ttfamily .destroy()}. {\ttfamily .destroy()} should abort any streaming that is in progress and clean up any associated resources. {\ttfamily .destroy()} will only be called after the stream has been {\ttfamily unpipe()}d from its destination.

Note that the stream returned from this method might not be the same object that is provided by {\ttfamily open\+Read\+Stream()}. The stream returned from this method might be {\ttfamily pipe()}d through one or more filter streams (for example, a zlib inflate stream).

\paragraph*{random\+Access\+Reader.\+read(buffer, offset, length, position, callback)}

Subclasses may implement this method. The default implementation uses {\ttfamily create\+Read\+Stream()} to fill the {\ttfamily buffer}.

This method should behave like {\ttfamily fs.\+read()}.

\paragraph*{random\+Access\+Reader.\+close(callback)}

Subclasses may implement this method. The default implementation is effectively {\ttfamily set\+Immediate(callback);}.

{\ttfamily callback} takes parameters {\ttfamily (err)}.

This method is called once the all streams returned from {\ttfamily \+\_\+read\+Stream\+For\+Range()} have ended, and no more {\ttfamily \+\_\+read\+Stream\+For\+Range()} or {\ttfamily read()} requests will be issued to this object.

\subsection*{How to Avoid Crashing}

When a malformed zipfile is encountered, the default behavior is to crash (throw an exception). If you want to handle errors more gracefully than this, be sure to do the following\+:


\begin{DoxyItemize}
\item Provide {\ttfamily callback} parameters where they are allowed, and check the {\ttfamily err} parameter.
\item Attach a listener for the {\ttfamily error} event on any {\ttfamily Zip\+File} object you get from {\ttfamily open()}, {\ttfamily from\+Fd()}, {\ttfamily from\+Buffer()}, or {\ttfamily from\+Random\+Access\+Reader()}.
\item Attach a listener for the {\ttfamily error} event on any stream you get from {\ttfamily open\+Read\+Stream()}.
\end{DoxyItemize}

\subsection*{Limitations}

\subsubsection*{No Streaming Unzip A\+PI}

Due to the design of the .zip file format, it\textquotesingle{}s impossible to interpret a .zip file from start to finish (such as from a readable stream) without sacrificing correctness. The Central Directory, which is the authority on the contents of the .zip file, is at the end of a .zip file, not the beginning. A streaming A\+PI would need to either buffer the entire .zip file to get to the Central Directory before interpreting anything (defeating the purpose of a streaming interface), or rely on the Local File Headers which are interspersed through the .zip file. However, the Local File Headers are explicitly denounced in the spec as being unreliable copies of the Central Directory, so trusting them would be a violation of the spec.

Any library that offers a streaming unzip A\+PI must make one of the above two compromises, which makes the library either dishonest or nonconformant (usually the latter). This library insists on correctness and adherence to the spec, and so does not offer a streaming A\+PI.

\subsubsection*{Limitted Z\+I\+P64 Support}

For Z\+I\+P64, only zip files smaller than {\ttfamily 8\+PiB} are supported, not the full {\ttfamily 16\+EiB} range that a 64-\/bit integer should be able to index. This is due to the Java\+Script Number type being an I\+E\+EE 754 double precision float.

The Node.\+js {\ttfamily fs} module probably has this same limitation.

\subsubsection*{Z\+I\+P64 Extensible Data Sector Is Ignored}

The spec does not allow zip file creators to put arbitrary data here, but rather reserves its use for P\+K\+W\+A\+RE and mentions something about Z390. This doesn\textquotesingle{}t seem useful to expose in this library, so it is ignored.

\subsubsection*{No Multi-\/\+Disk Archive Support}

This library does not support multi-\/disk zip files. The multi-\/disk fields in the zipfile spec were intended for a zip file to span multiple floppy disks, which probably never happens now. If the \char`\"{}number of this disk\char`\"{} field in the End of Central Directory Record is not {\ttfamily 0}, the {\ttfamily open()}, {\ttfamily from\+Fd()}, {\ttfamily from\+Buffer()}, or {\ttfamily from\+Random\+Access\+Reader()} {\ttfamily callback} will receive an {\ttfamily err}. By extension the following zip file fields are ignored by this library and not provided to clients\+:


\begin{DoxyItemize}
\item Disk where central directory starts
\item Number of central directory records on this disk
\item Disk number where file starts
\end{DoxyItemize}

\subsubsection*{No Encryption Support}

Currently, the presence of encryption is not even checked, and encrypted zip files will cause undefined behavior.

\subsubsection*{Local File Headers Are Ignored}

Many unzip libraries mistakenly read the Local File Header data in zip files. This data is officially defined to be redundant with the Central Directory information, and is not to be trusted. Aside from checking the signature, yauzl ignores the content of the Local File Header.

\subsubsection*{No C\+R\+C-\/32 Checking}

This library provides the {\ttfamily crc32} field of {\ttfamily Entry} objects read from the Central Directory. However, this field is not used for anything in this library.

\subsubsection*{version\+Needed\+To\+Extract Is Ignored}

The field {\ttfamily version\+Needed\+To\+Extract} is ignored, because this library doesn\textquotesingle{}t support the complete zip file spec at any version,

\subsubsection*{No Support For Obscure Compression Methods}

Regarding the {\ttfamily compression\+Method} field of {\ttfamily Entry} objects, only method {\ttfamily 0} (stored with no compression) and method {\ttfamily 8} (deflated) are supported. Any of the other 15 official methods will cause the {\ttfamily open\+Read\+Stream()} {\ttfamily callback} to receive an {\ttfamily err}.

\subsubsection*{Data Descriptors Are Ignored}

There may or may not be Data Descriptor sections in a zip file. This library provides no support for finding or interpreting them.

\subsubsection*{Archive Extra Data Record Is Ignored}

There may or may not be an Archive Extra Data Record section in a zip file. This library provides no support for finding or interpreting it.

\subsubsection*{No Language Encoding Flag Support}

Zip files officially support charset encodings other than C\+P437 and U\+T\+F-\/8, but the zip file spec does not specify how it works. This library makes no attempt to interpret the Language Encoding Flag.

\subsection*{Change History}


\begin{DoxyItemize}
\item 2.\+4.\+1
\begin{DoxyItemize}
\item Fix error handling.
\end{DoxyItemize}
\item 2.\+4.\+0
\begin{DoxyItemize}
\item Add Z\+I\+P64 support. \href{https://github.com/thejoshwolfe/yazl/issues/6}{\tt issue \#6}
\item Add {\ttfamily lazy\+Entries} option. \href{https://github.com/thejoshwolfe/yazl/issues/22}{\tt issue \#22}
\item Add {\ttfamily read\+Stream.\+destroy()} method. \href{https://github.com/thejoshwolfe/yazl/issues/26}{\tt issue \#26}
\item Add {\ttfamily from\+Random\+Access\+Reader()}. \href{https://github.com/thejoshwolfe/yazl/issues/14}{\tt issue \#14}
\item Add {\ttfamily examples/unzip.\+js}.
\end{DoxyItemize}
\item 2.\+3.\+1
\begin{DoxyItemize}
\item Documentation updates.
\end{DoxyItemize}
\item 2.\+3.\+0
\begin{DoxyItemize}
\item Check that {\ttfamily uncompressed\+Size} is correct, or else emit an error. \href{https://github.com/thejoshwolfe/yazl/issues/13}{\tt issue \#13}
\end{DoxyItemize}
\item 2.\+2.\+1
\begin{DoxyItemize}
\item Update dependencies.
\end{DoxyItemize}
\item 2.\+2.\+0
\begin{DoxyItemize}
\item Update dependencies.
\end{DoxyItemize}
\item 2.\+1.\+0
\begin{DoxyItemize}
\item Remove dependency on {\ttfamily iconv}.
\end{DoxyItemize}
\item 2.\+0.\+3
\begin{DoxyItemize}
\item Fix crash when trying to read a 0-\/byte file.
\end{DoxyItemize}
\item 2.\+0.\+2
\begin{DoxyItemize}
\item Fix event behavior after errors.
\end{DoxyItemize}
\item 2.\+0.\+1
\begin{DoxyItemize}
\item Fix bug with using {\ttfamily iconv}.
\end{DoxyItemize}
\item 2.\+0.\+0
\begin{DoxyItemize}
\item Initial release. 
\end{DoxyItemize}
\end{DoxyItemize}