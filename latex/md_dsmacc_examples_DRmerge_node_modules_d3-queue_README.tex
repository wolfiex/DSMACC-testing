A {\bfseries queue} evaluates zero or more {\itshape deferred} asynchronous tasks with configurable concurrency\+: you control how many tasks run at the same time. When all the tasks complete, or an error occurs, the queue passes the results to your {\itshape await} callback. This library is similar to \href{https://github.com/caolan/async}{\tt Async.\+js}’s \href{https://github.com/caolan/async#paralleltasks-callback}{\tt parallel} (when {\itshape concurrency} is infinite), \href{https://github.com/caolan/async#seriestasks-callback}{\tt series} (when {\itshape concurrency} is 1), and \href{https://github.com/caolan/async#queue}{\tt queue}, but features a much smaller footprint\+: as of release 2, d3-\/queue is about 700 bytes gzipped, compared to 4,300 for Async.

Each task is defined as a function that takes a callback as its last argument. For example, here’s a task that says hello after a short delay\+:


\begin{DoxyCode}
function delayedHello(callback) \{
  setTimeout(function() \{
    console.log("Hello!");
    callback(null);
  \}, 250);
\}
\end{DoxyCode}


When a task completes, it must call the provided callback. The first argument to the callback should be null if the task is successful, or the error if the task failed. The optional second argument to the callback is the return value of the task. (To return multiple values from a single callback, wrap the results in an object or array.)

To run multiple tasks simultaneously, create a queue, {\itshape defer} your tasks, and then register an {\itshape await} callback to be called when all of the tasks complete (or an error occurs)\+:


\begin{DoxyCode}
var q = d3.queue();
q.defer(delayedHello);
q.defer(delayedHello);
q.await(function(error) \{
  if (error) throw error;
  console.log("Goodbye!");
\});
\end{DoxyCode}


Of course, you can also use a {\ttfamily for} loop to defer many tasks\+:


\begin{DoxyCode}
var q = d3.queue();

for (var i = 0; i < 1000; ++i) \{
  q.defer(delayedHello);
\}

q.awaitAll(function(error) \{
  if (error) throw error;
  console.log("Goodbye!");
\});
\end{DoxyCode}


Tasks can take optional arguments. For example, here’s how to configure the delay before hello and provide a name\+:


\begin{DoxyCode}
function delayedHello(name, delay, callback) \{
  setTimeout(function() \{
    console.log("Hello, " + name + "!");
    callback(null);
  \}, delay);
\}
\end{DoxyCode}


Any additional arguments provided to \href{#queue_defer}{\tt {\itshape queue}.defer} are automatically passed along to the task function before the callback argument. You can also use method chaining for conciseness, avoiding the need for a local variable\+:


\begin{DoxyCode}
d3.queue()
    .defer(delayedHello, "Alice", 250)
    .defer(delayedHello, "Bob", 500)
    .defer(delayedHello, "Carol", 750)
    .await(function(error) \{
      if (error) throw error;
      console.log("Goodbye!");
    \});
\end{DoxyCode}


The \href{https://github.com/maxogden/art-of-node#callbacks}{\tt asynchronous callback pattern} is very common in Node.\+js, so Queue works directly with many \mbox{\hyperlink{classNode}{Node}} A\+P\+Is. For example, to \href{https://nodejs.org/dist/latest/docs/api/fs.html#fs_fs_stat_path_callback}{\tt stat two files} concurrently\+:


\begin{DoxyCode}
d3.queue()
    .defer(fs.stat, \_\_dirname + "/../Makefile")
    .defer(fs.stat, \_\_dirname + "/../package.json")
    .await(function(error, file1, file2) \{
      if (error) throw error;
      console.log(file1, file2);
    \});
\end{DoxyCode}


You can also make abortable tasks\+: these tasks return an object with an {\itshape abort} method which terminates the task. So, if a task calls \href{https://developer.mozilla.org/en-US/docs/Web/API/WindowTimers/setTimeout}{\tt set\+Timeout} on start, it can call \href{https://developer.mozilla.org/en-US/docs/Web/API/WindowTimers/clearTimeout}{\tt clear\+Timeout} on abort. For example\+:


\begin{DoxyCode}
function delayedHello(name, delay, callback) \{
  var id = setTimeout(function() \{
    console.log("Hello, " + name + "!");
    callback(null);
  \}, delay);
  return \{
    abort: function() \{
      clearTimeout(id);
    \}
  \};
\}
\end{DoxyCode}


When you call \href{#queue_abort}{\tt {\itshape queue}.abort}, any in-\/progress tasks will be immediately aborted; in addition, any pending (not-\/yet-\/started) tasks will not be started. Note that you can also use {\itshape queue}.abort {\itshape without} abortable tasks, in which case pending tasks will be cancelled, though active tasks will continue to run. Conveniently, the \href{https://github.com/d3/d3-request}{\tt d3-\/request} library implements abort atop X\+M\+L\+Http\+Request. For example\+:


\begin{DoxyCode}
var q = d3.queue()
    .defer(d3.request, "http://www.google.com:81")
    .defer(d3.request, "http://www.google.com:81")
    .defer(d3.request, "http://www.google.com:81")
    .awaitAll(function(error, results) \{
      if (error) throw error;
      console.log(results);
    \});
\end{DoxyCode}


To abort these requests, call {\ttfamily q.\+abort()}.

\subsection*{Installing}

If you use N\+PM, {\ttfamily npm install d3-\/queue}. If you use Bower, {\ttfamily bower install d3-\/queue}. Otherwise, download the \href{https://github.com/d3/d3-queue/releases/latest}{\tt latest release}. You can also load directly from \href{https://d3js.org}{\tt d3js.\+org}, either as a \href{https://d3js.org/d3-queue.v3.min.js}{\tt standalone library} or as part of \href{https://github.com/d3/d3}{\tt D3 4.\+0}. A\+MD, Common\+JS, and vanilla environments are supported. In vanilla, a {\ttfamily d3} global is exported\+:


\begin{DoxyCode}
<script src="https://d3js.org/d3-queue.v3.min.js"></script>
<script>

var q = d3.queue();

</script>
\end{DoxyCode}


\href{https://tonicdev.com/npm/d3-queue}{\tt Try d3-\/queue in your browser.}

\subsection*{A\+PI Reference}

\href{#queue}{\tt \#} d3.{\bfseries queue}(\mbox{[}{\itshape concurrency}\mbox{]}) \href{https://github.com/d3/d3-queue/blob/master/src/queue.js}{\tt $<$$>$}

Constructs a new queue with the specified {\itshape concurrency}. If {\itshape concurrency} is not specified, the queue has infinite concurrency. Otherwise, {\itshape concurrency} is a positive integer. For example, if {\itshape concurrency} is 1, then all tasks will be run in series. If {\itshape concurrency} is 3, then at most three tasks will be allowed to proceed concurrently; this is useful, for example, when loading resources in a web browser.

\href{#queue_defer}{\tt \#} {\itshape queue}.{\bfseries defer}({\itshape task}\mbox{[}, {\itshape arguments}…\mbox{]}) \href{https://github.com/d3/d3-queue/blob/master/src/queue.js#L20}{\tt $<$$>$}

Adds the specified asynchronous {\itshape task} callback to the queue, with any optional {\itshape arguments}. The {\itshape task} is a function that will be called when the task should start. It is passed the specified optional {\itshape arguments} and an additional {\itshape callback} as the last argument; the callback must be invoked by the task when it finishes. The task must invoke the callback with two arguments\+: the {\itshape error}, if any, and the {\itshape result} of the task. To return multiple results from a single callback, wrap the results in an object or array.

For example, here’s a task which computes the answer to the ultimate question of life, the universe, and everything after a short delay\+:


\begin{DoxyCode}
function simpleTask(callback) \{
  setTimeout(function() \{
    callback(null, \{answer: 42\});
  \}, 250);
\}
\end{DoxyCode}


If the task calls back with an error, any tasks that were scheduled {\itshape but not yet started} will not run. For a serial queue (of {\itshape concurrency} 1), this means that a task will only run if all previous tasks succeed. For a queue with higher concurrency, only the first error that occurs is reported to the await callback, and tasks that were started before the error occurred will continue to run; note, however, that their results will not be reported to the await callback.

Tasks can only be deferred before \href{#queue_await}{\tt {\itshape queue}.await} or \href{#queue_awaitAll}{\tt {\itshape queue}.await\+All} is called. If a task is deferred after then, an error is thrown. If the {\itshape task} is not a function, an error is thrown.

\href{#queue_abort}{\tt \#} {\itshape queue}.{\bfseries abort}() \href{https://github.com/d3/d3-queue/blob/master/src/queue.js#L29}{\tt $<$$>$}

Aborts any active tasks, invoking each active task’s {\itshape task}.abort function, if any. Also prevents any new tasks from starting, and immediately invokes the \href{#queue_await}{\tt {\itshape queue}.await} or \href{#queue_awaitAll}{\tt {\itshape queue}.await\+All} callback with an error indicating that the queue was aborted. See the \href{#d3-queue}{\tt introduction} for an example implementation of an abortable task. Note that if your tasks are not abortable, any running tasks will continue to run, even after the await callback has been invoked with the abort error. The await callback is invoked exactly once on abort, and so is not called when any running tasks subsequently succeed or fail.

\href{#queue_await}{\tt \#} {\itshape queue}.{\bfseries await}({\itshape callback}) \href{https://github.com/d3/d3-queue/blob/master/src/queue.js#L33}{\tt $<$$>$}

Sets the {\itshape callback} to be invoked when all deferred tasks have finished. The first argument to the {\itshape callback} is the first error that occurred, or null if no error occurred. If an error occurred, there are no additional arguments to the callback. Otherwise, the {\itshape callback} is passed each result as an additional argument. For example\+:


\begin{DoxyCode}
d3.queue()
    .defer(fs.stat, \_\_dirname + "/../Makefile")
    .defer(fs.stat, \_\_dirname + "/../package.json")
    .await(function(error, file1, file2) \{ console.log(file1, file2); \});
\end{DoxyCode}


If all \href{#queue_defer}{\tt deferred} tasks have already completed, the callback will be invoked immediately. This method may only be called once, after any tasks have been deferred. If this method is called multiple times, or if it is called after \href{#queue_awaitAll}{\tt {\itshape queue}.await\+All}, an error is thrown. If the {\itshape callback} is not a function, an error is thrown.

\href{#queue_awaitAll}{\tt \#} {\itshape queue}.{\bfseries await\+All}({\itshape callback}) \href{https://github.com/d3/d3-queue/blob/master/src/queue.js#L39}{\tt $<$$>$}

Sets the {\itshape callback} to be invoked when all deferred tasks have finished. The first argument to the {\itshape callback} is the first error that occurred, or null if no error occurred. If an error occurred, there are no additional arguments to the callback. Otherwise, the {\itshape callback} is also passed an array of results as the second argument. For example\+:


\begin{DoxyCode}
d3.queue()
    .defer(fs.stat, \_\_dirname + "/../Makefile")
    .defer(fs.stat, \_\_dirname + "/../package.json")
    .awaitAll(function(error, files) \{ console.log(files); \});
\end{DoxyCode}


If all \href{#queue_defer}{\tt deferred} tasks have already completed, the callback will be invoked immediately. This method may only be called once, after any tasks have been deferred. If this method is called multiple times, or if it is called after \href{#queue_await}{\tt {\itshape queue}.await}, an error is thrown. If the {\itshape callback} is not a function, an error is thrown. 