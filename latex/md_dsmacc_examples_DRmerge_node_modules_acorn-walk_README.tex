An abstract syntax tree walker for the \href{https://github.com/estree/estree}{\tt E\+S\+Tree} format.

\subsection*{Community}

Acorn is open source software released under an \href{https://github.com/acornjs/acorn/blob/master/acorn-walk/LICENSE}{\tt M\+IT license}.

You are welcome to \href{https://github.com/acornjs/acorn/issues}{\tt report bugs} or create pull requests on \href{https://github.com/acornjs/acorn}{\tt github}. For questions and discussion, please use the \href{https://discuss.ternjs.net}{\tt Tern discussion forum}.

\subsection*{Installation}

The easiest way to install acorn is from \href{https://www.npmjs.com/}{\tt {\ttfamily npm}}\+:


\begin{DoxyCode}
npm install acorn-walk
\end{DoxyCode}


Alternately, you can download the source and build acorn yourself\+:


\begin{DoxyCode}
git clone https://github.com/acornjs/acorn.git
cd acorn
npm install
\end{DoxyCode}


\subsection*{Interface}

An algorithm for recursing through a syntax tree is stored as an object, with a property for each tree node type holding a function that will recurse through such a node. There are several ways to run such a walker.

{\bfseries simple}{\ttfamily (node, visitors, base, state)} does a \textquotesingle{}simple\textquotesingle{} walk over a tree. {\ttfamily node} should be the A\+ST node to walk, and {\ttfamily visitors} an object with properties whose names correspond to node types in the \href{https://github.com/estree/estree}{\tt E\+S\+Tree spec}. The properties should contain functions that will be called with the node object and, if applicable the state at that point. The last two arguments are optional. {\ttfamily base} is a walker algorithm, and {\ttfamily state} is a start state. The default walker will simply visit all statements and expressions and not produce a meaningful state. (An example of a use of state is to track scope at each point in the tree.)


\begin{DoxyCode}
const acorn = require("acorn")
const walk = require("acorn-walk")

walk.simple(acorn.parse("let x = 10"), \{
  Literal(node) \{
    console.log(`Found a literal: $\{node.value\}`)
  \}
\})
\end{DoxyCode}


{\bfseries ancestor}{\ttfamily (node, visitors, base, state)} does a \textquotesingle{}simple\textquotesingle{} walk over a tree, building up an array of ancestor nodes (including the current node) and passing the array to the callbacks as a third parameter.


\begin{DoxyCode}
const acorn = require("acorn")
const walk = require("acorn-walk")

walk.ancestor(acorn.parse("foo('hi')"), \{
  Literal(\_, ancestors) \{
    console.log("This literal's ancestors are:", ancestors.map(n => n.type))
  \}
\})
\end{DoxyCode}


{\bfseries recursive}{\ttfamily (node, state, functions, base)} does a \textquotesingle{}recursive\textquotesingle{} walk, where the walker functions are responsible for continuing the walk on the child nodes of their target node. {\ttfamily state} is the start state, and {\ttfamily functions} should contain an object that maps node types to walker functions. Such functions are called with {\ttfamily (node, state, c)} arguments, and can cause the walk to continue on a sub-\/node by calling the {\ttfamily c} argument on it with {\ttfamily (node, state)} arguments. The optional {\ttfamily base} argument provides the fallback walker functions for node types that aren\textquotesingle{}t handled in the {\ttfamily functions} object. If not given, the default walkers will be used.

{\bfseries make}{\ttfamily (functions, base)} builds a new walker object by using the walker functions in {\ttfamily functions} and filling in the missing ones by taking defaults from {\ttfamily base}.

{\bfseries full}{\ttfamily (node, callback, base, state)} does a \textquotesingle{}full\textquotesingle{} walk over a tree, calling the callback with the arguments (node, state, type) for each node

{\bfseries full\+Ancestor}{\ttfamily (node, callback, base, state)} does a \textquotesingle{}full\textquotesingle{} walk over a tree, building up an array of ancestor nodes (including the current node) and passing the array to the callbacks as a third parameter.


\begin{DoxyCode}
const acorn = require("acorn")
const walk = require("acorn-walk")

walk.full(acorn.parse("1 + 1"), node => \{
  console.log(`There's a $\{node.type\} node at $\{node.ch\}`)
\})
\end{DoxyCode}


{\bfseries find\+Node\+At}{\ttfamily (node, start, end, test, base, state)} tries to locate a node in a tree at the given start and/or end offsets, which satisfies the predicate {\ttfamily test}. {\ttfamily start} and {\ttfamily end} can be either {\ttfamily null} (as wildcard) or a number. {\ttfamily test} may be a string (indicating a node type) or a function that takes {\ttfamily (node\+Type, node)} arguments and returns a boolean indicating whether this node is interesting. {\ttfamily base} and {\ttfamily state} are optional, and can be used to specify a custom walker. Nodes are tested from inner to outer, so if two nodes match the boundaries, the inner one will be preferred.

{\bfseries find\+Node\+Around}{\ttfamily (node, pos, test, base, state)} is a lot like {\ttfamily find\+Node\+At}, but will match any node that exists \textquotesingle{}around\textquotesingle{} (spanning) the given position.

{\bfseries find\+Node\+After}{\ttfamily (node, pos, test, base, state)} is similar to {\ttfamily find\+Node\+Around}, but will match all nodes {\itshape after} the given position (testing outer nodes before inner nodes). 