{\bfseries More about brace expansion} (click to expand)

\begin{quote}
Bash-\/like brace expansion, implemented in Java\+Script. Safer than other brace expansion libs, with complete support for the Bash 4.\+3 braces specification, without sacrificing speed. \end{quote}


Please consider following this project\textquotesingle{}s author, \href{https://github.com/jonschlinkert}{\tt Jon Schlinkert}, and consider starring the project to show your \+:heart\+: and support.

\subsection*{Install}

Install with \href{https://www.npmjs.com/}{\tt npm}\+:


\begin{DoxyCode}
$ npm install --save braces
\end{DoxyCode}


\subsection*{v3.\+0.\+0 Released!!}

See the changelog for details.

\subsection*{Why use braces?}

Brace patterns make globs more powerful by adding the ability to match specific ranges and sequences of characters.


\begin{DoxyItemize}
\item {\bfseries Accurate} -\/ complete support for the \href{www.gnu.org/software/bash/}{\tt Bash 4.\+3 Brace Expansion} specification (passes all of the Bash braces tests)
\item {\bfseries \href{#benchmarks}{\tt fast and performant}} -\/ Starts fast, runs fast and \href{#performance}{\tt scales well} as patterns increase in complexity.
\item {\bfseries Organized code base} -\/ The parser and compiler are easy to maintain and update when edge cases crop up.
\item {\bfseries Well-\/tested} -\/ Thousands of test assertions, and passes all of the Bash, minimatch, and \href{https://github.com/juliangruber/brace-expansion}{\tt brace-\/expansion} unit tests (as of the date this was written).
\item {\bfseries Safer} -\/ You shouldn\textquotesingle{}t have to worry about users defining aggressive or malicious brace patterns that can break your application. Braces takes measures to prevent malicious regex that can be used for D\+DoS attacks (see \href{https://www.regular-expressions.info/catastrophic.html}{\tt catastrophic backtracking}).
\item \href{#lists}{\tt Supports lists} -\/ (aka \char`\"{}sets\char`\"{}) {\ttfamily a/\{b,c\}/d} =$>$ `\mbox{[}\textquotesingle{}a/b/d', \textquotesingle{}a/c/d\textquotesingle{}\mbox{]}{\ttfamily }
\item {\ttfamily \mbox{[}Supports sequences\mbox{]}(\#sequences) -\/ (aka \char`\"{}ranges\char`\"{})}\{01..03\}{\ttfamily =$>$}\mbox{[}\textquotesingle{}01\textquotesingle{}, \textquotesingle{}02\textquotesingle{}, \textquotesingle{}03\textquotesingle{}\mbox{]}{\ttfamily }
\item {\ttfamily \mbox{[}Supports steps\mbox{]}(\#steps) -\/ (aka \char`\"{}increments\char`\"{})}\{2..10..2\}{\ttfamily =$>$}\mbox{[}\textquotesingle{}2\textquotesingle{}, \textquotesingle{}4\textquotesingle{}, \textquotesingle{}6\textquotesingle{}, \textquotesingle{}8\textquotesingle{}, \textquotesingle{}10\textquotesingle{}\mbox{]}\`{}
\item \href{#escaping}{\tt Supports escaping} -\/ To prevent evaluation of special characters.
\end{DoxyItemize}

\subsection*{Usage}

The main export is a function that takes one or more brace {\ttfamily patterns} and {\ttfamily options}.


\begin{DoxyCode}
const braces = require('braces');
// braces(patterns[, options]);

console.log(braces(['\{01..05\}', '\{a..e\}']));
//=> ['(0[1-5])', '([a-e])']

console.log(braces(['\{01..05\}', '\{a..e\}'], \{ expand: true \}));
//=> ['01', '02', '03', '04', '05', 'a', 'b', 'c', 'd', 'e']
\end{DoxyCode}


\subsubsection*{Brace Expansion vs. Compilation}

By default, brace patterns are compiled into strings that are optimized for creating regular expressions and matching.

{\bfseries Compiled}


\begin{DoxyCode}
console.log(braces('a/\{x,y,z\}/b')); 
//=> ['a/(x|y|z)/b']
console.log(braces(['a/\{01..20\}/b', 'a/\{1..5\}/b'])); 
//=> [ 'a/(0[1-9]|1[0-9]|20)/b', 'a/([1-5])/b' ]
\end{DoxyCode}


{\bfseries Expanded}

Enable brace expansion by setting the {\ttfamily expand} option to true, or by using \href{#expand}{\tt braces.\+expand()} (returns an array similar to what you\textquotesingle{}d expect from Bash, or {\ttfamily echo \{1..5\}}, or \href{https://github.com/isaacs/minimatch}{\tt minimatch})\+:


\begin{DoxyCode}
console.log(braces('a/\{x,y,z\}/b', \{ expand: true \}));
//=> ['a/x/b', 'a/y/b', 'a/z/b']

console.log(braces.expand('\{01..10\}'));
//=> ['01','02','03','04','05','06','07','08','09','10']
\end{DoxyCode}


\subsubsection*{Lists}

Expand lists (like Bash \char`\"{}sets\char`\"{})\+:


\begin{DoxyCode}
console.log(braces('a/\{foo,bar,baz\}/*.js'));
//=> ['a/(foo|bar|baz)/*.js']

console.log(braces.expand('a/\{foo,bar,baz\}/*.js'));
//=> ['a/foo/*.js', 'a/bar/*.js', 'a/baz/*.js']
\end{DoxyCode}


\subsubsection*{Sequences}

Expand ranges of characters (like Bash \char`\"{}sequences\char`\"{})\+:


\begin{DoxyCode}
console.log(braces.expand('\{1..3\}'));                // ['1', '2', '3']
console.log(braces.expand('a/\{1..3\}/b'));            // ['a/1/b', 'a/2/b', 'a/3/b']
console.log(braces('\{a..c\}', \{ expand: true \}));     // ['a', 'b', 'c']
console.log(braces('foo/\{a..c\}', \{ expand: true \})); // ['foo/a', 'foo/b', 'foo/c']

// supports zero-padded ranges
console.log(braces('a/\{01..03\}/b'));   //=> ['a/(0[1-3])/b']
console.log(braces('a/\{001..300\}/b')); //=> ['a/(0\{2\}[1-9]|0[1-9][0-9]|[12][0-9]\{2\}|300)/b']
\end{DoxyCode}


See \href{https://github.com/jonschlinkert/fill-range}{\tt fill-\/range} for all available range-\/expansion options.

\subsubsection*{Steppped ranges}

Steps, or increments, may be used with ranges\+:


\begin{DoxyCode}
console.log(braces.expand('\{2..10..2\}'));
//=> ['2', '4', '6', '8', '10']

console.log(braces('\{2..10..2\}'));
//=> ['(2|4|6|8|10)']
\end{DoxyCode}


When the \href{#optimize}{\tt .optimize} method is used, or \href{#optionsoptimize}{\tt options.\+optimize} is set to true, sequences are passed to \href{https://github.com/jonschlinkert/to-regex-range}{\tt to-\/regex-\/range} for expansion.

\subsubsection*{Nesting}

Brace patterns may be nested. The results of each expanded string are not sorted, and left to right order is preserved.

$\ast$$\ast$\char`\"{}\+Expanded\char`\"{} braces$\ast$$\ast$


\begin{DoxyCode}
console.log(braces.expand('a\{b,c,/\{x,y\}\}/e'));
//=> ['ab/e', 'ac/e', 'a/x/e', 'a/y/e']

console.log(braces.expand('a/\{x,\{1..5\},y\}/c'));
//=> ['a/x/c', 'a/1/c', 'a/2/c', 'a/3/c', 'a/4/c', 'a/5/c', 'a/y/c']
\end{DoxyCode}


$\ast$$\ast$\char`\"{}\+Optimized\char`\"{} braces$\ast$$\ast$


\begin{DoxyCode}
console.log(braces('a\{b,c,/\{x,y\}\}/e'));
//=> ['a(b|c|/(x|y))/e']

console.log(braces('a/\{x,\{1..5\},y\}/c'));
//=> ['a/(x|([1-5])|y)/c']
\end{DoxyCode}


\subsubsection*{Escaping}

{\bfseries Escaping braces}

A brace pattern will not be expanded or evaluted if {\itshape either the opening or closing brace is escaped}\+:


\begin{DoxyCode}
console.log(braces.expand('a\(\backslash\)\(\backslash\)\{d,c,b\}e'));
//=> ['a\{d,c,b\}e']

console.log(braces.expand('a\{d,c,b\(\backslash\)\(\backslash\)\}e'));
//=> ['a\{d,c,b\}e']
\end{DoxyCode}


{\bfseries Escaping commas}

Commas inside braces may also be escaped\+:


\begin{DoxyCode}
console.log(braces.expand('a\{b\(\backslash\)\(\backslash\),c\}d'));
//=> ['a\{b,c\}d']

console.log(braces.expand('a\{d\(\backslash\)\(\backslash\),c,b\}e'));
//=> ['ad,ce', 'abe']
\end{DoxyCode}


{\bfseries Single items}

Following bash conventions, a brace pattern is also not expanded when it contains a single character\+:


\begin{DoxyCode}
console.log(braces.expand('a\{b\}c'));
//=> ['a\{b\}c']
\end{DoxyCode}


\subsection*{Options}

\subsubsection*{options.\+max\+Length}

{\bfseries Type}\+: {\ttfamily Number}

{\bfseries Default}\+: {\ttfamily 65,536}

{\bfseries Description}\+: Limit the length of the input string. Useful when the input string is generated or your application allows users to pass a string, et cetera.


\begin{DoxyCode}
console.log(braces('a/\{b,c\}/d', \{ maxLength: 3 \}));  //=> throws an error
\end{DoxyCode}


\subsubsection*{options.\+expand}

{\bfseries Type}\+: {\ttfamily Boolean}

{\bfseries Default}\+: {\ttfamily undefined}

{\bfseries Description}\+: Generate an \char`\"{}expanded\char`\"{} brace pattern (alternatively you can use the {\ttfamily braces.\+expand()} method, which does the same thing).


\begin{DoxyCode}
console.log(braces('a/\{b,c\}/d', \{ expand: true \}));
//=> [ 'a/b/d', 'a/c/d' ]
\end{DoxyCode}


\subsubsection*{options.\+nodupes}

{\bfseries Type}\+: {\ttfamily Boolean}

{\bfseries Default}\+: {\ttfamily undefined}

{\bfseries Description}\+: Remove duplicates from the returned array.

\subsubsection*{options.\+range\+Limit}

{\bfseries Type}\+: {\ttfamily Number}

{\bfseries Default}\+: {\ttfamily 1000}

{\bfseries Description}\+: To prevent malicious patterns from being passed by users, an error is thrown when {\ttfamily braces.\+expand()} is used or {\ttfamily options.\+expand} is true and the generated range will exceed the {\ttfamily range\+Limit}.

You can customize {\ttfamily options.\+range\+Limit} or set it to {\ttfamily Inifinity} to disable this altogether.

{\bfseries Examples}


\begin{DoxyCode}
// pattern exceeds the "rangeLimit", so it's optimized automatically
console.log(braces.expand('\{1..1000\}'));
//=> ['([1-9]|[1-9][0-9]\{1,2\}|1000)']

// pattern does not exceed "rangeLimit", so it's NOT optimized
console.log(braces.expand('\{1..100\}'));
//=> ['1', '2', '3', '4', '5', '6', '7', '8', '9', '10', '11', '12', '13', '14', '15', '16', '17', '18',
       '19', '20', '21', '22', '23', '24', '25', '26', '27', '28', '29', '30', '31', '32', '33', '34', '35', '36',
       '37', '38', '39', '40', '41', '42', '43', '44', '45', '46', '47', '48', '49', '50', '51', '52', '53', '54',
       '55', '56', '57', '58', '59', '60', '61', '62', '63', '64', '65', '66', '67', '68', '69', '70', '71', '72',
       '73', '74', '75', '76', '77', '78', '79', '80', '81', '82', '83', '84', '85', '86', '87', '88', '89', '90',
       '91', '92', '93', '94', '95', '96', '97', '98', '99', '100']
\end{DoxyCode}


\subsubsection*{options.\+transform}

{\bfseries Type}\+: {\ttfamily Function}

{\bfseries Default}\+: {\ttfamily undefined}

{\bfseries Description}\+: Customize range expansion.

{\bfseries Example\+: Transforming non-\/numeric values}


\begin{DoxyCode}
const alpha = braces.expand('x/\{a..e\}/y', \{
  transform(value, index) \{
    // When non-numeric values are passed, "value" is a character code.
    return 'foo/' + String.fromCharCode(value) + '-' + index;
  \}
\});
console.log(alpha);
//=> [ 'x/foo/a-0/y', 'x/foo/b-1/y', 'x/foo/c-2/y', 'x/foo/d-3/y', 'x/foo/e-4/y' ]
\end{DoxyCode}


{\bfseries Example\+: Transforming numeric values}


\begin{DoxyCode}
const numeric = braces.expand('\{1..5\}', \{
  transform(value) \{
    // when numeric values are passed, "value" is a number
    return 'foo/' + value * 2;
  \}
\});
console.log(numeric); 
//=> [ 'foo/2', 'foo/4', 'foo/6', 'foo/8', 'foo/10' ]
\end{DoxyCode}


\subsubsection*{options.\+quantifiers}

{\bfseries Type}\+: {\ttfamily Boolean}

{\bfseries Default}\+: {\ttfamily undefined}

{\bfseries Description}\+: In regular expressions, quanitifiers can be used to specify how many times a token can be repeated. For example, {\ttfamily a\{1,3\}} will match the letter {\ttfamily a} one to three times.

Unfortunately, regex quantifiers happen to share the same syntax as \href{#lists}{\tt Bash lists}

The {\ttfamily quantifiers} option tells braces to detect when \href{https://developer.mozilla.org/en-US/docs/Web/JavaScript/Reference/Global_Objects/RegExp#quantifiers}{\tt regex quantifiers} are defined in the given pattern, and not to try to expand them as lists.

{\bfseries Examples}


\begin{DoxyCode}
const braces = require('braces');
console.log(braces('a/b\{1,3\}/\{x,y,z\}'));
//=> [ 'a/b(1|3)/(x|y|z)' ]
console.log(braces('a/b\{1,3\}/\{x,y,z\}', \{quantifiers: true\}));
//=> [ 'a/b\{1,3\}/(x|y|z)' ]
console.log(braces('a/b\{1,3\}/\{x,y,z\}', \{quantifiers: true, expand: true\}));
//=> [ 'a/b\{1,3\}/x', 'a/b\{1,3\}/y', 'a/b\{1,3\}/z' ]
\end{DoxyCode}


\subsubsection*{options.\+unescape}

{\bfseries Type}\+: {\ttfamily Boolean}

{\bfseries Default}\+: {\ttfamily undefined}

{\bfseries Description}\+: Strip backslashes that were used for escaping from the result.

\subsection*{What is \char`\"{}brace expansion\char`\"{}?}

Brace expansion is a type of parameter expansion that was made popular by unix shells for generating lists of strings, as well as regex-\/like matching when used alongside wildcards (globs).

In addition to \char`\"{}expansion\char`\"{}, braces are also used for matching. In other words\+:


\begin{DoxyItemize}
\item \href{#brace-expansion}{\tt brace expansion} is for generating new lists
\item \href{#brace-matching}{\tt brace matching} is for filtering existing lists
\end{DoxyItemize}

$<$details$>$

There are two main types of brace expansion\+:


\begin{DoxyEnumerate}
\item {\bfseries lists}\+: which are defined using comma-\/separated values inside curly braces\+: {\ttfamily \{a,b,c\}}
\item {\bfseries sequences}\+: which are defined using a starting value and an ending value, separated by two dots\+: {\ttfamily a\{1..3\}b}. Optionally, a third argument may be passed to define a \char`\"{}step\char`\"{} or increment to use\+: {\ttfamily a\{1..100..10\}b}. These are also sometimes referred to as \char`\"{}ranges\char`\"{}.
\end{DoxyEnumerate}

Here are some example brace patterns to illustrate how they work\+:

{\bfseries Sets}


\begin{DoxyCode}
\{a,b,c\}       => a b c
\{a,b,c\}\{1,2\}  => a1 a2 b1 b2 c1 c2
\end{DoxyCode}


{\bfseries Sequences}


\begin{DoxyCode}
\{1..9\}        => 1 2 3 4 5 6 7 8 9
\{4..-4\}       => 4 3 2 1 0 -1 -2 -3 -4
\{1..20..3\}    => 1 4 7 10 13 16 19
\{a..j\}        => a b c d e f g h i j
\{j..a\}        => j i h g f e d c b a
\{a..z..3\}     => a d g j m p s v y
\end{DoxyCode}


{\bfseries Combination}

Sets and sequences can be mixed together or used along with any other strings.


\begin{DoxyCode}
\{a,b,c\}\{1..3\}   => a1 a2 a3 b1 b2 b3 c1 c2 c3
foo/\{a,b,c\}/bar => foo/a/bar foo/b/bar foo/c/bar
\end{DoxyCode}


The fact that braces can be \char`\"{}expanded\char`\"{} from relatively simple patterns makes them ideal for quickly generating test fixtures, file paths, and similar use cases.

\subsection*{Brace matching}

In addition to {\itshape expansion}, brace patterns are also useful for performing regular-\/expression-\/like matching.

For example, the pattern {\ttfamily foo/\{1..3\}/bar} would match any of following strings\+:


\begin{DoxyCode}
foo/1/bar
foo/2/bar
foo/3/bar
\end{DoxyCode}


But not\+:


\begin{DoxyCode}
baz/1/qux
baz/2/qux
baz/3/qux
\end{DoxyCode}


Braces can also be combined with \href{https://github.com/jonschlinkert/micromatch}{\tt glob patterns} to perform more advanced wildcard matching. For example, the pattern {\ttfamily $\ast$/\{1..3\}/$\ast$} would match any of following strings\+:


\begin{DoxyCode}
foo/1/bar
foo/2/bar
foo/3/bar
baz/1/qux
baz/2/qux
baz/3/qux
\end{DoxyCode}


\subsection*{Brace matching pitfalls}

Although brace patterns offer a user-\/friendly way of matching ranges or sets of strings, there are also some major disadvantages and potential risks you should be aware of.

\subsubsection*{tldr}

$\ast$$\ast$\char`\"{}brace bombs\char`\"{}$\ast$$\ast$


\begin{DoxyItemize}
\item brace expansion can eat up a huge amount of processing resources
\item as brace patterns increase {\itshape linearly in size}, the system resources required to expand the pattern increase exponentially
\item users can accidentally (or intentially) exhaust your system\textquotesingle{}s resources resulting in the equivalent of a DoS attack (bonus\+: no programming knowledge is required!)
\end{DoxyItemize}

For a more detailed explanation with examples, see the \href{#geometric-complexity}{\tt geometric complexity} section.

\subsubsection*{The solution}

Jump to the \href{#performance}{\tt performance section} to see how Braces solves this problem in comparison to other libraries.

\subsubsection*{Geometric complexity}

At minimum, brace patterns with sets limited to two elements have quadradic or {\ttfamily O(n$^\wedge$2)} complexity. But the complexity of the algorithm increases exponentially as the number of sets, {\itshape and elements per set}, increases, which is {\ttfamily O(n$^\wedge$c)}.

For example, the following sets demonstrate quadratic ({\ttfamily O(n$^\wedge$2)}) complexity\+:


\begin{DoxyCode}
\{1,2\}\{3,4\}      => (2X2)    => 13 14 23 24
\{1,2\}\{3,4\}\{5,6\} => (2X2X2)  => 135 136 145 146 235 236 245 246
\end{DoxyCode}


But add an element to a set, and we get a n-\/fold Cartesian product with {\ttfamily O(n$^\wedge$c)} complexity\+:


\begin{DoxyCode}
\{1,2,3\}\{4,5,6\}\{7,8,9\} => (3X3X3) => 147 148 149 157 158 159 167 168 169 247 248 
                                    249 257 258 259 267 268 269 347 348 349 357 
                                    358 359 367 368 369
\end{DoxyCode}


Now, imagine how this complexity grows given that each element is a n-\/tuple\+:


\begin{DoxyCode}
\{1..100\}\{1..100\}         => (100X100)     => 10,000 elements (38.4 kB)
\{1..100\}\{1..100\}\{1..100\} => (100X100X100) => 1,000,000 elements (5.76 MB)
\end{DoxyCode}


Although these examples are clearly contrived, they demonstrate how brace patterns can quickly grow out of control.

{\bfseries More information}

Interested in learning more about brace expansion?


\begin{DoxyItemize}
\item \href{http://www.linuxjournal.com/content/bash-brace-expansion}{\tt linuxjournal/bash-\/brace-\/expansion}
\item \href{https://rosettacode.org/wiki/Brace_expansion}{\tt rosettacode/\+Brace\+\_\+expansion}
\item \href{https://en.wikipedia.org/wiki/Cartesian_product}{\tt cartesian product}
\end{DoxyItemize}

$<$/details$>$

\subsection*{Performance}

Braces is not only screaming fast, it\textquotesingle{}s also more accurate the other brace expansion libraries.

\subsubsection*{Better algorithms}

Fortunately there is a solution to the \href{#brace-matching-pitfalls}{\tt \char`\"{}brace bomb\char`\"{} problem}\+: {\itshape don\textquotesingle{}t expand brace patterns into an array when they\textquotesingle{}re used for matching}.

Instead, convert the pattern into an optimized regular expression. This is easier said than done, and braces is the only library that does this currently.

{\bfseries The proof is in the numbers}

Minimatch gets exponentially slower as patterns increase in complexity, braces does not. The following results were generated using {\ttfamily braces()} and {\ttfamily minimatch.\+brace\+Expand()}, respectively.

\tabulinesep=1mm
\begin{longtabu} spread 0pt [c]{*{3}{|X[-1]}|}
\hline
\rowcolor{\tableheadbgcolor}\textbf{ {\bfseries Pattern}  }&\textbf{ {\bfseries braces}  }&\textbf{ {\bfseries \mbox{[}minimatch\mbox{]}\mbox{[}\mbox{]}}   }\\\cline{1-3}
\endfirsthead
\hline
\endfoot
\hline
\rowcolor{\tableheadbgcolor}\textbf{ {\bfseries Pattern}  }&\textbf{ {\bfseries braces}  }&\textbf{ {\bfseries \mbox{[}minimatch\mbox{]}\mbox{[}\mbox{]}}   }\\\cline{1-3}
\endhead
{\ttfamily \{1..9007199254740991\}}\mbox{[}$^\wedge$1\mbox{]}  &{\ttfamily 298 B} (5ms 459μs)  &N/A (freezes)   \\\cline{1-3}
{\ttfamily \{1..1000000000000000\}}  &{\ttfamily 41 B} (1ms 15μs)  &N/A (freezes)   \\\cline{1-3}
{\ttfamily \{1..100000000000000\}}  &{\ttfamily 40 B} (890μs)  &N/A (freezes)   \\\cline{1-3}
{\ttfamily \{1..10000000000000\}}  &{\ttfamily 39 B} (2ms 49μs)  &N/A (freezes)   \\\cline{1-3}
{\ttfamily \{1..1000000000000\}}  &{\ttfamily 38 B} (608μs)  &N/A (freezes)   \\\cline{1-3}
{\ttfamily \{1..100000000000\}}  &{\ttfamily 37 B} (397μs)  &N/A (freezes)   \\\cline{1-3}
{\ttfamily \{1..10000000000\}}  &{\ttfamily 35 B} (983μs)  &N/A (freezes)   \\\cline{1-3}
{\ttfamily \{1..1000000000\}}  &{\ttfamily 34 B} (798μs)  &N/A (freezes)   \\\cline{1-3}
{\ttfamily \{1..100000000\}}  &{\ttfamily 33 B} (733μs)  &N/A (freezes)   \\\cline{1-3}
{\ttfamily \{1..10000000\}}  &{\ttfamily 32 B} (5ms 632μs)  &{\ttfamily 78.\+89 MB} (16s 388ms 569μs)   \\\cline{1-3}
{\ttfamily \{1..1000000\}}  &{\ttfamily 31 B} (1ms 381μs)  &{\ttfamily 6.\+89 MB} (1s 496ms 887μs)   \\\cline{1-3}
{\ttfamily \{1..100000\}}  &{\ttfamily 30 B} (950μs)  &{\ttfamily 588.\+89 kB} (146ms 921μs)   \\\cline{1-3}
{\ttfamily \{1..10000\}}  &{\ttfamily 29 B} (1ms 114μs)  &{\ttfamily 48.\+89 kB} (14ms 187μs)   \\\cline{1-3}
{\ttfamily \{1..1000\}}  &{\ttfamily 28 B} (760μs)  &{\ttfamily 3.\+89 kB} (1ms 453μs)   \\\cline{1-3}
{\ttfamily \{1..100\}}  &{\ttfamily 22 B} (345μs)  &{\ttfamily 291 B} (196μs)   \\\cline{1-3}
{\ttfamily \{1..10\}}  &{\ttfamily 10 B} (533μs)  &{\ttfamily 20 B} (37μs)   \\\cline{1-3}
{\ttfamily \{1..3\}}  &{\ttfamily 7 B} (190μs)  &{\ttfamily 5 B} (27μs)   \\\cline{1-3}
\end{longtabu}


\subsubsection*{Faster algorithms}

When you need expansion, braces is still much faster.

\+\_\+(the following results were generated using {\ttfamily braces.\+expand()} and {\ttfamily minimatch.\+brace\+Expand()}, respectively)\+\_\+

\tabulinesep=1mm
\begin{longtabu} spread 0pt [c]{*{3}{|X[-1]}|}
\hline
\rowcolor{\tableheadbgcolor}\textbf{ {\bfseries Pattern}  }&\textbf{ {\bfseries braces}  }&\textbf{ {\bfseries \mbox{[}minimatch\mbox{]}\mbox{[}\mbox{]}}   }\\\cline{1-3}
\endfirsthead
\hline
\endfoot
\hline
\rowcolor{\tableheadbgcolor}\textbf{ {\bfseries Pattern}  }&\textbf{ {\bfseries braces}  }&\textbf{ {\bfseries \mbox{[}minimatch\mbox{]}\mbox{[}\mbox{]}}   }\\\cline{1-3}
\endhead
{\ttfamily \{1..10000000\}}  &{\ttfamily 78.\+89 MB} (2s 698ms 642μs)  &{\ttfamily 78.\+89 MB} (18s 601ms 974μs)   \\\cline{1-3}
{\ttfamily \{1..1000000\}}  &{\ttfamily 6.\+89 MB} (458ms 576μs)  &{\ttfamily 6.\+89 MB} (1s 491ms 621μs)   \\\cline{1-3}
{\ttfamily \{1..100000\}}  &{\ttfamily 588.\+89 kB} (20ms 728μs)  &{\ttfamily 588.\+89 kB} (156ms 919μs)   \\\cline{1-3}
{\ttfamily \{1..10000\}}  &{\ttfamily 48.\+89 kB} (2ms 202μs)  &{\ttfamily 48.\+89 kB} (13ms 641μs)   \\\cline{1-3}
{\ttfamily \{1..1000\}}  &{\ttfamily 3.\+89 kB} (1ms 796μs)  &{\ttfamily 3.\+89 kB} (1ms 958μs)   \\\cline{1-3}
{\ttfamily \{1..100\}}  &{\ttfamily 291 B} (424μs)  &{\ttfamily 291 B} (211μs)   \\\cline{1-3}
{\ttfamily \{1..10\}}  &{\ttfamily 20 B} (487μs)  &{\ttfamily 20 B} (72μs)   \\\cline{1-3}
{\ttfamily \{1..3\}}  &{\ttfamily 5 B} (166μs)  &{\ttfamily 5 B} (27μs)   \\\cline{1-3}
\end{longtabu}


If you\textquotesingle{}d like to run these comparisons yourself, see \href{test/support/generate.js}{\tt test/support/generate.\+js}.

\subsection*{Benchmarks}

\subsubsection*{Running benchmarks}

Install dev dependencies\+:


\begin{DoxyCode}
npm i -d && npm benchmark
\end{DoxyCode}


\subsubsection*{Latest results}

Braces is more accurate, without sacrificing performance.


\begin{DoxyCode}
# range (expanded)
  braces x 29,040 ops/sec ±3.69% (91 runs sampled))
  minimatch x 4,735 ops/sec ±1.28% (90 runs sampled)

# range (optimized for regex)
  braces x 382,878 ops/sec ±0.56% (94 runs sampled)
  minimatch x 1,040 ops/sec ±0.44% (93 runs sampled)

# nested ranges (expanded)
  braces x 19,744 ops/sec ±2.27% (92 runs sampled))
  minimatch x 4,579 ops/sec ±0.50% (93 runs sampled)

# nested ranges (optimized for regex)
  braces x 246,019 ops/sec ±2.02% (93 runs sampled)
  minimatch x 1,028 ops/sec ±0.39% (94 runs sampled)

# set (expanded) 
  braces x 138,641 ops/sec ±0.53% (95 runs sampled)
  minimatch x 219,582 ops/sec ±0.98% (94 runs sampled)

# set (optimized for regex)
  braces x 388,408 ops/sec ±0.41% (95 runs sampled)
  minimatch x 44,724 ops/sec ±0.91% (89 runs sampled)

# nested sets (expanded)
  braces x 84,966 ops/sec ±0.48% (94 runs sampled)
  minimatch x 140,720 ops/sec ±0.37% (95 runs sampled)

# nested sets (optimized for regex)
  braces x 263,340 ops/sec ±2.06% (92 runs sampled)
  minimatch x 28,714 ops/sec ±0.40% (90 runs sampled)
\end{DoxyCode}


\subsection*{About}

$<$details$>$ 

{\bfseries Contributing}

Pull requests and stars are always welcome. For bugs and feature requests, \href{../../issues/new}{\tt please create an issue}.

$<$/details$>$

$<$details$>$ 

{\bfseries Running Tests}

Running and reviewing unit tests is a great way to get familiarized with a library and its A\+PI. You can install dependencies and run tests with the following command\+:


\begin{DoxyCode}
$ npm install && npm test
\end{DoxyCode}


$<$/details$>$

$<$details$>$ 

{\bfseries Building docs}

\+\_\+(This project\textquotesingle{}s readme.\+md is generated by \href{https://github.com/verbose/verb-generate-readme}{\tt verb}, please don\textquotesingle{}t edit the readme directly. Any changes to the readme must be made in the .verb.\+md \char`\"{}.\+verb.\+md\char`\"{} readme template.)\+\_\+

To generate the readme, run the following command\+:


\begin{DoxyCode}
$ npm install -g verbose/verb#dev verb-generate-readme && verb
\end{DoxyCode}


$<$/details$>$

\subsubsection*{Contributors}

\tabulinesep=1mm
\begin{longtabu} spread 0pt [c]{*{2}{|X[-1]}|}
\hline
\rowcolor{\tableheadbgcolor}\multicolumn{2}{|p{(\linewidth-\tabcolsep*2-\arrayrulewidth*1)*2/2}|}{\cellcolor{\tableheadbgcolor}\textbf{ $\ast$$\ast$\+Commits$\ast$   }}\\\cline{1-2}
\endfirsthead
\hline
\endfoot
\hline
\rowcolor{\tableheadbgcolor}\multicolumn{2}{|p{(\linewidth-\tabcolsep*2-\arrayrulewidth*1)*2/2}|}{\cellcolor{\tableheadbgcolor}\textbf{ $\ast$$\ast$\+Commits$\ast$   }}\\\cline{1-2}
\endhead
197  &\href{https://github.com/jonschlinkert}{\tt jonschlinkert}   \\\cline{1-2}
4  &\href{https://github.com/doowb}{\tt doowb}   \\\cline{1-2}
1  &\href{https://github.com/es128}{\tt es128}   \\\cline{1-2}
1  &\href{https://github.com/eush77}{\tt eush77}   \\\cline{1-2}
1  &\href{https://github.com/hemanth}{\tt hemanth}   \\\cline{1-2}
1  &\href{https://github.com/wtgtybhertgeghgtwtg}{\tt wtgtybhertgeghgtwtg}   \\\cline{1-2}
\end{longtabu}


\subsubsection*{Author}

{\bfseries Jon Schlinkert}


\begin{DoxyItemize}
\item \href{https://github.com/jonschlinkert}{\tt Git\+Hub Profile}
\item \href{https://twitter.com/jonschlinkert}{\tt Twitter Profile}
\item \href{https://linkedin.com/in/jonschlinkert}{\tt Linked\+In Profile}
\end{DoxyItemize}

\subsubsection*{License}

Copyright © 2019, \href{https://github.com/jonschlinkert}{\tt Jon Schlinkert}. Released under the \mbox{[}M\+IT License\mbox{]}(L\+I\+C\+E\+N\+SE).





{\itshape This file was generated by \href{https://github.com/verbose/verb-generate-readme}{\tt verb-\/generate-\/readme}, v0.\+8.\+0, on April 08, 2019.} 