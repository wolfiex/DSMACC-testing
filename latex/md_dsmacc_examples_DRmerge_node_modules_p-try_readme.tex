\begin{quote}
Start a promise chain \end{quote}


\href{http://cryto.net/~joepie91/blog/2016/05/11/what-is-promise-try-and-why-does-it-matter/}{\tt How is it useful?}

\subsection*{Install}


\begin{DoxyCode}
$ npm install p-try
\end{DoxyCode}


\subsection*{Usage}


\begin{DoxyCode}
const pTry = require('p-try');

(async () => \{
    try \{
        const value = await pTry(() => \{
            return synchronousFunctionThatMightThrow();
        \});
        console.log(value);
    \} catch (error) \{
        console.error(error);
    \}
\})();
\end{DoxyCode}


\subsection*{A\+PI}

\subsubsection*{p\+Try(fn, ...arguments)}

Returns a {\ttfamily Promise} resolved with the value of calling {\ttfamily fn(...arguments)}. If the function throws an error, the returned {\ttfamily Promise} will be rejected with that error.

Support for passing arguments on to the {\ttfamily fn} is provided in order to be able to avoid creating unnecessary closures. You probably don\textquotesingle{}t need this optimization unless you\textquotesingle{}re pushing a {\itshape lot} of functions.

\paragraph*{fn}

The function to run to start the promise chain.

\paragraph*{arguments}

Arguments to pass to {\ttfamily fn}.

\subsection*{Related}


\begin{DoxyItemize}
\item \href{https://github.com/sindresorhus/p-finally}{\tt p-\/finally} -\/ {\ttfamily Promise\+::finally()} ponyfill -\/ Invoked when the promise is settled regardless of outcome
\item \href{https://github.com/sindresorhus/promise-fun}{\tt More…}
\end{DoxyItemize}

\subsection*{License}

M\+IT © \href{https://sindresorhus.com}{\tt Sindre Sorhus} 