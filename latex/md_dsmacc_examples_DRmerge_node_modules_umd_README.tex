

Universal Module Definition for use in automated build systems


\begin{DoxyItemize}
\item simple synchronous wrapping of a string
\item {\ttfamily return} style module support
\item Common\+JS support
\item prevents internal U\+M\+Ds from conflicting
\end{DoxyItemize}

\href{https://travis-ci.org/ForbesLindesay/umd}{\tt } \href{https://david-dm.org/ForbesLindesay/umd}{\tt } \href{https://www.npmjs.com/package/umd}{\tt }

\subsection*{Source Format}

In order for the U\+MD wrapper to work the source code for your module should {\ttfamily return} the export, e.\+g.


\begin{DoxyCode}
function method() \{
  //code
\}
method.helper = function () \{
  //code
\}
return method;
\end{DoxyCode}


For examples, see the examples directory. The Common\+JS module format is also supported by passing true as the second argument to methods.

\subsection*{A\+PI}

options\+:


\begin{DoxyItemize}
\item {\ttfamily common\+JS} (default\+: {\ttfamily false}) -\/ If common\+JS is {\ttfamily true} then it will accept Common\+JS source instead of source code which {\ttfamily return}s the module.
\end{DoxyItemize}

\subsubsection*{umd(name, source, \mbox{[}options\mbox{]})}

The {\ttfamily name} should the the name of the module. Use a string like name, all lower case with hyphens instead of spaces.

If {\ttfamily source} should be a string, that is wrapped in umd and returned as a string.

\subsubsection*{umd.\+prelude(module, \mbox{[}options\mbox{]})}

return the text which will be inserted before a module.

\subsubsection*{umd.\+postlude(module, \mbox{[}options\mbox{]})}

return the text which will be inserted after a module.

\subsection*{Command Line}


\begin{DoxyCode}
Usage: umd <name> <source> <destination> [options]

Pipe Usage: umd <name> [options] < source > destination

Options:

 -h --help     Display usage information
 -c --commonJS Use CommonJS module format
\end{DoxyCode}


You can easilly pipe unix commands together like\+:


\begin{DoxyCode}
cat my-module.js | umd my-module | uglify-js > my-module.umd.min.js
\end{DoxyCode}


\subsection*{Name Casing and Characters}

The {\ttfamily name} passed to {\ttfamily umd} will be converted to camel case ({\ttfamily my-\/library} becomes {\ttfamily my\+Library}) and may only contain\+:


\begin{DoxyItemize}
\item alphanumeric characters
\item \$
\item \+\_\+
\end{DoxyItemize}

The name may not begin with a number. Invalid characters will be stripped.

\subsection*{License}

M\+IT 