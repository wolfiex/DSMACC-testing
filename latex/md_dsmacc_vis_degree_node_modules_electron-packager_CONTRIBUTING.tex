Electron Packager is a community-\/driven project. As such, we welcome and encourage all sorts of contributions. They include, but are not limited to\+:


\begin{DoxyItemize}
\item Constructive feedback
\item \href{#questions-about-usage}{\tt Questions about usage}
\item \href{#before-opening-bug-reportstechnical-issues}{\tt Bug reports / technical issues}
\item Documentation changes
\item Feature requests
\item \href{#filing-pull-requests}{\tt Pull requests}
\end{DoxyItemize}

We strongly suggest that before filing an issue, you search through the existing issues to see if it has already been filed by someone else.

This project is a part of the Electron ecosystem. As such, all contributions to this project follow https\+://github.com/electron/electron/blob/master/\+C\+O\+D\+E\+\_\+\+O\+F\+\_\+\+C\+O\+N\+D\+U\+C\+T.\+md \char`\"{}\+Electron\textquotesingle{}s code of conduct\char`\"{} where appropriate.

\subsection*{Questions about usage}

If you have questions about usage, we encourage you to visit one of the several \href{https://github.com/electron/electron#community}{\tt community-\/driven sites}.

\subsection*{Before opening bug reports/technical issues}

\subsubsection*{Debugging}

One way to troubleshoot potential problems is to set the {\ttfamily D\+E\+B\+UG} environment variable before calling electron-\/packager. This will print debug information from the specified modules. The value of the environment variable is a comma-\/separated list of modules which support this logging feature. Known modules include\+:


\begin{DoxyItemize}
\item {\ttfamily electron-\/download}
\item {\ttfamily electron-\/osx-\/sign}
\item {\ttfamily electron-\/packager} (always use this one before filing an issue)
\item {\ttfamily extract-\/zip}
\item {\ttfamily get-\/package-\/info}
\end{DoxyItemize}

We use the \href{https://www.npmjs.com/package/debug#usage}{\tt {\ttfamily debug}} module for this functionality. It has examples on how to set environment variables if you don\textquotesingle{}t know how.

{\bfseries If you are using {\ttfamily npm run} to execute {\ttfamily electron-\/packager}, run the {\ttfamily electron-\/packager} command without using {\ttfamily npm run} and make a note of the output, because {\ttfamily npm run} does not print out error messages when a script errors.}

\subsection*{Contribution suggestions}

We use the label \href{https://github.com/electron-userland/electron-packager/issues?q=is%3Aopen+is%3Aissue+label%3A%22help+wanted%22}{\tt {\ttfamily help wanted}} in the issue tracker to denote fairly-\/well-\/scoped-\/out bugs or feature requests that the community can pick up and work on. If any of those labeled issues do not have enough information, please feel free to ask constructive questions. (This applies to any open issue.)

\subsection*{Filing Pull Requests}

Here are some things to keep in mind as you file pull requests to fix bugs, add new features, etc.\+:


\begin{DoxyItemize}
\item Travis CI is used to make sure that the project builds packages as expected on the supported platforms, using supported Node.\+js versions.
\item Unless it\textquotesingle{}s impractical, please write tests for your changes. This will help us so that we can spot regressions much easier.
\item If your PR changes the behavior of an existing feature, or adds a new feature, please add/edit the package\textquotesingle{}s documentation. Files that will likely need to be updated include {\ttfamily readme.\+md}, {\ttfamily docs/api.\+md}, and {\ttfamily usage.\+txt}.
\item This project uses the \href{https://www.npmjs.com/package/standard}{\tt Java\+Script Standard Style} as a coding convention. CI will fail if the PR does not conform to this standard.
\item One of the philosophies of the project is to keep the code base as small as possible. If you are adding a new feature, think about whether it is appropriate to go into a separate \mbox{\hyperlink{classNode}{Node}} module, and then be integrated into this project.
\item If you are contributing a nontrivial change, please add an entry to {\ttfamily N\+E\+W\+S.\+md}. The format is similar to the one described at \href{http://keepachangelog.com/}{\tt Keep a Changelog}.
\item Please {\bfseries do not} bump the version number in your pull requests, the maintainers will do that. Feel free to indicate whether the changes require a major, minor, or patch version bump, as prescribed by the \href{http://semver.org/}{\tt semantic versioning specification}.
\item Once your pull request is approved, please make sure your commits are rebased onto the latest commit in the master branch, and that you limit/squash the number of commits created to a \char`\"{}feature\char`\"{}-\/level. For instance\+:
\end{DoxyItemize}

bad\+:


\begin{DoxyCode}
commit 1: add foo option
commit 2: standardize code
commit 3: add test
commit 4: add docs
commit 5: add bar option
commit 6: add test + docs
\end{DoxyCode}


good\+:


\begin{DoxyCode}
commit 1: add foo option
commit 2: add bar option
\end{DoxyCode}


Squashing commits during discussion of the pull request is almost always unnecessary, and makes it more difficult for both the submitters and reviewers to understand what changed in between comments. However, rebasing is encouraged when practical, particularly when there\textquotesingle{}s a merge conflict.

If you are continuing the work of another person\textquotesingle{}s PR and need to rebase/squash, please retain the attribution of the original author(s) and continue the work in subsequent commits.

\subsubsection*{Running tests}

To run the test suite on your local machine, you\textquotesingle{}ll first need to do a little setup.

If you\textquotesingle{}re using mac\+OS\+:


\begin{DoxyCode}
TRAVIS\_OS\_NAME=osx ./test/ci/before\_install.sh
\end{DoxyCode}


If you\textquotesingle{}re using a Debian/\+Ubuntu-\/derived distribution of Linux with x86\+\_\+64 architecture\+:


\begin{DoxyCode}
TRAVIS\_OS\_NAME=linux ./test/ci/before\_install.sh
\end{DoxyCode}


Then you can install dependencies and run the suite\+:


\begin{DoxyCode}
npm install
npm test
\end{DoxyCode}


\subsubsection*{Creating test fixtures}

For some unit tests, a test fixture Electron project is required. Sometimes it\textquotesingle{}s OK to use an existing fixture, such as {\ttfamily basic}. If you need to add a new fixture\+:


\begin{DoxyEnumerate}
\item Create a new subdirectory in {\ttfamily test/fixtures/}.
\item Add a {\ttfamily package.\+json} with only the minimal configuration necessary for your test(s).
\item If necessary, add supporting files, such as the JS file specified in the {\ttfamily main} key in the {\ttfamily package.\+json} file.
\item Use {\ttfamily fixture\+Subdir} from {\ttfamily test/util.\+js} to reference the fixture subdirectory in your test.
\end{DoxyEnumerate}

\subsection*{For Collaborators}

Make sure to get a {\ttfamily \+:thumbsup\+:}, {\ttfamily +1} or {\ttfamily L\+G\+TM} from another collaborator before merging a PR.

\subsubsection*{Release process}


\begin{DoxyItemize}
\item if you aren\textquotesingle{}t sure if a release should happen, open an issue
\item make sure that {\ttfamily N\+E\+W\+S.\+md} is up to date
\item make sure the tests pass
\item {\ttfamily npm version $<$major$\vert$minor$\vert$patch$>$}
\item {\ttfamily git push \&\& git push -\/-\/tags} (or {\ttfamily git push} with {\ttfamily git config -\/-\/global push.\+follow\+Tags true} on latest git)
\item create a new Git\+Hub release from the pushed tag with the contents of {\ttfamily N\+E\+W\+S.\+md} for that version
\item close the milestone associated with the version if one is open
\item {\ttfamily npm publish} 
\end{DoxyItemize}