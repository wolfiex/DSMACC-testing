A stream that emits multiple other streams one after another.

{\bfseries NB} Currently {\ttfamily combined-\/stream} works with streams version 1 only. There is ongoing effort to switch this library to streams version 2. Any help is welcome. \+:) Meanwhile you can explore other libraries that provide streams2 support with more or less compatibility with {\ttfamily combined-\/stream}.


\begin{DoxyItemize}
\item \href{https://www.npmjs.com/package/combined-stream2}{\tt combined-\/stream2}\+: A drop-\/in streams2-\/compatible replacement for the combined-\/stream module.
\item \href{https://www.npmjs.com/package/multistream}{\tt multistream}\+: A stream that emits multiple other streams one after another.
\end{DoxyItemize}

\subsection*{Installation}


\begin{DoxyCode}
npm install combined-stream
\end{DoxyCode}


\subsection*{Usage}

Here is a simple example that shows how you can use combined-\/stream to combine two files into one\+:


\begin{DoxyCode}
var CombinedStream = require('combined-stream');
var fs = require('fs');

var combinedStream = CombinedStream.create();
combinedStream.append(fs.createReadStream('file1.txt'));
combinedStream.append(fs.createReadStream('file2.txt'));

combinedStream.pipe(fs.createWriteStream('combined.txt'));
\end{DoxyCode}


While the example above works great, it will pause all source streams until they are needed. If you don\textquotesingle{}t want that to happen, you can set {\ttfamily pause\+Streams} to {\ttfamily false}\+:


\begin{DoxyCode}
var CombinedStream = require('combined-stream');
var fs = require('fs');

var combinedStream = CombinedStream.create(\{pauseStreams: false\});
combinedStream.append(fs.createReadStream('file1.txt'));
combinedStream.append(fs.createReadStream('file2.txt'));

combinedStream.pipe(fs.createWriteStream('combined.txt'));
\end{DoxyCode}


However, what if you don\textquotesingle{}t have all the source streams yet, or you don\textquotesingle{}t want to allocate the resources (file descriptors, memory, etc.) for them right away? Well, in that case you can simply provide a callback that supplies the stream by calling a {\ttfamily next()} function\+:


\begin{DoxyCode}
var CombinedStream = require('combined-stream');
var fs = require('fs');

var combinedStream = CombinedStream.create();
combinedStream.append(function(next) \{
  next(fs.createReadStream('file1.txt'));
\});
combinedStream.append(function(next) \{
  next(fs.createReadStream('file2.txt'));
\});

combinedStream.pipe(fs.createWriteStream('combined.txt'));
\end{DoxyCode}


\subsection*{A\+PI}

\subsubsection*{Combined\+Stream.\+create(\mbox{[}options\mbox{]})}

Returns a new combined stream object. Available options are\+:


\begin{DoxyItemize}
\item {\ttfamily max\+Data\+Size}
\item {\ttfamily pause\+Streams}
\end{DoxyItemize}

The effect of those options is described below.

\subsubsection*{combined\+Stream.\+pause\+Streams = {\ttfamily true}}

Whether to apply back pressure to the underlaying streams. If set to {\ttfamily false}, the underlaying streams will never be paused. If set to {\ttfamily true}, the underlaying streams will be paused right after being appended, as well as when {\ttfamily delayed\+Stream.\+pipe()} wants to throttle.

\subsubsection*{combined\+Stream.\+max\+Data\+Size = {\ttfamily 2 $\ast$ 1024 $\ast$ 1024}}

The maximum amount of bytes (or characters) to buffer for all source streams. If this value is exceeded, {\ttfamily combined\+Stream} emits an `\textquotesingle{}error'\`{} event.

\subsubsection*{combined\+Stream.\+data\+Size = {\ttfamily 0}}

The amount of bytes (or characters) currently buffered by {\ttfamily combined\+Stream}.

\subsubsection*{combined\+Stream.\+append(stream)}

Appends the given {\ttfamily stream} to the combined\+Stream object. If {\ttfamily pause\+Streams} is set to \`{}true, this stream will also be paused right away.

{\ttfamily streams} can also be a function that takes one parameter called {\ttfamily next}. {\ttfamily next} is a function that must be invoked in order to provide the {\ttfamily next} stream, see example above.

Regardless of how the {\ttfamily stream} is appended, combined-\/stream always attaches an `\textquotesingle{}error'\`{} listener to it, so you don\textquotesingle{}t have to do that manually.

Special case\+: {\ttfamily stream} can also be a String or Buffer.

\subsubsection*{combined\+Stream.\+write(data)}

You should not call this, {\ttfamily combined\+Stream} takes care of piping the appended streams into itself for you.

\subsubsection*{combined\+Stream.\+resume()}

Causes {\ttfamily combined\+Stream} to start drain the streams it manages. The function is idempotent, and also emits a `\textquotesingle{}resume'\`{} event each time which usually goes to the stream that is currently being drained.

\subsubsection*{combined\+Stream.\+pause();}

If {\ttfamily combined\+Stream.\+pause\+Streams} is set to {\ttfamily false}, this does nothing. Otherwise a `\textquotesingle{}pause'\`{} event is emitted, this goes to the stream that is currently being drained, so you can use it to apply back pressure.

\subsubsection*{combined\+Stream.\+end();}

Sets {\ttfamily combined\+Stream.\+writable} to false, emits an `\textquotesingle{}end'\`{} event, and removes all streams from the queue.

\subsubsection*{combined\+Stream.\+destroy();}

Same as {\ttfamily combined\+Stream.\+end()}, except it emits a `\textquotesingle{}close'{\ttfamily event instead of }\textquotesingle{}end\textquotesingle{}\`{}.

\subsection*{License}

combined-\/stream is licensed under the M\+IT license. 