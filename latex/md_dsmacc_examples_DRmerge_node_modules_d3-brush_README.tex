Brushing is the interactive specification a one-\/ or two-\/dimensional selected region using a pointing gesture, such as by clicking and dragging the mouse. Brushing is often used to select discrete elements, such as dots in a scatterplot or files on a desktop. It can also be used to zoom-\/in to a region of interest, or to select continuous regions for \href{http://square.github.io/crossfilter/}{\tt cross-\/filtering data} or live histograms\+:

\href{http://bl.ocks.org/mbostock/0d20834e3d5a46138752f86b9b79727e}{\tt }

The d3-\/brush module implements brushing for mouse and touch events using \href{https://www.w3.org/TR/SVG/}{\tt S\+VG}. Click and drag on the brush selection to translate the selection. Click and drag on one of the selection handles to move the corresponding edge (or edges) of the selection. Click and drag on the invisible overlay to define a new brush selection, or click anywhere within the brushable region while holding down the M\+E\+TA (⌘) key. Holding down the A\+LT (⌥) key while moving the brush causes it to reposition around its center, while holding down S\+P\+A\+CE locks the current brush size, allowing only translation.

Brushes also support programmatic control. For example, you can listen to \href{#brush-events}{\tt {\itshape end} events}, and then initiate a transition with \href{#brush_move}{\tt {\itshape brush}.move} to snap the brush selection to semantic boundaries\+:

\href{http://bl.ocks.org/mbostock/6232537}{\tt }

\subsection*{Installing}

If you use N\+PM, {\ttfamily npm install d3-\/brush}. Otherwise, download the \href{https://github.com/d3/d3-brush/releases/latest}{\tt latest release}. You can also load directly from \href{https://d3js.org}{\tt d3js.\+org}, either as a \href{https://d3js.org/d3-brush.v1.min.js}{\tt standalone library} or as part of \href{https://github.com/d3/d3}{\tt D3 4.\+0}. A\+MD, Common\+JS, and vanilla environments are supported. In vanilla, a {\ttfamily d3} global is exported\+:


\begin{DoxyCode}
<script src="https://d3js.org/d3-color.v1.min.js"></script>
<script src="https://d3js.org/d3-dispatch.v1.min.js"></script>
<script src="https://d3js.org/d3-ease.v1.min.js"></script>
<script src="https://d3js.org/d3-interpolate.v1.min.js"></script>
<script src="https://d3js.org/d3-timer.v1.min.js"></script>
<script src="https://d3js.org/d3-selection.v1.min.js"></script>
<script src="https://d3js.org/d3-transition.v1.min.js"></script>
<script src="https://d3js.org/d3-drag.v1.min.js"></script>
<script src="https://d3js.org/d3-brush.v1.min.js"></script>
<script>

var brush = d3.brush();

</script>
\end{DoxyCode}


\href{https://tonicdev.com/npm/d3-brush}{\tt Try d3-\/brush in your browser.}

\subsection*{A\+PI Reference}

\href{#brush}{\tt \#} d3.{\bfseries brush}() \href{https://github.com/d3/d3-brush/blob/master/src/brush.js#L131}{\tt $<$$>$}

Creates a new two-\/dimensional brush.

\href{#brushX}{\tt \#} d3.{\bfseries brushX}() \href{https://github.com/d3/d3-brush/blob/master/src/brush.js#L123}{\tt $<$$>$}

Creates a new one-\/dimensional brush along the {\itshape x}-\/dimension.

\href{#brushY}{\tt \#} d3.{\bfseries brushY}() \href{https://github.com/d3/d3-brush/blob/master/src/brush.js#L127}{\tt $<$$>$}

Creates a new one-\/dimensional brush along the {\itshape y}-\/dimension.

\href{#_brush}{\tt \#} {\itshape brush}({\itshape group}) \href{https://github.com/d3/d3-brush/blob/master/src/brush.js#L142}{\tt $<$$>$}

Applies the brush to the specified {\itshape group}, which must be a \href{https://github.com/d3/d3-selection}{\tt selection} of S\+VG \href{https://www.w3.org/TR/SVG/struct.html#Groups}{\tt G elements}. This function is typically not invoked directly, and is instead invoked via \href{https://github.com/d3/d3-selection#selection_call}{\tt {\itshape selection}.call}. For example, to render a brush\+:


\begin{DoxyCode}
svg.append("g")
    .attr("class", "brush")
    .call(d3.brush().on("brush", brushed));
\end{DoxyCode}


Internally, the brush uses \href{https://github.com/d3/d3-selection#selection_on}{\tt {\itshape selection}.on} to bind the necessary event listeners for dragging. The listeners use the name {\ttfamily .brush}, so you can subsequently unbind the brush event listeners as follows\+:


\begin{DoxyCode}
group.on(".brush", null);
\end{DoxyCode}


The brush also creates the S\+VG elements necessary to display the brush selection and to receive input events for interaction. You can add, remove or modify these elements as desired to change the brush appearance; you can also apply stylesheets to modify the brush appearance. The structure of a two-\/dimensional brush is as follows\+:


\begin{DoxyCode}
<g class="brush" fill="none" pointer-events="all" style="-webkit-tap-highlight-color: rgba(0, 0, 0, 0);">
  <rect class="overlay" pointer-events="all" cursor="crosshair" x="0" y="0" width="960"
       height="500"></rect>
  <rect class="selection" cursor="move" fill="#777" fill-opacity="0.3" stroke="#fff"
       shape-rendering="crispEdges" x="112" y="194" width="182" height="83"></rect>
  <rect class="handle handle--n" cursor="ns-resize" x="107" y="189" width="192" height="10"></rect>
  <rect class="handle handle--e" cursor="ew-resize" x="289" y="189" width="10" height="93"></rect>
  <rect class="handle handle--s" cursor="ns-resize" x="107" y="272" width="192" height="10"></rect>
  <rect class="handle handle--w" cursor="ew-resize" x="107" y="189" width="10" height="93"></rect>
  <rect class="handle handle--nw" cursor="nwse-resize" x="107" y="189" width="10" height="10"></rect>
  <rect class="handle handle--ne" cursor="nesw-resize" x="289" y="189" width="10" height="10"></rect>
  <rect class="handle handle--se" cursor="nwse-resize" x="289" y="272" width="10" height="10"></rect>
  <rect class="handle handle--sw" cursor="nesw-resize" x="107" y="272" width="10" height="10"></rect>
</g>
\end{DoxyCode}


The overlay rect covers the brushable area defined by \href{#brush_extent}{\tt {\itshape brush}.extent}. The selection rect covers the area defined by the current \href{#brushSelection}{\tt brush selection}. The handle rects cover the edges and corners of the brush selection, allowing the corresponding value in the brush selection to be modified interactively. To modify the brush selection programmatically, use \href{#brush_move}{\tt {\itshape brush}.move}.

\href{#brush_move}{\tt \#} {\itshape brush}.{\bfseries move}({\itshape group}, {\itshape selection}) \href{https://github.com/d3/d3-brush/blob/master/src/brush.js#L189}{\tt $<$$>$}

Sets the active {\itshape selection} of the brush on the specified {\itshape group}, which must be a \href{https://github.com/d3/d3-selection}{\tt selection} or a \href{https://github.com/d3/d3-transition}{\tt transition} of S\+VG \href{https://www.w3.org/TR/SVG/struct.html#Groups}{\tt G elements}. The {\itshape selection} must be defined as an array of numbers, or null to clear the brush selection. For a \href{#brush}{\tt two-\/dimensional brush}, it must be defined as \mbox{[}\mbox{[}{\itshape x0}, {\itshape y0}\mbox{]}, \mbox{[}{\itshape x1}, {\itshape y1}\mbox{]}\mbox{]}, where {\itshape x0} is the minimum {\itshape x}-\/value, {\itshape y0} is the minimum {\itshape y}-\/value, {\itshape x1} is the maximum {\itshape x}-\/value, and {\itshape y1} is the maximum {\itshape y}-\/value. For an \href{#brushX}{\tt {\itshape x}-\/brush}, it must be defined as \mbox{[}{\itshape x0}, {\itshape x1}\mbox{]}; for a \href{#brushY}{\tt {\itshape y}-\/brush}, it must be defined as \mbox{[}{\itshape y0}, {\itshape y1}\mbox{]}. The selection may also be specified as a function which returns such an array; if a function, it is invoked for each selected element, being passed the current datum {\ttfamily d} and index {\ttfamily i}, with the {\ttfamily this} context as the current D\+OM element. The returned array defines the brush selection for that element.

\href{#brush_extent}{\tt \#} {\itshape brush}.{\bfseries extent}(\mbox{[}{\itshape extent}\mbox{]}) \href{https://github.com/d3/d3-brush/blob/master/src/brush.js#L521}{\tt $<$$>$}

If {\itshape extent} is specified, sets the brushable extent to the specified array of points \mbox{[}\mbox{[}{\itshape x0}, {\itshape y0}\mbox{]}, \mbox{[}{\itshape x1}, {\itshape y1}\mbox{]}\mbox{]}, where \mbox{[}{\itshape x0}, {\itshape y0}\mbox{]} is the top-\/left corner and \mbox{[}{\itshape x1}, {\itshape y1}\mbox{]} is the bottom-\/right corner, and returns this brush. The {\itshape extent} may also be specified as a function which returns such an array; if a function, it is invoked for each selected element, being passed the current datum {\ttfamily d} and index {\ttfamily i}, with the {\ttfamily this} context as the current D\+OM element. If {\itshape extent} is not specified, returns the current extent accessor, which defaults to\+:


\begin{DoxyCode}
function extent() \{
  var svg = this.ownerSVGElement || this;
  return [[0, 0], [svg.width.baseVal.value, svg.height.baseVal.value]];
\}
\end{DoxyCode}


This default implementation requires that the owner S\+VG element have defined \href{https://www.w3.org/TR/SVG/struct.html#SVGElementWidthAttribute}{\tt width} and \href{https://www.w3.org/TR/SVG/struct.html#SVGElementHeightAttribute}{\tt height} attributes rather than (for example) relying on C\+SS properties or the view\+Box attribute; S\+VG provides no programmatic method for retrieving the \href{https://www.w3.org/TR/SVG/coords.html#ViewportSpace}{\tt initial viewport size}. Alternatively, consider using \href{https://developer.mozilla.org/en-US/docs/Web/API/Element/getBoundingClientRect}{\tt {\itshape element}.get\+Bounding\+Client\+Rect}. (In Firefox, \href{https://developer.mozilla.org/en-US/docs/Web/API/Element/clientWidth}{\tt {\itshape element}.client\+Width} and \href{https://developer.mozilla.org/en-US/docs/Web/API/Element/clientHeight}{\tt {\itshape element}.client\+Height} is zero for S\+VG elements!)

The brush extent determines the size of the invisible overlay and also constrains the brush selection; the brush selection cannot go outside the brush extent.

\href{#brush_filter}{\tt \#} {\itshape brush}.{\bfseries filter}(\mbox{[}{\itshape filter}\mbox{]}) \href{https://github.com/d3/d3-brush/blob/master/src/brush.js#L525}{\tt $<$$>$}

If {\itshape filter} is specified, sets the filter to the specified function and returns the brush. If {\itshape filter} is not specified, returns the current filter, which defaults to\+:


\begin{DoxyCode}
function filter() \{
  return !event.button;
\}
\end{DoxyCode}


If the filter returns falsey, the initiating event is ignored and no brush gesture is started. Thus, the filter determines which input events are ignored. The default filter ignores mousedown events on secondary buttons, since those buttons are typically intended for other purposes, such as the context menu.

\href{#brush_handleSize}{\tt \#} {\itshape brush}.{\bfseries handle\+Size}(\mbox{[}{\itshape size}\mbox{]}) \href{https://github.com/d3/d3-brush/blob/master/src/brush.js#L529}{\tt $<$$>$}

If {\itshape size} is specified, sets the size of the brush handles to the specified number and returns the brush. If {\itshape size} is not specified, returns the current handle size, which defaults to six. This method must be called before \href{#_brush}{\tt applying the brush} to a selection; changing the handle size does not affect brushes that were previously rendered.

\href{#brush_on}{\tt \#} {\itshape brush}.{\bfseries on}({\itshape typenames}, \mbox{[}{\itshape listener}\mbox{]}) \href{https://github.com/d3/d3-brush/blob/master/src/brush.js#L533}{\tt $<$$>$}

If {\itshape listener} is specified, sets the event {\itshape listener} for the specified {\itshape typenames} and returns the brush. If an event listener was already registered for the same type and name, the existing listener is removed before the new listener is added. If {\itshape listener} is null, removes the current event listeners for the specified {\itshape typenames}, if any. If {\itshape listener} is not specified, returns the first currently-\/assigned listener matching the specified {\itshape typenames}, if any. When a specified event is dispatched, each {\itshape listener} will be invoked with the same context and arguments as \href{https://github.com/d3/d3-selection#selection_on}{\tt {\itshape selection}.on} listeners\+: the current datum {\ttfamily d} and index {\ttfamily i}, with the {\ttfamily this} context as the current D\+OM element.

The {\itshape typenames} is a string containing one or more {\itshape typename} separated by whitespace. Each {\itshape typename} is a {\itshape type}, optionally followed by a period ({\ttfamily .}) and a {\itshape name}, such as {\ttfamily brush.\+foo} and {\ttfamily brush.\+bar}; the name allows multiple listeners to be registered for the same {\itshape type}. The {\itshape type} must be one of the following\+:


\begin{DoxyItemize}
\item {\ttfamily start} -\/ at the start of a brush gesture, such as on mousedown.
\item {\ttfamily brush} -\/ when the brush moves, such as on mousemove.
\item {\ttfamily end} -\/ at the end of a brush gesture, such as on mouseup.
\end{DoxyItemize}

See \href{https://github.com/d3/d3-dispatch#dispatch_on}{\tt {\itshape dispatch}.on} and \href{#brush-events}{\tt Brush Events} for more.

\href{#brushSelection}{\tt \#} d3.{\bfseries brush\+Selection}({\itshape node}) \href{https://github.com/d3/d3-brush/blob/master/src/brush.js#L118}{\tt $<$$>$}

Returns the current brush selection for the specified {\itshape node}. Internally, an element’s brush state is stored as {\itshape element}.\+\_\+\+\_\+brush; however, you should use this method rather than accessing it directly. If the given {\itshape node} has no selection, returns null. Otherwise, the {\itshape selection} is defined as an array of numbers. For a \href{#brush}{\tt two-\/dimensional brush}, it is \mbox{[}\mbox{[}{\itshape x0}, {\itshape y0}\mbox{]}, \mbox{[}{\itshape x1}, {\itshape y1}\mbox{]}\mbox{]}, where {\itshape x0} is the minimum {\itshape x}-\/value, {\itshape y0} is the minimum {\itshape y}-\/value, {\itshape x1} is the maximum {\itshape x}-\/value, and {\itshape y1} is the maximum {\itshape y}-\/value. For an \href{#brushX}{\tt {\itshape x}-\/brush}, it is \mbox{[}{\itshape x0}, {\itshape x1}\mbox{]}; for a \href{#brushY}{\tt {\itshape y}-\/brush}, it is \mbox{[}{\itshape y0}, {\itshape y1}\mbox{]}.

\subsubsection*{Brush Events}

When a \href{#brush_on}{\tt brush event listener} is invoked, \href{https://github.com/d3/d3-selection#event}{\tt d3.\+event} is set to the current brush event. The {\itshape event} object exposes several fields\+:


\begin{DoxyItemize}
\item {\ttfamily target} -\/ the associated \href{#brush}{\tt brush behavior}.
\item {\ttfamily type} -\/ the string “start”, “brush” or “end”; see \href{#brush_on}{\tt {\itshape brush}.on}.
\item {\ttfamily selection} -\/ the current \href{#brushSelection}{\tt brush selection}.
\item {\ttfamily source\+Event} -\/ the underlying input event, such as mousemove or touchmove. 
\end{DoxyItemize}