Package your \href{http://electron.atom.io}{\tt Electron} app into O\+S-\/specific bundles ({\ttfamily .app}, {\ttfamily .exe}, etc.) via Java\+Script or the command line.

\href{https://travis-ci.org/electron-userland/electron-packager}{\tt } \href{https://ci.appveyor.com/project/electron-userland/electron-packager}{\tt } \href{https://coveralls.io/github/electron-userland/electron-packager?branch=master}{\tt } \href{https://dependencyci.com/github/electron-userland/electron-packager}{\tt }

\subsection*{About}

Electron Packager is a command line tool and Node.\+js library that bundles Electron-\/based application source code with a renamed Electron executable and supporting files into folders ready for distribution.

Note that packaged Electron applications can be relatively large. A zipped barebones OS X Electron application is around 40\+MB.

\subsubsection*{Electron Packager is an \href{http://openopensource.org/}{\tt O\+P\+EN Open Source Project}}

Individuals making significant and valuable contributions are given commit-\/access to the project to contribute as they see fit. This project is more like an open wiki than a standard guarded open source project.

See https\+://github.com/electron-\/userland/electron-\/packager/blob/master/\+C\+O\+N\+T\+R\+I\+B\+U\+T\+I\+N\+G.\+md \char`\"{}\+C\+O\+N\+T\+R\+I\+B\+U\+T\+I\+N\+G.\+md\char`\"{} and \href{http://openopensource.org/}{\tt openopensource.\+org} for more details.

\subsubsection*{\href{https://github.com/electron-userland/electron-packager/blob/master/NEWS.md}{\tt https\+://github.\+com/electron-\/userland/electron-\/packager/blob/master/\+N\+E\+W\+S.\+md} \char`\"{}\+Release Notes\char`\"{}}

\subsection*{Supported Platforms}

Electron Packager is known to run on the following {\bfseries host} platforms\+:


\begin{DoxyItemize}
\item Windows (32/64 bit)
\item OS X (also known as mac\+OS)
\item Linux (x86/x86\+\_\+64)
\end{DoxyItemize}

It generates executables/bundles for the following {\bfseries target} platforms\+:


\begin{DoxyItemize}
\item Windows (also known as {\ttfamily win32}, for both 32/64 bit)
\item OS X (also known as {\ttfamily darwin}) / \href{http://electron.atom.io/docs/v0.36.0/tutorial/mac-app-store-submission-guide/}{\tt Mac App Store} (also known as {\ttfamily mas})\textsuperscript{$\ast$}
\item Linux (for x86, x86\+\_\+64, and armv7l architectures)
\end{DoxyItemize}

\textsuperscript{$\ast$} {\itshape Note for OS X / M\+AS target bundles\+: the {\ttfamily .app} bundle can only be signed when building on a host OS X platform.}

\subsection*{Installation}

This module requires Node.\+js 4.\+0 or higher to run.


\begin{DoxyCode}
# for use in npm scripts
npm install electron-packager --save-dev

# for use from cli
npm install electron-packager -g
\end{DoxyCode}


\subsubsection*{Building Windows apps from non-\/\+Windows platforms}

Building an Electron app for the Windows target platform requires editing the {\ttfamily Electron.\+exe} file. Currently, Electron Packager uses \href{https://github.com/atom/node-rcedit}{\tt node-\/rcedit} to accomplish this. A Windows executable is bundled in that \mbox{\hyperlink{classNode}{Node}} package and needs to be run in order for this functionality to work, so on non-\/\+Windows host platforms, \href{https://www.winehq.org/}{\tt Wine} 1.\+6 or later needs to be installed. On OS X, it is installable via \href{http://brew.sh/}{\tt Homebrew}.

\subsection*{Usage}

\subsubsection*{From the Command Line}

Running electron-\/packager from the command line has this basic form\+:


\begin{DoxyCode}
electron-packager <sourcedir> <appname> --platform=<platform> --arch=<arch> [optional flags...]
\end{DoxyCode}


This will\+:


\begin{DoxyItemize}
\item Find or download the correct release of Electron
\item Use that version of Electron to create a app in {\ttfamily $<$out$>$/$<$appname$>$-\/$<$platform$>$-\/$<$arch$>$} $\ast$(this can be customized via an optional flag)$\ast$
\end{DoxyItemize}

{\ttfamily -\/-\/platform} and {\ttfamily -\/-\/arch} can be omitted, in two cases\+:


\begin{DoxyItemize}
\item If you specify {\ttfamily -\/-\/all} instead, bundles for all valid combinations of target platforms/architectures will be created.
\item Otherwise, a single bundle for the host platform/architecture will be created.
\end{DoxyItemize}

For an overview of the other optional flags, run {\ttfamily electron-\/packager -\/-\/help} or see \href{https://github.com/electron-userland/electron-packager/blob/master/usage.txt}{\tt usage.\+txt}. For detailed descriptions, see the https\+://github.com/electron-\/userland/electron-\/packager/blob/master/docs/api.\+md \char`\"{}\+A\+P\+I documentation\char`\"{}.

If {\ttfamily appname} is omitted, this will use the name specified by \char`\"{}product\+Name\char`\"{} or \char`\"{}name\char`\"{} in the nearest package.\+json.

{\bfseries Characters in the Electron app name which are not allowed in all target platforms\textquotesingle{} filenames (e.\+g., {\ttfamily /}), will be replaced by hyphens ({\ttfamily -\/}).}

You should be able to launch the app on the platform you built for. If not, check your settings and try again.

{\bfseries Be careful} not to include {\ttfamily node\+\_\+modules} you don\textquotesingle{}t want into your final app. If you put them in the {\ttfamily dev\+Dependencies} section of {\ttfamily package.\+json}, by default none of the modules related to those dependencies will be copied in the app bundles. (This behavior can be turned off with the {\ttfamily -\/-\/no-\/prune} flag.) In addition, folders like {\ttfamily .git} and {\ttfamily node\+\_\+modules/.bin} will be ignored by default. You can use {\ttfamily -\/-\/ignore} to ignore files and folders via a regular expression ({\itshape not} a \href{https://en.wikipedia.org/wiki/Glob_%28programming%29}{\tt glob pattern}). Examples include {\ttfamily -\/-\/ignore=\textbackslash{}.gitignore} or {\ttfamily -\/-\/ignore=\char`\"{}\textbackslash{}.\+git(ignore$\vert$modules)\char`\"{}}.

\paragraph*{Example}

Let\textquotesingle{}s assume that you have made an app based on the \href{https://github.com/electron/electron-quick-start}{\tt electron-\/quick-\/start} repository on a OS X host platform with the following file structure\+:


\begin{DoxyCode}
foobar
├── package.json
├── index.html
├── […other files, like LICENSE…]
└── script.js
\end{DoxyCode}


…and that the following is true\+:


\begin{DoxyItemize}
\item {\ttfamily electron-\/packager} is installed globally
\item {\ttfamily product\+Name} in {\ttfamily package.\+json} has been set to {\ttfamily Foo Bar}
\item The {\ttfamily electron} module is in the {\ttfamily dev\+Dependencies} section of {\ttfamily package.\+json}, and set to the exact version of {\ttfamily 1.\+4.\+15}.
\item {\ttfamily npm install} for the {\ttfamily Foo Bar} app has been run at least once
\end{DoxyItemize}

When one runs the following command for the first time in the {\ttfamily foobar} directory\+:


\begin{DoxyCode}
electron-packager .
\end{DoxyCode}


{\ttfamily electron-\/packager} will do the following\+:


\begin{DoxyItemize}
\item Use the current directory for the {\ttfamily sourcedir}
\item Infer the {\ttfamily appname} from the {\ttfamily product\+Name} in {\ttfamily package.\+json}
\item Infer the {\ttfamily app\+Version} from the {\ttfamily version} in {\ttfamily package.\+json}
\item Infer the {\ttfamily platform} and {\ttfamily arch} from the host, in this example, {\ttfamily darwin} platform and {\ttfamily x64} arch.
\item Download the darwin x64 build of Electron 1.\+4.\+15 (and cache the downloads in {\ttfamily $\sim$/.electron})
\item Build the OS X {\ttfamily Foo Bar.\+app}
\item Place {\ttfamily Foo Bar.\+app} in {\ttfamily foobar/\+Foo Bar-\/darwin-\/x64/} (since an {\ttfamily out} directory was not specified, it used the current working directory)
\end{DoxyItemize}

The file structure now looks like\+:


\begin{DoxyCode}
foobar
├── Foo Bar-darwin-x64
│   ├── Foo Bar.app
│   │   └── […Mac app contents…]
│   ├── LICENSE
│   └── version
├── […other application bundles, like "Foo Bar-win32-x64" (sans quotes)…]
├── package.json
├── index.html
├── […other files, like LICENSE…]
└── script.js
\end{DoxyCode}


The {\ttfamily Foo Bar.\+app} folder generated can be executed by a system running OS X, which will start the packaged Electron app. This is also true of the Windows x64 build on a system running a new enough version of Windows for a 64-\/bit system (via {\ttfamily Foo Bar-\/win32-\/x64/\+Foo Bar.\+exe}), and so on.

\subsubsection*{\href{https://github.com/electron-userland/electron-packager/blob/master/docs/api.md}{\tt https\+://github.\+com/electron-\/userland/electron-\/packager/blob/master/docs/api.\+md} \char`\"{}\+Programmatic A\+P\+I\char`\"{}}

\subsection*{Related}


\begin{DoxyItemize}
\item \href{https://www.npmjs.com/package/electron-forge}{\tt Electron Forge} -\/ creates, builds, and distributes modern Electron applications
\item \href{https://github.com/Urucas/electron-packager-interactive}{\tt electron-\/packager-\/interactive} -\/ an interactive C\+LI for electron-\/packager
\item \href{https://www.npmjs.com/package/electron-packager-plugin-non-proprietary-codecs-ffmpeg}{\tt electron-\/packager-\/plugin-\/non-\/proprietary-\/codecs-\/ffmpeg} -\/ replaces the normal version of F\+Fmpeg in Electron with a version without proprietary codecs
\item \href{https://github.com/electron/electron-rebuild}{\tt electron-\/rebuild} -\/ rebuild native Node.\+js modules against the packaged Electron version
\item \href{https://github.com/sindresorhus/grunt-electron}{\tt grunt-\/electron} -\/ grunt plugin for electron-\/packager 
\end{DoxyItemize}