Ever noticed how sometimes Java\+Script doesn’t display numbers the way you expect? Like, you tried to print tenths with a simple loop\+:


\begin{DoxyCode}
for (var i = 0; i < 10; i++) \{
  console.log(0.1 * i);
\}
\end{DoxyCode}


And you got this\+:


\begin{DoxyCode}
0
0.1
0.2
0.30000000000000004
0.4
0.5
0.6000000000000001
0.7000000000000001
0.8
0.9
\end{DoxyCode}


Welcome to \href{https://en.wikipedia.org/wiki/Double-precision_floating-point_format}{\tt binary floating point}! ಠ\+\_\+ಠ

Yet rounding error is not the only reason to customize number formatting. A table of numbers should be formatted consistently for comparison; above, 0.\+0 would be better than 0. Large numbers should have grouped digits (e.\+g., 42,000) or be in scientific or metric notation (4.\+2e+4, 42k). Currencies should have fixed precision (\$3.\+50). Reported numerical results should be rounded to significant digits (4021 becomes 4000). Number formats should appropriate to the reader’s locale (42.\+000,00 or 42,000.\+00). The list goes on.

Formatting numbers for human consumption is the purpose of d3-\/format, which is modeled after Python 3’s \href{https://docs.python.org/3/library/string.html#format-specification-mini-language}{\tt format specification mini-\/language} (\href{https://www.python.org/dev/peps/pep-3101/}{\tt P\+EP 3101}). Revisiting the example above\+:


\begin{DoxyCode}
var f = d3.format(".1f");
for (var i = 0; i < 10; i++) \{
  console.log(f(0.1 * i));
\}
\end{DoxyCode}


Now you get this\+:


\begin{DoxyCode}
0.0
0.1
0.2
0.3
0.4
0.5
0.6
0.7
0.8
0.9
\end{DoxyCode}


But d3-\/format is much more than an alias for \href{https://developer.mozilla.org/en-US/docs/Web/JavaScript/Reference/Global_Objects/Number/toFixed}{\tt number.\+to\+Fixed}! A few more examples\+:


\begin{DoxyCode}
d3.format(".0%")(0.123);  // rounded percentage, "12%"
d3.format("($.2f")(-3.5); // localized fixed-point currency, "(£3.50)"
d3.format("+20")(42);     // space-filled and signed, "                 +42"
d3.format(".^20")(42);    // dot-filled and centered, ".........42........."
d3.format(".2s")(42e6);   // SI-prefix with two significant digits, "42M"
d3.format("#x")(48879);   // prefixed lowercase hexadecimal, "0xbeef"
d3.format(",.2r")(4223);  // grouped thousands with two significant digits, "4,200"
\end{DoxyCode}


See \href{#locale_format}{\tt {\itshape locale}.format} for a detailed specification, and try running \href{#formatSpecifier}{\tt d3.\+format\+Specifier} on the above formats to decode their meaning.

\subsection*{Installing}

If you use N\+PM, {\ttfamily npm install d3-\/format}. Otherwise, download the \href{https://github.com/d3/d3-format/releases/latest}{\tt latest release}. You can also load directly from \href{https://d3js.org}{\tt d3js.\+org}, either as a \href{https://d3js.org/d3-format.v1.min.js}{\tt standalone library} or as part of \href{https://github.com/d3/d3}{\tt D3 4.\+0}. A\+MD, Common\+JS, and vanilla environments are supported. In vanilla, a {\ttfamily d3} global is exported\+:


\begin{DoxyCode}
<script src="https://d3js.org/d3-format.v1.min.js"></script>
<script>

var format = d3.format(".2s");

</script>
\end{DoxyCode}


Locale files are hosted on \href{https://unpkg.com/}{\tt unpkg} and can be loaded using \href{https://github.com/d3/d3-request/blob/master/README.md#json}{\tt d3.\+json}. For example, to set Russian as the default locale\+:


\begin{DoxyCode}
d3.json("https://unpkg.com/d3-format@1/locale/ru-RU.json", function(error, locale) \{
  if (error) throw error;

  d3.formatDefaultLocale(locale);

  var format = d3.format("$,");

  console.log(format(1234.56)); // 1 234,56 руб.
\});
\end{DoxyCode}


\href{https://tonicdev.com/npm/d3-format}{\tt Try d3-\/format in your browser.}

\subsection*{A\+PI Reference}

\label{_format}%
\# d3.{\bfseries format}({\itshape specifier}) \href{https://github.com/d3/d3-format/blob/master/src/defaultLocale.js#L4}{\tt $<$$>$}

An alias for \href{#locale_format}{\tt {\itshape locale}.format} on the \href{#formatDefaultLocale}{\tt default locale}.

\label{_formatPrefix}%
\# d3.{\bfseries format\+Prefix}({\itshape specifier}, {\itshape value}) \href{https://github.com/d3/d3-format/blob/master/src/defaultLocale.js#L5}{\tt $<$$>$}

An alias for \href{#locale_formatPrefix}{\tt {\itshape locale}.format\+Prefix} on the \href{#formatDefaultLocale}{\tt default locale}.

\label{_locale_format}%
\# {\itshape locale}.{\bfseries format}({\itshape specifier}) \href{https://github.com/d3/d3-format/blob/master/src/locale.js#L18}{\tt $<$$>$}

Returns a new format function for the given string {\itshape specifier}. The returned function takes a number as the only argument, and returns a string representing the formatted number. The general form of a specifier is\+:


\begin{DoxyCode}
[​[fill]align][sign][symbol][0][width][,][.precision][type]
\end{DoxyCode}


The {\itshape fill} can be any character. The presence of a fill character is signaled by the {\itshape align} character following it, which must be one of the following\+:


\begin{DoxyItemize}
\item {\ttfamily $>$} -\/ Forces the field to be right-\/aligned within the available space. (Default behavior).
\item {\ttfamily $<$} -\/ Forces the field to be left-\/aligned within the available space.
\item {\ttfamily $^\wedge$} -\/ Forces the field to be centered within the available space.
\item {\ttfamily =} -\/ like {\ttfamily $>$}, but with any sign and symbol to the left of any padding.
\end{DoxyItemize}

The {\itshape sign} can be\+:


\begin{DoxyItemize}
\item {\ttfamily -\/} -\/ nothing for zero or positive and a minus sign for negative. (Default behavior.)
\item {\ttfamily +} -\/ a plus sign for zero or positive and a minus sign for negative.
\item {\ttfamily (} -\/ nothing for zero or positive and parentheses for negative.
\item {\ttfamily  } (space) -\/ a space for zero or positive and a minus sign for negative.
\end{DoxyItemize}

The {\itshape symbol} can be\+:


\begin{DoxyItemize}
\item {\ttfamily \$} -\/ apply currency symbols per the locale definition.
\item {\ttfamily \#} -\/ for binary, octal, or hexadecimal notation, prefix by {\ttfamily 0b}, {\ttfamily 0o}, or {\ttfamily 0x}, respectively.
\end{DoxyItemize}

The {\itshape zero} ({\ttfamily 0}) option enables zero-\/padding; this implicitly sets {\itshape fill} to {\ttfamily 0} and {\itshape align} to {\ttfamily =}. The {\itshape width} defines the minimum field width; if not specified, then the width will be determined by the content. The {\itshape comma} ({\ttfamily ,}) option enables the use of a group separator, such as a comma for thousands.

Depending on the {\itshape type}, the {\itshape precision} either indicates the number of digits that follow the decimal point (types {\ttfamily f} and {\ttfamily \%}), or the number of significant digits (types {\ttfamily ​}, {\ttfamily e}, {\ttfamily g}, {\ttfamily r}, {\ttfamily s} and {\ttfamily p}). If the precision is not specified, it defaults to 6 for all types except {\ttfamily ​} (none), which defaults to 12. Precision is ignored for integer formats (types {\ttfamily b}, {\ttfamily o}, {\ttfamily d}, {\ttfamily x}, {\ttfamily X} and {\ttfamily c}). See \href{#precisionFixed}{\tt precision\+Fixed} and \href{#precisionRound}{\tt precision\+Round} for help picking an appropriate precision.

The available {\itshape type} values are\+:


\begin{DoxyItemize}
\item {\ttfamily e} -\/ exponent notation.
\item {\ttfamily f} -\/ fixed point notation.
\item {\ttfamily g} -\/ either decimal or exponent notation, rounded to significant digits.
\item {\ttfamily r} -\/ decimal notation, rounded to significant digits.
\item {\ttfamily s} -\/ decimal notation with an \href{#locale_formatPrefix}{\tt SI prefix}, rounded to significant digits.
\item {\ttfamily \%} -\/ multiply by 100, and then decimal notation with a percent sign.
\item {\ttfamily p} -\/ multiply by 100, round to significant digits, and then decimal notation with a percent sign.
\item {\ttfamily b} -\/ binary notation, rounded to integer.
\item {\ttfamily o} -\/ octal notation, rounded to integer.
\item {\ttfamily d} -\/ decimal notation, rounded to integer.
\item {\ttfamily x} -\/ hexadecimal notation, using lower-\/case letters, rounded to integer.
\item {\ttfamily X} -\/ hexadecimal notation, using upper-\/case letters, rounded to integer.
\item {\ttfamily c} -\/ converts the integer to the corresponding unicode character before printing.
\item {\ttfamily ​} (none) -\/ like {\ttfamily g}, but trim insignificant trailing zeros.
\end{DoxyItemize}

The type {\ttfamily n} is also supported as shorthand for {\ttfamily ,g}. For the {\ttfamily g}, {\ttfamily n} and {\ttfamily ​} (none) types, decimal notation is used if the resulting string would have {\itshape precision} or fewer digits; otherwise, exponent notation is used. For example\+:


\begin{DoxyCode}
d3.format(".2")(42);  // "42"
d3.format(".2")(4.2); // "4.2"
d3.format(".1")(42);  // "4e+1"
d3.format(".1")(4.2); // "4"
\end{DoxyCode}


\label{_locale_formatPrefix}%
\# {\itshape locale}.{\bfseries format\+Prefix}({\itshape specifier}, {\itshape value}) \href{https://github.com/d3/d3-format/blob/master/src/locale.js#L127}{\tt $<$$>$}

Equivalent to \href{#locale_format}{\tt {\itshape locale}.format}, except the returned function will convert values to the units of the appropriate \href{https://en.wikipedia.org/wiki/Metric_prefix#List_of_SI_prefixes}{\tt SI prefix} for the specified numeric reference {\itshape value} before formatting in fixed point notation. The following prefixes are supported\+:


\begin{DoxyItemize}
\item {\ttfamily y} -\/ yocto, 10⁻²⁴
\item {\ttfamily z} -\/ zepto, 10⁻²¹
\item {\ttfamily a} -\/ atto, 10⁻¹⁸
\item {\ttfamily f} -\/ femto, 10⁻¹⁵
\item {\ttfamily p} -\/ pico, 10⁻¹²
\item {\ttfamily n} -\/ nano, 10⁻⁹
\item {\ttfamily µ} -\/ micro, 10⁻⁶
\item {\ttfamily m} -\/ milli, 10⁻³
\item {\ttfamily ​} (none) -\/ 10⁰
\item {\ttfamily k} -\/ kilo, 10³
\item {\ttfamily M} -\/ mega, 10⁶
\item {\ttfamily G} -\/ giga, 10⁹
\item {\ttfamily T} -\/ tera, 10¹²
\item {\ttfamily P} -\/ peta, 10¹⁵
\item {\ttfamily E} -\/ exa, 10¹⁸
\item {\ttfamily Z} -\/ zetta, 10²¹
\item {\ttfamily Y} -\/ yotta, 10²⁴
\end{DoxyItemize}

Unlike \href{#locale_format}{\tt {\itshape locale}.format} with the {\ttfamily s} format type, this method returns a formatter with a consistent SI prefix, rather than computing the prefix dynamically for each number. In addition, the {\itshape precision} for the given {\itshape specifier} represents the number of digits past the decimal point (as with {\ttfamily f} fixed point notation), not the number of significant digits. For example\+:


\begin{DoxyCode}
var f = d3.formatPrefix(",.0", 1e-6);
f(0.00042); // "420µ"
f(0.0042); // "4,200µ"
\end{DoxyCode}


This method is useful when formatting multiple numbers in the same units for easy comparison. See \href{#precisionPrefix}{\tt precision\+Prefix} for help picking an appropriate precision, and \href{http://bl.ocks.org/mbostock/9764126}{\tt bl.\+ocks.\+org/9764126} for an example.

\label{_formatSpecifier}%
\# d3.{\bfseries format\+Specifier}({\itshape specifier}) \href{https://github.com/d3/d3-format/blob/master/src/formatSpecifier.js}{\tt $<$$>$}

Parses the specified {\itshape specifier}, returning an object with exposed fields that correspond to the \href{#locale_format}{\tt format specification mini-\/language} and a to\+String method that reconstructs the specifier. For example, {\ttfamily format\+Specifier(\char`\"{}s\char`\"{})} returns\+:


\begin{DoxyCode}
\{
  "fill": " ",
  "align": ">",
  "sign": "-",
  "symbol": "",
  "zero": false,
  "width": undefined,
  "comma": false,
  "precision": 6,
  "type": "s"
\}
\end{DoxyCode}


This method is useful for understanding how format specifiers are parsed and for deriving new specifiers. For example, you might compute an appropriate precision based on the numbers you want to format using \href{#precisionFixed}{\tt precision\+Fixed} and then create a new format\+:


\begin{DoxyCode}
var s = d3.formatSpecifier("f");
s.precision = precisionFixed(0.01);
var f = d3.format(s);
f(42); // "42.00";
\end{DoxyCode}


\label{_precisionFixed}%
\# d3.{\bfseries precision\+Fixed}({\itshape step}) \href{https://github.com/d3/d3-format/blob/master/src/precisionFixed.js}{\tt $<$$>$}

Returns a suggested decimal precision for fixed point notation given the specified numeric {\itshape step} value. The {\itshape step} represents the minimum absolute difference between values that will be formatted. (This assumes that the values to be formatted are also multiples of {\itshape step}.) For example, given the numbers 1, 1.\+5, and 2, the {\itshape step} should be 0.\+5 and the suggested precision is 1\+:


\begin{DoxyCode}
var p = d3.precisionFixed(0.5),
    f = d3.format("." + p + "f");
f(1);   // "1.0"
f(1.5); // "1.5"
f(2);   // "2.0"
\end{DoxyCode}


Whereas for the numbers 1, 2 and 3, the {\itshape step} should be 1 and the suggested precision is 0\+:


\begin{DoxyCode}
var p = d3.precisionFixed(1),
    f = d3.format("." + p + "f");
f(1); // "1"
f(2); // "2"
f(3); // "3"
\end{DoxyCode}


Note\+: for the {\ttfamily \%} format type, subtract two\+:


\begin{DoxyCode}
var p = Math.max(0, d3.precisionFixed(0.05) - 2),
    f = d3.format("." + p + "%");
f(0.45); // "45%"
f(0.50); // "50%"
f(0.55); // "55%"
\end{DoxyCode}


\label{_precisionPrefix}%
\# d3.{\bfseries precision\+Prefix}({\itshape step}, {\itshape value}) \href{https://github.com/d3/d3-format/blob/master/src/precisionPrefix.js}{\tt $<$$>$}

Returns a suggested decimal precision for use with \href{#locale_formatPrefix}{\tt {\itshape locale}.format\+Prefix} given the specified numeric {\itshape step} and reference {\itshape value}. The {\itshape step} represents the minimum absolute difference between values that will be formatted, and {\itshape value} determines which SI prefix will be used. (This assumes that the values to be formatted are also multiples of {\itshape step}.) For example, given the numbers 1.\+1e6, 1.\+2e6, and 1.\+3e6, the {\itshape step} should be 1e5, the {\itshape value} could be 1.\+3e6, and the suggested precision is 1\+:


\begin{DoxyCode}
var p = d3.precisionPrefix(1e5, 1.3e6),
    f = d3.formatPrefix("." + p, 1.3e6);
f(1.1e6); // "1.1M"
f(1.2e6); // "1.2M"
f(1.3e6); // "1.3M"
\end{DoxyCode}


\label{_precisionRound}%
\# d3.{\bfseries precision\+Round}({\itshape step}, {\itshape max}) \href{https://github.com/d3/d3-format/blob/master/src/precisionRound.js}{\tt $<$$>$}

Returns a suggested decimal precision for format types that round to significant digits given the specified numeric {\itshape step} and {\itshape max} values. The {\itshape step} represents the minimum absolute difference between values that will be formatted, and the {\itshape max} represents the largest absolute value that will be formatted. (This assumes that the values to be formatted are also multiples of {\itshape step}.) For example, given the numbers 0.\+99, 1.\+0, and 1.\+01, the {\itshape step} should be 0.\+01, the {\itshape max} should be 1.\+01, and the suggested precision is 3\+:


\begin{DoxyCode}
var p = d3.precisionRound(0.01, 1.01),
    f = d3.format("." + p + "r");
f(0.99); // "0.990"
f(1.0);  // "1.00"
f(1.01); // "1.01"
\end{DoxyCode}


Whereas for the numbers 0.\+9, 1.\+0, and 1.\+1, the {\itshape step} should be 0.\+1, the {\itshape max} should be 1.\+1, and the suggested precision is 2\+:


\begin{DoxyCode}
var p = d3.precisionRound(0.1, 1.1),
    f = d3.format("." + p + "r");
f(0.9); // "0.90"
f(1.0); // "1.0"
f(1.1); // "1.1"
\end{DoxyCode}


Note\+: for the {\ttfamily e} format type, subtract one\+:


\begin{DoxyCode}
var p = Math.max(0, d3.precisionRound(0.01, 1.01) - 1),
    f = d3.format("." + p + "e");
f(0.01); // "1.00e-2"
f(1.01); // "1.01e+0"
\end{DoxyCode}


\subsubsection*{Locales}

\label{_formatLocale}%
\# d3.{\bfseries format\+Locale}({\itshape definition}) \href{https://github.com/d3/d3-format/blob/master/src/locale.js}{\tt $<$$>$}

Returns a {\itshape locale} object for the specified {\itshape definition} with \href{#locale_format}{\tt {\itshape locale}.format} and \href{#locale_formatPrefix}{\tt {\itshape locale}.format\+Prefix} methods. The {\itshape definition} must include the following properties\+:


\begin{DoxyItemize}
\item {\ttfamily decimal} -\/ the decimal point (e.\+g., {\ttfamily \char`\"{}.\char`\"{}}).
\item {\ttfamily thousands} -\/ the group separator (e.\+g., {\ttfamily \char`\"{},\char`\"{}}).
\item {\ttfamily grouping} -\/ the array of group sizes (e.\+g., {\ttfamily \mbox{[}3\mbox{]}}), cycled as needed.
\item {\ttfamily currency} -\/ the currency prefix and suffix (e.\+g., {\ttfamily \mbox{[}\char`\"{}\$\char`\"{}, \char`\"{}\char`\"{}\mbox{]}}).
\item {\ttfamily numerals} -\/ optional; an array of ten strings to replace the numerals 0-\/9.
\item {\ttfamily percent} -\/ optional; the percent suffix (defaults to {\ttfamily \char`\"{}\%\char`\"{}}).
\end{DoxyItemize}

Note that the {\itshape thousands} property is a misnomer, as the grouping definition allows groups other than thousands.

\label{_formatDefaultLocale}%
\# d3.{\bfseries format\+Default\+Locale}({\itshape definition}) \href{https://github.com/d3/d3-format/blob/master/src/defaultLocale.js}{\tt $<$$>$}

Equivalent to \href{#formatLocale}{\tt d3.\+format\+Locale}, except it also redefines \href{#format}{\tt d3.\+format} and \href{#formatPrefix}{\tt d3.\+format\+Prefix} to the new locale’s \href{#locale_format}{\tt {\itshape locale}.format} and \href{#locale_formatPrefix}{\tt {\itshape locale}.format\+Prefix}. If you do not set a default locale, it defaults to \href{https://github.com/d3/d3-format/blob/master/locale/en-US.json}{\tt U.\+S. English}. 